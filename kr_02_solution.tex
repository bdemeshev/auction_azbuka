\begin{enumerate}

\item
\begin{enumerate}
\item Да, является супермодулярной. Проверяем два свойства:
\begin{enumerate}
\item $ af(\vec{x}) $ — ок.
\item $ f(\vec{x})+g(\vec{x}) $ — ок.
\end{enumerate}

\item Сразу ответ: $ f(a,b)=(n-1)(n-2)C_{n-3}^{4}C_{n-7}^{4}3a^{2}3b^{2}(a^{3}-b^{3})^{4}(b^{3})^{n-11}(1-a^{3})^{4} $.
\end{enumerate}


\item Сразу начнем с чудо-замены, $ b_{1}=b(a) $. Сначала упростим событие $ W_{1} $:\index{чудо-замена}
\begin{equation}
W_{1}=\{Bid_{2}<b(a)\}=\{b(X_{2})<b(a)\}=\{X_{2}<a\}.
\end{equation}

А теперь прибыль:
\begin{multline}
\pi(x,b(a))=\E(X_{1}X_{2}1_{W_{1}}|X_{1}=x)-b(a)\E(1_{W_{1}}|X_{1}=x)=\\
=x\E(X_{2}1_{X_{2}<a}|X_{1}=x)-b(a)\E(1_{X_{2}<a}|X_{1}=x)=\\
=x\int_{0}^{a}x_{2}\frac{f(x,x_{2})}{f(x)} \, dx_{2}-b(a)\int_{0}^{a}\frac{f(x,x_{2})}{f(x)} \, dx_{2}.
\end{multline}


Сокращаем на $ f(x) $ и берём производную по $ a $:
\begin{equation}
\frac{\partial \pi}{\partial a}=xaf(x,a)-b'(a)\int_{0}^{a}f(x,x_{2}) \, dx_{2}-b(a)f(x,a)=0.
\end{equation}


Мы хотим, чтобы оптимальной стратегий первого была $ b_{1}=b(x) $, то есть чтобы $ a=x $:
\begin{equation}
\frac{\partial \pi}{\partial a}=x^{2}f(x,x)-b'(x)\int_{0}^{x}f(x,x_{2})dx_{2}-b(x)f(x,x)=0.
\end{equation}

Остаётся подставить:
\begin{equation}
\begin{array}{c}
f(x,x)=\frac{7}{8}+\frac{1}{2}x^{2}; \\
\int_{0}^{x}f(x,x_{2})dx_{2}=\frac{7}{8}x+\frac{1}{4}x^{3}.
\end{array}
\end{equation}


\item  Никакой разницы с кнопочным аукционом в данном случае нет. Игроков-то всего два! Значит, равновесие Нэша имеет вид $ b(x)=x^{2} $. \index{аукцион!кнопочный}

Доказательство.

Если второй игрок использует такую стратегию и первый выигрывает аукцион, то его прибыль равна:
\begin{equation}
X_{1}X_{2}-X_{2}^{2}=(X_{1}-X_{2})X_{2}.
\end{equation}
Мы видим, что прибыль положительна, только если $ X_{1}>X_{2} $. Использование первым игроком функции $ b(x)=x^{2} $ будет обеспечивать его выигрыш только в ситуации $ X_{1}>X_{2} $, значит, это и есть равновесие Нэша.


\item Стратегия описывается двумя функциями: $ b^{3}(x)=2x^{2} $. Если при использовании такой стратегии игрок вышел на цене $ p $, значит, его $ x=\sqrt{p/2} $. Получаем $ b^{2}(x,p)=x^{2}+x\sqrt{p/2}$. Доказательство аналогично лекции:

Если остальные игроки используют эти стратегии и первый выигрывает аукцион, то его выигрыш равен
\begin{multline}
X_{1}(X_{2}+X_{3})-b^{2}(Y_{1},b^{3}(Y_{2}))=\\
=X_{1}(Y_{1}+Y_{2})-(Y_{1}^{2}+Y_{1}Y_{2})=(X_{1}-Y_{1})(Y_{1}+Y_{2}).
\end{multline}
Мы видим, что выигрыш положительный, только если $ X_{1}>Y_{1} $. Использование первым игроком правил $ b^{3}() $ и $ b^{2}() $ приводит к выигрышу, только если $ X_{1}>Y_{1} $, значит, это и есть равновесие.


\item  Пример равновесия Нэша. Обозначим максимальную ценность $ v_{max} $, а вторую по величине ценность — $v_{sec}$. Возможно, эти две ценности совпадают. Те игроки, чья ценность $ v_{i}<v_{max} $, делают ставку $ b_{i}=v_{i}-1 $. Если лидер один, то он делает ставку $ b=v_{sec} $, иначе каждый лидер делает ставку $ b=v_{max}-1 $.


\end{enumerate}
