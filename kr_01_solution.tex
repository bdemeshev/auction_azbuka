

\begin{enumerate}
\item

\begin{equation}
\Cov(Pay_{1}, Pay_{2})=\E(Pay_{1}\cdot Pay_{2})-\E(Pay_{1})\E(Pay_{2}).
\end{equation}

На вычитаемое способ аукциона влиять не может в силу теоремы об одинаковой доходности. Сосредоточимся на $\E(Pay_{1}\cdot Pay_{2})$. В аукционе первой цены никакие два игрока не могут платить одновременно, поэтому произведение выплат всегда равно нулю, то есть $ \E(Pay_{1}\cdot Pay_{2})=0 $. В аукционе «Платят все»\index{аукцион!платят все} произведение выплат строго положительно, поэтому $ \E(Pay_{1}Pay_{2})>0 $. Значит, способ проведения аукциона может влиять на ковариацию.



\item  Ожидаемая прибыль:

\begin{multline}
\pi(x,b_{1})=(x-\E(b(X_{2})|b(X_{2})<b_{1}))\cdot \P(b(X_{2})<b_{1})+\\
+b_{1}\P(b(X_{2})>b_{1}).
\end{multline}

После чудо-замены $ b_{1}=b(a) $ и упрощения вероятностей:

\begin{equation}
\pi=(x-\E(b(X_{2})|X_{2}<a))\P(X_{2}<a)+b(a)(1-\P(X_{2}<a)),
\end{equation}

и
\begin{equation}
\pi=xF(a)-\E(b(X_{2})\cdot 1_{X_{2}<a}))+b(a)(1-F(a)).
\end{equation}

В записи с интегралом:
\begin{equation}
\pi=xF(a)-\int_{0}^{a}b(t)f(t) \, dt+b(a)(1-F(a)).
\end{equation}

Приравниваем производную к нулю:
\begin{equation}
xf(a)-b(a)f(a)-b(a)f(a)+b'(a)(1-F(a))=0.
\end{equation}

Для случая равномерного распределения:
\begin{equation}
x-2b(x)+b'(x)(1-x)=0.
\end{equation}

Подбором коэффициентов находим линейное решение:
\begin{equation}
b(x)=\frac{1}{3}x+\frac{1}{6}.
\end{equation}


\item Составляем табличку и видим, что стратегия $ b_{1}=X_{1} $ нестрого доминирует остальные стратегии.


\item  Составляем табличку и видим, что стратегия $ b_{1}=X_{1} $ нестрого доминирует остальные стратегии.

\item Предположим, что стратегия имеет вид\index{аукцион!с дискретными ставками}: если ценность ниже порога $ x^{*} $, то делать ставку 0, иначе делать ставку $ 0.5 $.

Осталось найти $ x^{*} $.

Допустим, что второй игрок использует такую стратегию.

Если первый сделает ставку ноль, то его ожидаемый выигрыш будет равен:
\begin{equation}
\pi(x,0)=x^{*}\frac{1}{2}x.
\end{equation}

Если первый сделает ставку $0.5$, то его ожидаемый выигрыш будет равен:
\begin{equation}
\pi(x,0.5)=(x-0.5)x^{*}+\frac{1}{2}(x-0.5)(1-x^{*})=\frac{1}{2}(x-0.5)(x^{*}+1).
\end{equation}

Находим условие, при котором $ \pi(x,0.5)>\pi(x,0) $, получаем:

\begin{equation}
x>\frac{1}{2}(x^{*}+1).
\end{equation}

Значит, правая часть представляет собой $ x^{*} $. Решаем уравнение $x^{*}=\frac{1}{2}(x^{*}+1)  $, получаем $ x^{*}=1 $, то есть вне зависимости от ценности игрокам имеет смысл ставить 0.


\end{enumerate}
