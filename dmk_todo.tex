
тудушки:
- вступление
- впечатление о прошедшем курсе (про трейлеры контрольный)
- расширить алфавитный указатель
- вытащить на свет задачи в комментариях
- список литературы с комментариями
- убрать нумерацию формул?








%\item Есть $ 3 $ города. Рядом с одним из них нужно построить мусоросжигательный завод. Жители города рядом с которым будет построен завод получат отрицательную полезность $ U_{i}=-X_{i} $. Остальные получат полезность 0. Величины $ X_{i}\sim U[0;1] $ и независимы. Каждый город знает своё $ X_{i} $. Города одновременно называют требуемую компенсацию $ b_{i} $. Завод строится у того города, у которого $ b_{i} $ меньше. Остальные города выплачивают компенсацию поровну.
%\begin{enumerate}
%\item Найдите равновесие Нэша
%\item Как надо изменить этот механизм, чтобы он стал правдивым?
%\end{enumerate}



%\item Есть один покупатель и один продавец. Ценности товара: $ X_{1} $ — для покупателя, $ X_{2} $ — для продавца. Величины $ X_{i} $ независимы и равномерны на $ [0;1] $.
%\begin{enumerate}
%\item Опишите механизм VCG применительно к этой задаче. то есть предполагается, что игроки объявляют числа $ b_{i}\in [0;1] $ и механизм должен определять, кому отдать товар и какие платежи должны сделать игроки в зависимости от $ b_{i} $.
%\item Каков средний баланс механизма VCG в этой задаче?
%\item Предположим, что вместо VCG используется такой механизм: игроки одновременно называют желаемые цены, $ b_{1} $ и $ b_{2} $. Если $ b_{1}>b_{2} $, то обмен происходит по цене $ 0.5(b_{1}+b_{2}) $. Найдите равновесие Нэша.
%\item Верно ли, что при втором механизме обмен происходит если и только если $ X_{1}>X_{2} $?
%\end{enumerate}

%Подсказка: равновесие Нэша будет в линейных стратегиях
