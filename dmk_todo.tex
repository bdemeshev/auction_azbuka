2.	Flavio M. Menezes, Paulo K. Monteiro, An introduction to auction theory
2.	Vijay Krishna, Auction theory
2.	Paul Klemperer, Auctions Theory and Practice
2.	Paul Milgrom, Putting auction theory to work
книжка на русском?
книжка от Захарова

тудушки:
- вступление
- разделить контрольные и решения
- формат дмк
- впечатление о прошедшем курсе (про трейлеры контрольный)



\begin{mydef}
Игрок не склонен к риску, если его функция полезности $ u() $ удовлетворяет условиями $ u'>0 $ и $ u''\leq 0 $.
\end{mydef}

Для удобства мы будем считать, что $ u(0)=0 $.

Аукцион второй цены. Что изменится, если агенты будут несклонны к риску?

По сравнению со случаем нейтральности к риску в табличках произойдет только одно изменение: вместо $ X_{1}-b_{1} $ в табличке будет стоять $ u(X_{1}-b_{1}) $. Там, где был ноль, там ноль и останется: $ u(0)=0 $. В силу того, что $ u'>0 $ знак $ u(X_{1}-b_{1}) $ совпадает со знаком $ X_{1}-b_{1} $. Поэтому стратегия $ b_{1}=X_{1} $ по прежнему нестрого доминирует все остальные стратегии. Кстати, мы не использовали  тот факт, что $u''\leq 0 $, а значит тоже равновесие Нэша остаётся и для случая склонных к риску агентов.






%\item Есть $ 3 $ города. Рядом с одним из них нужно построить мусоросжигательный завод. Жители города рядом с которым будет построен завод получат отрицательную полезность $ U_{i}=-X_{i} $. Остальные получат полезность 0. Величины $ X_{i}\sim U[0;1] $ и независимы. Каждый город знает своё $ X_{i} $. Города одновременно называют требуемую компенсацию $ b_{i} $. Завод строится у того города, у которого $ b_{i} $ меньше. Остальные города выплачивают компенсацию поровну.
%\begin{enumerate}
%\item Найдите равновесие Нэша
%\item Как надо изменить этот механизм, чтобы он стал правдивым?
%\end{enumerate}



%\item Есть один покупатель и один продавец. Ценности товара: $ X_{1} $ — для покупателя, $ X_{2} $ — для продавца. Величины $ X_{i} $ независимы и равномерны на $ [0;1] $.
%\begin{enumerate}
%\item Опишите механизм VCG применительно к этой задаче. то есть предполагается, что игроки объявляют числа $ b_{i}\in [0;1] $ и механизм должен определять, кому отдать товар и какие платежи должны сделать игроки в зависимости от $ b_{i} $.
%\item Каков средний баланс механизма VCG в этой задаче?
%\item Предположим, что вместо VCG используется такой механизм: игроки одновременно называют желаемые цены, $ b_{1} $ и $ b_{2} $. Если $ b_{1}>b_{2} $, то обмен происходит по цене $ 0.5(b_{1}+b_{2}) $. Найдите равновесие Нэша.
%\item Верно ли, что при втором механизме обмен происходит если и только если $ X_{1}>X_{2} $?
%\end{enumerate}

%Hint: равновесие Нэша будет в линейных стратегиях
