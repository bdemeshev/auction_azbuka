2.	Flavio M. Menezes, Paulo K. Monteiro, An introduction to auction theory
2.	Vijay Krishna, Auction theory
2.	Paul Klemperer, Auctions Theory and Practice
2.	Paul Milgrom, Putting auction theory to work
книжка на русском?
книжка от Захарова

тудушки:
- вступление
- ---
- ё!
- разделить контрольные и решения
- xelatex
- формат дмк
- впечатление о прошедшем курсе (про трейлеры контрольный)



\begin{mydef}
Игрок не склонен к риску, если его функция полезности $ u() $ удовлетворяет условиями $ u'>0 $ и $ u''\leq 0 $.
\end{mydef}

Для удобства мы будем считать, что $ u(0)=0 $.

Аукцион второй цены. Что изменится, если агенты будут несклонны к риску?

По сравнению со случаем нейтральности к риску в табличках произойдет только одно изменение: вместо $ X_{1}-b_{1} $ в табличке будет стоять $ u(X_{1}-b_{1}) $. Там, где был ноль, там ноль и останется: $ u(0)=0 $. В силу того, что $ u'>0 $ знак $ u(X_{1}-b_{1}) $ совпадает со знаком $ X_{1}-b_{1} $. Поэтому стратегия $ b_{1}=X_{1} $ по прежнему нестрого доминирует все остальные стратегии. Кстати, мы не использовали  тот факт, что $u''\leq 0 $, а значит тоже равновесие Нэша остаётся и для случая склонных к риску агентов.
