\begin{enumerate}


\item Пусть $  V $ — общая ценность товара для двух игроков, равномерна на $ [0;1] $. Величины $ R_{1} $ и $ R_{2} $ — независимы между собой и с $ V $ и равномерны на $ [0.5;1.5] $. Игроки получают сигналы $ X_{i}=V\cdot R_{i} $.
\begin{enumerate}
\item Найдите совместную функцию плотности $ X_{1} $ и $ X_{2} $. Верно ли, что $ X_{1} $ и $ X_{2} $ аффилированны?
\item Найдите $ v(x,y)=\E(V|X_{1}=x,Y_{1}=y) $
\item Найдите совместную функцию плотности $ X_{1} $ и $ Y_{1} $, $ g(x,y) $
\end{enumerate}


\item На аукционе продаётся картина, которая равновероятно является «Джокондой» Леонардо да Винчи или её подделкой. За неё торгуются $ n $ покупателей. Ценность картины для всех покупателей одинакова, $ V_{1}=V_{2}=\ldots=V_{n}=V $ и равна 1, если это оригинал и 0, если подделка.

Если $ V=0 $, то сигналы $ X_{i} $ условно независимы и равномерны на $ [0;1] $. Если $ V=1 $, то сигналы $ X_{i} $ условно независимы и имеют функцию плотности $ f(x|V=1)=2x $ при  $x\in [0;1] $
\begin{enumerate}
\item Найдите совместную функцию плотности всех $ X_{i} $. Верно ли, что все $ X_{i} $ аффилированны?
\item Найдите $ v(x,y)=\E(V|X_{1}=x,Y_{1}=y) $
\item Найдите совместную функцию плотности $ X_{1} $ и $ Y_{1} $, $ g(x,y) $
\end{enumerate}



\item На аукционе второй цены присутствуют $ n $ покупателей. Ценности совпадают с сигналами, $ V_{i}=X_{i} $; сигналы $ X_{i} $ независимы и равномерны на $ [0;1] $. На аукционе продаётся $k$ одинаковых чудо-швабр, $ 1<k<n $. Каждому покупателю нужна только одна чудо-швабра. Покупатели одновременно делают свои ставки. Чудо-швабры достаются по одной каждому из $ k $ покупателей с самыми высокими ставками. Каждый из $ k $ победителей платит организатору наибольшую проигравшую ставку.

Найдите равновесие Нэша.

\item На аукционе первой цены присутствуют $ n $ покупателей. Ценности совпадают с сигналами, $ V_{i}=X_{i} $; сигналы $ X_{i} $ независимы и равномерны на $ [0;1] $. На аукционе продаётся $k$ одинаковых чудо-швабр, $ 1<k<n $. Каждому покупателю нужна только одна чудо-швабра. Покупатели одновременно делают свои ставки. Чудо-швабры достаются по одной каждому из $ k $ покупателей с самыми высокими ставками. Эти $ k $ победителей платят свои ставки организатору.

Найдите дифференциальное уравнение, которому удовлетворяет равновесная стратегия.

Hint: Когда продавался один товар, то условие победы первого игрока — $ Y_{1}<a $, а если продаётся $ k $ товаров, то условие победы первого игрока $ Y_{?}<a $.

\item Существуют ли неаффилированные случайные величины $ X_{1} $ и $ X_{2} $ такие, что $Cov(X_{1},X_{2})>0  $?

\end{enumerate}
