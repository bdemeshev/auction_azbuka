\begin{enumerate}

\item
\begin{equation}
MR(x)=x-\frac{1-x^{a}}{ax^{a-1}}=x\left(1+\frac{1}{a}\right)-\frac{1}{ax^{a-1}}
\end{equation}
Даже без производной видно, что функция возрастает. Оптимальным будет аукцион второй цены с резервной ценой:
\begin{equation}
r=\left(\frac{1}{a+1}\right)^{1/a}
\end{equation}

\item Составляем табличку:

\begin{tabular}{c|cccc}
& (Ш,К) & (К,Ш) & (КШ,-) & (-,КШ) \\
\hline
Покупатель 1 & 0.5 & $ 0.8X_{1} $ & $ 0.5+X_{1} $ & 0 \\
Покупатель 2 & $ X_{2} $ & 0.8 & 0 & $ 0.8+0.8X_{2} $ \\
Сумма & $ 0.5+X_{2} $& $ 0.8+0.8X_{1} $ & $ 0.5+X_{1} $ & $ 0.8+0.8X_{2} $ \\
\end{tabular}

Покупатели одновременно декларируют свои значения $ X_{i} $. Мы знаем, что в механизме VCG им будет оптимально говорить правду. Механизм VCG максимизирует сумму полезностей. В данном случае мы замечаем, что $ 0.8+0.8X_{1}>0.5+X_{1} $ при любых $ X_{1} \in [0;1]$. И аналогично для $ X_{2} $. Поэтому правило выбора решения имеет вид:

Если $ X_{1}>X_{2} $, то комод — первому, и шкаф — второму. Если $ X_{1}<X_{2} $, то комод и шкаф — второму.

Осталось правило платежей:

Если $ X_{1}>X_{2} $, то первый платит $ 0.8X_{2} $, а второй — $ 0.5+0.2X_{1} $.

Если $ X_{1}<X_{2} $, то первый платит 0, а второй — $ 0.5+X_{1} $.

Получаем выручку продавца:
\begin{equation}
R=(0.5+0.2X_{1}+0.8X_{2})1_{X_{1}>X_{2}}+(0.5+X_{1})1_{X_{1}<X_{2}}
%=0.5+(0.2X_{1}+0.8X_{2})1_{X_{1}>X_{2}}+X_{1}(1-1_{X_{1}<X_{2}})=0.5+X_{1}+(0.8X_{2}-0.8X_{1})1_{X_{1}>X_{2}}
\end{equation}

Находим:
\begin{equation}
\E(X_{1}1_{X_{1}>X_{2}})=\int_{0}^{1}\int_{0}^{x_{1}}x_{1} \cdot 1 \cdot dx_{2}dx_{1}=1/3
\end{equation}

Аналогично, $ \E(X_{1}1_{X_{1}<X_{2}})=1/6 $.

Получаем, что средняя выручка равна:
\begin{equation}
\E(R)=0.5\cdot \frac{1}{2}+0.2\cdot \frac{1}{3}+0.8\cdot \frac{1}{6}+0.5\cdot \frac{1}{2}+\frac{1}{6}=\frac{13}{15}
\end{equation}

\item  Каждый город одновременно декларирует свой ущерб.

Правило принятия решения: завод построить рядом с городом, сообщившим наименьший ущерб.

Правило платежей: Город рядом с которым строят завод должен получить компенсацию в размере минимума ущербов остальных городов. Остальные города ничего не платят и не получают.

Автоматически получаем, что механизм VCG требует вливания средств извне. так как компенсация равна не самому маленькому ущербу, а ущербу второму по малости, то: компенсация всегда больше ущерба.

Функция плотности: $ p(y)=n\cdot 1\cdot (n-1)y(1-y)^{n-2} $.

Средняя компесация равна (для взятия интеграла можно сделать замену $ z=1-y $):
\begin{equation}
\E(K)=\int_{0}^{1}y\cdot n(n-1)y(1-y)^{n-2}dy=\frac{2}{n+1}
\end{equation}




\item  Поскольку третий игрок ничего не знает, а только видит, сколько игроков осталось в игре, то его стратегия описывается двумя числами, $ b_{3}^{3} $ и $ b_{3}^{2} $. Эти числа говорят, до какой цены давить кнопку, если в игре осталось три и два игрока.

Стратегия первого игрока описывается тремя функциями: $ b_{1}^{3}(x) $ — до какой цены давить кнопку, если в игре три игрока, $b_{1}^{2a}(x,p)$ — до какой цены давить кнопку, если в игре двое: я и второй; $b_{1}^{2b}(x,p)$ — до какой цены давить кнопку, если в игре двое: я и третий. Стратегия второго игрока имеет такой же вид.

Поскольку ценности независимы, то никакой полезной информации от наблюдения за ценами выхода других игроков мы не получаем. Следовательно, стратегию третьего игрока можно заменить одним числом $ b_{3} $, а стратегию первого — одной функцией $b_{1}(x)$.

Получаем аукцион второй цены. Игроки ориентируются на ожидаемый выигрыш. Поэтому с точки зрения третьего игрока его ценность равна 0.5. то есть равновесие Нэша имеет вид $ b_{3}=0.5 $; $ b_{1}(x)=x $; $ b_{2}(x)=x $.


\end{enumerate}
