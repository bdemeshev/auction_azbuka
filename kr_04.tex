\begin{enumerate}

\item На аукционе участвуют $ n $ игроков. Ценности независимы, $ X_{i}=V_{i}$. Пусть функция распределения сигналов имеет вид $ F(x)=x^{a} $ на $ [0;1] $, где $ a $ — это некая константа, $ a\geq 1 $.
\begin{enumerate}
\item Найдите $ MR(x) $. Является ли $ MR(x) $ возрастающей?
\item Постройте оптимальный аукцион. \index{аукцион!оптимальный}
\end{enumerate}

\item Петя переезжает на новую квартиру, поэтому продаёт свои старые шкаф и комод (варианта взять их с собой у него нет).  Потенциальных покупателей двое. Первый покупатель знает значение $ X_{1} $, второй — значение $ X_{2} $. Величины  $ X_{1} $ и  $ X_{2} $ независимы и равномерны на $ [0;1] $. Полезности первого игрока: от шкафа — $ 0.5 $, от комода — $ 0.8X_{1} $, от шкафа и комода — $ 0.5+X_{1} $. Полезности второго игрока: от шкафа — $ 0.8 $, от комода — $ X_{2} $, от шкафа и комода — $ 0.8+0.8X_{2}$. \index{задача!о продаже шкафа и комода}
\begin{enumerate}
\item Чётко опишите механизм VCG применительно к этой задаче.
\item Какова средняя прибыль продавца при использовании механизма VCG?
\end{enumerate}

\item Есть $ n $ городов. Рядом с одним из них нужно построить мусоросжигательный завод\index{задача!о мусоросжигательном заводе}. Жители города, рядом с которым будет построен завод, получат отрицательную полезность $ U_{i}=-X_{i} $. Остальные получат полезность 0. Величины $ X_{i}\sim U[0;1] $ и независимы. Каждый город знает своё $ X_{i} $.
\begin{enumerate}
\item Опишите механизм VCG применительно к этой задаче. То есть предполагается, что игроки объявляют числа $ b_{i}\in [0;1] $ и механизм должен определять, у какого города строить завод и какие платежи должны сделать игроки в зависимости от $ b_{i} $.
\item Выпишите функцию плотности для компенсации, которую получают жители города, рядом с которым будет построен мусоросжигательный завод.
\item Сходится ли баланс у механизма VCG в этом случае? Если нет, то сколько в среднем нужно вложить средств извне в этот механизм?
\item Что больше, компенсация или ущерб от строительства завода в механизме VCG?
\end{enumerate}


\item Кнопочный аукцион и три игрока. Ценности $ V_{1} $, $ V_{2} $ и $ V_{3} $ равномерны на $ [0;1] $ и независимы. Первый и второй игроки знают значение своих ценностей, то есть $ X_{1}=V_{1} $ и $ X_{2}=V_{2} $. А третий игрок не знает значения своей ценности, а знает только закон распределения. \index{аукцион!кнопочный}
\begin{enumerate}
\item Что собой представляют стратегии игроков в этом случае? Почему их можно упростить?
\item Найдите равновесие Нэша.
\end{enumerate}


\end{enumerate}
