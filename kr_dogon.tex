\begin{enumerate}
\item Кнопка «Buy now!»

Два игрока торгуются за товар на кнопочном аукционе с возможностью немедленной покупки товара. Ценности $ X_{i}=V_{i} $ независимы и равномерны на $ [0;1] $. Каждый игрок знает свою ценность $ X_{i} $. Продавец дает игрокам возможность купить товар немедленно по фиксированной цене $ a $. Подробнее. В начале аукциона текущая цена равна нулю и оба игрока жмут на свои кнопки. Текущая цена растет с течением времени. Кто первый отпустил свою кнопку, тот проиграл. В этот момент аукцион заканчивается и победитель получает товар по текущей цене. Но в любой момент пока аукцион не закончился, любой игрок может сказать: «Покупаю по цене $ a $». В этом случае ему достаётся товар по цене $ a $ и аукцион заканчивается.
\begin{enumerate}
\item Что является стратегией игрока на этом аукционе?
\item Найдите равновесие Нэша
\item Изменится ли ожидаемый доход продавца, если аукцион будет проводится по обычным правилам аукциона второй цены? Применима ли теорема об одинаковой доходности?
\end{enumerate}

\item Есть шесть покупателей. У продавца две чудо-швабры. Каждый покупатель хочет только одну чудо-швабру. Продавец решил продавать эти две чудо швабры путем двух последовательных аукционов первой цены, на каждом из которых будет выставляться одна чудо-швабра. Каждый игрок знает ценность чудо-швабры для себя, $ X_{i}=V_{i} $. Ценности независимы и равномерны на $ [0;1] $. Ценности не меняются со временем. Когда проводится второй аукцион известна только ставка, которую сделал победитель первого.

\begin{enumerate}
\item Что является стратегией игрока в этой игре?
\item Найдите равновесие Нэша
\item Верно ли, что средние цены на обоих аукционах равны?
\item Какова вероятность того, что на первом аукционе цена будет больше, чем на втором?
\item Изменится ли ожидаемый доход продавца, если чудо-швабры будут продаваться на двух последовательных аукционах второй цены? Применима ли в данном случае теорема об одинаковой доходности или её небольшая вариация?
\end{enumerate}

\item В моделях аукциона первой и второй цены с независимыми, равномерными на $ [0;1] $ ценностями покупателей сравните дисперсию выигрыша продавца.

%Можно ли сказать сделать какой-то вывод\footnote{Этот вопрос является исследовательским. Возможно он очень легкий или наоборот очень сложный — не знаю. Оценивается любое продвижение вперед.}  для произвольного регулярного распределения ценностей?

\item Может ли цена расти с ростом предложения?

Рассмотрим кнопочный аукцион, в котором участвуют три игрока. Продавец продает две одинаковых чудо-швабры. Каждому игроку нужна только одна чудо-швабра. Ценность чудо-швабры для всех игроков одинакова и равна $ V=X_{1}+X_{2}+X_{3} $. Каждый из игроков знает только своё $ X_{i} $. Сигналы $ X_{i} $ независимы и имеют регулярное распределение $ F(t) $ на отрезке $ [0;1] $. Чудо-швабры по одной достаются тем игрокам, кто отпустил кнопку позже всех. При этом платят они за неё цену, на которой отпустил кнопку самый слабый игрок.

\begin{enumerate}
\item Найдите равновесие Нэша
\item Найдите равновесие Нэша в случае когда продаётся всего одна чудо-швабра
\item Существует ли пример распределения $ F(t) $ при котором средняя цена чудо-швабры в  случае двух чудо-швабр выше, чем в случае одной чудо-швабры?
\end{enumerate}



\item Может ли цена расти с падением спроса?

Рассмотрим кнопочный аукцион, в котором хотят участвовать три игрока. Продается одна чудо-швабра. Ценность чудо-швабры для всех игроков одинакова и равна $ V=X_{1}+X_{2}+X_{3} $. Каждый из трёх потенциальных игроков знает только своё $ X_{i} $. Сигналы $ X_{i} $ независимы и имеют регулярное распределение $ F(t) $ на отрезке $ [0;1] $. Перед началом аукциона продавец случайным образом выбирает одного игрока и говорит: «Ты мне не нравишься, поэтому ты в аукционе не участвуешь». Оставшиеся двое участвуют в аукционе.

\begin{enumerate}
\item Найдите равновесие Нэша
\item Существует ли пример распределения $ F(t) $ при котором средняя цена в случае удаления одного из игроков выше, чем в случае когда участвуют все трое желающих?
\end{enumerate}

Hint: Задача 7 из лекции 3




\end{enumerate}
