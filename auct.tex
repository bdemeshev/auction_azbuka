\documentclass[11pt, openany]{book}

\usepackage[
  paperwidth=160mm,
  paperheight=220mm,
  headheight=14mm,
  left=10mm,
  right=10mm,
  top=20mm,
  bottom=20mm
]{geometry}


%\pagestyle{headings}


\usepackage{etex} % расширение классического tex
% в частности позволяет подгружать гораздо больше пакетов, чем мы и займёмся далее

\usepackage{verbatim} % для многострочных комментариев
\usepackage{imakeidx} % для создания предметных указателей
%\indexsetup{level=\chapter}

%%%%% russian xetex
\usepackage{fontspec}
\usepackage{polyglossia}

\setmainlanguage{russian}
\setotherlanguages{english}

% download "Linux Libertine" fonts:
% http://www.linuxlibertine.org/index.php?id=91&L=1
\setmainfont{Linux Libertine O} % or Helvetica, Arial, Cambria
% why do we need \newfontfamily:
% http://tex.stackexchange.com/questions/91507/
\newfontfamily{\cyrillicfonttt}{Linux Libertine O}
%%%%% end russian xetex

\usepackage{fancyhdr}

\pagestyle{fancy}
\fancyhf{}
\fancyhead[LE,RO]{\thepage}
\fancyhead[LO]{\leftmark}
\fancyhead[RE]{\rightmark}

\usepackage{setspace}
\usepackage{amsmath, amsfonts, amssymb, amsthm}
\usepackage{mathrsfs} % sudo yum install texlive-rsfs
\usepackage{dsfont} % sudo yum install texlive-doublestroke
\usepackage{array, multicol, multirow, bigstrut} % sudo yum install texlive-multirow
\usepackage{indentfirst} % установка отступа в первом абзаце главы
\usepackage{bm}
\usepackage{bbm} % шрифт с двойными буквами


\usepackage{answers} % дележка ответов и вопросов

\usepackage{dcolumn} % центрирование по разделителю для apsrtable

\usepackage{xcolor}

% создание гиперссылок в pdf
\usepackage[unicode,
colorlinks=true,
linkcolor=black,
anchorcolor=black,
citecolor=black,
filecolor=black,
menucolor=black,
urlcolor=black,
hyperindex,
breaklinks]{hyperref}
% moving to xelatex: pdftex option removed

% свешиваем пунктуацию
% теперь знаки пунктуации могут вылезать за правую границу текста, при этом текст выглядит ровнее
\usepackage{microtype}

\usepackage{textcomp}  % Чтобы в формулах можно было русские буквы писать через \text{}



\usepackage{etoolbox} % нужно для ifdef





\usepackage{float}
\usepackage{longtable}
\usepackage{soulutf8}  %%% error if ommitted but does not interfere with knitr

\usepackage{enumitem} % дополнительные плюшки для списков
%  например \begin{enumerate}[resume] позволяет продолжить нумерацию в новом списке

\usepackage{mathtools}
\usepackage{cancel, xspace} % sudo yum install texlive-cancel


\usepackage{numprint} % sudo yum install texlive-numprint
\npthousandsep{,}\npthousandthpartsep{}\npdecimalsign{.}


\usepackage{subfigure} % для создания нескольких рисунков внутри одного


% \usepackage{tikz-3dplot}
\usepackage{tikz, pgfplots} % язык для рисования графики из latex'a
\usepackage{tikzsymbols} % смайлики :)
% \usetikzlibrary{trees} % tikz-прибамбас для рисовки деревьев
% \usepackage{tikz-qtree} % альтернативный tikz-прибамбас для рисовки деревьев
% \usetikzlibrary{arrows} % tikz-прибамбас для рисовки стрелочек подлиннее


\usepackage{todonotes} % для вставки в документ заметок о том, что осталось сделать
% \todo{Здесь надо коэффициенты исправить}
% \missingfigure{Здесь будет Последний день Помпеи}
% \listoftodos — печатает все поставленные \todo'шки


% более красивые таблицы
\usepackage{booktabs}
% заповеди из докупентации:
% 1. Не используйте вертикальные линни
% 2. Не используйте двойные линии
% 3. Единицы измерения - в шапку таблицы
% 4. Не сокращайте .1 вместо 0.1
% 5. Повторяющееся значение повторяйте, а не говорите "то же"

%\usepackage{asymptote}

% вместо горизонтальной делаем косую черточку в нестрогих неравенствах
\renewcommand{\le}{\leqslant}
\renewcommand{\ge}{\geqslant}
\renewcommand{\leq}{\leqslant}
\renewcommand{\geq}{\geqslant}

% делаем короче интервал в списках
\setlength{\itemsep}{0pt}
\setlength{\parskip}{0pt}
\setlength{\parsep}{0pt}


% DEFS
\def \mbf{\mathbf}
\def \msf{\mathsf}
\def \mbb{\mathbb}
\def \tbf{\textbf}
\def \tsf{\textsf}
\def \ttt{\texttt}
\def \tbb{\textbb}

\def \wh{\widehat}
\def \wt{\widetilde}
\def \ni{\noindent}
\def \ol{\overline}
\def \cd{\cdot}
\def \bl{\bigl}
\def \br{\bigr}
\def \Bl{\Bigl}
\def \Br{\Bigr}
\def \fr{\frac}
\def \bs{\backslash}
\def \lims{\limits}
\def \arg{{\operatorname{arg}}}
\def \dist{{\operatorname{dist}}}
\def \VC{{\operatorname{VCdim}}}
\def \card{{\operatorname{card}}}
\def \sgn{{\operatorname{sign}\,}}
\def \sign{{\operatorname{sign}\,}}
\def \xfs{(x_1,\ldots,x_{n-1})}
\def \Tr{{\operatorname{\mbf{Tr}}}}
\DeclareMathOperator*{\argmin}{arg\,min}
\DeclareMathOperator*{\argmax}{arg\,max}
\DeclareMathOperator*{\amn}{arg\,min}
\DeclareMathOperator*{\amx}{arg\,max}

\DeclareMathOperator{\Cov}{Cov}
\DeclareMathOperator{\Var}{Var}
\let\P\relax
\DeclareMathOperator{\P}{\mathbb{P}}
\DeclareMathOperator{\E}{\mathbb{E}}


\def \xfs{(x_1,\ldots,x_{n-1})}
\def \ti{\tilde}
\def \wti{\widetilde}


\def \mL{\mathcal{L}}
\def \mW{\mathcal{W}}
\def \mH{\mathcal{H}}
\def \mC{\mathcal{C}}
\def \mE{\mathcal{E}}
\def \mN{\mathcal{N}}
\def \mA{\mathcal{A}}
\def \mB{\mathcal{B}}
\def \mU{\mathcal{U}}
\def \mV{\mathcal{V}}
\def \mF{\mathcal{F}}

\def \R{\mbb R}
\def \N{\mbb N}
\def \Z{\mbb Z}
\def \D{\msf{D}}
\def \I{\mbf{I}}

\def \a{\alpha}
\def \b{\beta}
\def \t{\tau}
\def \dt{\delta}
\def \e{\varepsilon}
\def \ga{\gamma}
\def \kp{\varkappa}
\def \la{\lambda}
\def \sg{\sigma}
\def \sgm{\sigma}
\def \tt{\theta}
\def \ve{\varepsilon}
\def \Dt{\Delta}
\def \La{\Lambda}
\def \Sgm{\Sigma}
\def \Sg{\Sigma}
\def \Tt{\Theta}
\def \Om{\Omega}
\def \om{\omega}


\def \ni{\noindent}
\def \lq{\glqq}
\def \rq{\grqq}
\def \lbr{\linebreak}
\def \vsi{\vspace{0.1cm}}
\def \vsii{\vspace{0.2cm}}
\def \vsiii{\vspace{0.3cm}}
\def \vsiv{\vspace{0.4cm}}
\def \vsv{\vspace{0.5cm}}
\def \vsvi{\vspace{0.6cm}}
\def \vsvii{\vspace{0.7cm}}
\def \vsviii{\vspace{0.8cm}}
\def \vsix{\vspace{0.9cm}}
\def \VSI{\vspace{1cm}}
\def \VSII{\vspace{2cm}}
\def \VSIII{\vspace{3cm}}


\newcommand{\grad}{\mathrm{grad}}
\newcommand{\dx}[1]{\,\mathrm{d}#1} % для интеграла: маленький отступ и прямая d
\newcommand{\ind}[1]{\mathbbm{1}_{\{#1\}}} % Индикатор события
%\renewcommand{\to}{\rightarrow}
\newcommand{\eqdef}{\mathrel{\stackrel{\rm def}=}}
\newcommand{\iid}{\mathrel{\stackrel{\rm i.\,i.\,d.}\sim}}
\newcommand{\const}{\mathrm{const}}
 % use the local copy


\newcommand{\winpro}{\phi} % специальный знак для \pi из Кришны — не используется
\newcommand{\indef}[1]{\textbf{#1}}

\AddEnumerateCounter{\asbuk}{\russian@alph}{щ} % для списков с русскими буквами
\setlist[enumerate, 2]{label=\asbuk*),ref=\asbuk*}




\numberwithin{equation}{page} % уравнения нумеруются на каждой стр. отдельно

\newtheorem{myth}[equation]{Теорема} % нумерация сквозная с уравнениями

\theoremstyle{definition} % убирает курсив и что-то ещё наверное делает ;)
\newtheorem{mydef}[equation]{Определение}

\theoremstyle{definition}
\newtheorem{myex}[equation]{Пример}

%\newtheorem{assertion}{Утверждение}
%\newtheorem{lemma}{Лемма}

\theoremstyle{definition}
\newtheorem*{myproof}{Доказательство}


\begin{comment}
% todo list
\begin{enumerate}
\item Решение догонялки
\item Дописать про афиллированные с.в.
\item Нарисовать картинки
\item Сделать вводную по вероятностям с вопросами:
$P(X_{2}<a|X_{1}=x)$, $ E(X_{2}1_{X_{2}<a}|X_{1}=x) $, $ p(x_{2}|x_{1}) $ если известна $ p(x_{1},x_{2}) $
\item Сделать вводную по взятию производных от интеграла:
$ f(a)=\int_{0}^{a}g(x)dx $ найти $ f'(a) $. И более забубенистое: $ f(a)=\int_{0}^{a}\int_{y}^{1}p(x,y)dxdy $
\item Предупреждение: у непрерывных величин $ P(X=x)=0$
\item Переоформить теорему о равновесии на аукционе первой цены так \ldots  это решение диф. ура, с b(0)=0. Кстати там может быть пропущен знак минус в формуле решения. Проверить.
\item Проверить нет ли где ошибки связанной с тем, что (например) нет совместной плоности у пары $ X_{2} $, $ Y_{2} $. Сделать вопрос с подвохом. Типа найдите плотность.
\item Список сопоставляющий обозначения: Vishna, Klemperer?, Introduction to auctions\ldots
\end{enumerate}
\end{comment}






\title{Моделирование аукционов. Азбука. }
\author{Борис Демешев}
\date{Москва, 2016 г.}



\makeindex[intoc] % команда для создания предметного указателя
\bibliographystyle{plain} % стиль оформления ссылок



\begin{document}

\maketitle

\setcounter{page}{3}

\tableofcontents{}

\chapter*{Предисловие}
% ненумерованная глава не добавляется по умолчанию в оглавление
% и не меняет верхний колонтитул
% приходится делать это ручками
\addcontentsline{toc}{chapter}{Предисловие}
\markboth{Предисловие}{}

Эта книга — подробный конспект лекций курса по моделированию аукционов. Курс был прочитан дистанционно в НИУ-ВШЭ в 2011 году. Самое важное отличие книги от других книг по моделированию аукционов — огромное количество задач с решениями!

Из книги любопытный читатель узнает, например:
\begin{itemize}
\item что аукцион — это не обязательно «дядя с молоточком»;
\item как устроены самые крупные аукционы;
\item почему не всегда победитель аукциона платит ту сумму, которую поставил;
\item как влияют на прибыль организатора разные правила проведения аукциона;
\item при каких условиях может нарушиться закон спроса.
\end{itemize}

В книге четыре главы. Первая — про разные виды аукционов и теорему об~эквивалентности доходностей, которая утверждает, что при независимости игроков все аукционы приносят одинаковый доход организатору аукциона. Вторая глава~— техническая. Её цель — заполнить пробелы по теории вероятностей, рассказать технику решения задач с~помощью о-малых и объяснить концепцию аффилированных сигналов. Третья глава посвящена сравнению доходностей организатора аукциона в~случае, когда игроки зависимы из-за того, что получают общую информацию о~товаре. Четвёртая глава излагает аукционы с помощью общего языка теории механизмов. В~главе вводится механизм Викри—Кларка—Гровса и доказывается оптимальность аукциона второй цены с резервной ценой.

Моделирование аукционов — довольно сложная и сильно математизированная дисциплина. Именно из-за теоретической сложности она редко встречается в программе бакалавриата. Моей целью было сделать этот курс максимально доступным для бакалавров. Поэтому я старался, во-первых, снизить входные требования к уровню подготовки, во-вторых, включить в курс максимальное количество задач с решениями.

Для чтения книги требуется немного теории игр и теории вероятностей. Из теории игр — понимание равновесия Нэша. Из теории вероятностей — умение считать условные вероятности и математические ожидания дискретных и непрерывных случайных величин.

Выражаю большую благодарность рецензентам Вадиму Львовичу Шагину и Николаю Петровичу Пильнику за ценные замечания.

Для удобства поиска внутри книги нумерация формул идёт постраничная. Например, формула (27.3) — это третья формула на 27-й странице. Конец доказательства обозначается значком $\square$.

Читателю могут также оказаться полезными видеолекции курса, \url{vimeo.com/album/1530587}, и блог, бывший активным в 2011 году, \url{auctiontheory.wordpress.com/}.

Удачи в освоении теории аукционов!

\begin{flushright}
  Борис Демешев \\
  \href{mailto:bdemeshev@hse.ru}{bdemeshev@hse.ru} \\
  Национальный исследовательский университет\\
  «Высшая школа экономики»
\end{flushright}


\input{lecture_01.tex}

\section{Контрольная работа 1}



\begin{enumerate}
\item Предположим, что условия теоремы об одинаковых доходностях выполнены.
\begin{enumerate}
\item Может ли выбор механизма проведения аукциона влиять на ковариацию выплат двух разных игроков?
\item  Найдите ковариацию выплат первого и второго игрока в аукционе первой цены с независимыми и равномерными на $ [0;1] $ ценностями. Hint: можно пользоваться тем, что средняя выплата равна $ \frac{n-1}{n(n+1)} $.
\end{enumerate}


\item «Наследство»\index{Наследство} по типу аукциона второй цены. Двум сыновьям достался земельный участок в наследство. Отец не хотел, чтобы участок был разделен, поэтому по завещанию установлены следущие правила: два брата одновременно делают ставки. Участок получает тот, кто сделал большую ставку. При этом получивший участок выплачивает проигравшему меньшую из двух ставок. Ценности участка для игроков независимы и равномерны на $ [0;1] $.

Найдите равновесие Нэша.


\item Рассмотрим аукцион второй цены. Предположим, что ценности независимы и имеют регулярное распределение. Агенты не нейтральны к риску. Их отношение к риску отражается функцией полезности $ u() $. Про $ u() $ известно, что она непрерывна, строго возрастает и для удобства $ u(0)=0 $. то есть если игрок получает товар ценностью $ x $ и платит продавцу $ m $, то его полезность равна $ u(x-m) $.

Найдите равновесие Нэша.



\item Рассмотрим аукцион второй цены с резервной ставкой $ r $. Резервная ставка — это минимальная цена за которую продавец согласен расстаться с товаром. Если все игроки сделали ставки ниже $ r $, товар остаётся у продавца, никто ничего не платит. Если хотя бы один игрок сделал ставку выше $ r $, то товар достаётся игроку сделавшему самую высокую ставку и платит он максимум между второй по величине ставкой и $ r $. Константа $ r $ общеизвестна всем игрокам. Ценности независимы и имеют регулярное распределение. Агенты нейтральны к риску.

Найдите равновесие Нэша.


\item Рассмотрим аукцион первой цены с двумя игроками. Ценности независимы и равномерны на $ [0;1] $. Но ставку можно сделать только 0 или 0.5. Если ставки игроков совпали, то товар достаётся случайно выбираемому игроку за соответствующую плату.

Найдите равновесие Нэша.

\end{enumerate}


\section{Решение контрольной работы 1}



\begin{enumerate}
\item

\begin{equation}
\Cov(Pay_{1}, Pay_{2})=\E(Pay_{1}\cdot Pay_{2})-\E(Pay_{1})\E(Pay_{2})
\end{equation}

На вычитаемое способ аукциона влиять не может в силу теоремы об одинаковой доходности. Сосредоточимся на $\E(Pay_{1}\cdot Pay_{2})$. В аукционе первой цены никакие два игрока не могут платить одновременно, поэтому произведение выплат всегда равно нулю, то есть $ \E(Pay_{1}\cdot Pay_{2})=0 $. В аукционе «Платят все»\index{аукцион!платят все} произведение выплат строго положительно, поэтому $ \E(Pay_{1}Pay_{2})>0 $. Значит, способ проведения аукциона может влиять на ковариацию.



\item  Ожидаемая прибыль:

\begin{multline}
\pi(x,b_{1})=(x-\E(b(X_{2})|b(X_{2})<b_{1}))\cdot \P(b(X_{2})<b_{1})+\\
+b_{1}\P(b(X_{2})>b_{1})
\end{multline}

После чудо-замены $ b_{1}=b(a) $ и упрощения вероятностей:

\begin{equation}
\pi=(x-\E(b(X_{2})|X_{2}<a))\P(X_{2}<a)+b(a)(1-\P(X_{2}<a))
\end{equation}

Или:
\begin{equation}
\pi=xF(a)-\E(b(X_{2})\cdot 1_{X_{2}<a}))+b(a)(1-F(a))
\end{equation}

В записи с интегралом:
\begin{equation}
\pi=xF(a)-\int_{0}^{a}b(t)f(t)dt+b(a)(1-F(a))
\end{equation}

Приравниваем производную к нулю:
\begin{equation}
xf(a)-b(a)f(a)-b(a)f(a)+b'(a)(1-F(a))=0
\end{equation}

Для случая равномерного распределения:
\begin{equation}
x-2b(x)+b'(x)(1-x)=0
\end{equation}

Подбором коэффициентов находим линейное решение:
\begin{equation}
b(x)=\frac{1}{3}x+\frac{1}{6}
\end{equation}


\item Составляем табличку и видим, что стратегия $ b_{1}=X_{1} $ нестрого доминирует остальные стратегии.


\item  Составляем табличку и видим, что стратегия $ b_{1}=X_{1} $ нестрого доминирует остальные стратегии.

\item Предположим, что стратегия имеет вид\index{аукцион!с дискретными ставками}:

Если ценность ниже порога $ x^{*} $, то делать ставку 0, иначе делать ставку $ 0.5 $.

Осталось найти $ x^{*} $.

Допустим, что второй игрок использует такую стратегию.

Если первый сделает ставку ноль, то его ожидаемый выигрыш будет равен:
\begin{equation}
\pi(x,0)=x^{*}\frac{1}{2}x
\end{equation}

Если первый сделает ставку $0.5$, то его ожидаемый выигрыш будет равен:
\begin{equation}
\pi(x,0.5)=(x-0.5)x^{*}+(x-0.5)\frac{1}{2}(1-x^{*})=(x-0.5)\frac{1}{2}(x^{*}+1)
\end{equation}

Находим условие, при котором $ \pi(x,0.5)>\pi(x,0) $, получаем:

\begin{equation}
x>\frac{1}{2}(x^{*}+1)
\end{equation}

Значит, правая часть представляет собой $ x^{*} $. Решаем уравнение $x^{*}=\frac{1}{2}(x^{*}+1)  $, получаем $ x^{*}=1 $. то есть вне зависимости от ценности игрокам имеет смысл ставить 0.


\end{enumerate}



\input{lecture_02.tex}

\section{Контрольная работа 2}

\begin{enumerate}

\item Техническая задача.
\begin{enumerate}
\item Известно, что $ f(\vec{x}) $ и $ g(\vec{x}) $ — супермодулярные функции\index{супермодулярная функция}, а $ a>0 $ и $ b>0 $ — константы. Верно ли, что $ af(\vec{x})+bg(\vec{x}) $ — супермодулярная функция?

\item Пусть $ X_{1} $,\ldots, $ X_{n} $ независимы и имеют функцию плотности $ f(t)=3t^{2} $ на $ [0;1] $. Случайные $ Y_{1} $, \ldots, $ Y_{n-1} $ — это есть упорядоченные по убыванию случайные величины $ X_{2} $, \ldots, $ X_{n} $. С помощью о-малых или без неё найдите совместную функцию плотности $ f_{Y_{5},Y_{10}}(a,b) $.\index{о-малые}
\end{enumerate}

Следующие две задачи очень похожи, разница в них только в типе аукциона\ldots

\item На аукционе первой цены продаётся участок\index{задача!о продаже участка}. Потенциальных покупателей двое. Ценность участка для каждого игрока определяется его площадью. Первый покупатель знает ширину участка $ X_{1} $, а второй — длину $X_{2}$. Совместная фукнция плотности $ X_{1} $ и $ X_{2} $ имеет вид $ f(x_{1},x_{2})=\frac{7}{8}+\frac{1}{2}x_{1}x_{2} $. Найдите дифференциальное уравнение, которому подчиняется равновесная стратегия игрока.

\item На аукционе второй цены продаётся участок\index{задача!о продаже участка}. Потенциальных покупателей двое. Ценность участка для каждого игрока определяется его площадью. Первый покупатель знает ширину участка $ X_{1} $, а второй — длину $X_{2}$. Совместная фукнция плотности $ X_{1} $ и $ X_{2} $ имеет вид $ f(x_{1},x_{2})=\frac{7}{8}+\frac{1}{2}x_{1}x_{2} $. Найдите равновесие Нэша.


\item Найдите равновесие Нэша в случае кнопочного аукциона\index{аукцион!кнопочный}. Сигналы $ X_{i} $ игроков имеют совместную функцию плотности $ f(x_{1},x_{2},x_{3})=7/8+x_{1}x_{2}x_{3} $ при $ x_{1},x_{2},x_{3}\in[0;1] $. Ценности определяеются по формулам:
\begin{equation}
\begin{array}{c}
V_{1}=X_{1}(X_{2}+X_{3}) \\
V_{2}=X_{2}(X_{1}+X_{3}) \\
V_{3}=X_{3}(X_{1}+X_{2})
\end{array}
\end{equation}


\item Ценности игроков одинаково распределены, независимы, распределение ценностей дискретно\index{ценности!дискретные}: $ X_{i}$ равновероятно принимает натуральное значение от 1 до 100 включительно. Игроки одновременно делают ставки. Значения всех ценностей общеизвестны всем игрокам ещё до ставок! Разрешаются только целые неотрицательные ставки. Товар достаётся игроку, сделавшему наивысшую ставку. Если таких игроков несколько, то победитель выбирается из них равновероятно. Победитель платит сделанную им ставку. Найдите хотя бы одно равновесие Нэша в чистых стратегиях, в котором игроки не используют нестрого доминируемых стратегий.


\end{enumerate}


\section{Решение контрольной работы 2}

\begin{enumerate}

\item
\begin{enumerate}
\item Да, является супермодулярной. Проверяем два свойства:
\begin{enumerate}
\item $ af(\vec{x}) $ — ок.
\item $ f(\vec{x})+g(\vec{x}) $ — ок.
\end{enumerate}

\item Сразу ответ: $ f(a,b)=(n-1)(n-2)C_{n-3}^{4}C_{n-7}^{4}3a^{2}3b^{2}(a^{3}-b^{3})^{4}(b^{3})^{n-11}(1-a^{3})^{4} $.
\end{enumerate}


\item Сразу начнем с чудо-замены, $ b_{1}=b(a) $. Сначала упростим событие $ W_{1} $:\index{чудо-замена}
\begin{equation}
W_{1}=\{Bid_{2}<b(a)\}=\{b(X_{2})<b(a)\}=\{X_{2}<a\}.
\end{equation}

А теперь прибыль:
\begin{multline}
\pi(x,b(a))=\E(X_{1}X_{2}1_{W_{1}}|X_{1}=x)-b(a)\E(1_{W_{1}}|X_{1}=x)=\\
=x\E(X_{2}1_{X_{2}<a}|X_{1}=x)-b(a)\E(1_{X_{2}<a}|X_{1}=x)=\\
=x\int_{0}^{a}x_{2}\frac{f(x,x_{2})}{f(x)} \, dx_{2}-b(a)\int_{0}^{a}\frac{f(x,x_{2})}{f(x)} \, dx_{2}.
\end{multline}


Сокращаем на $ f(x) $ и берём производную по $ a $:
\begin{equation}
\frac{\partial \pi}{\partial a}=xaf(x,a)-b'(a)\int_{0}^{a}f(x,x_{2}) \, dx_{2}-b(a)f(x,a)=0.
\end{equation}


Мы хотим, чтобы оптимальной стратегий первого была $ b_{1}=b(x) $, то есть чтобы $ a=x $:
\begin{equation}
\frac{\partial \pi}{\partial a}=x^{2}f(x,x)-b'(x)\int_{0}^{x}f(x,x_{2})dx_{2}-b(x)f(x,x)=0.
\end{equation}

Остаётся подставить:
\begin{equation}
\begin{array}{c}
f(x,x)=\frac{7}{8}+\frac{1}{2}x^{2}; \\
\int_{0}^{x}f(x,x_{2})dx_{2}=\frac{7}{8}x+\frac{1}{4}x^{3}.
\end{array}
\end{equation}


\item  Никакой разницы с кнопочным аукционом в данном случае нет. Игроков-то всего два! Значит, равновесие Нэша имеет вид $ b(x)=x^{2} $. \index{аукцион!кнопочный}

Доказательство.

Если второй игрок использует такую стратегию и первый выигрывает аукцион, то его прибыль равна:
\begin{equation}
X_{1}X_{2}-X_{2}^{2}=(X_{1}-X_{2})X_{2}.
\end{equation}
Мы видим, что прибыль положительна, только если $ X_{1}>X_{2} $. Использование первым игроком функции $ b(x)=x^{2} $ будет обеспечивать его выигрыш только в ситуации $ X_{1}>X_{2} $, значит, это и есть равновесие Нэша.


\item Стратегия описывается двумя функциями: $ b^{3}(x)=2x^{2} $. Если при использовании такой стратегии игрок вышел на цене $ p $, значит, его $ x=\sqrt{p/2} $. Получаем $ b^{2}(x,p)=x^{2}+x\sqrt{p/2}$. Доказательство аналогично лекции:

Если остальные игроки используют эти стратегии и первый выигрывает аукцион, то его выигрыш равен
\begin{multline}
X_{1}(X_{2}+X_{3})-b^{2}(Y_{1},b^{3}(Y_{2}))=\\
=X_{1}(Y_{1}+Y_{2})-(Y_{1}^{2}+Y_{1}Y_{2})=(X_{1}-Y_{1})(Y_{1}+Y_{2}).
\end{multline}
Мы видим, что выигрыш положительный, только если $ X_{1}>Y_{1} $. Использование первым игроком правил $ b^{3}() $ и $ b^{2}() $ приводит к выигрышу, только если $ X_{1}>Y_{1} $, значит, это и есть равновесие.


\item  Пример равновесия Нэша. Обозначим максимальную ценность $ v_{max} $, а вторую по величине ценность — $v_{sec}$. Возможно, эти две ценности совпадают. Те игроки, чья ценность $ v_{i}<v_{max} $, делают ставку $ b_{i}=v_{i}-1 $. Если лидер один, то он делает ставку $ b=v_{sec} $, иначе каждый лидер делает ставку $ b=v_{max}-1 $.


\end{enumerate}


\input{lecture_03.tex}

\section{Контрольная работа 3}

\begin{enumerate}


\item Пусть $  V $ — общая ценность товара для двух игроков, равномерна на $ [0;1] $. Величины $ R_{1} $ и $ R_{2} $ — независимы между собой и с $ V $ и равномерны на $ [0.5;1.5] $. Игроки получают сигналы $ X_{i}=V\cdot R_{i} $.
\begin{enumerate}
\item Найдите совместную функцию плотности $ X_{1} $ и $ X_{2} $. Верно ли, что $ X_{1} $ и $ X_{2} $ аффилированны?
\item Найдите $ v(x,y)=\E(V|X_{1}=x,Y_{1}=y) $
\item Найдите совместную функцию плотности $ X_{1} $ и $ Y_{1} $, $ g(x,y) $
\end{enumerate}


\item На аукционе продаётся картина, которая равновероятно является «Джокондой» Леонардо да Винчи или её подделкой. За неё торгуются $ n $ покупателей. Ценность картины для всех покупателей одинакова, $ V_{1}=V_{2}=\ldots=V_{n}=V $ и равна 1, если это оригинал и 0, если подделка.

Если $ V=0 $, то сигналы $ X_{i} $ условно независимы и равномерны на $ [0;1] $. Если $ V=1 $, то сигналы $ X_{i} $ условно независимы и имеют функцию плотности $ f(x|V=1)=2x $ при  $x\in [0;1] $
\begin{enumerate}
\item Найдите совместную функцию плотности всех $ X_{i} $. Верно ли, что все $ X_{i} $ аффилированны?
\item Найдите $ v(x,y)=\E(V|X_{1}=x,Y_{1}=y) $
\item Найдите совместную функцию плотности $ X_{1} $ и $ Y_{1} $, $ g(x,y) $
\end{enumerate}



\item На аукционе второй цены присутствуют $ n $ покупателей. Ценности совпадают с сигналами, $ V_{i}=X_{i} $; сигналы $ X_{i} $ независимы и равномерны на $ [0;1] $. На аукционе продаётся $k$ одинаковых чудо-швабр, $ 1<k<n $. Каждому покупателю нужна только одна чудо-швабра. Покупатели одновременно делают свои ставки. Чудо-швабры достаются по одной каждому из $ k $ покупателей с самыми высокими ставками. Каждый из $ k $ победителей платит организатору наибольшую проигравшую ставку.

Найдите равновесие Нэша.

\item На аукционе первой цены присутствуют $ n $ покупателей. Ценности совпадают с сигналами, $ V_{i}=X_{i} $; сигналы $ X_{i} $ независимы и равномерны на $ [0;1] $. На аукционе продаётся $k$ одинаковых чудо-швабр, $ 1<k<n $. Каждому покупателю нужна только одна чудо-швабра. Покупатели одновременно делают свои ставки. Чудо-швабры достаются по одной каждому из $ k $ покупателей с самыми высокими ставками. Эти $ k $ победителей платят свои ставки организатору.

Найдите дифференциальное уравнение, которому удовлетворяет равновесная стратегия.

Подсказка: Когда продавался один товар, то условие победы первого игрока — $ Y_{1}<a $, а если продаётся $ k $ товаров, то условие победы первого игрока $ Y_{?}<a $.

\item Существуют ли неаффилированные случайные величины $ X_{1} $ и $ X_{2} $ такие, что $Cov(X_{1},X_{2})>0  $?

\end{enumerate}


\section{Решение контрольной  работы 3}

\input{kr_03_solution.tex}

\section{Домашняя работа 3}

Контрольная номер 3 оказалась чересчур сложной, поэтому студентам была предложена вместо неё домашняя работа 3.


\begin{enumerate}


\item Техническая задача.
\begin{enumerate}
\item Выразите $ (a+c)\vee (b+c) $ через $ a\vee b $. Выразите $ (a+c)\wedge (b+c) $ через $ a\wedge b $.
\item Случайные величины $ Z_{1} $, \ldots , $ Z_{n} $ аффилированы между собой. Случайные величины $ W_{1} $, \ldots , $ W_{k} $ — аффилированы между собой. Набор случайных величин $ Z_{1} $, \ldots , $ Z_{n} $ не зависит от набора $ W_{1} $, \ldots , $ W_{k} $. Верно ли, что набор случайных величин $ Z_{1} $, \ldots , $ Z_{n} $, $ W_{1} $, \ldots , $ W_{k} $ аффилирован?
\end{enumerate}


\item Пусть $  V $ — общая ценность товара для двух игроков, равномерна на $ [1;2] $. Величины $ R_{1} $ и $ R_{2} $ — независимы между собой и с $ V $ и равномерны на $ [-0.5;0.5] $. По смыслу: $ R_{1} $ и $ R_{2} $ — это ошибки игроков при подсчете ценности товара $ V $. Игроки получают сигналы $ X_{i}=V+R_{i} $, то есть игроки знают ценность  $ V $ с ошибкой.
\begin{enumerate}
\item Найдите совместную функцию плотности $ X_{1} $ и $ X_{2} $. Верно ли, что $ X_{1} $ и $ X_{2} $ аффилированны?
\item Найдите $ v(x,y)=\E(V|X_{1}=x,Y_{1}=y) $. Найдите равновесие Нэша на аукционе второй цены.
\item Найдите совместную функцию плотности $ X_{1} $ и $ Y_{1} $, $ g(x,y) $
\end{enumerate}

Hint: В решении контрольной есть похожая задача. А $ g(x,y) $ можно неплохо упростить пользуясь предыдущей задачей.

Поскольку игроков всего двое, то $ g(x,y)$ — это просто совместная функция плотности $ X_{1} $ и $ X_{2} $.


\item Пусть $ R_{1} $, $ R_{2} $ и $ S $ — равномерны на $ [0;1] $ и независимы. Ценность товара для первого игрока, $ V_{1}=0.8X_{1}+0.2X_{2} $ и для второго — $ V_{2}=0.8X_{2}+0.2X_{1} $. Первый игрок получает сигнал $ X_{1}=S+R_{1} $. Второй игрок получает сигнал $ X_{2}=S+R_{2} $.
\begin{enumerate}
\item Найдите $ g(x,y) $, $ R(y|x) $ и $ v(x,y)=\E(V|X_{1}=x,Y_{1}=y) $
\item Используя предыдущие функции найдите равновесие Нэша на аукционе второй цены, первой цены и кнопочном аукционе
\end{enumerate}

% ??? (занудно) В рамках предыдущей задачи найдите $ g(y|x) $, $ G(y|x) $, $ R(y|x) $, $ R(x|x) $. Найдите равновесие Нэша на аукционе первой цены и кнопочном аукционе.

%Hint: Обратите внимание, что в лекции $ X_{i} $ принимает значения на $ [0;1] $. Здесь $ X_{i} \in [0.5;2.5] $, поэтому при подсчете $ G(y|x) $ нижний предел интегрирования не равен 0.


\item Продолжение задачи 2 с контрольной (можно использовать все полученные в ней результаты).
На аукционе продаётся картина, которая равновероятно является «Джокондой» Леонардо да Винчи или её подделкой. За неё торгуются $ n $ покупателей. Ценность картины для всех покупателей одинакова, $ V_{1}=V_{2}=\ldots =V_{n}=V $ и равна 1, если это оригинал и 0, если подделка.

Если $ V=0 $, то сигналы $ X_{i} $ условно независимы и равномерны на $ [0;1] $. Если $ V=1 $, то сигналы $ X_{i} $ условно независимы и имеют функцию плотности $ f(x|V=1)=2x $ при  $x\in [0;1] $

\begin{enumerate}
\item Найдите равновесие Нэша на аукционе второй цены
\item Найдите $ \E(V|X_{1}=x_{1},X_{2}=x_{2},X_{3}=x_{3}\ldots X_{n}=x_{n}) $
\item С помощью предыдущего пункта найдите функции $ b^{n}(x) $,  $ b^{n-1}(x,p_{n}) $  и $ b^{n-2}(x,p_{n-1},p_{n}) $ в равновесии Нэша на кнопочном аукционе
\end{enumerate}


%\item
% Хм: страшные интегралы лезут отовсюду..
%Товар имеет общую ценность $ V $ для трёх игроков, $ V $ равномерно на $ [0;1] $. При фиксированном $ V=v $ сигналы независимы и имеют функцию плостности:
%\begin{equation}
%f(x|v)=2v\cdot x^{2v-1}, \quad x\in[0;1]
%\end{equation}
%\begin{enumerate}
%\item Какой ожидаемый сигнал получают игроки, если $ V=0 $? $ V=0.5 $? $ V=1 $?
%\item Найдите $ g(x,y) $
%\item Найдите $ v(x,y)=\E(V|X_{1}=x,Y_{1}=y) $ и равновесие Нэша на аукционе второй цены
%\item Найдите $ \E(V|X_{1}=x_{1},X_{2}=x_{2},X_{3}=x_{3} )$ и равновесие Нэша на кнопочном аукционе
%\end{enumerate}

%Hint: Можно пользоваться тем, что $  $


\end{enumerate}


\section{Решение домашней работы 3}

\begin{enumerate}


\item
\begin{equation}
(a+c)\vee (b+c)=a\vee b +c
\end{equation}

\begin{equation}
(a+c)\wedge (b+c)=a\wedge b +c
\end{equation}

Да, набор $ Z_{1} $, \ldots , $ Z_{n} $, $ W_{1} $, \ldots , $ W_{k} $ аффилирован. В силу независимости логарифм совместной функции плотности разлагается в сумму логарифмов:
\begin{equation}
\ln f_{Z,W}(z_{1},\ldots ,z_{n},w_{1},\ldots ,w_{k})= \ln f_{Z}(z_{1},\ldots ,z_{n})+\ln f_{W}(w_{1},\ldots ,w_{k})
\end{equation}
И смешанные производные равны либо нулю, либо неотрицательны в силу аффилированности $ Z_{i} $ между собой и $ W_{i} $ между собой.

\item  Поскольку игроков всего двое, то $ g(x,y)$ — это просто совместная функция плотности $ X_{1} $ и $ X_{2} $.

Находим условную совместную плотность:
\begin{equation}
p(x_{1},x_{2}|v)=1, \quad x_{1},x_{2}\in [v-0.5;v+0.5]
\end{equation}

Значит:
\begin{equation}
p(x_{1},x_{2},v)=1, \quad x_{1},x_{2}\in [v-0.5;v+0.5],v\in [1;2]
\end{equation}

Заметим, что область, где плотность положительна, можно описать условием:
\begin{equation}
v\in [(x_{1}-0.5)\vee (x_{2}-0.5); (x_{1}+0.5)\wedge (x_{2}+0.5)]=[v_{min};v_{max}]
\end{equation}

Интегрируем по $ v $ и получаем:
\begin{equation}
p(x_{1},x_{2})=\int_{v_{min}}^{v_{max}} 1 dv= v_{max}-v_{min}=x_{1}\wedge x_{2}-x_{1}\vee x_{2}+1
\end{equation}

\begin{multline}
\E(V|X_{1}=x_{1}, X_{2}=x_{2})=\int v p(v|x_{1},x_{2})dv=\\
\int v \frac{p(x_{1},x_{2},v)}{p(x_{1},x_{2})} dv=\frac{\int v p(x_{1},x_{2},v) dv }{p(x_{1},x_{2})}
\end{multline}

В числителе:
\begin{equation}
\int_{v_{min}}^{v_{max}}vdv=\frac{v_{max}^{2}-v_{min}^{2}}{2}
\end{equation}

Значит, в итоге:
\begin{equation}
v(x_{1},x_{2})=\frac{v_{max}^{2}-v_{min}^{2}}{2\cdot (v_{max}-v_{min})}=\frac{v_{max}+v_{min}}{2}= \frac{x_{1}\wedge x_{2}+x_{1}\vee x_{2}}{2}
\end{equation}

Равновесие Нэша на аукционе второй цены:
\begin{equation}
v(x,x)=x
\end{equation}


\item Игроков всего два, значит, $ g(x,y) $ — просто совместная функция плотности $ X_{1} $ и $ X_{2} $.

\begin{equation}
p(x_{1},x_{2}|s)=1\cdot 1, \quad x_{1},x_{2}\in [s;s+1]
\end{equation}

Следовательно:
\begin{equation}
p(x_{1},x_{2},s)=1\cdot 1, \quad x_{1},x_{2}\in [s;s+1], s\in [0;1]
\end{equation}

Заметим, что область, где плотность положительна, можно описать условием:
\begin{equation}
s\in [x_{1}\vee x_{2}-1; x_{1}\wedge x_{2}]=[s_{min};s_{max}]
\end{equation}

Интегрируем по $ s $ и получаем:
\begin{equation}
p(x_{1},x_{2})=\int_{s_{min}}^{s_{max}} 1 ds= s_{max}-s_{min}=x_{1}\wedge x_{2}-x_{1}\vee x_{2}+1
\end{equation}

Плотность обращается в ноль за пределами участка $ 0\leq x_{1},x_{2}\leq 2 $, $ x_{1}-1\leq x_{2} \leq x_{1}+1 $.

Чтобы найти $ R(y|x) $ вспоминаем что это такое:
\begin{equation}
R(y|x)=\frac{g(x,y)}{\int_{0}^{y}g(x,t)dt}
\end{equation}

Возникает четыре случая для $ R(y|x) $\ldots

%\begin{equation}
%R(y|x)=
%\begin{cases}
%\frac{1+y-x}{y+0.5y^{2}-xy}, \quad x<1,y<x \\
%\frac{1+x-y}{-2x^{2}+y+yx-0.5y^{2}}, \quad x<1,y>x \\
%\frac{1+y-x}{y+0.5y^{2}-xy+0.5x^{2}+0.5-x}, \quad x>1,y<x \\
%\frac{1+x-y}{0.5+y+yx-0.5y^{2}-x-x^{2}+0.5x^{2}}, \quad x>1,y>x
%\end{cases}
%\end{equation}

%Нам нужна $ R(x|x) $. Находим, что $ g(x,x)=1 $, и $ g(x,t)=t-x+1 $ при $ t<x $ .

%\begin{equation}
%R(x|x)=
%\begin{cases}
%\frac{1}{x-0.5x^{2}}, \quad x<1 \\
%2, \quad x>1
%\end{cases}
%\end{equation}

К сожалению, в явном виде хорошего мало. Стандартная максимизация с чудо-заменой дает дифференциальное уравнение:
\begin{equation}
(0.8x-b'(x))\int_{0}^{x}p(x,x_{2})dx_{2}+x-b(x)=0
\end{equation}

Возникает два случая из-за ломаной $ p(x_{1},x_{2}) $\ldots

Если $ x\in [0;1] $, то:
\begin{equation}
(0.8x-b'(x))\cdot (x-0.5x^{2})+x-b(x)=0
\end{equation}
Из этого уравнения надо выбрать решение с $ b(0)=0 $.

Если $ x\in [1;2] $, то:
\begin{equation}
(0.8x-b'(x))\cdot 0.5+x-b(x)=0
\end{equation}
Из этого уравнения надо выбрать решение непрерывно склеивающееся с первым в точке $x=1$.


Находим $ v(x,y) $:
\begin{equation}
v(x,y)=\E(V_{1}|X_{1}=x,Y_{1}=y)=\E(V_{1}|X_{1}=x,X_{2}=y)=0.8x+0.2y
\end{equation}

Равновесие Нэша на аукционе второй цены:
\begin{equation}
b(x)=v(x,x)=x
\end{equation}
Кнопочный аукцион совпадает с аукционом второй цены.


\item В решении контрольной 3 мы получили результат:
\begin{equation}
v(x,y)=\frac{4xy^{n-1}}{1+4xy^{n-1}}
\end{equation}

Следовательно, равновесие Нэша на аукционе второй цены:
\begin{equation}
b(x)=v(x,x)=\frac{4x^{n}}{1+4x^{n}}
\end{equation}

Можно отметить, что функция растет с ростом $ x $ и падает с ростом $ n $.

Теперь рассмотрим $ A=\{X_{1}\in[x_{1};x_{1}+\Delta] \cap \ldots  \cap X_{n}\in[x_{n};x_{n}+\Delta]\} $. Как и в решении задачи с контрольной:
\begin{multline}
\E(V|A)=\P(V=1|A)=\frac{\P(V=1 \cap A)}{\P(A)}=\\
=\frac{\P(A|V=1)\cdot \P(V=1)}{\P(A)}=\frac{0.5\P(A|V=1)}{\P(A)}
\end{multline}

Согласно методу о-малых аналогичная формула справедлива для плотностей:
\begin{multline}
\E(V|X_{1}=x_{1},X_{2}=x_{2},\ldots ,X_{n}=x_{n})=\\
=\frac{0.5\cdot 2^{n}\Pi_{i=1}^{n}x_{i}}{0.5+0.5\cdot 2^{n}\Pi_{i=1}^{n}x_{i}}
=\frac{2^{n}\Pi_{i=1}^{n}x_{i}}{1+ 2^{n}\Pi_{i=1}^{n}x_{i}}
\end{multline}

Теперь частично находим стратегию на кнопочном аукционе:
\begin{equation}
b^{n}(x)=\frac{2^{n}x^{n}}{1+2^{n}x^{n}}
\end{equation}

Если все игроки используют эту функцию, то чтобы игрок вышел на цене $ p $ ценность должна равняться:
\begin{equation}
x=\frac{1}{2}\left(\frac{p}{1-p} \right)^{1/n}
\end{equation}

Подставляя один такой $ x $ в ожидаемую ценность получаем:
\begin{equation}
b^{n-1}(x,p_{n})=\frac{2^{n-1}x^{n-1}\left(\frac{p_{n}}{1-p_{n}} \right)^{1/n}}{1+2^{n-1}x^{n-1}\left(\frac{p_{n}}{1-p_{n}} \right)^{1/n}}
\end{equation}

Если второй выходит на цене $ p_{n-1} $, то его ценность была равна:
\begin{equation}
x=\frac{1}{2}\left(\frac{p_{n}}{1-p_{n}} \right)^{1/n(n-1)}\left(\frac{p_{n-1}}{1-p_{n-1}} \right)^{1/(n-1)}
\end{equation}

Значит:
\begin{equation}
b^{n-2}(x,p_{n-1},p_{n})=\frac{2^{n-2}x^{n-2}\left(\frac{p_{n}}{1-p_{n}} \right)^{1/n(n-1)}\left(\frac{p_{n-1}}{1-p_{n-1}} \right)^{1/(n-1)}}{1+2^{n-2}x^{n-2}\left(\frac{p_{n}}{1-p_{n}} \right)^{1/n(n-1)}\left(\frac{p_{n-1}}{1-p_{n-1}} \right)^{1/(n-1)}}
\end{equation}


%\item
% Хм: страшные интегралы лезут отовсюду..
%Товар имеет общую ценность $ V $ для трёх игроков, $ V $ равномерно на $ [0;1] $. При фиксированном $ V=v $ сигналы независимы и имеют функцию плостности:
%\begin{equation}
%f(x|v)=2v\cdot x^{2v-1}, \quad x\in[0;1]
%\end{equation}
%\begin{enumerate}
%\item Какой ожидаемый сигнал получают игроки, если $ V=0 $? $ V=0.5 $? $ V=1 $?
%\item Найдите $ g(x,y) $
%\item Найдите $ v(x,y)=\E(V|X_{1}=x,Y_{1}=y) $ и равновесие Нэша на аукционе второй цены
%\item Найдите $ \E(V|X_{1}=x_{1},X_{2}=x_{2},X_{3}=x_{3} )$ и равновесие Нэша на кнопочном аукционе
%\end{enumerate}

%Подсказка: Можно пользоваться тем, что $  $


\end{enumerate}



\input{lecture_04.tex}

\section{Контрольная работа 4}

\begin{enumerate}

\item На аукционе участвуют $ n $ игроков. Ценности независимы, $ X_{i}=V_{i}$. Пусть функция распределения сигналов имеет вид $ F(x)=x^{a} $ на $ [0;1] $, где $ a $ — это некая константа, $ a\geq 1 $.
\begin{enumerate}
\item Найдите $ MR(x) $. Является ли $ MR(x) $ возрастающей?
\item Постройте оптимальный аукцион.
\end{enumerate}

\item Петя переезжает на новую квартиру, поэтому продает свои старые шкаф и комод (варианта взять их с собой у него нет).  Потенциальных покупателей двое. Первый покупатель знает значение $ X_{1} $, второй — значение $ X_{2} $. Величины  $ X_{1} $ и  $ X_{2} $ независимы и равномерны на $ [0;1] $. Полезности первого игрока: от шкафа — $ 0.5 $, от комода — $ 0.8X_{1} $, от шкафа и комода — $ 0.5+X_{1} $. Полезности второго игрока: от шкафа — $ 0.8 $, от комода — $ X_{2} $, от шкафа и комода — $ 0.8+0.8X_{2}$
\begin{enumerate}
\item Четко опишите механизм VCG применительно к этой задаче.
\item Какова средняя прибыль продавца при использовании механизма VCG?
\end{enumerate}

\item Есть $ n $ городов. Рядом с одним из них нужно построить мусоросжигательный завод. Жители города рядом с которым будет построен завод получат отрицательную полезность $ U_{i}=-X_{i} $. Остальные получат полезность 0. Величины $ X_{i}\sim U[0;1] $ и независимы. Каждый город знает своё $ X_{i} $.
\begin{enumerate}
\item Опишите механизм VCG применительно к этой задаче. то есть предполагается, что игроки объявляют числа $ b_{i}\in [0;1] $ и механизм должен определять, у какого города строить завод и какие платежи должны сделать игроки в зависимости от $ b_{i} $.
\item Выпишите функцию плотности для компенсации, которую получают жители города рядом с которым будет построен мусоросжигательный завод.
\item Сходится ли баланс у механизма VCG в этом случае? Если нет, то сколько в среднем нужно вложить средств извне в этот механизм?
\item Что больше: компенсация или ущерб от строительства завода в механизме VCG?
\end{enumerate}


\item Кнопочный аукцион и три игрока. Ценности $ V_{1} $, $ V_{2} $ и $ V_{3} $ равномерны на $ [0;1] $ и независимы. Первый и второй игрок знают значение своих ценностей, то есть $ X_{1}=V_{1} $ и $ X_{2}=V_{2} $. А третий игрок — не знает значения своей ценности, а знает только закон распределения.
\begin{enumerate}
\item Что собой представляют стратегии игроков в этом случае? Почему их можно упростить?
\item Найдите равновесие Нэша
\end{enumerate}


\end{enumerate}


\section{Решение контрольной работы 4}

\begin{enumerate}

\item
\begin{equation}
MR(x)=x-\frac{1-x^{a}}{ax^{a-1}}=x\left(1+\frac{1}{a}\right)-\frac{1}{ax^{a-1}}.
\end{equation}
Даже без производной видно, что функция возрастает. Оптимальным будет аукцион второй цены с резервной ценой: \index{аукцион!оптимальный}
\begin{equation}
r=\left(\frac{1}{a+1}\right)^{1/a}.
\end{equation}

\item Составляем табличку:

\begin{tabular}{c|cccc}
& (Ш,К) & (К,Ш) & (КШ,-) & (-,КШ) \\
\hline
Покупатель 1 & 0.5 & $ 0.8X_{1} $ & $ 0.5+X_{1} $ & 0 \\
Покупатель 2 & $ X_{2} $ & 0.8 & 0 & $ 0.8+0.8X_{2} $ \\
Сумма & $ 0.5+X_{2} $& $ 0.8+0.8X_{1} $ & $ 0.5+X_{1} $ & $ 0.8+0.8X_{2} $ \\
\end{tabular}

Покупатели одновременно декларируют свои значения $ X_{i} $. Мы знаем, что в механизме VCG им будет оптимально говорить правду. Механизм VCG максимизирует сумму полезностей. В данном случае мы замечаем, что $ 0.8+0.8X_{1}>0.5+X_{1} $ при любых $ X_{1} \in [0;1]$. И аналогично для $ X_{2} $. Поэтому правило выбора решения имеет вид:

Если $ X_{1}>X_{2} $, то комод — первому и шкаф — второму. Если $ X_{1}<X_{2} $, то комод и шкаф — второму.

Осталось правило платежей:

Если $ X_{1}>X_{2} $, то первый платит $ 0.8X_{2} $, а второй — $ 0.5+0.2X_{1} $.

Если $ X_{1}<X_{2} $, то первый платит 0, а второй — $ 0.5+X_{1} $.

Получаем выручку продавца:
\begin{equation}
R=(0.5+0.2X_{1}+0.8X_{2})1_{X_{1}>X_{2}}+(0.5+X_{1})1_{X_{1}<X_{2}}.
%=0.5+(0.2X_{1}+0.8X_{2})1_{X_{1}>X_{2}}+X_{1}(1-1_{X_{1}<X_{2}})=0.5+X_{1}+(0.8X_{2}-0.8X_{1})1_{X_{1}>X_{2}}
\end{equation}

Находим:
\begin{equation}
\E(X_{1}1_{X_{1}>X_{2}})=\int_{0}^{1}\int_{0}^{x_{1}}x_{1} \cdot 1 \, dx_{2}dx_{1}=1/3.
\end{equation}

Аналогично $ \E(X_{1}1_{X_{1}<X_{2}})=1/6 $.

Получаем, что средняя выручка равна
\begin{equation}
\E(R)=0.5\cdot \frac{1}{2}+0.2\cdot \frac{1}{3}+0.8\cdot \frac{1}{6}+0.5\cdot \frac{1}{2}+\frac{1}{6}=\frac{13}{15}.
\end{equation}

\item  Каждый город одновременно декларирует свой ущерб. \index{задача!о мусоросжигательном заводе}

Правило принятия решения: завод построить рядом с городом, сообщившим наименьший ущерб.

Правило платежей: город, рядом с которым строят завод, должен получить компенсацию в размере минимума ущербов остальных городов. Остальные города ничего не платят и не получают.

Автоматически получаем, что механизм VCG требует вливания средств извне. Так как компенсация равна не самому маленькому ущербу, а ущербу второму по малости, то компенсация всегда больше ущерба.

Функция плотности: $ p(y)=n\cdot 1\cdot (n-1)y(1-y)^{n-2} $.

Средняя компенсация равна (для взятия интеграла можно сделать замену $ z=1-y $):
\begin{equation}
\E(K)=\int_{0}^{1}y\cdot n(n-1)y(1-y)^{n-2} \, dy=\frac{2}{n+1}.
\end{equation}


\item  Поскольку третий игрок ничего не знает, а только видит, сколько игроков осталось в игре, то его стратегия описывается двумя числами, $ b_{3}^{3} $ и $ b_{3}^{2} $. Эти числа говорят, до какой цены давить кнопку, если в игре осталось три и два игрока. \index{аукцион!кнопочный}

Стратегия первого игрока описывается тремя функциями: $ b_{1}^{3}(x) $ — до какой цены давить кнопку, если в игре три игрока, $b_{1}^{2a}(x,p)$ — до какой цены давить кнопку, если в игре двое: я и второй; $b_{1}^{2b}(x,p)$ — до какой цены давить кнопку, если в игре двое: я и третий. Стратегия второго игрока имеет такой же вид.

Поскольку ценности независимы, то никакой полезной информации от наблюдения за ценами выхода других игроков мы не получаем. Следовательно, стратегию третьего игрока можно заменить одним числом $ b_{3} $, а стратегию первого — одной функцией $b_{1}(x)$.

Получаем аукцион второй цены. Игроки ориентируются на ожидаемый выигрыш. Поэтому с точки зрения третьего игрока его ценность равна 0.5, то есть равновесие Нэша имеет вид $ b_{3}=0.5 $; $ b_{1}(x)=x $; $ b_{2}(x)=x $.


\end{enumerate}



\section{Догонялка}

Тем, кто по уважительной причине пропустил какую-либо из контрольных, предлагается догонялка:

\input{kr_dogon.tex}

\section{Подсказки к догонялке}

\begin{enumerate}

\item[4.] Количество чудо-швабр обозначим буквой $k$. \index{задача!о продаже чудо-швабр}

$ k=1 $: $ b^{3}(x)=3x $, $ b^{2}(x,p_{3})=2x+p/3$

$ k=2 $. Поскольку аукцион заканчивается при выходе первого игрока, то стратегия определяется функцией $ b^{3}(x)$.

Поскольку мы такой аукцион не решали, то используем стандартный подход с максимизацией прибыли:
\begin{multline}
\E(Profit_{1}|X_{1}=x,Bid_{1}:=b_{1})=\\
=\E((X_{1}+X_{2}+X_{3}-b(Y_{2}))1_{b_{1}>b(Y_{2})}|X_{1}=x,Bid_{1}:=b_{1})
\end{multline}

Чудо-замена $b_{1}=b(a)$ и независимость $ X_{i} $ дают нам:
\begin{multline}
\pi_{1}(x,b(a))=\E((x+X_{2}+X_{3})-b(Y_{2}))1_{a>Y_{2}})=\\
=x\P(Y_{2}<a)+2\E(X_{2}\cdot 1_{Y_{2}<a})-\E(b(Y_{2})\cdot 1_{Y_{2}<a})
\end{multline}

Сосредоточимся на $\E(X_{2}\cdot 1_{Y_{2}<a}) $:
\begin{multline}
\E(X_{2}\cdot 1_{Y_{2}<a})=\E(X_{2}\cdot 1_{X_{2}\wedge X_{3}<a})=\\
=\E(X_{2}\cdot 1_{X_{2}<a}\cdot 1_{X_{2}<X_{3}})+\E(X_{2}\cdot 1_{X_{3}<a}\cdot 1_{X_{3}<X_{2}})
\end{multline}


\end{enumerate}


\section{Прочие задачи}

Осторожно! Эти задачи не проверялись на наличие приличного решения.

\input{more_problems.tex}


\chapter*{Впечатления о курсе}
\addcontentsline{toc}{chapter}{Впечатления о курсе}

Курс был прочитан дистанционно зимой 2011 года. Я находился в Бельгии, а слушатели — в Москве. Для курса был организован блог \url{auctiontheory.wordpress.com}. Из 45 записавшихся на факультатив 12 добровольно отписались, а остальные 33 богатыря дошли до победы!

Курс состоял из четырёх больших лекций. Раз в две недели я выкладывал новую видеолекцию, письменный конспект к ней и оперативно отвечал на вопросы в блоге. Примерно через десять дней после лекции ассистенты проводили очную контрольную работу. Итого было четыре контрольные работы. Каждая письменная работа имела 25\%-ный вес в итоговой оценке.

Маленькие находки:

В 2011 году я выбрал видеохостинг vimeo, хотя сейчас выбрал бы youtube. Очень много времени уходило на конвертацию видео из формата камеры в формат, пригодный для vimeo. А youtube выполняет конвертацию автоматически.

Лекции, естественно, содержали опечатки, которые читатели находили. Я исправлял опечатки, делал дополнительные пояснения там, где неудачно написано. Иногда из-за правки в тексте возникала новая нумерация формул, и, когда следующий комментирующий писал «в формуле (12) опечатка», было не ясно, какую конкретно формулу (12) надо смотреть. В результате я пришёл к следущей процедуре. При исправлении опечаток вывешиваю новую версию лекции. При этом старая версия лекции остаётся доступной. Внутри лекции написана её версия. И при обсуждении можно сказать «в формуле (12) версии 2 опечатка».

Несмотря на понимание сложности курса, я сильно недооценил трудность задач. В результате третья контрольная была написана на низкий средний балл, и я дал вместо неё домашнюю работу. Кроме того, вместо всех плохо написанных работ я предложил домашнюю работу-«догонялку» в конце курса. Для выравнивания результатов контрольной работы и домашней работы я использовал простое масштабирование — вычитал выборочное среднее и делил на выборочное стандартное отклонение.

Чтобы развивать умение отличать строгое решение от правдоподобного рассуждения, я предложил студентам после контрольной работы выставлять себе оценку за каждую решенную задачу. Тем, кто верно себя оценивал, я добавлял небольшой бонус. Принесло ли это пользу студентам — не знаю, надеюсь, что да. А я наглядно увидел эффект Даннинга—Крюгера: сильные студенты гораздо точнее оценивают свои знания, чем слабые.

Поначалу я писал лекции, используя редактор lyx, но потом перешёл на простой тех. Одна из идей была в том, чтобы вывешивать студентам лекции в формате lyx. Смысл идеи в том, чтобы, не тратя времени на освоение теха, студенты смогли бы его освоить с помощью lyx. Но от этой идеи пришлось отказаться, курс и без того был сложным.

Ассистенты, проводившие контрольные работы, не знали теории аукционов, поэтому ответить на вопросы во время проведения или дать подсказку было некому. Чтобы снизить волнение и страх контрольной, я вывешивал предварительно «тизер». Отчасти «тизер» помогал студентам по ключевым словам понять, к чему готовиться. В частности, в нём чётко прописаны правила проведения работы. На контрольных разрешалось использовать в качестве официальной шпаргалки заранее подготовленный лист А4.

Дистанционно курс вести трудно. Вживую гораздо приятнее. Очень помогало обсуждение лекций и задач в блоге. Большое спасибо тридцати трём «богатырям»!!

Завершу впечатления о курсе «тизером» последней письменной работы:

\subsection*{Моделирование аукционов. Контрольная работа 4}

\begin{enumerate}
\item Можно пользоваться калькулятором. Вопрос в том, нужно ли?
\item Можно решать задачи в любом порядке.
\item С собой можно принести один лист А4, где заранее могут быть написаны (именно написаны, а не напечатаны) любые формулы, теоремы или комментарии.
\item Продолжительность работы 1 час 20 минут.
\item Условия нельзя забрать с собой. Условия и решения открыто доступны на \url{auctiontheory.wordpress.com} после окончания контрольной.
\item Обсуждать задачи во время работы нельзя.
\item Человек, проводящий контрольную, не будет отвечать на вопросы по тексту задач.
\item Скорее всего, в задачах нет очепяток. Если, по твоему мнению, опечатка есть, то её нужно исправить самому, исходя из своего представления о хорошей задаче. При этом нужно четко отразить этот факт перед началом решения. Например: «По-моему, в тексте есть опечатка и вместо \ldots должно быть \ldots». Твоя гипотеза об опечатках является личной и не подлежит обсуждению во время работы.
\item Насколько подробно всё расписывать — решай сам, исходя из конкретной ситуации. Очевидно, что в примере $ 1+2+3=\ldots $ ответ можно написать сразу, а взятие интеграла $ \int x^{5}\cos(x)\, dx $ требует каких-то промежуточных записей.
\item Паниковать на контрольной строжайше запрещено!
%\item Каждая из 5 задач весит 5 баллов.
\item Для каждой задачи обязательно нужно спрогнозировать свою оценку. Не надо скромничать, лучше попытаться объективно оценить своё решение.  За неверное оценивание баллы снижаться не будут, а верное оценивание даст возможность чему-то научиться. Опыт показывает, что оценка своих собственных решений позволяет резко улучшить их качество. Прогноз своей оценки пишем в табличку!
\item Не забудь подписать свою работу. Пожалуйста!

\end{enumerate}

\begin{tabular}{ccccccc}
\toprule
Задача & 1 & 2 & 3 & 4 & Итого \\
\midrule
Максимальный балл & 5 & 5 & 10 & 5 & 25 \\
Прогноз оценки &  &  &  &  &   \\
Оценка &  &  &  &  &   \\
\bottomrule
\end{tabular}
\newpage

\begin{enumerate}

\item На аукционе участвуют $n$ игроков. ...... . ...... . . .. ... .......... ........ ........ ........ ........ .......... ........... ............ .......... .......... .......... ...... ........ ....... ....... ..... ......
\begin{enumerate}
\item Найдите $ MR(x) $. ...... ........
\item Постройте оптимальный аукцион.
\end{enumerate}

\item Петя переезжает на новую квартиру, поэтому ..... ......... ........... ........... ........ ........... ............ ......... ............. ............. ............ .......... .......... ........... ........... ......... ........ .......... ........... ...... Потенциальных покупателей двое. Первый покупатель знает значение $ X_{1} $, второй — значение $ X_{2} $. Величины  $ X_{1} $ и  $ X_{2} $ независимы и равномерны на $ [0;1] $. Полезности первого игрока: ..... ......... ............ .......... ........... .......... .......... ............ Полезности первого игрока: ............ ............. ............. ............. .............
\begin{enumerate}
\item Четко опишите механизм VCG применительно к этой задаче.
\item Какова средняя прибыль продавца при использовании механизма VCG?
\end{enumerate}


\item Есть $ n $ городов. ........ ............... .............. ............ .............. Жители города ........... ............. ............... .............. ................ ........ ......... получат ........... полезность ........ ......... .......... .......... ....... .........  Величины $ X_{i}\sim U[0;1] $ и независимы. Каждый город знает свое $ X_{i} $.
\begin{enumerate}
\item Опишите механизм VCG применительно к этой задаче. То есть предполагается, что игроки объявляют числа $ b_{i}\in [0;1] $ и механизм должен определять, ........... ........... ............. .............. ............. ............ ...........
\item Выпишите функцию плотности для .......... ............. ................ ............... ........... .......... ........... ............ .........
\item Сходится ли баланс у механизма VCG в этом случае? Если нет, то сколько в среднем нужно вложить средств извне в этот механизм?
\item Что больше: .......... или ..... .......... .......... в механизме VCG?
\end{enumerate}


%\item Есть $ 3 $ города. Рядом с одним из них нужно построить мусоросжигательный завод. Жители города рядом с которым будет построен завод получат отрицательную полезность $ U_{i}=-X_{i} $. Остальные получат полезность 0. Величины $ X_{i}\sim U[0;1] $ и независимы. Каждый город знает свое $ X_{i} $. Города одновременно называют требуемую компенсацию $ b_{i} $. Завод строится у того города, у которого $ b_{i} $ меньше. Остальные города выплачивают компенсацию поровну.
%\begin{enumerate}
%\item Найдите равновесие Нэша
%\item Как надо изменить этот механизм, чтобы он стал правдивым?
%\end{enumerate}



%\item Есть один покупатель и один продавец. Ценности товара: $ X_{1} $ --- для покупателя, $ X_{2} $ --- для продавца. Величины $ X_{i} $ независимы и равномерны на $ [0;1] $.
%\begin{enumerate}
%\item Опишите механизм VCG применительно к этой задаче. Т.е. предполагается, что игроки объявляют числа $ b_{i}\in [0;1] $ и механизм должен определять, кому отдать товар и какие платежи должны сделать игроки в зависимости от $ b_{i} $.
%\item Каков средний баланс механизма VCG в этой задаче?
%\item Предположим, что вместо VCG используется такой механизм: игроки одновременно называют желаемые цены, $ b_{1} $ и $ b_{2} $. Если $ b_{1}>b_{2} $, то обмен происходит по цене $ 0.5(b_{1}+b_{2}) $. Найдите равновесие Нэша.
%\item Верно ли, что при втором механизме обмен происходит если и только если $ X_{1}>X_{2} $?
%\end{enumerate}

%Подсказка: равновесие Нэша будет в линейных стратегиях



\item Кнопочный аукцион и три игрока. Ценности $ V_{1} $, $ V_{2} $ и $ V_{3} $ ........ ........ ............. ............. ............. .............. ........... .......... ........... .......... ........... ........... ......... ........... .......
\begin{enumerate}
\item Что собой представляют стратегии игроков в этом случае? Почему их можно упростить?
\item Найдите равновесие Нэша.
\end{enumerate}
\end{enumerate}




\printindex



\nocite{nikolenko:tem, menesez:iat, krishna:at, milgrom:patw, klemperer:atp}

\bibliography{opit}
\addcontentsline{toc}{chapter}{Литература}




\end{document}
