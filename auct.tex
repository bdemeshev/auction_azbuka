\documentclass[11pt, openany]{book}

\usepackage[margin=10mm, paperwidth=160mm, paperheight=220mm, includehead]{geometry}


\pagestyle{headings}


\usepackage{etex} % расширение классического tex
% в частности позволяет подгружать гораздо больше пакетов, чем мы и займёмся далее

\usepackage{verbatim} % для многострочных комментариев
\usepackage{makeidx} % для создания предметных указателей

%%%%% russian xetex
\usepackage{fontspec}
\usepackage{polyglossia}

\setmainlanguage{russian}
\setotherlanguages{english}

% download "Linux Libertine" fonts:
% http://www.linuxlibertine.org/index.php?id=91&L=1
\setmainfont{Linux Libertine O} % or Helvetica, Arial, Cambria
% why do we need \newfontfamily:
% http://tex.stackexchange.com/questions/91507/
\newfontfamily{\cyrillicfonttt}{Linux Libertine O}
%%%%% end russian xetex



\usepackage{setspace}
\usepackage{amsmath, amsfonts, amssymb, amsthm}
\usepackage{mathrsfs} % sudo yum install texlive-rsfs
\usepackage{dsfont} % sudo yum install texlive-doublestroke
\usepackage{array, multicol, multirow, bigstrut} % sudo yum install texlive-multirow
\usepackage{indentfirst} % установка отступа в первом абзаце главы
\usepackage{bm}
\usepackage{bbm} % шрифт с двойными буквами


\usepackage{answers} % дележка ответов и вопросов

\usepackage{dcolumn} % центрирование по разделителю для apsrtable

% создание гиперссылок в pdf
\usepackage[unicode, colorlinks=true, urlcolor=blue, hyperindex, breaklinks]{hyperref}
% moving to xelatex: pdftex option removed

% свешиваем пунктуацию
% теперь знаки пунктуации могут вылезать за правую границу текста, при этом текст выглядит ровнее
\usepackage{microtype}

\usepackage{textcomp}  % Чтобы в формулах можно было русские буквы писать через \text{}



\usepackage{etoolbox} % нужно для ifdef


\usepackage{xcolor}


\usepackage{float}
\usepackage{longtable}
\usepackage{soulutf8}  %%% error if ommitted but does not interfere with knitr

\usepackage{enumitem} % дополнительные плюшки для списков
%  например \begin{enumerate}[resume] позволяет продолжить нумерацию в новом списке

\usepackage{mathtools}
\usepackage{cancel, xspace} % sudo yum install texlive-cancel


\usepackage{numprint} % sudo yum install texlive-numprint
\npthousandsep{,}\npthousandthpartsep{}\npdecimalsign{.}


\usepackage{subfigure} % для создания нескольких рисунков внутри одного


\usepackage{tikz}
\usepackage{tikz-3dplot}
\usepackage{pgfplots} % язык для рисования графики из latex'a
\usetikzlibrary{trees} % tikz-прибамбас для рисовки деревьев
\usepackage{tikz-qtree} % альтернативный tikz-прибамбас для рисовки деревьев
\usetikzlibrary{arrows} % tikz-прибамбас для рисовки стрелочек подлиннее


\usepackage{todonotes} % для вставки в документ заметок о том, что осталось сделать
% \todo{Здесь надо коэффициенты исправить}
% \missingfigure{Здесь будет Последний день Помпеи}
% \listoftodos — печатает все поставленные \todo'шки


% более красивые таблицы
\usepackage{booktabs}
% заповеди из докупентации:
% 1. Не используйте вертикальные линни
% 2. Не используйте двойные линии
% 3. Единицы измерения - в шапку таблицы
% 4. Не сокращайте .1 вместо 0.1
% 5. Повторяющееся значение повторяйте, а не говорите "то же"




% вместо горизонтальной делаем косую черточку в нестрогих неравенствах
\renewcommand{\le}{\leqslant}
\renewcommand{\ge}{\geqslant}
\renewcommand{\leq}{\leqslant}
\renewcommand{\geq}{\geqslant}

% делаем короче интервал в списках
\setlength{\itemsep}{0pt}
\setlength{\parskip}{0pt}
\setlength{\parsep}{0pt}


% DEFS
\def \mbf{\mathbf}
\def \msf{\mathsf}
\def \mbb{\mathbb}
\def \tbf{\textbf}
\def \tsf{\textsf}
\def \ttt{\texttt}
\def \tbb{\textbb}

\def \wh{\widehat}
\def \wt{\widetilde}
\def \ni{\noindent}
\def \ol{\overline}
\def \cd{\cdot}
\def \bl{\bigl}
\def \br{\bigr}
\def \Bl{\Bigl}
\def \Br{\Bigr}
\def \fr{\frac}
\def \bs{\backslash}
\def \lims{\limits}
\def \arg{{\operatorname{arg}}}
\def \dist{{\operatorname{dist}}}
\def \VC{{\operatorname{VCdim}}}
\def \card{{\operatorname{card}}}
\def \sgn{{\operatorname{sign}\,}}
\def \sign{{\operatorname{sign}\,}}
\def \xfs{(x_1,\ldots,x_{n-1})}
\def \Tr{{\operatorname{\mbf{Tr}}}}
\DeclareMathOperator*{\argmin}{arg\,min}
\DeclareMathOperator*{\argmax}{arg\,max}
\DeclareMathOperator*{\amn}{arg\,min}
\DeclareMathOperator*{\amx}{arg\,max}
\def \cov{{\operatorname{Cov}}}

\DeclareMathOperator{\Cov}{Cov}

\def \xfs{(x_1,\ldots,x_{n-1})}
\def \ti{\tilde}
\def \wti{\widetilde}


\def \mL{\mathcal{L}}
\def \mW{\mathcal{W}}
\def \mH{\mathcal{H}}
\def \mC{\mathcal{C}}
\def \mE{\mathcal{E}}
\def \mN{\mathcal{N}}
\def \mA{\mathcal{A}}
\def \mB{\mathcal{B}}
\def \mU{\mathcal{U}}
\def \mV{\mathcal{V}}
\def \mF{\mathcal{F}}

\def \R{\mbb R}
\def \N{\mbb N}
\def \Z{\mbb Z}
\def \P{\mbb{P}}
%\def \p{\mbb{P}}
\def \E{\mbb{E}}
\def \D{\msf{D}}
\def \I{\mbf{I}}

\def \a{\alpha}
\def \b{\beta}
\def \t{\tau}
\def \dt{\delta}
\def \e{\varepsilon}
\def \ga{\gamma}
\def \kp{\varkappa}
\def \la{\lambda}
\def \sg{\sigma}
\def \sgm{\sigma}
\def \tt{\theta}
\def \ve{\varepsilon}
\def \Dt{\Delta}
\def \La{\Lambda}
\def \Sgm{\Sigma}
\def \Sg{\Sigma}
\def \Tt{\Theta}
\def \Om{\Omega}
\def \om{\omega}


\def \ni{\noindent}
\def \lq{\glqq}
\def \rq{\grqq}
\def \lbr{\linebreak}
\def \vsi{\vspace{0.1cm}}
\def \vsii{\vspace{0.2cm}}
\def \vsiii{\vspace{0.3cm}}
\def \vsiv{\vspace{0.4cm}}
\def \vsv{\vspace{0.5cm}}
\def \vsvi{\vspace{0.6cm}}
\def \vsvii{\vspace{0.7cm}}
\def \vsviii{\vspace{0.8cm}}
\def \vsix{\vspace{0.9cm}}
\def \VSI{\vspace{1cm}}
\def \VSII{\vspace{2cm}}
\def \VSIII{\vspace{3cm}}


\newcommand{\grad}{\mathrm{grad}}
\newcommand{\dx}[1]{\,\mathrm{d}#1} % для интеграла: маленький отступ и прямая d
\newcommand{\ind}[1]{\mathbbm{1}_{\{#1\}}} % Индикатор события
%\renewcommand{\to}{\rightarrow}
\newcommand{\eqdef}{\mathrel{\stackrel{\rm def}=}}
\newcommand{\iid}{\mathrel{\stackrel{\rm i.\,i.\,d.}\sim}}
\newcommand{\const}{\mathrm{const}}
 % use the local copy


\newcommand{\winpro}{\phi} % специальный знак для \pi из Кришны
\newcommand{\indef}[1]{\textbf{#1}}

\numberwithin{equation}{page} % уравнения нумеруются на каждой стр. отдельно

\newtheorem{myth}[equation]{Теорема} % нумерация сквозная с уравнениями

\theoremstyle{definition} % убирает курсив и что-то ещё наверное делает ;)
\newtheorem{mydef}[equation]{Определение}

\theoremstyle{definition}
\newtheorem{myex}[equation]{Пример}

%\newtheorem{assertion}{Утверждение}
%\newtheorem{lemma}{Лемма}

\theoremstyle{definition}
\newtheorem*{myproof}{Доказательство}


\begin{comment}
% todo list
\begin{enumerate}
\item Решение догонялки
\item Сделать индекс
\item Дописать про афиллированные с.в.
\item Нарисовать картинки
\item Сделать вводную по вероятностям с вопросами:
$P(X_{2}<a|X_{1}=x)$, $ E(X_{2}1_{X_{2}<a}|X_{1}=x) $, $ p(x_{2}|x_{1}) $ если известна $ p(x_{1},x_{2}) $
\item Сделать вводную по взятию производных от интеграла:
$ f(a)=\int_{0}^{a}g(x)dx $ найти $ f'(a) $. И более забубенистое: $ f(a)=\int_{0}^{a}\int_{y}^{1}p(x,y)dxdy $
\item Предупреждение: у непрерывных величин $ P(X=x)=0$
\item Переоформить теорему о равновесии на аукционе первой цены так \ldots  это решение диф. ура, с b(0)=0. Кстати там может быть пропущен знак минус в формуле решения. Проверить.
\item Проверить нет ли где ошибки связанной с тем, что (например) нет совместной плоности у пары $ X_{2} $, $ Y_{2} $. Сделать вопрос с подвохом. Типа найдите плотность.
\item Список сопоставляющий обозначения: Vishna, Klemperer?, Introduction to auctions\ldots
\end{enumerate}
\end{comment}






\title{Моделирование аукционов. Азбука. }
\author{Борис Демешев}
\date{\today}



\makeindex % команда для создания предметного указателя
\bibliographystyle{plain} % стиль оформления ссылок



\begin{document}

\maketitle
\tableofcontents{}

\chapter{О книжке}

\chapter{Аукционы бывают разные}

%Что почитать:

%Базовый текст: IAT, chapter 3. В IAT поначалу автор использует обозначение $ v_{i} $ для ценностей, а потом меняет %обозначение на $ X_{i} $. Мы сразу будем использовать $ X_{i} $.


\section{Три аукциона и три модели}

\begin{itemize}

\item \indef{Английский аукцион}\index{аукцион!английский}. Именно этот аукцион описан в «12 стульях» Ильфа и Петрова. Игроки по очереди называют ставки. Каждая последующая ставка должна быть больше, чем предыдущая. Товар достаётся тому, кто назвал наибольшую цену. Победитель платит за товар столько, сколько он сам поставил\footnote{Про оплату комиссионого сбора в 12 стульях мы, конечно, помним.}. По этому принципу устроен самый крупный современный он-лайн аукцион товаров в Интернете, Ebay\index{аукцион!ebay}, \url{http://www.ebay.com/}.

\item \indef{Голландский аукцион цветов}\index{аукцион!голландский}. Большинство цветов, продающихся в России, были куплены на аукционе цветов в Голландии. Этот традиционный аукцион отличается от английского. Потенциальные покупатели цветов сидят в общем зале. Перед ними на стене — большие часы с одной стрелкой. Стрелка показывает текущую цену товара. Изначально цена высока и никто не хочет покупать. С движением стрелки цена опускается. Наступит момент, когда одного из покупателей цена наконец устроит. Он получает товар и платит соответствующую цену.

\item \indef{Аукцион интернет-рекламы}\index{аукцион!Интернет-рекламы}. Когда пользователь набирает в поисковике (в Яндексе, Гугле или любом другом) какое-нибудь слово, к примеру «НИУ-ВШЭ», поисковик выдает найденные страницы и рекламные ссылки. Естественно, рекламодатели платят за то, что поисковик показывает их рекламные ссылки. Более того, рекламные ссылки продаются на аукционе! Представим себе, что поисковик продаёт одно рекламное место. Желающие рекламодатели независимо друг от друга направляют свои заявки: «я готов платить за него 5 копеек за клик», «я готов платить 10 копеек за клик», «я готов платить 7 копеек за клик». Побеждает, естественно, тот, кто готов платить больше других. Но платит он не ту сумму, которую заявил в своей заявке! Победитель платит вторую по величине ставку! В нашем примере с тремя заявками рекламное место достаётся тому, кто был готов платить 10 копеек за клик, но платить он будет 7 копеек за клик. В реальности всё чуть сложнее. Например, рекламных мест может быть несколько, тогда тот, кому досталось второе по притягательности место, платит ставку того, кому досталось третье. Гугл продаёт свои рекламные ссылки на \url{adwords.google.com}.

\end{itemize}

Этим трём реальным примерам мы сопоставим три простых модели.


Общее между тремя моделями:

На аукционе продаётся единица неделимого товара, скажем одна морковка. За право получить морковку борятся $ n $ покупателей.

% Для $ i $-го покупателя морковка имеет некую ценность $ V_{i} $. Для начала мы предположим, что эти ценности — независимые случайные величины, и каждый покупатель знает значение своей $ V_{i} $ и не знает значения остальных ценностей.

%Ставку $ i $-го покупателя будем обозначать $ b_{i} $ (b от слова bid).


\begin{enumerate}
\item \indef{Кнопочный аукцион}\index{аукцион!кнопочный}. У каждого покупателя есть кнопка. Стартовая цена равна нулю. Изначально все покупатели давят на свои кнопки. Затем цена начинает расти. Как только игрок отпускает свою кнопку, он покидает аукцион. Аукцион прекращается, когда остаётся лишь один игрок, жмущий на кнопку. Товар достаётся этому игроку по цене, сложившейся на момент остановки.

Эта модель является упрощением реального английского аукциона. В реальности часто бывает, что игроки начинают активно играть лишь незадолго до окончания аукциона. Эта модель не предназначена для описания такого явления. В реальности игроки могут повышать текущую цену на произвольную величину, здесь же цена меняется непрерывно. Тем не менее, многие свойства английского аукциона кнопочная модель ловит.

\item \indef{Закрытый аукцион первой цены}\index{аукцион!первой цены}. Покупатели одновременно делают свои ставки. Товар достаётся тому покупателю, который назвал самую высокую цену. Победитель платит продавцу свою ставку.

Такой аукцион лучше всего подходит для моделирования голландского аукциона. Действительно, на голландском аукционе никакой другой информации, кроме того, по какой цене был продан товар, ни один из игроков не получает. Голландский аукцион и аукцион первой цены стратегически эквивалентны — множество стратегий у каждого игрока одно и то же, и функция выигрышей — такая же. Тем не менее, в реальности на цену аукциона влияет, например, такой фактор как скорость движения стрелки на часах.
\item \indef{Закрытый аукцион второй цены}\index{аукцион!второй цены}. Покупатели одновременно делают свои ставки. Товар достаётся тому покупателю, который назвал самую высокую цену. Победитель платит продавцу вторую по величине ставку, то есть наибольшую ставку сделанную покупателями за исключением его самого.

Эта модель хорошо подходит для аукциона интернет-рекламы с одним рекламным местом.
\end{enumerate}

Чтобы разграничить реальность и модели, мы будем использовать слова «Английский аукцион», «Голландский аукцион» для описания реальных явлений, а слова «аукцион первой цены», «аукцион второй цены», «кнопочный аукцион» — для описания моделей.

Иногда вводят понятия: \indef{открытый аукцион}\index{аукцион!открытый}, то есть аукцион, где игроки видят ставки других игроков и \indef{закрытый аукцион},\index{аукцион!закрытый} где игроки не видят ставок других игроков. По этой классификации кнопочный аукцион является открытым, а аукционы, где игроки делают ставки одновременно — закрытыми.

Для формального описания наших задач нам потребуется куча обозначений. Мы их будем вводить потихоньку, поэтому пугаться не стоит. Нужно всего лишь быть очень аккуратным и отличать заглавные и строчные буквы, например, $x$ и $X$.
% А листочек с обозначениями можно для удобства распечатать.
Для удобства вынесем все обозначения в отдельный список. Встречайте\ldots

\newpage

\begin{center}
\indef{Обозначения!}\index{список обозначений}
\end{center}

\indef{Событие:}
\begin{itemize}
\item $W_{i}$ — событие\index{список обозначений}, состоящее в том, что победителем аукциона стал игрок $ i $.
\end{itemize}


\indef{Случайные величины:}
\begin{itemize}
\item $ X_{i} $ — случайная величина, сигнал о ценности, получаемый игроком. Значение $ X_{i} $ известно игроку $ i $. Функцию распределения этой случайной величины обозначим $ F() $, а функцию плотности — $ f() $.

\item $ V_{i} $ — случайная величина, ценность товара для игрока $ i $. Если игрок точно знает ценность товара, то $ V_{i}=X_{i} $. Есть множество других возможностей, например, $ X_{i}=V_{i}+e_{i} $, где $ e_{i} $ — некая случайная ошибка.

\item $Bid_{i}$ — случайная величина, ставка, которую сделает игрок $ i $ в равновесии Нэша. В симметричном равновесии Нэша:
\begin{equation}
Bid_{i}=b(X_{i})
\end{equation}

\item $Pay_{i}$ — случайная величина, выплата, которую делает игрок $ i $ в равновесии Нэша.

\item $ R $ — случайная величина, доход продавца в равновесии Нэша:
\begin{equation}
R=Pay_{1}+Pay_{2}+\ldots+Pay_{n}
\end{equation}

\item $Profit_{i}$ — случайная величина, выигрыш  игрока $i$ в равновесии Нэша:
\begin{equation}
Profit_{i}=V_{i}\cdot 1_{W_{i}}-Pay_{i}
\end{equation}

\item $Y_{1}$, \ldots, $ Y_{n-1} $ — случайные величины $ X_{2} $, \ldots, $ X_{n} $ упорядоченные по убыванию. В частности, $ Y_{1}=\max\{X_{2},\ldots,X_{n}\} $ и $ Y_{n-1}=\min\{X_{2},\ldots,X_{n}\} $

\end{itemize}


\newpage
\indef{Детерминистические функции:}

\begin{itemize}
\item $ b(\cdot ) $ — неслучайная функция, зависимость ставки от сигнала в равновесии Нэша

\item $q(x)=\P(W_{1}|X_{1}=x)$ — вероятность выигрыша первого игрока, если $ X_{1}=x $ в равновесии Нэша

\item $pay_{1}(x)=\E(Pay_{1}|X_{1}=x)$ — средняя выплата первого игрока, если $ X_{1}=x $ в равновесии Нэша

\end{itemize}

При поиске равновесия Нэша полезно ещё одно обозначение. Мы ищем наилучший ответ первого игрока на действия остальных, поэтому все игроки кроме первого используют равновесные стратегии, а первый игрок ставит константу $b_1$ вне зависиости от сигнала.

\begin{itemize}

%\item $\widehat{q}(b)=\P(W_{1}|Bid_{1}=b_{1})$ — вероятность выигрыша, если используется стратегия $ Bid_{1}=b_{1} $.

%\item $\widehat{pay_{1}}(b_{1})=\E(Pay_{1}|Bid_{1}=b_{1})$ — средняя выплата, если используется стратегия $ Bid_{1}=b_{1} $.

\item $\pi_{1}(x,b_{1})=\E(Profit_{1}|X_{1}=x ; Bid_{1}=b_{1})$ — средний выигрыш первого игрока, если $ X_{1}=x $, он ставит константу $b_{1} $, а остальные игроки используют равновесные стратегии.
\end{itemize}

\newpage


Для полного описания моделей нужно сказать, как распределены ценности $ V_{i} $ и какую информацию $ X_{i} $ о ценностях получают игроки. Наиболее популярен анализ двух частных случаев:
\begin{enumerate}
\item Частные независимые ценности\index{ценности!частные независимые}. Каждый игрок знает, какую ценность товар представляет для него. то есть $X_{i}=V_{i}$.
Такая ситуация возникает, если товар сложно перепродать вне аукциона или игроки не собираются делать этого. Для простоты ценности предполагают независимыми случайными величинами.
\item Общая ценность\index{ценности!общая ценность}. Если товар можно легко перепродать и купить вне аукциона по стабильной цене, то ценность товара для каждого игрока определяется этой рыночной ценой товара. То есть $ V_{1}=V_{2}=V_{3}=\ldots=V_{n}=V $. При этом игроки могут не знать этого $ V $. Каждый игрок знает лишь свой сигнал $ X_{i} $, который зависит от $ V $ но не обязательно ему равен.
\end{enumerate}

В общем случае, который мы проанализируем, не будет ни равенства ценностей, ни независимости сигналов. Но начнем мы со случая частных независимых ценностей.

А теперь давайте найдём оптимальные стратегии игроков и средний доход продавца в трёх моделях.

\section{Поиск оптимальных стратегий}

Кратко напомним предпосылки. Сигнал $ X_{i} $, который получает от Природы игрок $i$, — это и есть ценность товара для него, то есть $ X_{i}=V_{i} $. Пусть ценности $ X_{i} $ будут независимыми и равномерными на отрезке $ [0;1] $ случайными величинами. Мы ограничимся поиском симметричного равновесия, то есть равновесия, где все игроки используют одинаковую стратегию. Фактические ставки при этом могут отличаться! Стратегия — это функция $b()$ от ценности, и даже если эти фукнции $ b() $ одиковые, величины $ b(X_{i}) $ будут разными в силу того, что ценности $ X_{i} $ будут разными.


До начала игры игроки ничем не отличаются: у них одинаковый закон распределения ценности товара, поэтому при анализе мы будем изучать поведение первого игрока.

Поехали!
\begin{enumerate}
\item Аукцион первой цены\index{аукцион!первой цены}.

Предположим, что есть некая равновесная стратегия $ b(x) $. Предположим также, что она дифференциируема и возрастает по $x$.

Первый игрок выигрывает, если его ставка больше всех остальных, то есть $ b_{1}>Bid_{i} $ для $ i\geq 2 $. Обозначим событие, состоящее в том, что первый игрок выиграл буквой $ W_{1} $. Его ожидаемый выигрыш равен:


\begin{equation}
\label{first_price_eq}
\pi_{1}(x,b_{1})=(x-b_{1})\P(W_{1}|X_{1}=x; Bid_{1}=b_{1})
\end{equation}

Вероятность:
\begin{equation}
\P(W_{1}|X_{1}=x; Bid_{1}=b_{1})=\P(b_{1}>Bid_{2} \cap b_{1}>Bid_{3} \cap \ldots \cap b_{1}>Bid_{n})
\end{equation}


Наша задача — найти равновесие Нэша, то есть такую ситуацию, когда использование стратегии $ b(x) $ является наилучшим действием, если остальные игроки используют такую же стратегию. Поэтому мы предположим, что все игроки кроме первого используют стратегию $ b(x) $, и найдём условие, при котором первому игроку тоже выгодно её использовать.

\begin{multline}
\P(W_{1}|X_{1}=x; Bid_{1}=b_{1})=\\
=\P(b_{1}>b(X_{2}) \cap b_{1}>b(X_{3}) \cap \ldots \cap b_{1}>b(X_{n}))
\end{multline}

В силу независимости случайных величин $ X_{i} $:

\begin{multline}
\P(W_{1}|X_{1}=x; Bid_{1}=b_{1})=\\
=\P(b_{1}>b(X_{2}))\cdot \P(b_{1}>b(X_{3}))\cdot \ldots\cdot \P(b_{1}>b(X_{n}))
\end{multline}

В силу одинакового закона распределения $ X_{i} $:
\begin{equation}
\P(W_{1}|X_{1}=x; Bid_{1}=b_{1})=\P(b_{1}>b(X_{2}))^{n-1}
\end{equation}



Далее следует начало красивого трюка!

\subsubsection*{Чудо-замена}\index{чудо-замена}

Все мы знаем, как делать замену переменных при решении задач. Входит, скажем, в уравнение переменная $ k $, а мы говорим, что вместо $k$ мы будем писать $ f(m) $. Или наоборот, входит в уравнение $ f(m) $, а мы говорим, что вместо $ f(m) $ будем писать $k$. Так вот сейчас мы сделаем замену. Мы заменим $ b_{1} $ на неизвестную (!) функцию!!!

Итак, мы делаем замену $ b_{1}:=b(a) $. С помощью этой замены мы упростим вероятность:

\begin{multline}
\P(b_{1}>b(X_{2}))^{n-1}=\P(b(a)>b(X_{2}))^{n-1}=\\
=\P(a>X_{2})^{n-1}=\P(X_{2}<a)^{n-1}=F(a)^{n-1}
\end{multline}

На всякий случай: $ F(a)=\P(X_{i}\leq a) $ — это функция распределения.

И наша прибыль имеет вид:

\begin{equation}
\pi_{1}=(x-b(a))(F(a))^{n-1}
\end{equation}


Вместо максимизации по $ b_{1} $ нам придется максимизировать по $ a $. Для нахождения оптимальной стратегии первого игрока приравниваем производную по $ a $ к нулю:

\begin{equation}
\frac{\partial \pi_{1}}{\partial a}=-b'(a)(F(a))^{n-1}+(x-b(a))(n-1)F(a)^{n-2}f(a)=0
\end{equation}


После упрощения:

\begin{equation}
-b'(a)F(a)+(x-b(a))(n-1)f(a)=0
\end{equation}


Завершение красивого трюка! Мы хотим потребовать, чтобы первому игроку тоже было оптимально использовать стратегию $ b() $. Ценность товара для первого игрока мы обозначили $ x $, значит, оптимальное $ b_{1} $ должно равняться $ b(x) $. А мы делали замену $ b_{1}=b(a) $. Значит, в точке оптимума $ b(x)=b(a) $ или $ x=a $.

\begin{equation}
\label{first_price_final_diffeq}
-b'(x)F(x)+(x-b(x))(n-1)f(x)=0
\end{equation}


Это дифференциальное уравнение можно решить в общем виде, но мы ограничимся нашим равномерным случаем.

Для равномерной случайной на отрезке $ [0;1] $ получаем $ f(x)=1 $ и $ F(x)=x $:

\begin{equation}
-b'(x)x+(x-b(x))(n-1)=0
\end{equation}

Это линейное дифференциальное уравнение\ldots Его можно решить стандартными методами, скажем, вариацией постоянной, а можно угадать вид решения. Мы пойдем путем угадывания, но я предполагаю, что все могут решить его честно! Раз фигурирует производная и первая степень $ x $, попробуем $b(x)=kx+m $:

\begin{equation}
-kx+(x-kx-m)(n-1)=0
\end{equation}

Собираем коэффициенты при $ x $:

\begin{equation}
-m(n-1)+x(-k+(1-k)(n-1))=0
\end{equation}

Это должно быть тождеством для любого $ x $, значит, $ m=0 $ и  $-k+(1-k)(n-1)=0$. Находим $ k $, $ k=\frac{n-1}{n} $.

Оптимальная стратегия первого и всех остальных игроков: $ b(x)=\frac{n-1}{n}x $.


Комментарии:

\begin{enumerate}
\item Так как $ \frac{n-1}{n}<1 $ игроки занижают свою истинную ценность в равновесии Нэша. Причем, чем меньше игроков, тем сильнее занижаются ставки по сравнению с субъективной ценностью товара.
\item Можно обойтись без красивого трюка. Для этого можно рассмотреть функцию, обратную к функции $b()$, и применить её внутри вероятности.
\item В равномерном случае можно обойтись и без дифференциальных уравнений. Для этого достаточно сделать удачную догадку до начала решения. То есть начать со слов: «Предположим, что оптимальная стратегия имеет вид $ b(x)=kx+m $,» — и максимизировать по $k$ и $m$.
\item Тот, кто попробует честно решить линейное дифференциальное уравнение, обнаружит, что общее решение имеет вид $ b(x)=c\cdot x^{-(n-1)}+\frac{n-1}{n}x $. Почему мы берём $ c=0 $? Наше дифференциальное уравнение является необходимым условием, полученным в предположении, что $ b(x) $ — возрастающая функция. Только решение при $ c=0 $ является возрастающим.
\item Мы проверили только необходимое условие максимума: первая производная равна нулю. Желающие могут взять вторую производную и убедиться, что она меньше нуля, как и положено в максимуме. Позже мы в общем случае докажем, что достаточное условие выполнено.
\item Мы доказали, что найденная $ b(x) $ — единственное симметричное равновесие Нэша, где $ b(x) $ — возрастающая функция. Мы не искали равновесия Нэша, где $ b(x) $ хотя бы иногда убывает.
\end{enumerate}


\begin{myex} Решение «в лоб», без чудо-замены.

Начнём с того, что прибыль представима в виде:
\begin{equation}
\pi(x,b_{1})=(x-b_{1})\P(b(X_{2})<b_{1})^{n-1}
\end{equation}

По нашим предположениям функция $ b() $ строго возрастает, значит, у неё есть обратная. Обозначим её $ b^{-1}() $:

\begin{equation}
\pi(x,b_{1})=(x-b_{1})\P(X_{2}<b^{-1}(b_{1}))^{n-1}=(x-b_{1})F(b^{-1}(b_{1}))^{n-1}
\end{equation}

Берём производную по $ b_{1} $:
\begin{multline}
\frac{\partial \pi(x,b_{1})}{\partial b_{1}}=-F(b^{-1}(b_{1}))^{n-1}+\\
+(x-b_{1})(n-1)F(b^{-1}(b_{1}))^{n-2}f(b^{-1}(b_{1}))\cdot \frac{db^{-1}(b_{1})}{db_{1}}=0
\end{multline}

Отсюда как-то неявно выражается $ b_{1} $ как функция от $ x $. Но мы на самом деле знаем ответ! Мы уже предположили, что ситуация, когда все игроки используют функцию $b()$, — это равновесие Нэша. Значит, если все игроки кроме первого используют $ b() $, то и первому игроку оптимально её использовать! Следовательно, решением должно являться $b_{1}=b(x)$. Поэтому при подстановке $b_{1}=b(x)$, должно получаться тождество, верное при любых $ x $.

При подстановке $ b_{1}=b(x) $ величина $ b^{-1}(b_{1}) $ превращается в $x$. Собственно, это и есть $ a $ при чудо-замене\ldots Получаем дифференциальное уравнение:

\begin{equation}
-F(x)^{n-1}+(x-b(x))(n-1)F(x)^{n-2}f(x)\left.\frac{db^{-1}(b_{1})}{db_{1}}\right|_{b_{1}=b(x)}=0
\end{equation}

Сокращаем $ F^{n-2} $:

\begin{equation}
-F(x)+(x-b(x))(n-1)f(x)\left.\frac{db^{-1}(b_{1})}{db_{1}}\right|_{b_{1}=b(x)}=0
\end{equation}

Остаётся вспомнить, что производная обратной функции — это единица делить на производную исходной функции, и наше уравнение совпадает с \ref{first_price_final_diffeq}.
\end{myex}

\item Аукцион второй цены\index{аукцион!второй цены}.
Раз все игроки одинаковые, ограничимся рассмотрением первого игрока. Результат аукциона для него зависит от его собственной ставки и от максимальной ставки остальных игроков. Ценность товара для первого игрока у нас обозначена $ X_{1} $. Обозначим максимальную ставку остальных игроков — $ m $. Величину $ X_{1} $ игрок знает, а $ m$ — нет. В наших обозначениях $ m=b(Y_{1}) $, но это не существенно.

Сравним две стратегии первого игрока: $b_{1}=X_{1}  $, $ b_{1}=X_{1}+\Delta $. Числа $ X_{1} $ и $ X_{1}+\Delta $ разбивают числовую прямую на три интервала. Неизвестное $ m $ попадет в один из этих трёх интервалов. Запишем выигрыш первого игрока в табличку. Если $ b_{1}<m $, то он ничего не платит и не получает товар. Если $ b_{1}>m $, то игрок получает товар ценностью $ X_{1} $ и платит $ m $:

\begin{tabular}{c|ccc}
& $ m \in (-\infty;X_{1}) $ & $ m \in (X_{1};X_{1}+\Delta) $ & $ m \in (X_{1}+\Delta;+\infty) $ \\
\hline
$ b_{1}=X_{1}$         & $ X_{1}-m $ & 0 & 0 \\
$ b_{1}=X_{1}+\Delta $ & $ X_{1}-m $ & $ X_{1}-m $ & 0 \\
\end{tabular}

Мы видим, что в двух случаях из трёх стратегии приносят одинаковый выигрыш. Различие есть только если $ m \in (X_{1};X_{1}+\Delta) $. Стратегия $ b_{1}=X_{1}$ приносит нулевой выигрыш, а стратегия $  b_{1}=X_{1}+\Delta  $ приносит выигрыш $ X_{1}-m<0 $. Значит, стратегия $ b_{1} $ нестрого доминирует любую стратегию вида $ b_{1}=X_{1}+\Delta $ при $ \Delta>0 $. Делать ставку выше своей ценности невыгодно!


Аналогично сравним стратегии $ b_{1}=X_{1} $ и $ b_{1}=X_{1}-\Delta $:

\begin{tabular}{c|ccc}
& $ m \in (-\infty;X_{1}-\Delta) $ & $ m \in (X_{1}-\Delta;X_{1}) $ & $ m \in (X_{1};+\infty) $ \\
\hline
$ b_{1}=X_{1}$         & $ X_{1}-m $ & $X_{1}-m$ & 0 \\
$ b_{1}=X_{1}-\Delta $ & $ X_{1}-m $ & 0 & 0 \\
\end{tabular}

На этот раз разница в выигрышах возникает если $ m \in (X_{1}-\Delta;X_{1}) $. Стратегия $ b_{1}=X_{1}$ приносит выигрыш $  X_{1}-m>0 $. Следовательно, стратегия $b_{1}$ нестрого доминирует любую стратегию вида $ b_{1}=X_{1}-\Delta $ при $ \Delta>0 $.

Вывод. Существует равновесие Нэша, в котором все игроки используют стратегию $ b(x)=x $, то есть правдиво декларируют свои ценности.

Комментарии:
\begin{enumerate}
\item Равномерность распределения нигде не использовалась в решении. Значит, наше рассуждение проходит для любого непрерывного закона распределения $ X $.
Почему нам важна непрерывность распределения? Надеюсь, кто-нибудь обратил внимание, что интервалы для $ m $ были открытые, мы не рассматривали случай, когда $ m $ идеально точно попадает в его границу. Если распределение ценностей непрерывно, то вероятность того, что $ m $ будет равняться конкретному числу равна нулю. Исключив эти случаи из рассмотрения, мы не изменили ожидаемую прибыль первого игрока, а, значит, не изменили его оптимальную стратегию.

В случае дискретного распределения доходностей очень важным становится правило, согласно которому распределяется товар, если ставки совпали. В непрерывном случае вероятность совпадения ставок равна нулю, и никакое правило распределения товара при «ничьей» не влияет на оптимальные стратегии.
\item Также в решении нигде не использовалась независимость $ X_{i} $. Значит, рассуждение проходит и для зависимых ценностей. Единственное ограничение: вероятность совпадения ценностей должна равняться нулю.
\end{enumerate}


\begin{myex}
Можно решить аукцион второй цены таким же способом, как и аукцион первой цены. В этом случае:
\begin{multline}
\pi(x,b_{1})=\E((X_{1}-b(Y_{1}))1_{b_{1}>b(Y_{1})}|X_{1}=x,Bid_{1}:=b_{1})=\\
=\E((x-b(Y_{1}))1_{b_{1}>b(Y_{1})})
\end{multline}
Поскольку величины $ X_{i} $ независимы, условное математическое ожидание превратилось в безусловное. Чудо-замена $b_{1}=b(a)$\index{чудо-замена} и предположение о возрастании функции $ b() $ позволяют упростить выражение:
\begin{multline}
\pi(x,b(a))=x\P(Y_{1}<a)+\E(b(Y_{1})1_{Y_{1}<a})=\\
=x\int_{0}^{a}p_{Y_{1}}(t)dt-\int_{0}^{a}b(t)p_{Y_{1}}(t)dt
\end{multline}
Берём производную по $ a $:
\begin{equation}
\frac{\partial \pi(x,b(a))}{\partial a}=xp_{Y_{1}}(a)-b(a)p_{Y_{1}}(a)
\end{equation}
Мы предположили, что $ b() $ — это равновесная стратегия, значит, при ценности $ x $ игроку должно быть оптимально ставить $ b_{1}=b(x) $. Кроме того мы делали замену $ b_{1}=b(a) $. Значит, производная обращается в ноль при $ a=x $:
\begin{equation}
xp_{Y_{1}}(x)-b(x)p_{Y_{1}}(x)=0
\end{equation}
И отсюда мы получаем решение $b(x)=x$.

Недостаток этого способа в том, что он говорит только что $ b(x)=x $ — равновесие Нэша. А способ с доминированием стратегий говорит, что это не просто равновесие, а равновесие в нестрого доминирующих стратегиях.
\end{myex}

\item Кнопочный аукцион\index{аукцион!кнопочный}.

Снова рассмотрим первого игрока. Если он видит, например, что другие игроки долго давят свои кнопки, он может сделать вывод, что их ценности товара высоки. Наблюдая за другими, он получает информацию о них, но не о себе! Его ценность не зависит от их ценностей! Ситуция резко изменится, когда мы будем рассматривать зависимые ценности в следующих лекциях. А пока наблюдение за другими не дает нашему игроку никакой полезной информации, кроме того, остался ли он уже один в игре, или ещё нет.

Естественно, как только игрок остался один в игре, победитель сразу определён. Следовательно, стратегия игрока не зависит, от того, сколько ещё игроков осталось кроме него. ещё двое или трое, или семеро — никакой разницы. Таким образом, ещё до начала аукциона, узнав свою ценность $ X $, игрок может уже спланировать свои действия: «я буду давить на кнопку до тех пор, пока цена не дойдет до некоей цены $ b $ или пока я не выиграю аукцион.»

Итак, действия игрока описывается его числом $ b_{i} $. Представим теперь, что игроки просто пишут свои $ b_{i} $ на бумажках, а на кнопки давят роботы, согласно этим $ b_{i} $. Кто победит на аукционе? Победит тот, кто написал наибольшее $ b_{i} $. А сколько он заплатит? Он заплатит вторую по величине ставку $ b_{i} $!

Получается, что при независимых ценностях, кнопочный аукцион полностью эквивалентен аукциону второй цены. А его мы уже решали. Оптимальная стратегия: $ b(x)=x $.

Когда ценности будут коррелированы, кнопочный аукцион будет отличаться от аукциона второй цены.

\end{enumerate}


\section{Теорема об одинаковой доходности}

Чтобы не повторяться, введём:

\begin{mydef}
Закон распределения случайной величины $ X $ назовём \indef{регулярным}\index{регулярное распределение}, если существуют такие числа $a$ и $b$, что функция распределения $ F $ строго возрастает и непрерывна на отрезке $ [a;b] $, $ F(a)=0 $ и $ F(b)=1 $.
\end{mydef}


В этой книге мы всегда рассматриваем регулярное распределение на отрезке $[0;1]$. Это нисколько нас не ограничивает, так как вопрос выбора начала и конца отрезка — это вопрос выбора масштаба, в котором измеряются денежные суммы. Может быть «один» — это один миллион рублей. Зато обозначения становятся проще.


\begin{myth} Теорема об одинаковой доходности. Revenue equivalence theorem. \index{теорема!об одинаковой доходности}


Если:

\begin{itemize}
\item[RE1.] На аукционе выставлен один товар
\item[RE2.] За право получить товар торгуются $ n $ игроков
\item[RE3.] Ценности товара для разных игроков одинаково распределенны и независимы
\item[RE4.] Ценности имеют регулярное распределение на отрезке $ [0;1] $
\item[RE5.] В равновесии товар достаётся тому игроку, для которого он ценнее всего
\item[RE6.] В равновесии средний выигрыш игрока с минимальной ценностью (у нас с ценностью 0) равен 0
\item[RE7.] Покупатели нейтральны к риску
\end{itemize}

То:

Средний доход продавца не зависит от конкретного механизма проведения аукциона.

\end{myth}


\begin{proof}

Доказательство состоит из трёх шагов. Двух простых и третьего, позапутаннее, — связанного с теоремой об огибающей\ldots

Шаг 1. Рассмотрим игрока с ценностью $ x $. Какова вероятность того, что он выиграет аукцион? Из требования RE5 следует, что это вероятность того, что ценность остальных игроков ниже $ x $. Применяя требования RE1-RE4 получаем, что искомая вероятность, обозначим её $ q(x) $, равна
\[ q(x)=F(x)^{n-1} \]

Шаг 2. Заметим, что средняя выручка продавца — это сумма средних платежей всех игроков.


Шаг 3. Оказывается, что средний платеж игрока однозначно выводится из вероятности, упомянутой в шаге 1 и условия RE6. А именно, вот-вот мы докажем, что средняя выплата игрока определяется по формуле:

\begin{equation}
\label{pay_eq}
pay(x)=xq(x)-\int_{0}^{x}q(t)dt
\end{equation}

Доказательство на самом деле занимает две или три строчки, но перед ними нужно ввести кучу обозначений. Итак\ldots

Мы выбрали конкретного игрока. Рассмотрим ситуацию, в которой остальные игроки используют равновесные стратегии. Вероятность того, что наш игрок выиграет аукцион, зависит только от его ставки $ b $, так как стратегии остальных зафикисрованы. Обозначим её $ \widehat{q}(b) $. Если игрок выигрывает аукцион, то он что-то платит, причём необязательно свою ставку! Если не выигрывает, то ничего не платит. Среднее значение этого платежа опять же зависит только от его ставки $ b $, обозначим его $ \widehat{pay}(b) $.

Средний выигрыш игрока определяется по формуле:

\begin{equation}
\pi(x,b)=x\widehat{q}(b)-\widehat{pay}(b)
\end{equation}

У игрока есть равновесная стратегия $ b(x) $, которая по определению равновесия Нэша, является наилучшим ответом на действия других игроков.

При подстановке этой наилучшей стратегии  $ b(x) $ вместо $ b $ мы получаем определения трёх новых функций:

\[ q(x):=\widehat{q}(b(x)) \]

\[ pay(x)=\widehat{pay}(b(x)) \]

\[ \pi^{*}(x)=\pi(x,b(x))=xq(x)-pay(x) \]

Для ясности: единственная разница между функциями $ \widehat{q}() $ и $ q() $ состоит в том, что первая зависит от ставки, а вторая — от ценности. Повесив на стену столько ружей, пора стрелять!

Находим производную:

\begin{equation}
\frac{d \pi^{*}(x)}{dx}=\frac{d \pi(x,b(x))}{dx}=\frac{\partial \pi}{\partial x}+\frac{\partial \pi}{\partial b}\frac{d b }{d x}
\end{equation}

А теперь вспомним, что оптимальная стратегия $ b(x) $ находится из условия $ \frac{\partial \pi}{\partial b}=0 $. Значит:

\begin{equation}
\frac{d \pi^{*}(x)}{dx}=\left.\frac{\partial \pi}{\partial x}\right|_{b=b(x)}
\end{equation}

Это, между нами говоря, была теорема об огибающей\index{теорема!об огибающей}. Упрощаем производную:

\begin{equation}
\left.\frac{\partial \pi(x,b)}{\partial x}\right|_{b=b(x)}=\widehat{q}(b)|_{b=b(x)}=q(x)
\end{equation}

Вот и всё! Кстати, это имеет легкую смысловую интерпретацию. Частная производная говорит нам, что случится с ожидаемым выигрышем, если мы будем менять ценность $x$, но не будем менять стратегию $ b $. Не меняя стратегию мы не влияем на вероятность выигрыша и на наш средний платеж организаторам аукциона. Естественно, мы должны получить вероятность выигрыша.

Осталось записать выражение в интегральной форме:

\begin{equation}
\pi^{*}(x)=\pi^{*}(0)+\int_{0}^{x}q(t)dt
\end{equation}

Условие RE6 говорит, что $ \pi^{*}(0)=0 $ и мы можем увидеть, что:

\begin{equation}
xq(x)-pay(x)=\int_{0}^{x}q(t)dt
\end{equation}

Что и требовалось доказать.
\end{proof}

Примечания.
\begin{enumerate}
\item Теорема говорит только о равенстве среднего дохода. Она не говорит, что доход продавца при данных конкретных ценностях покупателей не зависит от формы аукциона.
\item Теорема не говорит, что равновесие, в котором игрок с наивысшей ценностью получает товар, существует. Она, наоборот, опирается на существование такого равновесия: если такое равновесие есть, то средний доход продавца не зависит от формы проведения аукциона. Поэтому использовать теорему нужно аккуратно.

Если в аукционе товар достаётся тому, кто сделал наибольшую ставку, то часто помогает следующая цепочка рассуждений. Предположим, что равновесие, где товар достаётся игроку с наибольшей ценностью есть. Тогда мы можем использовать теорему. Она нам поможет, сейчас мы на примере это увидим, найти равновесную стратегию. Затем мы проверяем, что эта равновесная стратегия $b(x)$ является возрастающей функцией по $x$. И мы, подобно барону Мюнхаузену, вытащили сами себя за волосы! Если по правилам аукциона товар достаётся сделавшему наибольшую ставку, а наибольшая ставка соответствует наибольшей ценности, значит, теорему можно было применять!
\end{enumerate}


Наши три модели удовлетворяют условиям теоремы, поэтому средний доход продавца в них одинаковый:

\[ \E(R^{B})=\E(R^{FP})=\E(R^{SP}) \]

Средний доход продавца равен $ n $ умножить на среднюю выплату от первого игрока продавцу, поэтому воспользуемся формулой для средней выплаты от игрока продавцу:

\[ pay(x)=xq(x)-\int_{0}^{x}q(t)dt \]

При равномерном распределении ценностей, то есть $ q(x)=F(x)^{n-1}=x^{n-1} $:

\begin{equation}
pay(x)=\frac{n-1}{n}x^{n}
\end{equation}

Воспользуемся равномерностью второй раз:

\begin{equation}
\E(pay(X_{1}))=\E\left(\frac{n-1}{n}X_{1}^{n}\right)=\frac{n-1}{n}\int_{0}^{1} t^{n} dt=\frac{n-1}{n(n+1)}
\end{equation}

Умножаем на $ n $ и получаем:

\begin{equation}
\E(R^{B})=\E(R^{FP})=\E(R^{SP})=\frac{n-1}{n+1}
\end{equation}


А сейчас мы с помощью этой теоремы в два счета получим решение аукциона первой цены не только для равномерного случая.


\begin{myex} \label{use_ret} Решение аукциона первой цены для случай произвольного регулярного распределения ценностей\index{аукцион!первой цены}.

Предположим, что есть некое равновесие, и теорему об эквивалетности доходностей можно применять\index{теорема!об одинаковой доходности}. На аукционе первой цене средний платеж игрока равен его ставке помноженной на вероятность выигрыша:

\begin{equation}
\label{first_price_pay_eq}
pay(x)=b(x)q(x)
\end{equation}

Таким образом, уравнение \ref{pay_eq} имеет вид:

\begin{equation}
xq(x)-b(x)q(x)=\int_{0}^{x}q(t)dt
\end{equation}

Отсюда мы находим оптимальную стратегию:

\begin{equation}
\label{first_price_b_eq}
b(x)=x-\frac{\int_{0}^{x}q(t)dt}{q(x)}
\end{equation}

Напомним, что условия RE1-RE5 говорят нам, что $ q(x)=F(x)^{n-1} $. Это и есть решение аукциона первой цены для произвольной регулярной $ F $.

Остаётся проверить, что теорему можно было применять! Берём производную $ \frac{db(x)}{dx} $ и убеждаемся, что она строго положительна!

Можно проверить, что при подстановке $ F(t)=t $ на отрезке $ [0;1] $ мы получаем наш результат для равномерно распределенных ценностей.
\end{myex}




\section{Пример с коррелированными ценностями}


Сравним на примере доходы аукциона первой и второй цены при коррелированных ценностях\index{ценности!коррелированные}.

В аукционе участвуют два игрока. Предположим, что совместная функция плотностей на множестве $ x_{1},x_{2}\in [0;1] $ имеет вид:

\begin{equation}
f(x_{1},x_{2})=x_{1}+x_{2}
\end{equation}

Наша задача найти равновесие Нэша для аукциона первой и для аукциона второй цены, а также среднюю выручку продавца.

\begin{enumerate}
\item Аукцион первой цены\index{аукцион!первой цены}. Считаем ожидаемый выигрыш первого игрока, если его ценность равна $ x $, а ставит он $ b_{1} $. Отличие от формулы \ref{first_price_eq} состоит в том, что знание $ x $ содержит в себе информацию о ценности, и следовательно, ставке, второго игрока. Поэтому мы используем условную вероятность:

\begin{equation}
\pi_{1}(x,b_{1})=(x-b_{1})\P(b_{1}>Bid_{2} |X_{1}=x)
\end{equation}

Предположим, что существует симметричное равновесие Нэша, в котором все игроки делают ставки согласно функции $ b(x) $. Предположим, что эта функция дифференциируема и возрастает по $ x $.

Снова рассмотрим ситуацию, в которой все игроки кроме первого используют функцию $ b() $ для своих ставок, и найдём оптимальное поведение первого игрока. В нашем частном случае «все остальные» — это только второй игрок. то есть $ Bid_{2}=b(X_{2}) $.

Снова сделаем магическую замену $ b_{1} $ на пока неизвестную функцию $ b(a) $. В силу предположения о возрастании  $ b(x) $ условие $ b(a)>b(X_{2}) $ равносильны тому, что $ a>X_{2} $.

После магической подстановки\index{чудо-замена} наша прибыль имеет вид:

\begin{equation}
\pi_{1}=(x-b(a))\P(a>X_{2} |X_{1}=x)
\end{equation}


Чтобы брать производную по $a$, вспомним немного теорию вероятностей:

Из совместной функции плотности $f(x_{1}, x_{2})$ можно найти:

Сначала частную функцию плотности $X_{1}$:

\begin{equation}
f_{X_{1}}(x_{1}):=\int_{0}^{1} f(x_{1},x_{2}) dx_{2}
\end{equation}

Взяв интеграл, в нашем случае получаем $ f_{X_{1}}(x_{1})=\frac{1}{2}+x_{1} $.

А затем находим и условную функцию плотности по принципу:

\begin{equation}
f(x_{2}|x_{1}):=\frac{f(x_{1},x_{2})}{f_{X_{1}}(x_{1})}
\end{equation}

Условная функция плотности имеет вид $ f(x_{2}|x_{1})=\frac{x_{1}+x_{2}}{1/2+x_{1}} $

Из условной функции плотности можно получить условную функцию распределения:

\begin{equation}
F(x_{2}|x_{1}):=\int_{0}^{x_{2}} f(t|x_{1}) dt
\end{equation}

В нашем случае получаем $F(x_{2}|x_{1}):=\frac{x_{1}x_{2}+x_{2}^{2}/2}{1/2+x_{1}}$.

Но ведь $ \P(X_{2}<a|X_{1}=x) $ — это и есть условная функция распределения, $ F_{X_{2}|X_{1}}(a|x) $.

Значит, наша прибыль в вероятностных терминах записывается как:

\begin{equation}
\pi_{1}=(x-b(a)) F_{X_{2}|X_{1}}(a|x)
\end{equation}

Берём производную по $ a $ и приравниваем к нулю:

\begin{equation}
\frac{\partial \pi_{1}}{\partial a}=-b'(a) F_{X_{2}|X_{1}}(a|x)+(x-b(a))f_{X_{2}|X_{1}}(a|x)=0
\end{equation}

Снова завершаем магический трюк. Мы должны потребовать, чтобы оптимальной стратегий первого игрока была бы функция $b_{1}=b(x)$. Но мы использовали замену $ b_{1}=b(a) $, значит, $ x=a $:


\begin{equation}
-b'(x) F_{X_{2}|X_{1}}(x|x)+(x-b(x))f_{X_{2}|X_{1}}(x|x)=0
\end{equation}

В результате мы получили дифференциальное уравнение:

\begin{equation}
b'(x)=(x-b(x))\frac{f_{X_{2}|X_{1}}(x|x)}{F_{X_{2}|X_{1}}(x|x)}
\end{equation}

В нашем частном случае: $ f(x|x)=\frac{2x}{1/2+x} $, $ F(x|x)=\frac{1.5x^2}{1/2+x} $.

Получаем дифференциальное уравнение:

\begin{equation}
b'=(x-b(x))\frac{4}{3x}
\end{equation}

Мы же везунчики, правда у него будет линейное решение?

Можно подбором получить $ b(x)=\frac{4}{7}x $.

Если же решать честно, то общее решение имеет вид  $ b(x)=c\cdot x^{-4/3}+\frac{4}{7}x $. Но мы ищем стратегию, которая возрастает по $ x $, поэтому $ c=0 $.


Настала очередь дохода продавца. Он равен удвоенной ожидаемой выплате первого игрока.
Первый игрок платит $ \frac{4}{7}X_{1} $, только если $ X_{1}>X_{2} $. Значит:

\begin{equation}
\E(R^{FP})=2\E\left(\frac{4}{7}X_{1}\cdot 1_{X_{1}>X_{2}}\right)
\end{equation}

Задача свелась к двойному интегралу:

\begin{equation}
\E(\frac{4}{7}X_{1}\cdot 1_{X_{1}>X_{2}})=\int_{0}^{1} \int_{0}^{x_{1}}  \frac{4}{7}x_{1}  f(x_{1},x_{2}) dx_{2} dx_{1}
\end{equation}

Подставляем функцию плотности и после взятия интеграла получаем:

\begin{equation}
\E(R^{FP})=\frac{3}{7}\approx 0.4286
\end{equation}


\item Аукцион второй цены\index{аукцион!второй цены}. Логика решения такая же, как и в случае независимых ценностей. Рассматриваем поведение первого игрока. Поскольку игроков всего два, $ m=b_{2} $. Отличие от случая независимыми ценностями состоит в том, что случайные величины $ m $ и  $ X_{1} $ зависимы. Но этот факт никак не влияет на логику решения. Поэтому снова оказывается, что стратегия $ b_{1}=X_{1} $ нестрого доминирует любую другую стратегию первого игрока.

Считаем ожидаемый доход продавца. Он равен удвоенной ожидаемой выплате первого игрока.
Первый игрок платит $ X_{2} $ только если $ X_{1}>X_{2} $. Значит:

\begin{equation}
\E(R^{SP})=2\E(X_{2}\cdot 1_{X_{1}>X_{2}})
\end{equation}

Задача свелась к двойному интегралу:

\begin{equation}
\E(X_{2}\cdot 1_{X_{1}>X_{2}})=\int_{0}^{1} \int_{0}^{x_{1}}  x_{2}  f(x_{1},x_{2}) dx_{2} dx_{1}
\end{equation}

Подставляем функцию плотности и после взятия интеграла получаем:

\begin{equation}
\E(R^{SP})=5/12\approx 0.4167
\end{equation}

\end{enumerate}

Таким образом, для данного совместного распределения доходностей аукционы первой и второй цены не одинаково выгодны для продавца! Мы исследуем подробнее случай связанных доходностей позже.



\section{Задачи}


\begin{enumerate}
\item В моделях аукциона первой и второй цены с независимыми, равномерными на $ [0;1] $ ценностями покупателей приведите примеры
\begin{enumerate}
\item Вектора ценностей, при котором для продавца лучше аукцион первой цены
\item Вектора ценностей, при котором для продавца лучше аукцион второй цены
\end{enumerate}


\item Рассмотрите покупателей с независимыми ценностями, имеющими функцию плотности $ f(x)=2x $ на отрезке $ [0;1] $. Найдите в явном виде оптимальные стратегии и среднюю прибыль продавца.



\item Докажите, что формулу $ pay(x)=xq(x)-\int_{0}^{x}q(t)dt$ можно представить в виде:

\begin{equation}
pay(x)=pay(0)+\int_{0}^{x}t \cdot \frac{dq(t)}{dt}\cdot dt
\end{equation}



\item Аукцион «Платят все!» \index{аукцион!платят все}. Покупатели одновременно делают ставки. Товар достаётся тому, кто назвал наибольшую ставку, но платят все игроки. Каждый платит свою ставку. Ценности товара для покупателей имеют независимое регулярное распределение с функцией распределения $ F() $.

Используя трюк с теоремой об одинаковых доходностях (см. пример \ref{use_ret}), найдите оптимальные стратегии игроков, $ b(x) $.

Какой вид имеют оптимальные стратегии, если $ X_{i} $ равномерны на $ [0;1] $? Чему в этом случае будет равен ожидаемый доход продавца?




\item Наследство\index{задача!о наследстве}. Двум сыновьям достался земельный участок в наследство. Отец не хотел, чтобы участок был разделен, поэтому по завещанию установлены следующие правила: два брата одновременно делают ставки. Участок получает тот, кто сделал б\'{о}льшую ставку. При этом получивший участок выплачивает свою ставку проигравшему. Найдите равновесие Нэша, если ценности участка независимы и равномерны на $ [0;1] $.

\item Аукцион «Победитель платит чужую среднюю»\index{аукцион!победитель платит чужую среднюю}. Покупатели одновременно делают ставки. Товар достаётся тому, кто назвал наибольшую ставку. Победитель платит среднюю арифметическую ставок остальных игроков. Ценности товара для покупателей независимы и равномерно распределены на $ [0;1] $. Найдите оптимальные стратегии игроков, $ b(x) $, средний доход продавца.

Подсказка: Может быть, у дифференциального уравнения есть простое линейное решение?


\item Аукцион с дискретными ценностями\index{аукцион!с дискретными ценностями}\index{ценности!дискретные}. В аукционе участвуют два покупателя. Ценности товара для покупателей независимы и имеют дискретное распределение: $ X_{i} $ равновероятно принимает значения $ 0 $ и $ 1 $. Аукцион проходит по следующим правилам: продавец предлагает товар по цене $ a $, где $ a $ — это некая константа, $ a\in (0;2/3) $. Игроки одновременно решают, подходит ли им эта цена или нет. Если один сказал «да», а другой «нет», то товар достаётся тому, кто сказал «да», и он платит величину $a$. Если оба сказали «нет», то товар отдается бесплатно случайно выбираемому игроку. Если оба сказали «да», то товар отдается по цене $ a $ случайно выбираемому игроку.

\begin{enumerate}
\item Найдите хотя бы одно равновесие Нэша.

\item Зависит ли равновесный доход аукциониста от $ a $? Применима ли теорема об одинаковой доходности и почему?

\item При каком $a$ доход продавца будет максимальным?
\end{enumerate}


Подсказка: Для тех кто забыл теорию игр :) Во-первых, равновесия бывают в смешанных стратегиях, во-вторых, чистых стратегий у каждого игрока здесь четыре. Стратегия — это функция от ценности, значит, у игрока есть, например, стратегия «если $ X=0 $, то говорю „нет“, если $X=1$, то говорю „да“».

Ещё подсказка: Если задача кажется слишком сложной, решите её для конкретного $ a $, скажем, для $a=0.1$, а затем попробуйте снова. Это, кстати, один из немногих универсальных приемов решения всех задач: «Я не хочу решать эту задачу, поэтому буду решать более простую!»


\end{enumerate}


\section{Решения задач}

\begin{enumerate}
\item В качестве примера возьмём аукцион двух игроков.

\begin{enumerate}
\item Если $ X_{1}=0.2 $, $ X_{2}=0.6 $, то на аукционе второй цены продавец получит $ 0.2 $ (вторую по величине ставку), а на аукционе первой цены продавец получит $ 0.3 $ (так как оптимальная стратегия имеет вид $ b(x)=\frac{n-1}{n}x=\frac{1}{2}x $).

\item Если $ X_{1}=0.4 $, $ X_{2}=0.6 $, то на аукционе второй цены продавец получит $ 0.4 $ (вторую по величине ставку), а на аукционе первой цены продавец получит $ 0.3 $ (так как оптимальная стратегия имеет вид $ b(x)=\frac{n-1}{n}x=\frac{1}{2}x $).
\end{enumerate}

\item Подставляем $ F(x)=x^{2} $, значит, $ q(x)=(x^{2})^{n-1}=x^{2n-2} $. Подставляем $ q(x) $ в \ref{first_price_b_eq}. Получаем $ b(x)=\frac{2n-2}{2n-1}x $.

Полученную $ b(x) $ подставляем в \ref{first_price_pay_eq}. Получаем $ pay(x)=\frac{2n-2}{2n-1}x^{2n-1} $.

Считаем ожидаемую выплату первого игрока:

\begin{multline}
\E(pay(X_{1})=\int_{0}^{1}pay(t)f(t)dt=\int_{0}^{1}\frac{2n-2}{2n-1}t^{2n-1}\cdot 2tdt=\\
=\frac{4(n-1)}{(2n-1)(2n+1)}
\end{multline}

И умножаем на число игроков:

\begin{equation}
\E(R^{FP})=\frac{4n(n-1)}{(2n-1)(2n+1)}
\end{equation}

\item Используем формулу интегрирования по частям.


\item Используем уравнение из теоремы об одинаковой доходности\index{теорема!об одинаковой доходности}

\begin{equation}
xq(x)-pay(x)=\int_{0}^{x}q(t)dt
\end{equation}

В данном случае $ pay(x)=b(x) $, так как ставка платится вне зависимости от того, кому достанется товар. Значит:

\begin{equation}
\label{NE_all_pay}
b(x)=xq(x)-\int_{0}^{x}q(t)dt
\end{equation}
где $ q(x)=F(x)^{n-1} $

Можно решить и по-другому, явно выписав задачу максимизации игрока и получив дифференциальное уравнение.


\item Эта игра не аукцион в чистом виде, так как игрок тоже может получить деньги.
Выписываем прибыль первого игрока:
\begin{equation}
\pi_{1}=(x-b_{1})\P(b(X_{2})<b_{1})+\E(b(X_{2})\cdot 1_{b(X_{2})>b_{1}})
\end{equation}
Прибавка в прибыли — это ожидаемый платеж от второго игрока первому.

После чудо-замены:
\begin{equation}
\pi_{1}=(x-b(a))\P(X_{2}<a)+\E(b(X_{2})\cdot 1_{X_{2}>a})
\end{equation}

Запишем математическое ожидание в виде интеграла:
\begin{equation}
\pi_{1}=(x-b(a))F(a)+\int_{a}^{1}b(t)f(t)dt
\end{equation}

После взятия производной по $ a $:
\begin{equation}
-b'(a)F(a)+(x-b(a))f(a)-b(a)f(a)=0
\end{equation}

Требуем оптимальности стратегии  $ b_{1}=b(x) $:
\begin{equation}
-b'(x)F(x)+(x-b(x))f(x)-b(x)f(x)=0
\end{equation}

Для равномерного случая получаем:
\begin{equation}
-b'(x)x+x-b(x)-b(x)=0
\end{equation}

Находим общее решение и замечаем, что $ c=0 $ или сразу подбираем линейное решение $ b(x)=kx $:
\begin{equation}
-kx+x-2kx=0
\end{equation}

Получаем $ k=1/3 $ и равновесие Нэша вида $ b(x)=x/3 $.



\item

Победа первого игрока, событие $ W_{1}=\{b(X_{2})<b_{1}\cap \ldots\cap b(X_{n})<b_{1}\} $. Чудо-замена  $b_{1}=b(a)$ позволяет упростить его до $ W_{1}=\{X_{2}<a\cap\ldots\cap X_{n}<a\} $.

Функция прибыли:
\begin{equation}
\pi_{1}=\left(x-\E\left(\frac{b(X_{2})+\ldots+b(X_{n})}{n-1} \right|\left.\vphantom{\frac{b(X_{2})+\ldots+b(X_{n})}{n-1}} W_{1}\right)\right)\cdot \P(W_{1})
\end{equation}

В силу того, что $ X_{i} $ одинаково распределены и независимы, $ \P(W_{1})=F(a)^{n-1} $ и:
\begin{equation}
\pi_{1}=(x-\E\left(b(X_{2})|X_{2}<a)\right)\cdot F(a)^{n-1}
\end{equation}

Воспользуемся тем, что $ \E(b(X_{2})|X_{2}<a)\cdot \P(X_{2}<a)=\E(b(X_{2})\cdot 1_{X_{2}<a}) $:
\begin{equation}
\pi_{1}=x\cdot F(a)^{n-1}-\E\left(b(X_{2})\cdot 1_{X_{2}<a})\right)\cdot F(a)^{n-2}
\end{equation}

Заметим, что математическое ожидание равно интегралу:
\begin{equation}
\E(b(X_{2})\cdot 1_{X_{2}<a})=\int_{0}^{a}b(t)f(t)dt
\end{equation}

Стало быть, производная от математического ожидания равна:

\begin{equation}
\frac{d \E(b(X_{2})\cdot 1_{X_{2}<a})}{d a}=b(a)f(a)
\end{equation}

Теперь легко находим производную прибыли:

\begin{multline}
\pi_{1}=F(a)^{n-1}+x\cdot (n-1) F(a)^{n-2}f(a)-\\
-\E\left(b(X_{2})\cdot 1_{X_{2}<a})\right)\cdot (n-2)F(a)^{n-3}f(a)-b(a)f(a)F(a)^{n-2}
\end{multline}

Приравняв к нулю и завершив чудо-замену замечанием, что $ a=x $, получаем уравнение:
\begin{equation}
(n-1)xF(x)f(x)-(n-2)\int_{0}^{x}b(t)f(t)dt f(x)-b(x)f(x)F(x)=0
\end{equation}

Мы получили интегральное уравнение, то есть уравнение с интегралами, а не с производными. Но его можно свести к дифференциальному, сделав замену $ y(x)=\int_{0}^{x}b(t)f(t)dt $, тогда $ b(x)f(x)=y'(x) $. В общем виде дальше мы его решать не будем, а вспомним, что у нас $ f(x)=1 $ и $ F(x)=x $:

\begin{equation}
(n-1)x^{2}-(n-2)\int_{0}^{x}b(t)dt -b(x)x=0
\end{equation}

Вместо возможной замены $ y(x)=\int_{0}^{x}b(t)dt $ мы возьмём производную от обеих частей уравнения:
\begin{equation}
(n-1)2x-(n-2)b(x)-b(x)-b'(x)x=0
\end{equation}

Далее есть два варианта действий. Либо найти общее решение и заметить, что нужно взять $ c=0 $, либо подобрать линейное решение $ b(x)=kx $:

\begin{equation}
(n-1)2x-(n-2)kx-kx-kx=0
\end{equation}

Получаем $ k=\frac{2(n-1)}{n} $ и равновесие Нэша со стратегиями вида $ b(x)=\frac{2(n-1)}{n}x $. Кстати, при $ n=2 $ мы получаем аукцион второй цены, а наше решение, как и следует, дает $ b(x)=x $.

Альтернативное решение по принципу «Мне повезёт и без дифуров»:

А вдруг оптимальная стратегия линейна, то есть имеет вид $ b(x)=kx $? Подставим эту функцию сразу в прибыль, до чудо-замены:

\begin{multline}
\pi_{1}=\left(x-\E\left(\frac{kX_{2}+\ldots+kX_{n}}{n-1}\right|\left.\vphantom{\frac{kX_{2}+\ldots+kX_{n}}{n-1}}kX_{2}<b_{1}\cap \ldots kX_{n}<b_{1}\right)\right)\cdot \\
\cdot \P(kX_{2}<b_{1}\cap \ldots kX_{n}<b_{1}))
\end{multline}

Упрощаем математическое ожидание и вероятность:

\begin{equation}
\pi_{1}=(x-k\E\left(X_{2}|X_{2}<b_{1}/k\right)\cdot \P(X_{2}<b_{1}/k\cap \ldots \cap X_{n}<b_{1}/k))
\end{equation}

Теперь условное математическое ожидание легко считается. Условие $ X_{2}<b_{1}/k $ и априорная равномерность $ X_{2} $ равносильны тому, что $ X_{2} $ равномерно на $ [0;b_{1}/k] $. Значит, условное математическое ожидание равное $ \frac{b_{1}}{2k} $. Получаем уравнение

\begin{equation}
\pi_{1}=\left( x-\frac{b_{1}}{2}\right)\cdot (b_{1}/k)^{n-1}
\end{equation}

Без чудо-замены берём производную по $ b_{1} $. То есть сразу ищем оптимальную ставку:

\begin{equation}
k^{1-n}\left( \left( x-\frac{b_{1}}{2} \right)\cdot (n-1)b_{1}^{n-2}-\frac{1}{2}b_{1}^{n-1} \right)=0
\end{equation}

Выражаем $ b_{1} $ и получаем $ b_{1}=\frac{2(n-1)}{n}x $. Поскольку она имеет предположенный вид, то все шаги были верными.


Считаем средний доход продавца. Поскольку все условия теоремы об одинаковой доходности выполнены\index{теорема!об одинаковой доходности}, то ответ совпадает с найденным в лекции для аукциона первой цены:
\begin{equation}
 \E(R^{MO})=\frac{n-1}{n+1}
\end{equation}


\item
Произвольная смешанная стратегия имеет вид:

Если $ X=0 $, то говорить «да» с вероятностью $ p_{0} $; если $ X=1 $, то говорить «да» с вероятностью $ p_{1} $.

Пусть моя ценность равна 0. Если я скажу «нет», то ничего не заплачу и, возможно, получу товар с нулевой для меня ценностью, значит, мой ожидаемый выигрыш равен 0. Если я скажу «да», то с положительной вероятностью мне придется платить $ a $, и мой ожидаемый выигрыш меньше 0. Значит, если ценность равна нулю, оптимально говорит  «нет». То есть $ p_{0}=0 $.

Пусть моя ценность равна 1. Если я скажу «нет», то получу в среднем:

\begin{equation}
(1-0.5p_{1})\cdot \frac{1}{2}
\end{equation}


Если я скажу «да», то получу в среднем:

\begin{equation}
(1-0.5p_{1})\cdot (1-a)+0.5p_{1}\cdot \frac{1-a}{2}
\end{equation}

Чтобы смешанная стратегия была оптимальной, мне нужно быть безразличным между чистыми стратегиями, так как иначе я выбрал бы чистую. Приравниваем эти два выигрыша и находим $ p_{1} $.

Формально получается $ p_{1}=4-\frac{2}{a}$. Возникает несколько случаев\ldots

Если $ a\in (0;1/2) $, то равенство достигается только при $ p_{1}<0 $. Это означает, что при всех $ p_{1}\in [0;1] $ говорить «да» выгоднее, чем «нет». Следовательно, в равновесии каждый игрок использует стратегию: «Если $ x=0 $, то „нет“, если $ x=1 $, то „да“».

Если $ a\in (1/2;2/3) $, то равенство достигается только при $ p_{1}=4-\frac{2}{a} $. В равновесии каждый игрок использует стратегию: «Если $ x=0 $, то „нет“, если $ x=1 $, то „да“ с вероятностью $ p_{1}=4-\frac{2}{a}  $».

Рассмотрим случай $ a=1/2 $. Если $ x=1 $, то «да» лучше, чем «нет» при $ p_{1}\in (0;1] $ и игрок безразличен между «да» и «нет» при $ p_{1}=0 $. Получаем равновесие с парой стратегий: «Если $ x=0 $, то „нет“, если $ x=1 $, то „да“». И ещё одно равновесие с парой стратегий: «Всегда „нет“».

\end{enumerate}


\section{Контрольная 1}



\begin{enumerate}
\item Предположим, что условия теоремы об одинаковых доходностях выполнены.
\begin{enumerate}
\item Может ли выбор механизма проведения аукциона влиять на ковариацию выплат двух разных игроков?
\item  Найдите ковариацию выплат первого и второго игрока в аукционе первой цены с независимыми и равномерными на $ [0;1] $ ценностями. Подсказка: можно пользоваться тем, что средняя выплата равна $ \frac{n-1}{n(n+1)} $.
\end{enumerate}


\item «Наследство»\index{Наследство} по типу аукциона второй цены.

Двум сыновьям достался земельный участок в наследство. Отец не хотел, чтобы участок был разделен, поэтому по завещанию установлены следущие правила: два брата одновременно делают ставки. Участок получает тот, кто сделал большую ставку. При этом получивший участок выплачивает проигравшему меньшую из двух ставок. Ценности участка для игроков независимы и равномерны на $ [0;1] $.

Найдите равновесие Нэша.


\item Рассмотрим аукцион второй цены. Предположим, что ценности независимы и имеют регулярное распределение. Агенты не нейтральны к риску. Их отношение к риску отражается функцией полезности $ u() $. Про функцию $ u() $ известно, что она непрерывна, строго возрастает и для удобства отмасштабирована, $ u(0)=0 $. Если игрок получает товар ценностью $ x $ и платит продавцу $ m $, то его полезность равна $ u(x-m) $.

Найдите равновесие Нэша.



\item Рассмотрим аукцион второй цены с резервной ставкой $ r $. Резервная ставка — это минимальная цена, за которую продавец согласен расстаться с товаром. Если все игроки сделали ставки ниже $ r $, товар остаётся у продавца, никто ничего не платит. Если хотя бы один игрок сделал ставку выше $ r $, то товар достаётся игроку, сделавшему самую высокую ставку, и платит он максимум между второй по величине ставкой и $ r $. Константа $ r $ общеизвестна всем игрокам. Ценности независимы и имеют регулярное распределение. Агенты нейтральны к риску.

Найдите равновесие Нэша.


\item Рассмотрим аукцион первой цены с двумя игроками. Ценности независимы и равномерны на $ [0;1] $. Но ставку можно сделать только 0 или 0.5. Если ставки игроков совпали, то товар достаётся случайно выбираемому игроку за соответствующую плату.

Найдите равновесие Нэша.

\end{enumerate}


\section{Решение контрольной 1}



\begin{enumerate}
\item

\begin{equation}
\Cov(Pay_{1}, Pay_{2})=\E(Pay_{1}\cdot Pay_{2})-\E(Pay_{1})\E(Pay_{2}).
\end{equation}

На вычитаемое способ аукциона влиять не может в силу теоремы об одинаковой доходности. Сосредоточимся на $\E(Pay_{1}\cdot Pay_{2})$. В аукционе первой цены никакие два игрока не могут платить одновременно, поэтому произведение выплат всегда равно нулю, то есть $ \E(Pay_{1}\cdot Pay_{2})=0 $. В аукционе «Платят все»\index{аукцион!платят все} произведение выплат строго положительно, поэтому $ \E(Pay_{1}Pay_{2})>0 $. Значит, способ проведения аукциона может влиять на ковариацию.



\item  Ожидаемая прибыль:

\begin{multline}
\pi(x,b_{1})=(x-\E(b(X_{2})|b(X_{2})<b_{1}))\cdot \P(b(X_{2})<b_{1})+\\
+b_{1}\P(b(X_{2})>b_{1}).
\end{multline}

После чудо-замены $ b_{1}=b(a) $ и упрощения вероятностей:

\begin{equation}
\pi=(x-\E(b(X_{2})|X_{2}<a))\P(X_{2}<a)+b(a)(1-\P(X_{2}<a)),
\end{equation}

и
\begin{equation}
\pi=xF(a)-\E(b(X_{2})\cdot 1_{X_{2}<a}))+b(a)(1-F(a)).
\end{equation}

В записи с интегралом:
\begin{equation}
\pi=xF(a)-\int_{0}^{a}b(t)f(t) \, dt+b(a)(1-F(a)).
\end{equation}

Приравниваем производную к нулю:
\begin{equation}
xf(a)-b(a)f(a)-b(a)f(a)+b'(a)(1-F(a))=0.
\end{equation}

Для случая равномерного распределения:
\begin{equation}
x-2b(x)+b'(x)(1-x)=0.
\end{equation}

Подбором коэффициентов находим линейное решение:
\begin{equation}
b(x)=\frac{1}{3}x+\frac{1}{6}.
\end{equation}


\item Составляем табличку и видим, что стратегия $ b_{1}=X_{1} $ нестрого доминирует остальные стратегии.


\item  Составляем табличку и видим, что стратегия $ b_{1}=X_{1} $ нестрого доминирует остальные стратегии.

\item Предположим, что стратегия имеет вид\index{аукцион!с дискретными ставками}: если ценность ниже порога $ x^{*} $, то делать ставку 0, иначе делать ставку $ 0.5 $.

Осталось найти $ x^{*} $.

Допустим, что второй игрок использует такую стратегию.

Если первый сделает ставку ноль, то его ожидаемый выигрыш будет равен:
\begin{equation}
\pi(x,0)=x^{*}\frac{1}{2}x.
\end{equation}

Если первый сделает ставку $0.5$, то его ожидаемый выигрыш будет равен:
\begin{equation}
\pi(x,0.5)=(x-0.5)x^{*}+\frac{1}{2}(x-0.5)(1-x^{*})=\frac{1}{2}(x-0.5)(x^{*}+1).
\end{equation}

Находим условие, при котором $ \pi(x,0.5)>\pi(x,0) $, получаем:

\begin{equation}
x>\frac{1}{2}(x^{*}+1).
\end{equation}

Значит, правая часть представляет собой $ x^{*} $. Решаем уравнение $x^{*}=\frac{1}{2}(x^{*}+1)  $, получаем $ x^{*}=1 $, то есть вне зависимости от ценности игрокам имеет смысл ставить 0.


\end{enumerate}



\chapter{Общая ценность, аффилированные сигналы}


\section{Напоминалка по теории вероятностей}



\begin{mydef}
Индикатором события $ A $ называется случайная величина, которая равна 1, если $ A $ произошло и 0, если $ A $ не произошло. Обозначают индикатор $ A $ так: $ 1_{A} $.
\end{mydef}

Проверьте, что вы понимаете, что $ \E(1_{A})=\P(A) $.

С помощью индикаторов легко определить условное ожидание:

\begin{mydef}
Если $ \P(A)>0 $, то условным ожидание случайной величины $ X $ при условии события $ A $ называют число:
\[ \E(X|A):=\frac{\E(X\cdot 1_{A})}{\P(A)} \]
\label{posit_condition}
\end{mydef}

Если вам знакомо альтернативное определение условного ожидания, убедитесь, что оно совпадает с этим на паре примеров. Альтернативное определение основано на идее: условное ожидание считается так же, как и безусловное, только вместо безусловной вероятности используется условная.

\begin{myex} Задан закон распределения случайной величины $X$:


\begin{tabular}{c|cccc}
Prob & 0.1 & 0.2 & 0.3 & 0.4 \\
\hline
$X$ & 6 & 2 & 3 & 1 \\
\end{tabular}

Найдите $ \E(X|X>2) $.

Решение:

$\P(X>2)=0.4$

Составляем табличку для $ X\cdot 1_{X>2} $:

\begin{tabular}{c|cccc}
Prob & 0.1 & 0.2 & 0.3 & 0.4 \\
\hline
$1_{X>2}$ & 1 & 0 & 1 & 0 \\
$X\cdot 1_{X>2}$ & 6 & 0 & 3 & 0 \\
\end{tabular}

Находим $ \E(X\cdot 1_{X>2})=1.5 $. Значит $ \E(X|X>2)=1.5/0.4=3.75 $

\end{myex}


\begin{myex} Пусть $ X $ распределено экспоненциально с параметром $ \lambda=1 $, то есть функция плотности $ X $ при $ t\geq 0 $ имеет вид:
\[ p_{X}(t)=e^{-t} \]
Найдите $ \E(X|X<5) $.

Решение. Находим $ \P(X<5)=\int_{0}^{5}e^{-x}dx=1-e^{-5} $. Затем $ \E(X\cdot 1_{X<5})=\int_{0}^{5} xe^{-x}dx=1-6e^{-5} $. И, следовательно, $ \E(X|X<5)=\frac{e^{5}-6}{e^{5}-1} $

\end{myex}


Часто приходится иметь дело с условным ожиданием виде $ \E(Y|X=x) $. В случае, когда $ X $ дискретна и $ \P(X=x)>0 $ мы без проблем применяем определение \ref{posit_condition}. Однако, если, $ X $ непрерывна и $ \P(X=x)=0 $, у нас возникают проблемы. Впрочем, если $ \P(X\in [x;x+\Delta x])>0$, то наши проблемы легко решаются:
\begin{mydef}
Если $ \P(X\in [x;x+\Delta x])>0 $, то мы определяем условное ожидание $ \E(Y|X=x) $ по формуле:
\begin{equation}
\E(Y|X=x):=\lim_{\Delta x\to 0}\E(Y|X\in [x;x+\Delta x])
\end{equation}
\end{mydef}

Для практических вычислений мы редко (почти никогда в этих лекциях) будем пользоваться определением. Нам будет достаточно четырех свойств:

\begin{itemize}
\item Мат. ожидание суммы равно сумме мат. ожиданий
\begin{equation}
\E(X+Y|A)=\E(X|A)+\E(Y|A)
\end{equation}
\item Константу можно выносить за знак мат. ожидания
\begin{equation}
\E(cX|A)=c\E(X|A)
\end{equation}
\item Значения известной случайной величины можно подставлять:
\begin{equation}
\E(f(X,Y)|X=x)=\E(f(x,Y)|X=x)
\end{equation}
\item Если случайная величина $ X $ и событие $ A $ независимы, то
\begin{equation}
\E(X|A)=\E(X)
\end{equation}

\end{itemize}

Для величин имеющих совместную функцию плотности можно указать способ считать $ \E(Y|X=x) $ без предельного перехода:
\begin{myth} Если пара случайных величин $ X $ и $ Y $ имеет совместную функцию плотности $ f(x,y) $, то
\begin{equation}
\E(Y|X=x)=\int_{-\infty}^{+\infty}yf(y|x)dy
\end{equation}
\end{myth}

\begin{proof}
\begin{multline}
\E(Y|X\in [x;x+\Delta x])=\frac{\E(Y1_{X\in[x;x+\Delta x]})}{\P(X\in[x;x+\Delta x])}=\frac{\int_{-\infty}^{\infty}\int_{x}^{x+\Delta x} y f(x,y)dx dy }{f_{X}(x)\cdot \Delta x+o(\Delta x)}= \\
=\frac{\int_{-\infty}^{\infty}y\int_{x}^{x+\Delta x}  f(x,y)dx dy }{f_{X}(x)\cdot \Delta x+o(\Delta x)}=\frac{\int_{-\infty}^{\infty}y (f(x,y)\Delta x +o(\Delta x) ) dy }{f_{X}(x)\cdot \Delta x+o(\Delta x)}= \\
=\int_{-\infty}^{\infty}y \frac{f(x,y)\Delta x +o(\Delta x) }{f_{X}(x)\cdot \Delta x+o(\Delta x)} dy
\end{multline}

При $ \Delta x\to 0 $ указанный интеграл стремится к $ \int_{-\infty}^{+\infty}yf(y|x)dy $.
\end{proof}













\section{Большая сила о-малых!}

В теории вероятностей часто возникает примерно такая задача:

Известна функция плотности случайной величины $ X $, $ p_{X}(t) $. Также известно, как $ Y $ выражается через $ X $, то есть известно, что $ Y=f(X) $. Причем функция $ f $ — монотонная и дифференцируемая. Нужно найти функцию плотности $ Y $, $ p_{Y}(t) $.

Есть два способа решения.

Первый — стандартный, без о-малых и их силы. Нужно знать только, что функция плотности — это производная от функции распределения:

\[ p_{Y}(y)=\frac{d\P(Y\leq y)}{dy}=\frac{d\P(X\leq f^{-1}(y))}{dy}=p_{X}(f^{-1}(y))\cdot \frac{df^{-1}(y)}{dy} \]

Или, если считать, что $ y=f(x) $:

\begin{equation}
\frac{dy}{dx}p_{Y}(y)=p_{X}(x)
\end{equation}

\begin{myex} Пусть $ X $ имеет функцию плотности $ p_{X}(x)=2x $ на отрезке $ [0;1] $ и $ Y=X^{3} $. Найдите функцию плотности $ Y $.

Решение: Здесь $ y=x^{3} $, значит $ y'=3x^{2} $ и $ x=y^{1/3} $. Значит:
\begin{equation}
3x^{2}p_{Y}(y)=2x
\end{equation}

Подставляем вместо $ x=y^{1/3} $:
\begin{equation}
3y^{2/3}p_{Y}(y)=2y^{1/3}
\end{equation}

Итого:
\begin{equation}
p_{Y}(y)=\frac{2}{3}y^{-1/3}
\end{equation}
\end{myex}

Теперь магия о-малых \index{О-малые}!

Какой смысл в функции плотности? Вероятность того, что $ X $ лежит в отрезке небольшой длины примерно равна произведению длины этого отрезка на значение плотности:

\begin{equation}
\P(X\in [x;x+\Delta x])\approx p(x)\Delta x
\end{equation}

Здесь $ \Delta x $ — это небольшое число. Если быть точным, то:
\begin{equation}
\P(X\in [x;x+\Delta x])=p_{X}(x)\Delta x+o(\Delta x)
\end{equation}


На всякий случай,
\begin{itemize}
\item $ o(\Delta x) $ — это такая функция от $ \Delta x $, что:
\begin{equation}
\lim_{\Delta x\to 0} \frac{o(\Delta x)}{\Delta x}=0
\end{equation}
\item $ o(1) $ — это такая функция от $ \Delta x $, что
\begin{equation}
\lim_{\Delta x \to 0} o(1)=0
\end{equation}


\end{itemize}



Поскольку $ f $ монотонная, то событию $X\in [x;x+\Delta x]  $ соответствует событие $Y\in [y;y+\Delta y]$, где конечно, $ y=f(x) $ и $ \Delta y= f(x+\Delta x)-f(x) $.

Аналогично:
\begin{equation}
\P(Y\in [y;y+\Delta y])=p_{Y}(y)\Delta y+o(\Delta y)
\end{equation}

Приравниваем две вероятности:
\begin{equation}
p_{X}(x)\Delta x+o(\Delta x)=p_{Y}(y)\Delta y+o(\Delta y)
\end{equation}

Делим на $ \Delta x $:
\begin{equation}
p_{X}(x)+\frac{o(\Delta x)}{\Delta x}=p_{Y}(y)\frac{\Delta y}{\Delta x}+\frac{o(\Delta y)}{\Delta x}
\end{equation}

Устремляем о-малое к нулю и по определению о-малого получаем:
\begin{equation}
p_{X}(x)=p_{Y}(y)\frac{dy}{dx}
\end{equation}

Продолжаем осваивать большую силу о-малых!

Пусть $ X_{1} $, \ldots, $ X_{n} $ — имеют регулярную функцию распределения $ F() $ на $ [0;1] $ и независимы. Напомним, что мы вводили обозначения $ Y_{1} $, $ Y_{2} $, \ldots, $ Y_{n-1} $ — это величины $ X_{2} $, $ X_{3} $, \ldots, $ X_{n} $, отсортированные в порядке убывания. В частности, $ Y_{1}=\max\{X_{2},\ldots,X_{n}\} $ — наибольшая ставка сделанная всеми игроками кроме первого.

\begin{myex} Найдите функцию плотности $ Y_{1} $.

Решение. Прежде чем доставать из ножен о-малые вспомним два простых факта:

\begin{equation}
\P(X_{1}\leq x)=F(x)
\end{equation}

\begin{equation}
\P(X_{1}\in [x_{1};x_{2}])=F(x_{2})-F(x_{1})
\end{equation}


%В аукционе первой цены судьба первого игрока зависит от $ X_{1} $ и $ Y_{1} $\ldots

Вместо плотности $ p_{Y_{1}}(z) $ мы ищем вероятность $ \P(Y_{1}\in [z;z+\Delta z]) $. При маленьких $ \Delta z $ вероятность и плотность связаны:
\begin{equation}
\P(Y_{1}\in [z;z+\Delta z])\approx p_{Y_{1}}(z)\Delta z
\end{equation}


Расчехлим о-малые:

\begin{equation}
\P(X_{2}<X_{1}|X_{1}\in [z;z+\Delta z])=F(z)+o(1)
\end{equation}

Чуть-чуть помедитируйте над этим равенством. При $ \Delta z\to 0 $ правая и левая части становятся похожи на $ F(z) $. Значит верное равенство.

А теперь мы одним махом выпишем ответ!

Что значит $ Y_{1}\in [z;z+\Delta z] $? Это означает, что одна из величин $ X_{i} $ попала в этот интервал. У нас есть $ (n-1) $ возможностей выбрать эту одну. А остальные $ (n-2) $ случайные величины должны быть меньше избранной! Смотрите на ответ:

\begin{equation}
\P(Y_{1}\in [z;z+\Delta z])=(n-1)\cdot (F(z+\Delta z)-F(z)) \cdot (F(z)+o(1))^{n-2}
\end{equation}

По сомножителям:
\begin{itemize}
\item $ (n-1) $ — это число способов выбрать ту случайную величину, которая будет максимумом. Скажем мы выбрали $ X_{3} $
\item $ F(z+\Delta z)-F(z) $ — это вероятность того, что $ X_{3} $ попадет в интервал $ [z;z+\Delta z] $.
\item $ F(z)+o(1)=\P(X_{i}<X_{3}|X_{3}\in [z;z+\Delta z]) $
\end{itemize}

Чтобы получить функцию плотности: делим на $ \Delta z $ и устремляем $ \Delta z $ к нулю!

\begin{equation}
p_{Y_{1}}(z)=(n-1)\cdot f(z)\cdot F(z)^{n-2}
\end{equation}


На всякий случай напомню стандартный способ без о-малых:
\begin{multline}
p_{Y_{1}}(z)=\frac{d\P(Y_{1}\leq z)}{dz}=\frac{d \P(X_{2}<z\cap\ldots\cap X_{n}<z)}{dz}=\\
=\frac{dF(z)^{n-1}}{dz}=(n-1)f(z)F(z)^{n-2}
\end{multline}
\end{myex}



Если при поиске отдельной функции плотности можно обойтись без о-малых, то при переходе к совместной функции плотности о-малые впереди на лихом коне!

\begin{myex}
Найдите совместную функцию плотности для пары  $ Y_{1} $ и $ Y_{3} $.

Вместо плотности легче искать вероятность:

\begin{equation}
 \P(Y_{1}\in [w;w+\Delta w] \cap Y_{3} \in [z+\Delta z] )=p(w,z)\Delta w\Delta z+o(\Delta w \Delta z)
\end{equation}


Что значит $ Y_{1} \in [w;w+\Delta w] \cap Y_{3} \in [z+\Delta z]$? Это означает, что одна из величин $ X_{2} $, \ldots, $ X_{n} $ попала в отрезок $ [w;w+\Delta w] $. У нас есть $ (n-1) $ возможностей выбрать эту одну. Ещё одна величина попала в $ [z;z+\Delta z] $. Эту одну можно выбрать $ (n-2) $ способом. Ещё одна попала между ними, то есть в отрезок $ [w+o(1);z+o(1)] $. Эту одну можно выбрать $(n-3)$ способами. Оставшиеся $ (n-4) $ случайные величины должны быть меньше $ Y_{3} $, они лежат на отрезке $ [0;w+o(1)] $:

\begin{multline}
\P(Y_{1}\in [w;w+\Delta w] \cap Y_{3}\in [z;z+\Delta z] ) =\\
= (n-1)(n-2)(n-3) (F(w+\Delta w)-F(w)) (F(z+\Delta z)-F(z))\cdot \\
\cdot (F(z)-F(w)+o(1)) (F(z)+o(1))^{n-4}
\end{multline}

На всякий случай объясняем по сомножителям:
\begin{itemize}
\item $(n-1)  $ — это число способов выбрать ту случайную величину, которая будет максимумом. Скажем мы выбрали $ X_{3} $
\item $ (n-2) $ — это число способов выбрать $ Y_{3} $ среди оставшихся. Скажем мы выбрали $ X_{7} $.
\item $ (n-3) $ — это число способов выбрать $ Y_{2} $. её надо оговаривать особо, так как она должна лечь между $ Y_{1} $ и $ Y_{3} $. Пусть это оказалась $ X_{9} $.
\item $(F(w+\Delta w)-F(w))$ — это $ \P(X_{3}\in [w;w+\Delta w]) $
\item $(F(z+\Delta z)-F(z))$ — это $ \P(X_{7}\in [z;z+\Delta z] ) $
\item $(F(w)-F(z)+o(1))$ — это $ \P(X_{9}\in [w+o(1);z+o(1)] ) $
\item $ (F(z)+o(1))^{n-4} $ — это вероятность того, что оставшиеся $ X_{i} $ меньше $ X_{7} $
\end{itemize}

Делим на $ \Delta w $, $ \Delta z $ и устремляем их к нулю.

\begin{equation}
p_{Y_{1},Y_{3}}(w,z)=(n-1)(n-2)(n-3)f(z)f(w)(F(w)-F(z))F(z)^{n-4}
\end{equation}

\end{myex}










\section{Старые формулы на вероятностном языке}


\begin{quotation}
Сова стала объяснять, что такое Необходимая или Соответствующая Спинная Мускулатура. Она уже объясняла это когда-то Пуху и Кристоферу Робину и с тех пор ожидала удобного случая, чтобы повторить объяснения, потому что это такая штука, которую вы спокойно можете объяснять два раза, не опасаясь, что кто-нибудь поймёт, о чём вы говорите.

Алан Милн в переводе Бориса Заходера.
\end{quotation}

Мы быстро повторим то, что сделали на первой лекции:
\begin{myth}
Если сигналы совпадают с ценностями, то есть $ X_{i}=V_{i} $, то ожидаемая прибыль первого игрока может быть записана как:
\begin{enumerate}
\item $\pi_{1}(x,b(a))=(x-b(a))\cdot (F(a))^{n-1}$, если $ X_{i} $ независимы
\item $\pi_{1}(x,b(a))=(x-b(a))\cdot \P(Y_{1}<a|X_{1}=x)$, если $ X_{i} $ зависимы
\end{enumerate}
\end{myth}

\begin{proof}
Для аукциона первой цены:
\begin{equation}
Pay_{1}=Bid_{1}\cdot 1_{W_{1}}
\end{equation}

Значит:
\begin{equation}
Profit_{1}=X_{1}\cdot 1_{W_{1}}-Pay_{1}=(X_{1}-Bid_{1})\cdot 1_{W_{1}}
\end{equation}



\begin{multline}
\E(Profit_{1}|X_{1}=x ; Bid_{1}=b_{1})=\\
=\E((X_{1}-Bid_{1})\cdot 1_{W_{1}}|X_{1}= x ; Bid_{1}=b_{1})=\\
= (x-b_{1})\E(1_{W_{1}}|X_{1}=x ; Bid_{1}=b_{1})=\\
=(x-b_{1})\P(W_{1}|X_{1}=x ; Bid_{1}=b_{1})
\end{multline}

Озаботимся величиной $ \P(W_{1}|X_{1}=x ; Bid_{1}=b_{1}) $. Если ценности независимы, то:
\begin{multline}
\P(W_{1}|X_{1}=x ; Bid_{1}=b_{1})=\P(W_{1}|Bid_{1}=b_{1})= \\
=\P(Bid_{2}<Bid_{1}\cap \ldots ; Bid_{n}<Bid_{1}|Bid_{1}=b_{1})=\\
=\P(Bid_{2}<b_{1}\cap \ldots ; Bid_{n}<b_{1})=\\
=\P(Bid_{2}<b_{1})^{n-1}=\P(b(X_{2})<b_{1})^{n-1}
\end{multline}

Дальше чудо-замена $ b_{1}=b(a) $. Уточняем: $ b_{1} $ — константа, $ a $ — константа, $ b() $ — неизвестная, но детерминистическая функция.

\begin{equation}
\P(b(X_{2})<b_{1})^{n-1}=\P(b(X_{2})<b(a))^{n-1}=\P(X_{2}<a)^{n-1}=(F(a))^{n-1}
\end{equation}

Итого, для аукциона первой цены с независимыми ценностями:
\begin{equation}
\pi_{1}(x,b(a))=(x-b(a))\cdot (F(a))^{n-1}
\end{equation}

Что меняется, если ценности зависимы?

Единственное отличие состоит в том, что:
\begin{equation}
\P(W_{1}|X_{1}=x ; Bid_{1}=b_{1}) \neq \P(W_{1}|Bid_{1}=b_{1})
\end{equation}

На этот раз вероятность упрощается не так сильно:
\begin{multline}
\P(W_{1}|X_{1}=x ; Bid_{1}=b_{1})=\\
=\P(Bid_{2}<Bid_{1}\cap \ldots \cap Bid_{n}<Bid_{1}|X_{1}=x ; Bid_{1}=b_{1})=\\
=\P(Bid_{2}<b_{1}\cap \ldots \cap Bid_{n}<b_{1}|X_{1}=x)=\\
=\P(b(X_{2})<b_{1}\cap \ldots \cap b(X_{n})<b_{1}|X_{1}=x)
\end{multline}

И ещё чуть-чуть после чудо-замены:
\begin{multline}
\P(b(X_{2})<b_{1}\cap \ldots \cap b(X_{n})<b_{1}|X_{1}=x)=\\
\P(b(X_{2})<b(a)\cap \ldots \cap b(X_{n})<b(a)|X_{1}=x)=\\
\P(X_{2}<a\cap \ldots \cap X_{n}<a|X_{1}=x)
\end{multline}

А эту вероятность можно посчитать, если известна совместная функция плотности ценностей. С использованием обозначения $ Y_{1}=\max\{X_{2},\ldots, X_{n}\} $ можно записать её короче:
\begin{equation}
\P(X_{2}<a\cap \ldots \cap X_{n}<a|X_{1}=x)=\P(Y_{1}<a|X_{1}=x)
\end{equation}

Итого, для аукциона первой цены с зависимыми ценностями:
\begin{equation}
\pi_{1}(x,b(a))=(x-b(a))\cdot \P(Y_{1}<a|X_{1}=x)
\end{equation}
\end{proof}

А сейчас мы увидим, как с помощью мат. ожидания записать уже знакомые нам вещи. А именно:

\begin{myth} Если предпосылки теоремы об одинаковой доходности выполнены, то:
\label{probabilistic_interpretation}
\begin{itemize}
\item Для произвольного аукциона $ q(x)=\P(Y_{1}<x) $
\item Для произвольного аукциона $ pay(x)=\E(Y_{1}\cdot 1_{Y_{1}<x}) $
\item Для аукциона первой цены $ b(x)=\E(Y_{1}|Y_{1}<x) $
\end{itemize}
\end{myth}

\begin{proof}

Первое. Из предпосылки о том, что товар достаётся тому покупателю, у которого выше ценность немедленно следует, что $ q(x)=\P(Y_{1}<x) $.

Второе. В одном из упражнений первой лекции мы установили:
\begin{equation}
pay(x)=pay(0)+\int_{0}^{x}t \cdot \frac{dq(t)}{dt}\cdot dt
\end{equation}

Если предпосылки теоремы об одинаковой доходности выполнены, то:
\begin{equation}
pay(x)=\int_{0}^{x}t \cdot \frac{dq(t)}{dt}\cdot dt
\end{equation}


Вспоминаем, что:
\begin{equation}
q(x)=F(x)^{n-1}
\end{equation}

Стало быть
\begin{equation}
\frac{dq(x)}{dx}=(n-1)f(x)F(x)^{n-2}
\end{equation}

И!!! Мы видим, что это есть функция плотности $ Y_{1} $!!!:
\begin{equation}
\frac{dq(x)}{dx}=p_{Y_{1}}(x)
\end{equation}

то есть для любого аукциона подходящего в теорему об одинаковой доходности:
\begin{equation}
pay(x)=\int_{0}^{x}t \cdot p_{Y_{1}}(t)\cdot dt=\E(Y_{1}\cdot 1_{Y_{1}<x})
\end{equation}



Третье. На аукционе первой цены:
\begin{equation}
pay(x)=b(x)\cdot q(x)
\end{equation}

Пользуемся первыми двумя результатами и получаем:
\begin{equation}
\E(Y_{1}1_{Y_{1}<x})=b(x)\cdot \P(Y_{1}<x)
\end{equation}

Отсюда немедленно следует, что:
\begin{equation}
b(x)=\frac{\E(Y_{1}1_{Y_{1}<x})}{\P(Y_{1}<x)}=\E(Y_{1}|Y_{1}<x)
\end{equation}



\end{proof}







\section{Просто разные примеры}


\begin{myex} Потренируем о-малую мускулу. Найдем равновесие Нэша в аукционе третьей цены\index{Аукцион третьей цены}. Его правила таковы. Есть $ n $ игроков, они одновременно делают ставки. Товар получает тот, кто назвал самую высокую ставку, но платит от не свою ставку, а третью по величине ставку. Предполагаем, что ценность и сигнал — это одно и то же, то есть $ V_{i}=X_{i} $, а сами сигналы $ X_{i} $ независимы и имеют регулярное распределение на $ [0;1] $.


Мы только что доказали, что при выполнении теоремы об одинаковой доходности:
\begin{equation}
pay_{1}(x)=\E(Y_{1}\cdot 1_{Y_{1}<x})=\int_{0}^{x}tp_{Y_{1}}(t)dt
\end{equation}

Из этого следует, что:
\begin{equation}
\frac{dpay_{1}(x)}{dx}=xp_{Y_{1}}(x)
\end{equation}


С другой стороны на аукционе третьей цены первый игрок платит третью по величине ставку, значит вторую по величине ставку игроков не считая себя.

\begin{equation}
pay_{1}(x)=\E(Pay_{1}|X_{1}=x)=\E(b(Y_{2})\cdot 1_{W_{1}}|X_{1}=x)
\end{equation}

Если мы предположим, что в равновесии функция $ b() $ строго возрастает, то
\begin{equation}
W_{1}=\{b(X_{1})>b(X_{2})\cap \ldots\cap  b(X_{1})>b(X_{n})\}=\{X_{1}>Y_{1}\}
\end{equation}

так как $ X_{i} $ независимы, нашу функцию выплат можно записать с помощью безусловного мат. ожидания:
\begin{equation}
pay_{1}(x)=\E(b(Y_{2})\cdot 1_{Y_{1}<X_{1}}|X_{1}=x)=\E(b(Y_{2})\cdot 1_{Y_{1}<x})
\end{equation}

Судя по формуле нам нужна совместная функция плотности $ Y_{1} $ и $ Y_{2} $. О-малые приходят на помощь:

\begin{equation}
p(y_{1},y_{2})=(n-1)(n-2)f(y_{1})f(y_{2})F(y_{2})^{n-3}
\end{equation}

Следует уточнить, что эта формула верна при $0<y_{2}<y_{1}<1$. При остальных $ y_{1} $ и $ y_{2} $ плотность равна нулю.

Для нахождения математического ожидания выписываем страшный двойной интеграл:
\begin{multline}
pay_{1}(x)=\E(b(Y_{2})\cdot 1_{Y_{1}<x})=\int_{0}^{1}\int_{0}^{y_{1}} b(y_{2})1_{y_{1}<x} p(y_{1},y_{2})dy_{2}dy_{1}=\\
=\int_{0}^{x}\int_{0}^{y_{1}} b(y_{2})p(y_{1},y_{2})dy_{2}dy_{1}=\\
=\int_{0}^{x}\int_{0}^{y_{1}} b(y_{2}) (n-1)(n-2)f(y_{1})f(y_{2})F(y_{2})^{n-3} dy_{2}dy_{1}=\\
(n-1)(n-2)\int_{0}^{x}f(y_{1})\int_{0}^{y_{1}} b(y_{2}) f(y_{2})F(y_{2})^{n-3} dy_{2}dy_{1}
\end{multline}

Конечно, интегрировать это мы не будем. Мы наоборот, возьмем производную по $ x $ два раза. Берём производную первый раз:
\begin{equation}
\frac{dpay_{1}(x)}{dx}=(n-1)(n-2)f(x)\int_{0}^{x} b(y_{2}) f(y_{2})F(y_{2})^{n-3} dy_{2}
\end{equation}

С другой стороны,
\begin{equation}
\frac{dpay_{1}(x)}{dx}=xp_{Y_{1}}(x)=x(n-1)f(x)(F(x))^{n-2}
\end{equation}

После сокращения $ (n-1) $ и $ f(x) $ мы получили уравнение:
\begin{equation}
(n-2)\int_{0}^{x} b(y_{2}) f(y_{2})F(y_{2})^{n-3} dy_{2}=x(F(x))^{n-2}
\end{equation}

Чтобы избавиться от интеграла берём ещё раз производную по $ x $ от обеих частей.
\begin{equation}
(n-2) b(x) f(x)F(x)^{n-3}=F(x)^{n-2}+(n-2)xf(x)F(x)^{n-3}
\end{equation}

Выражаем $ b(x) $:
\begin{equation}
b(x)=\frac{F(x)}{(n-2)f(x)}+x
\end{equation}

При равномерном распределении наша формула превращается в:
\begin{equation}
b(x)=\frac{n-1}{n-2}x
\end{equation}

Она возрастает по $ x $, значит теорему об одинаковой доходности действительно можно было применять.
\end{myex}


\begin{myex} Общая ценность.

Предположим, что на аукционе первой цены продаётся участок с домом. Торгуются два игрока. Ценность участка с домом одинакова для обоих игроков. Только они её не совсем полностью знают. Один игрок хорошо разбирается в домах, а второй — в земельных участках. то есть Природа сообщает первому игроку ценность дома, а второму — ценность участка.

Введем обозначения:

\begin{itemize}
\item $ V_{i} $ — случайная величина, ценность товара для игрока $ i $
\item $ X_{i} $ — случайная величина, сигнал от Природы, который получает игрок $ i $
\end{itemize}

В нашем случае: $ V_{1}=V_{2}=X_{1}+X_{2} $ — это ценности товара, а $ X_{1} $ — сигнал, часть ценности, известная первому игроку и $ X_{2} $ — сигнал, часть ценности, известная второму игроку.

Предположим, что $ X_{i} $ независимы и равномерны на $ [0;1] $. Найдите равновесие Нэша.

Решение:

В этом случае прибыль равна:

\begin{equation}
Profit_{1}=(V_{1}-Bid_{1})\cdot 1_{W_{1}}
\end{equation}

Находим нашу детерминистическую функцию:

\begin{multline}
\E(Profit_{1}|X_{1}=x; Bid_{1}=b_{1})=\\
=\E((x+X_{2})-b_{1})\cdot 1_{W_{1}}|X_{1}=x; Bid_{1}=b_{1})=\\
=(x-b_{1})\P(W_{1}|X_{1}=x; Bid_{1}=b_{1})+\E(X_{2}\cdot 1_{W_{1}}|X_{1}=x; Bid_{1}=b_{1})
\end{multline}


Здесь нужно быть очень аккуратным так как $ \E(X\cdot Y) $ в общем случае не совпадает с $ \E(X)\cdot \E(Y) $!

В силу независимости $ X_{i} $ и того, что $ Bid_{2}=b(X_{2}) $ упрощаем $\P(W_{1}|X_{1}=x ; Bid_{1}=b_{1})$:
\begin{multline}
\P(W_{1}|X_{1}=x; Bid_{1}=b_{1})=\\
=\P(Bid_{2}<Bid_{1}|X_{1}=x; Bid_{1}=b_{1})=\P(Bid_{2}<b_{1})
\end{multline}

Вспомнив чудо-замену $ b_{1}=b(a) $ получаем:
\begin{equation}
\P(Bid_{2}<b_{1})=\P(b(X_{2})<b(a))=\P(X_{2}<a)=F(a)
\end{equation}

Упрощаем: $ \E(X_{2}\cdot 1_{W_{1}}|X_{1}=x; Bid_{1}=b_{1}) $:
\begin{multline}
\E(X_{2}\cdot 1_{W_{1}}|X_{1}=x; Bid_{1}=b_{1})=\E(X_{2}\cdot 1_{Bid_{2}<Bid_{1}}|X_{1}=x; Bid_{1}=b_{1})=\\
=\E(X_{2}\cdot 1_{b(X_{2})<b_{1}}|X_{1}=x)=\E(X_{2}\cdot 1_{b(X_{2})<b_{1}})
\end{multline}

В последнем переходе мы использовали то, что $ X_{1} $ и $ X_{2} $ независимы.

И, чудо-замена,

\begin{equation}
\E(X_{2}\cdot 1_{b(X_{2})<b_{1}})=\E(X_{2}\cdot 1_{b(X_{2})<b(a)})=\E(X_{2}\cdot 1_{X_{2}<a})=\int_{0}^{a}tf(t)dt
\end{equation}

Собираем все вместе:

\begin{equation}
\pi_{1}(x,b(a))=(x-b(a))F(a)+\int_{0}^{a}tf(t)dt
\end{equation}

Берём производную по $ a $:
\begin{equation}
-b'(a)F(a)+(x-b(a))f(a)+af(a)=0
\end{equation}

так как мы ищем равновесие Нэша, оптимальное $b_{1}=b(x)  $, и значит, оптимальное $ a=x $:

\begin{equation}
-b'(x)F(x)+(x-b(x))f(x)+xf(x)=0
\end{equation}

В случае равномерных ценностей:
\begin{equation}
-b'(x)x+(x-b(x))+x=0
\end{equation}

Это линейное дифференциальное уравнение, общее решение имеет вид $ b(x)=x+\frac{c}{x} $. Единственное ограниченное на $ [0;1] $ решение — это $ b(x)=x $. Почему нас не интересуют неограниченные решения? Потому, что оно заведомо не оптимально. Если стратегия $ b(x) $ принимает значения больше единицы, то стратегия $ \min\{b(x),1\} $ окажется лучше. Стратегия $ b(x) $ с положительной вероятностью приводит к ситуации, когда мы выигрываем товар, но обязаны заплатить за него сумму выше 1, то есть мы получаем отрицательный выигрыш. Стратегия $ \min\{b(x),1\} $ этого недостатка лишена, а во всем прочем точно копирует стратегию $ b(x) $.

\end{myex}


\begin{myex} Кнопочный аукцион с зависимыми ценностями.
У нас имеется три игрока, $ V_{1}=V_{2}=V_{3}=X_{1}+X_{2}+X_{3} $, ценности независимы и равномерны на $ [0;1] $. Найдите равновесные Нэша и средний доход продавца.

Самое сложное — это понять, что является здесь стратегией игрока. Напомню правила. Текущая цена равна времени прошедшему с момента начала аукциона. В начале все игроки жмут свои кнопки. Каждый сам решает, когда ему отпустить кнопку. Как только кнопку отпускает предпоследний игрок, аукцион оканчивается. Победителем считается тот, кто продолжает давить. Он получает товар, по текущей цене на конец аукциона. Во время аукциона игроки знают кто и когда его покидает. Стратегия игрока может учитывать эту информацию. Если ценности зависимы, так оно и окажется.

Рассмотрим первого игрока. Стратегия должна говорить игроку: до которого времени давить кнопку, если никто другой не вышел, и до которого времени давить кнопку, если один другой игрок вышел на цене $ p $. При всем при этом стратегия может учитывать известное игроку значение $ X_{1} $. В результате стратегия игрока в симметричном случае описывается двумя (!) функциями $ (b^{3}(x), b^{2}(x,p)) $:
\begin{itemize}
\item $ b^{3}(x) $ — до которого времени давить кнопку, если в игре 3 игрока
\item $ b^{2}(x,p) $ — до которого времени давить кнопку, если в игре 2 игрока, и один ушел на цене $ p $
\end{itemize}


Сейчас мы предъявим равновесные стратегии, а затем докажем, что они действительно равновесные.

Пусть:
\begin{itemize}
\item $ b^{3}(x)=x+x+x=3x$
\item $ b^{2}(x,p)=x+x+\frac{p}{3}=2x+\frac{p}{3}$
\end{itemize}

Откуда взялись эти стратегии?


Первая, $ b^{3}() $, получилась подстановкой $ x $ в платежную функцию вместо каждого $ X_{i} $.

Чтобы понять вторую, $ b^{2}(x,p) $, давайте ещё раз вспомним, что такое равновесие Нэша.  Равновесие Нэша — это набор стратегий, такой что игрокам не выгодно в одиночку их менять, даже если они расскажут друг другу о своих стратегиях. то есть при поиске равновесия Нэша можно считать стратегии общеизвестными!

Теперь допустим, что текущая цена равна $ p $, и остальные два игрока используют функцию $ b^{3}() $. И вдруг кто-то из них выходит. Это означает, что для него $ b^{3}(x)=p $. Значит для уходящего игрока $ x=\frac{p}{3} $. В равновесии Нэша этот вывод могут сделать все игроки! то есть можно восстановить сигнал $ x $ уходящего игрока, зная цену на которой он вышел. И мы вводим новую функцию $ b^{2}(x,p) $, в которой сигнал ушедшего игрока выражен через $ p $, а сигналы остающихся игроков по-прежнему заменены на одно и то же $ x $.

Чем они хороши?

Сначала определим, в каком порядке будут выходить игроки, если все используют указанную стратегию. Сначала выйдет тот игрок, у кого минимальное значение $ b^{3}(X_{i}) $. Поскольку $ b^{3}() $ строго возрастающая функция первым выйдет игрок с минимальным $ X_{i} $. Допустим это произошло на цене $ p $. Следующим игроком из двух оставшихся выйдет тот, у кого $ b^{2}(X_{i},p) $ меньше. Но $ b_{2}(x,p) $ строго возрастает по $ x $, а значение $ p $ одинаковое, значит вторым выйдет тот из оставшихся игроков, у кого $ X_{i} $ меньше. Мораль. Если игроки используют эти стратегии, то они выходят в порядке возрастания ценностей и товар получает тот, у кого $ X_{i} $ наибольшее. Это хорошее свойство указанной стратегии, но ещё не доказательство оптимальности.

Почему же они все-таки равновесны?

Рассматриваем ситуацию, когда все игроки кроме первого используют указанную стратегию. Мы сейчас посчитаем какой выигрыш получает первый игрок, если тоже использует указанную стратегию и сколько он получит, если отклонится.

Пусть первый игрок также использует стратегию $ (b^{3}(x), b^{2}(x,p)) $. Возможно две ситуации:
\begin{itemize}
\item Первый игрок выигрывает аукцион.

Это происходит, если его ценность выше других, то есть $ X_{1}>Y_{1} $. В этом случае первый выход из игры происходит при цене $ b^{3}(Y_{2}) $, а второй — при цене $ b^{2}(Y_{1},b^{3}(Y_{2})) $. Значит первый игрок заплатит $ b^{2}(Y_{1},b^{3}(Y_{2})) $. И выигрыш первого игрока равен:
\begin{multline}
V_{1}-b^{2}(Y_{1},b^{3}(Y_{2}))=X_{1}+X_{2}+X_{3}-b^{2}(Y_{1},b^{3}(Y_{2}))=\\
=X_{1}+Y_{1}+Y_{2}-Y_{1}-Y_{1}-Y_{2}=X_{1}-Y_{1}>0
\end{multline}
Если игрок захочет отклонится, то есть не выиграть аукцион, то он получит 0. Значит отклоняться не выгодно.

\item Первый игрок проигрывает аукцион.

Это происходит, если его ценность не максимальная, то есть $ X_{1}<Y_{1} $. В этом случае его доход равен 0. Сколько игрок получит, если захочет отклонится, то есть выиграть?
\begin{multline}
 V_{1}-b^{2}(Y_{1},b^{3}(Y_{2}))=X_{1}+X_{2}+X_{3}-b^{2}(Y_{1},b^{3}(Y_{2}))=\\
 =X_{1}+Y_{1}+Y_{2}-Y_{1}-Y_{1}-Y_{2}=X_{1}-Y_{1}<0
\end{multline}
Отклоняться не выгодно.
\end{itemize}


Считаем среднюю выручку продавца:
\begin{equation}
\E(R)=\E(b^{2}(Y_{1},b^{3}(Y_{2}))|X_{1}>Y_{1})=\E((2Y_{1}+Y_{2})|X_{1}>Y_{1})=\ldots=1.25
\end{equation}

Замечания:
\begin{itemize}
\item При определении стратегий была важна связь между $ X_{i} $ и $ V_{i} $, но не совместная функция плотности $ X_{i} $. В решении равномерность распределения не использовалась!
\item Если после окончания игры игроки раскроют значения своих $ X_{i} $ друг другу, то никто не пожалеет о выбранной стратегии. Проигравшие не жалеют, так как при известных $ X_{i} $ им ничего не светит. Выигравший не жалеет, так как при фиксированных стратегиях других игроков выиграть за меньшую сумму он не мог.
\item Аукцион как и раньше очень сильно похож на аукцион второй цены
\end{itemize}




\end{myex}



\section{Супермодулярные функции}

Введем более короткие обозначения для максимума и минимума:

\begin{equation}
x\wedge y\wedge z =\min\{x,y,z\}
\end{equation}

\begin{equation}
x\vee y\vee z =\max\{x,y,z\}
\end{equation}

Запомнить эти обозначения легко поняв их происхождение. Кто-то когда-то давно заметил, что минимум и максимум нескольких чисел ведут себя точно так же, как пересечение и объединение множеств:

Убедитесь, что для множеств верно:

\begin{equation}
(A \cap B)\cup C=(A\cup C)\cap (B\cup C)
\end{equation}

А для чисел верно:
\begin{equation}
(a \wedge b)\vee c=(a\vee c)\wedge (b\vee c)
\end{equation}

Совпадение это не случайно. Если какая-то формула для множеств верна, то она верна и для подмножеств числовой прямой вида $ (-\infty;t] $. А на таких подмножествах аналогия совершенно прямая: если $ A=(-\infty;a] $ и $ B=(-\infty;b] $, то $ A\cup B= (-\infty;a\vee b] $ и $ A\cap B=(-\infty;a\wedge b] $.

Докажите, что, например, верны формулы:

\begin{equation}
1_{A}\wedge 1_{B}=1_{A\cap B}
\end{equation}

\begin{equation}
1_{A}\vee 1_{B}=1_{A\cup B}
\end{equation}


%Эти значки называются мит и джойн (?)\ldots

Аналогичные операции применяются и к векторам:

\begin{mydef} Если $ \vec{x}=(x_{1},\ldots, x_{n}) $ и $ \vec{y}=(y_{1},\ldots, y_{n}) $, то:
\begin{equation}
\vec{x}\wedge\vec{y}=(x_{1}\wedge y_{1}, \ldots, x_{n}\wedge y_{n})
\end{equation}

\begin{equation}
\vec{x}\vee\vec{y}=(x_{1}\vee y_{1}, \ldots, x_{n}\vee y_{n})
\end{equation}

\end{mydef}


В экономическом моделировании то там, то сям возникают супермодулярные функции:

\begin{mydef} Функция $ f() $ называется \indef{супермодулярной}\index{Супермодулярная функция}, если для любых $ \vec{x} $ и $ \vec{y} $
\begin{equation}
f(\vec{x}\wedge\vec{y})+f(\vec{x}\vee\vec{y})\geq f(\vec{x})+f(\vec{y})
\end{equation}

\end{mydef}

\begin{myex} Функция $ f(x_{1},x_{2})=x_{1} $ является супермодулярной. Действительно, левая часть равна:
\begin{equation}
f(\vec{x}\wedge\vec{y})+f(\vec{x}\vee\vec{y})=x_{1}\wedge y_{1}+x_{1}\vee y_{1}=x_{1}+y_{1}
\end{equation}
Правая часть равна:
\begin{equation}
f(\vec{x})+f(\vec{y})=x_{1}+y_{1}
\end{equation}
\end{myex}

\begin{myex} Функция $ f(x)=x^{3} $ является супермодулярной:
\begin{multline}
f(x\wedge y)+f(x \vee y)=(x\wedge y)^{3}+(x\vee y)^{3}=x^{3}\wedge y^{3} + x^{3}\vee y^{3}=\\
=x^{3}+y^{3}=f(x)+f(y)
\end{multline}
\end{myex}

\begin{myex} Функция $ f(x_{1},x_{2})=-x_{1}x_{2} $ не является супермодулярной:
\begin{equation}
f((1,2)\wedge (2,1))+f((1,2)\vee (2,1))=f(2,2)+f(1,1)=-4-1=-5
\end{equation}
\begin{equation}
f(1,2)+f(2,1)=-2-2=-4
\end{equation}


\end{myex}




\begin{myth} Если у функции $ f $ существуют вторые производные, то она является супермодулярной если и только если для $ i\neq j $:
\label{supermod_crit}
\begin{equation}
\frac{\partial^{2}f}{\partial x_{i}\partial x_{j}}\geq 0
\end{equation}
\end{myth}

Сначала мы докажем, что эта теорема верна для функции двух переменных.
\begin{proof} Случай двух переменных.

Рассмотрим функцию $ f(x,y) $ и четыре точки на плоскости, $ a'>a $ и $ b'>b $:


\begin{tikzpicture}
    \draw[style=help lines,color=gray] (-0.1,-0.1) grid (4.1,4.1);
    \draw[->] (-0.1,0) -- (4.2,0) node[right] {$b_1$};
    \draw[->] (0,-0.1) -- (0,4.2) node[above] {$b_2$};
    \draw[color=black,dashed] (1,0)--(1,4) ;
    \draw[color=black,dashed] (3,0)--(3,4) ;
    \draw[color=black,dashed] (0,1)--(4,1) ;
    \draw[color=black,dashed] (0,3)--(4,3) ;
	\node[below] at (1,0) {$a$};
	\node[below] at (3,0) {$a'$};
	\node[left] at (0,1) {$b$};
	\node[left] at (0,3) {$b'$};
	\node at (1.5,0.7) {$(a,b)$};
	\node at (3.5,0.7) {$(a',b)$};
	\node at (1.5,2.7) {$(a,b')$};
	\node at (3.5,2.7) {$(a',b')$};
\end{tikzpicture}

Применяя определение супермодулярности к паре $ (a,b') $ и $ (a',b) $ мы получаем:
\begin{equation}
f((a',b)\wedge(a,b')) + f((a',b)\vee(a,b'))\geq f(a',b) + f(a,b')
\end{equation}

Применительно к нашим точкам:
\begin{equation}
f(a,b)+f(a',b')\geq f(a',b)+f(a,b')
\end{equation}

Или:
\begin{equation}
f(a',b')-f(a',b)\geq f(a,b')-f(a,b)
\end{equation}

Что можно озвучить словами так: разность $ f(x,b')-f(x,b) $ растет по $ x $ при $ b'>b $.

%Следовательно, $ \ln (f(x,b')/f(x,b)) $ растет по $ x $ при $ b'>b $.

Значит при $ b'>b $:
\begin{equation}\label{first_deriv}
\frac{\partial }{\partial x}(f(x,b')-f(x,b))=\frac{\partial }{\partial x}f(x,b')-\frac{\partial }{\partial x} f(x,b) \geq 0
\end{equation}

По определению, вторая производная, это:
\begin{equation}
\frac{\partial^{2}}{\partial x \partial y}f(x,b)=\lim_{\Delta b\to 0}\frac{\frac{\partial }{\partial x}f(x,b+\Delta b)-\frac{\partial }{\partial x} f(x,b)}{\Delta b}
\end{equation}
Если взять $ b'=b+\Delta b $ и $ \Delta b>0 $ мы получаем положительные числитель и знаменатель. Значит и сама вторая производная будет положительной.

В обратную сторону. Если вторая производная всюду неотрицательна, то первая производная разности растет по $ b $, то есть мы получаем формулу \ref{first_deriv}. Остальные переходы равносильны.

\end{proof}


Чтобы получить доказательство для произвольного случая нам понадобится такая теорема полезная и сама по себе:
\begin{myth}
Функция $ f(x_{1},\ldots,x_{n}) $ является супермодулярной если и только если она является супермодулярной функцией от любых двух своих переменных при фиксированных значениях остальных переменных.
\end{myth}

\begin{proof}

Мы докажем теорему для функции трёх переменных. В общем случае доказательство ничем не сложнее, просто навешивается куча индексов.

Итак, пусть у нас есть супермодулярная функция трёх аргументов, $ f(a,b,c) $.

Заметим, что рассмотрев её как функцию двух аргументов, зафиксировав любой третий мы получим снова супермодулярную функцию. Например, если $ g(a,b):=f(a,b,c) $, то $ g $ — супермодулярна:

\begin{multline}
g((x_{1},x_{2})\vee (y_{1},y_{2}))+g((x_{1},x_{2})\wedge (y_{1},y_{2}))=\\
f((x_{1},x_{2},c)\vee (y_{1},y_{2},c))+f((x_{1},x_{2},c)\wedge (y_{1},y_{2},c))\geq \\
f((x_{1},x_{2},c))+f((y_{1},y_{2},c))=g(x_{1},x_{2})+g(y_{1},y_{2})
\end{multline}

В обратную сторону чуть посложнее\ldots Пусть при рассмотрении любых двух переменных мы получаем супермодулярную функцию.

Рассмотрим две точки, $(a',b',c) $ и $ (a,b,c') $. Штрихованная переменная больше, $ a'>a $, $ b'>b $, $ c'>c $.

Получаем:
\begin{multline}
f((a',b',c)\vee (a,b,c'))+f((a',b',c)\wedge (a,b,c'))-f(a',b',c)-f(a,b,c')=\\
f(a',b',c')+f(a,b,c)-f(a',b',c)-f(a,b,c')=\\
f(a',b',c')+f(a,b,c)-f(a',b,c)-f(a,b',c')+\\
+[f(a',b,c)-f(a',b,c)]+[f(a',b,c')-f(a',b,c')]=\\
[f(a',b',c')+f(a',b,c)-f(a',b,c')-f(a',b',c)]+\\
+[f(a,b,c)+f(a',b,c')-f(a',b,c)-f(a,b,c')]\geq 0
\end{multline}

При доказательстве для $ n>3 $ можно пользоваться предположением индукции, то есть тем, что для $ n-1 $ теорема уже доказана.
\end{proof}

Нам супермодулярные функции понадобятся для описания зависимости между сигналами  $X_{i}$. Мы хотим, чтобы сигналы были связаны положительно между собой. Например, если выставлен на торги хороший лот, то все участники считают его хорошим. А выставлен плохой — все более или менее видят, что он плох. Более формально:

%Мы предположим, что игрок не полностью знает ценность товара для себя. то есть каждый игрок получает от Природы сигнал $ X_{i} $. Отныне $ X_{i} $ — это не ценность товара для игрока $ i $!

\begin{mydef}
Случайные величины $ X_{1} $, \ldots, $ X_{n} $ с совместной функцией плотности $ f(x_{1},\ldots,x_{n}) $ называются \indef{аффилированными}\index{Аффилированные случайные величины}, если $ \ln(f(\vec{x})) $ — супермодулярная функция.
\end{mydef}

Чуть позже (теорема \ref{prop_affiliated}) мы докажем, что из аффилированности случайных величин $ X $ и $ Y $ следует например, что:
\begin{enumerate}
\item $ Cov(X,Y)\geq 0 $
\item Функция $g(x)= \E(Y|X=x) $ неотрицательно зависит от $ x $
\end{enumerate}

В силу того, что логарифм произведения равен сумме логарифмов, это определение эквивалентно следующему:
\begin{mydef}
Случайные величины $ X_{1} $, \ldots, $ X_{n} $ с совместной функцией плотности $ f(x_{1},\ldots,x_{n}) $ называются \indef{аффилированными}\index{Аффилированные случайные величины}, если

\begin{equation}
f(\vec{x}\wedge\vec{y})\cdot f(\vec{x}\vee\vec{y})\geq f(\vec{x})\cdot f(\vec{y})
\end{equation}
\end{mydef}




%Упражнение:




% Известно, что аффилированы случайные величины $ X_{1} $, \ldots, $ X_{n} $. Верно ли, что аффилированы случайные величины $ X_{1} $, \ldots, $ X_{n-1} $?


%\begin{myth}
%Если аффилированы $ X_{1} $, \ldots, $ X_{n} $, то аффилированы $ X_{1} $, $ Y_{1} $, $ Y_{2} $, \ldots, $ Y_{n-1} $
%\end{myth}

%\begin{proof}
%\ldots
%\end{proof}

\section{Задачи}
\begin{enumerate}

\item Пусть $ A $ и $ B $ — это события. Верно ли, что $ \E(1_{A}|B)=\P(A|B) $?

\item Пусть $ A $ и $ B $ — это независимые события. Верно ли, что $ \E(1_{A}|B)=\E(1_{A})  $?

\item С помощью о-малых найдите:

\begin{enumerate}
\item функцию плотности минимума остальных ставок, $ p_{Y_{n-1}}(t) $
\item функцию плотности $ p_{Y_{6}}(t) $
\item совместную функцию плотности $ p_{Y_{1},Y_{n-1}}(a,b) $
\item совместную функцию плотности $ p_{Y_{3},Y_{6}}(a,b) $
\item совместную функцию плотности $ p_{Y_{n-2},Y_{n-1}}(a,b) $
\item совместную функцию плотности $ p_{X_{1},Y_{1}}(a,b) $
\end{enumerate}

\item Рассмотрим аукцион первой цены двух игроков. Ценности независимы и имеют функцию плотности $f(t)=2t  $ на $ [0;1] $.

Найдите:
\begin{enumerate}
\item Равновесие Нэша, то есть детерминистические стратегии $ b(x) $
\item Детерминистическую функцию $ pay_{1}(x) $
\item Детерминистическую функцию $ \widehat{pay_{1}}(b) $
\item Детерминистическую функцию $ q_{1}(x) $
\item Детерминистическую функцию $ \widehat{q_{1}}(b) $
\item Функцию распределения случайной величины $ Pay_{1} $
\item $ \E(Pay_{1}) $, $ Cov(Pay_{1},Pay_{2}) $, $ Var(Pay_{1}) $
\item Функцию распределения случайной величины $ R $
\item $ \E(R) $, $ Var(R) $, $ Cov(R,Pay_{1}) $
\item Тонкая разница! Если применить детерминистическую функцию $ pay_{1}(x) $ к случайной величине $ X_{1} $, то мы получим случайную величину $ pay_{1}(X_{1}) $. Временно обозначим её $ L_{1} $ (и забудем это обозначение после этого упражнения). Найдите функцию распределения $ L_{1} $, $ \E(L_{1}) $, $ Var(L_{1}) $, $ Cov(L_{1},L_{2}) $

\end{enumerate}

\item Предположим, что на аукционе первой цены продаётся участок с домом. Торгуются два игрока. Природа сообщает первому игроку ценность дома, $ X_{1} $, а второму — ценность участка, $ X_{2} $. При этом ценности игроков определяются по формулам: $ V_{1}=X_{1}+0.5X_{2} $ — первому важнее дом, $ V_{2}=0.5X_{1}+X_{2} $ — второму важнее участок. Ценности независимы и равномерны на $ [0;1] $. Найдите равновесие Нэша.

\item Сколько функций потребуется (и от каких переменных каждая из них зависит) чтобы описать стратегию в симметричном равновесии Нэша в кнопочном аукционе с 4-мя игроками? С 5-ю? С $ n $ игроками?

\item Сколько функций потребуется (и от каких переменных каждая из них зависит) чтобы описать стратегию в несимметричном равновесии Нэша в кнопочном аукционе с 4-мя игроками? С 5-ю? С $ n $ игроками?

\item Рассмотрим кнопочный аукцион с тремя игроками. Каждый игрок знает своё $ X_{i} $, а ценности определяются по правилу:
\begin{equation}
V_{1}=X_{1}+X_{2}\cdot X_{3} \\
V_{2}=X_{2}+X_{1}\cdot X_{3} \\
V_{3}=X_{3}+X_{1}\cdot X_{2} \\
\end{equation}
Сигналы $ X_{i} $ независимы и имеют равномерное распределение на $ [0;1] $. Найдите равновесие Нэша и средний доход продавца.

\item Найдите $ (1,2,3)\wedge (5,0,2) $ и $ (1,2,3)\vee (5,0,2) $

\item Являеются ли супермодулярными функции:
\begin{enumerate}
\item $ f(x_{1},\ldots,x_{n})=x_{1}+\ldots+x_{n} $
\item $ f(x_{1},\ldots,x_{n})=x_{1}\cdot \ldots \cdot x_{n} $ при положительных $ x_{i} $
\item $ f(a,b,c)=a\wedge b\wedge c $
\item $ f(a,b,c)=a\vee b\vee c $
\end{enumerate}

\item Существует ли функция одной переменной не являющаяся супермодулярной?

\item Верно ли, что $ X_{i} $ аффилированы если:

\begin{enumerate}
\item $ X_{1} $, \ldots, $ X_{n} $ — независимы
\item $ f(x,y)=x+y $ при $ x,y\in [0;1] $
\item $ f(x,y)=\frac{1+4xy}{2} $ при $ x,y\in [0;1] $
\end{enumerate}

\item Пара случайных величин $ X $ и $ Y $ имеет совместное нормальное распределение с корреляцией $ \rho $. При каких $ \rho $ они будут аффилированы?

\end{enumerate}


\section{Решения задач}
\begin{enumerate}

\item Да. $ \E(1_{A}|B)=\E(1_{A}1_{B})/\P(B)=\P(A\cap B)/\P(B)=\P(A|B) $

\item Да. $ \E(1_{A}|B)=\P(A|B)=\P(A)=\E(1_{A})$

\item
\begin{enumerate}
\item $ p_{Y_{n-1}}(t)=(n-1)f(t)(1-F(t))^{n-2} $
\item $ p_{Y_{6}}(t)=(n-1)f(t)C^{5}_{n-1}(1-F(t))^{5}(F(t))^{n-7} $
\item $ p_{Y_{1},Y_{n-1}}(a,b)=(n-1)(n-2)f(a)f(b)(F(a)-F(b))^{n-3} $
\item $ p_{Y_{3},Y_{6}}(a,b)=(n-1)(n-2)C_{n-3}^{2}C_{n-5}^{2}f(a)f(b) (F(b))^{n-7}(1-F(a))^{2}(F(a)-F(b))^{2} $
\item $ p_{Y_{n-2},Y_{n-1}}(a,b)=(n-1)(n-2)f(a)f(b)(1-F(a))^{n-3} $
\item $ p_{X_{1},Y_{1}}(a,b)=f(a)(n-1)f(b)(F(b))^{n-2} $
\end{enumerate}


\item

\begin{enumerate}
\item Стратегии $ b(x) $ мы уже искали в предыдущей домашней работе, $ b(x)=\frac{2}{3}x $
\item Функцию $ pay_{1}(x) $ проще всего найти через теорему \ref{probabilistic_interpretation}:
\begin{equation}
pay_{1}(x)=\int_{0}^{x}t(n-1)f(t)(F(t))^{n-2}dt=\frac{2}{3}x^{3}
\end{equation}
\item Функция $ \widehat{q_{1}}(b) $ — это вероятность того, что первый выиграет, если поставит $ b $. Значит:
\begin{multline}
\widehat{q_{1}}(b)=\P(Bid_{2}<b)=\P\left(\frac{2}{3}X_{2}<b\right)=\\
=\P\left( X_{2}<\frac{3}{2}b\right)=F\left(\frac{3}{2}b\right)=\left(\frac{3}{2}b\right)^{2}
\end{multline}
\item Ищем $ \widehat{pay_{1}}(b) $:
\begin{equation}
\widehat{pay_{1}}(b)=\E(b\cdot 1_{Bid_{2}<b})=b\cdot \P(Bid_{2}<b)=b\left(\frac{3}{2}b\right)^{2}
\end{equation}
\item Функция $ q_{1}(x) $ — это вероятность того, что первый выиграет при ценности $ x $, значит $ q_{1}(x)=x^{2} $
\item Cлучайная величина $ Pay_{1} $ — интересный зверь. Она не является ни дискретной, ни непрерывной. Заметим, что в равновесии с вероятностью $ 0.5 $ первый проигрывает аукцион и не платит ничего. Значит у функции распределения скачок высотой 0.5 в точке $ t=0 $! А в остальных точках — она непрерывна. Замечаем, что область значений $ Pay_{1} $ — отрезок $ [0;2/3] $\ldots

Более детально рассматриваем точки $ t\in (0;2/3) $
\begin{multline}
\P(Pay_{1}\leq t)=\P(Pay_{1}\leq t \cap X_{1}>X_{2})+\P(Pay_{1}\leq t \cap X_{1}<X_{2})=\\
=\P(Pay_{1}\leq t \cap X_{1}>X_{2})+\P(X_{1}<X_{2})=\\
=\P(Pay_{1}\leq t \cap X_{1}>X_{2})+0.5
\end{multline}

Считаем первую вероятность отдельно:
\begin{multline}
\P(Pay_{1}\leq t \cap X_{1}>X_{2})=\P(\frac{2}{3}X_{1}\leq t \cap X_{1}>X_{2})=\\
\P(X_{1}\leq 1.5t \cap X_{1}>X_{2})=\int_{0}^{1.5t}\int_{0}^{x_{1}} 2x_{1}2x_{2}dx_{2}dx_{1}=\frac{81}{32}t^{4}
\end{multline}

Итого, получаем функцию распределения:
\begin{equation}
F_{Pay_{1}}(t)=
\begin{cases}
0, t<0 \\
0.5+\frac{81}{32}t^{4}, t\in [0;2/3] \\
1, t>2/3 \\
\end{cases}
\end{equation}

Функции плотности у $ Pay_{1} $ нет! Функция распределения разрывна. Тем не менее выпишем производную:
\begin{equation}
f_{Pay_{1}}(t)=0.5d(0)+\frac{81}{8}t^{3}
\end{equation}
В начале формулы идет некое мифическое $ 0.5d(0) $ — это просто условная запись. Она нужна чтобы помнить, что у  $ F(t) $ в точке $ t=0 $ был скачок высотой $ 0.5 $.


\item $ \E(Pay_{1}) $, , $ Var(Pay_{1}) $

Считаем мат. ожидание. Это легко. От дискретной части появляется $ 0\cdot 0.5 $. От непрерывной — интеграл.
\begin{equation}
\E(Pay_{1})=0\cdot 0.5+\int_{0}^{2/3}t\cdot \frac{81}{8}t^{3} dt=\ldots=4/15
\end{equation}

Для дисперсии нам нужно:
\begin{equation}
\E(Pay_{1}^{2})=0^{2}\cdot 0.5+\int_{0}^{2/3}t^{2}\cdot \frac{81}{8}t^{3} dt=\ldots=4/27
\end{equation}

Дисперсия равна:
\begin{equation}
Var(Pay_{1})=\E(Pay_{1}^{2})-\E(Pay_{1})^{2}=4/27-(4/15)^{2}=52/675
\end{equation}


Два игрока никогда не платят одновременно, поэтому $ Pay_{1}\cdot Pay_{2} $ тождественно равно нулю. Отсюда:
\begin{multline}
Cov(Pay_{1},Pay_{2})=\E(Pay_{1}Pay_{2})-\E(Pay_{1})\E(Pay_{2})=\\
=0-(4/15)^{2}
\end{multline}
\item Cлучайная величина $ R$ — это не что иное, как максимум из ставок:
\begin{equation}
R=\max\left\{ \frac{2}{3}X_{1},\frac{2}{3}X_{2} \right\}=\frac{2}{3}\max\{X_{1},X_{2}\}
\end{equation}
Функция распределения:
\begin{multline}
F_{R}(t)=\P(R\leq t)=\P\left(\frac{2}{3}\max\{X_{1},X_{2}\}\leq t\right)=\\
=\P\left(X_{1}<\frac{3}{2}t \cap X_{2}<\frac{3}{2}t\right)=F\left(\frac{3}{2}t\right)^{2}=\left(\frac{3}{2}t\right)^{4}
\end{multline}
Это непрерывная случайная величина, и у неё есть функция плотности:
\begin{equation}
f_{R}(t)=4\left(\frac{3}{2}t\right)^{3}\frac{3}{2}
\end{equation}

\item Для нахождения $ \E(R) $, $ Var(R) $, $ Cov(R,Pay_{1}) $ вспоминаем, что $ R=Pay_{1}+Pay_{2} $.
\begin{equation}
\E(R)=\E(Pay_{1}+Pay_{2})=2\E(Pay_{1})
\end{equation}
\begin{equation}
Var(R)=Var(Pay_{1})+Var(Pay_{2})+2Cov(Pay_{1},Pay_{2})
\end{equation}

\begin{multline}
Cov(R,Pay_{1})=Cov(Pay_{1}+Pay_{2},Pay_{1})=\\
=Var(Pay_{1})+Cov(Pay_{2},Pay_{1})
\end{multline}
\item В нашем случае $ L_{1}=\frac{2}{3}X_{1}^{3} $.

Функция распределения:
\begin{multline}
F_{L_{1}}(t)=\P(L_{1}<t)=\P\left(\frac{2}{3}X_{1}^{3}<t\right)=\\
=\P(X_{1}<(1.5t)^{1/3})=F((1.5t)^{1/3})=(1.5t)^{2/3}
\end{multline}

В данном случае речь идет о непрерывной случайной величине, у неё есть функция плотности:
\begin{equation}
f_{L_{1}}(t)=1.5\frac{2}{3}(1.5t)^{-1/3}
\end{equation}

Мат. ожидание:
\begin{equation}
\E(L_{1})=\int_{0}^{1}\frac{2}{3}t^{3}2tdt=\frac{4}{15}
\end{equation}

Дисперсия:
\begin{equation}
Var(L_{1})=\E(L_{1}^{2})-\E(L_{1})^{2}=\frac{1}{9}-\left(\frac{4}{15}\right)^{2}=\frac{1}{25}
\end{equation}

Поскольку $ X_{1} $ и $ X_{2} $ независимы, $ pay(X_{1}) $ и $ pay(X_{2}) $ тоже независимы, и $ Cov(L_{1},L_{2})=0 $
\end{enumerate}

Заметим, что $ \E(Pay_{1})=\E(pay_{1}(X_{1})) $, но $Var(Pay_{1})\neq Var(pay(X_{1}))$! Ковариации также различаются\ldots то есть $ Pay_{1} $ и $ pay_{1}(X_{1}) $ — это разные случайные величины!


\item От рассмотренного примера отличается только коэффициентом $ 0.5 $ перед $ \E(X_{2}1_{W_{1}}|\ldots) $. Значит финальное дифф. уравнение имеет вид:
\begin{equation}
-b'(x)F(x)+(x-b(x))f(x)+0.5\cdot xf(x)=0
\end{equation}
При равномерном распределении:
\begin{equation}
-b'(x)x+(x-b(x))+0.5\cdot x=0
\end{equation}

Подбираем сразу линейное решение, получаем $ b(x)=0.75x $

\item Для четырех. $ b^{4}(x) $, $ b^{3}(x,p_{4}) $, $ b^{2}(x,p_{3},p_{4}) $ (названия переменных объясняются соглашением, что $ p_{i} $ — это моменты выхода игроков в порядке убывания)

Для пяти. $ b^{5}(x) $, $ b^{4}(x,p_{5}) $, $ b^{3}(x,p_{4},p_{5}) $, $ b^{2}(x,p_{3},p_{4},p_{5}) $

Для $ n $ игроков. Нужна $ (n-1) $ функция, $ b^{n}(x) $, $ b^{n-1}(x,p_{n}) $, \ldots, $ b^{2}(x,p_{3},\ldots,p_{n}) $

\item Начнём все-таки с трех. Заметим, что у каждого игрока свои стратегии! Например, рассмотрим первого:
\begin{enumerate}
\item $b_{1}^{3}(x)$ — до какого момента давить на кнопку, если в игре 3 игрока.
\item $b_{1}^{2:-2}(x,p) $ — до какого момента давить на кнопку, если в игре 2 игрока и первым вышел второй.
\item $b_{1}^{2:-3}(x,p) $ — до какого момента давить на кнопку, если в игре 2 игрока и первым вышел третий.
\end{enumerate}

Для четырех (функции описывают, до какого момента давить на кнопку)
\begin{enumerate}
\item $b_{1}^{4}(x)$ — если все четверо в игре
\item $b_{1}^{3:2}(x,p) $ — если вышел только второй на цене $ p $
\item $b_{1}^{3:3}(x,p) $ — если вышел только третий на цене $ p $
\item $b_{1}^{3:4}(x,p) $ — если вышел только четвертый на цене $ p $

\item $b_{1}^{2:2,3}(x,p_{3},p_{4}) $ — если сначала вышел второй, на цене $ p_{4} $, а затем — третий, на цене $ p_{3} $
\item $b_{1}^{2:2,4}(x,p_{3},p_{4}) $
\item $b_{1}^{2:3,4}(x,p_{3},p_{4}) $
\item $b_{1}^{2:3,2}(x,p_{3},p_{4}) $
\item $b_{1}^{2:4,2}(x,p_{3},p_{4}) $
\item $b_{1}^{2:4,3}(x,p_{3},p_{4}) $
\end{enumerate}
Итого: $1+3+6=10$ функций.

Для пяти: $1+4+12+24=41$ функция.

Для произвольного $ n $: $ C_{n-1}^{1}1!+C_{n-1}^{2}2!+C_{n-1}^{3}3!+C_{n-1}^{4}4!+\ldots C_{n-1}^{n-2}(n-2)! $


\item Сначала равновесие. Первая функция — $ b^{3}(x)=x+x^{2} $. Со второй чуть посложнее\ldots
Если игрок вышел на цене $ p $, то $ x+x^{2}=p $. Решаем квадратное уравнение, берём корень из $ [0;1] $:
\begin{equation}
x=\frac{\sqrt{1+4p}-1}{2}
\end{equation}
И получаем:
\begin{equation}
b^{2}(x,p)=x+x\frac{\sqrt{1+4p}-1}{2}=x\frac{\sqrt{1+4p}+1}{2}
\end{equation}

Пусть $ Z_{1} $ — наибольшая из всех $ X_{i} $, $ Z_{2} $ — вторая по величине, а $ Z_{3} $ — самая маленькая. Тогда побеждает игрок с сигналом $ Z_{1} $. При этом он платит $ b^{2}(Z_{2},b(Z_{3}))=Z_{2}+Z_{2}Z_{3} $.

Значит:
\begin{equation}
\E(R)=\E(Z_{2}+Z_{2}Z_{3})=\E(Z_{2})+\E(Z_{2}Z_{3})
\end{equation}

Для их расчета привлекаем банду о-малых.


Ищем функцию плотности $ Z_{2} $. Три способа выбрать $ Z_{2} $, одна из двух $ X_{i} $ должна быть больше избранной (сомножитель $ (1-t) $), другая — меньше (сомножитель $ t $):
\begin{equation}
p_{Z_{2}}(t)=3\cdot 2\cdot t(1-t)
\end{equation}

Совместная функция плотности положительна только при $ a>b $ и равна:
\begin{equation}
p_{Z_{2},Z_{3}}(a,b)=6(1-a)
\end{equation}

После интегрирования первой получаем очевидный результат, что мат. ожидание медианы трёх равномерных случайных величин равно половине:
\begin{equation}
\E(Z_{2})=\ldots=1/2
\end{equation}

И после интегрирования второй:
\begin{equation}
\E(Z_{2}Z_{3})=\ldots=\frac{3}{20}
\end{equation}

Итого: $ \E(R)=\frac{13}{20} $

\item Ответ: $ (1,2,3)\wedge (5,0,2)=(1,0,2) $ и $ (1,2,3)\vee (5,0,2)=(5,2,3) $

\item К первым двум можно применять теорему \ref{supermod_crit}.
\begin{enumerate}
\item $ \frac{\partial^{2} f}{\partial x_{i}\partial x_{j}}=0 $, супермодулярна
\item $ \frac{\partial^{2} f}{\partial x_{i}\partial x_{j}}>0 $, супермодулярна
\item да, супермодулярна. Достаточно, например, рассмотреть два случая:
\begin{enumerate}
\item $ x_{1}>y_{1} $, $ x_{2}>y_{2} $, $x_{3}>y_{3}$. Здесь $ f(\vec{x}\wedge \vec{y})=f(\vec{y}) $ и $ f(\vec{x}\vee \vec{y})=f(\vec{x}) $ и неравенство выполнено.
\item $ x_{1}>y_{1} $, $ x_{2}>y_{2} $, $x_{3}<y_{3}$ Здесь нужно немного помучиться с рассмотрением разных порядков\ldots
\end{enumerate}
\item нет, не супермодулярна. Например: $ \vec{x}=(3,3,1) $ и $ \vec{y}=(2,2,4) $. Тогда: $ \vec{x}\wedge \vec{y}=(2,2,1) $ и $ \vec{x}\vee\vec{y}=(3,3,4) $. Находим, что: $ f(\vec{x}\wedge \vec{y})=2 $, $ f(\vec{x}\vee \vec{y})=4 $, $ f(\vec{x})=3 $, $ f(\vec{y})=4 $

\end{enumerate}

\item Нет. Если $ x=y $, то $ x\wedge y=x\vee y=x=y $  и $ f(x\wedge y)+f(x\vee y)= f(x)+f(y) $. Если $ x\neq y $, то одно из этих чисел больше, а другое — меньше. А это значит, что $ x=x\wedge y $ и $ y=x \vee y $. Или наоборот, $ x=x\vee y $ и $ y=x\wedge y $.

\item
\begin{enumerate}
\item Независимые случайные величины аффилированы: $ \ln(f(x,y,z)=\ln(f(x))+\ln(f(y))+\ln(f(z)) $
\item Если $ f(x,y)=x+y $, то
\begin{equation}
\frac{\partial^{2} \ln(f)}{\partial x \partial y}=-\frac{1}{(x+y)^{2}}<0
\end{equation}
Значит, величины не аффилированны. Кстати, в этом примере ради интереса можно посчитать ковариацию: $ \E(X)=\E(Y)=7/12 $, $ \E(XY)=1/3 $, $ Cov(X,Y)=-1/144 $.
\item Если  $ f(x,y)=\frac{1+4xy}{2} $, то:
\begin{equation}
\frac{\partial^{2} \ln(f)}{\partial x \partial y}=\ldots=\frac{4}{(1+4xy)^{2}}\geq 0
\end{equation}
Случайные величины аффилированны

\end{enumerate}

\item Для начала заметим, что ответ на вопрос не зависит от мат. ожидания и дисперсии. Это связано с тем, что в силу теоремы \ref{supermod_crit} нужна только неотрицательность смешанной второй производной логарифма плотности при любых $ x $ и $ y $. Выбор мат. ожидания — это перенос графика плотности вдоль осей, выбор дисперсии — это растяжение графика плотности вдоль осей. Если была где-то точка с отрицательной второй смешанной производной, то она просто поменяет свои координаты.

Рассматриваем случай с нулевым мат. ожиданием и единичной дисперсией. Ковариационная матрица имеет вид:
\begin{equation}
S=\left(\begin{array}{cc}
1 & \rho \\
\rho & 1 \\
\end{array}
\right)
\end{equation}

Функция плотности равна:

\begin{equation}
p(x,y)=\frac{1}{2\pi \det(S)} \exp\left(-\frac{1}{2}(x\; y)S^{-1}(x\; y)^{t}\right)
\end{equation}
После логарифмирования получаем (нам интересна только смешанная вторая производная, поэтому следим только за коэффициентом при $ xy $):

\begin{equation}
\ln (p(x,y))=\ldots+\frac{\rho}{1-\rho^{2}}xy
\end{equation}
Значит двумерное совместное нормальное распределение аффилированно если корреляция неотрицательная.


\end{enumerate}


\section{Контрольная 2}

\begin{enumerate}

\item Техническая задача.
\begin{enumerate}
\item Известно, что функции $ f(\vec{x}) $ и $ g(\vec{x}) $ супермодулярные\index{супермодулярная функция}, а константы $a$ и $b$ положительные. Верно ли, что функция $ af(\vec{x})+bg(\vec{x}) $ супермодулярная?

\item Пусть $ X_{1} $,\ldots, $ X_{n} $ независимы и имеют функцию плотности $ f(t)=3t^{2} $ на $ [0;1] $. Случайные $ Y_{1} $, \ldots, $ Y_{n-1} $ — это есть упорядоченные по убыванию случайные величины $ X_{2} $, \ldots, $ X_{n} $. С помощью о-малых или без неё найдите совместную функцию плотности $ f_{Y_{5},Y_{10}}(a,b) $.\index{о-малые}
\end{enumerate}

Следующие две задачи очень похожи, разница в них только в типе аукциона\ldots

\item На аукционе первой цены продаётся участок\index{задача!о продаже участка}. Потенциальных покупателей двое. Ценность участка для каждого игрока определяется его площадью. Первый покупатель знает ширину участка $ X_{1} $, а второй — длину $X_{2}$. Совместная фукнция плотности $ X_{1} $ и $ X_{2} $ имеет вид $ f(x_{1},x_{2})=\frac{7}{8}+\frac{1}{2}x_{1}x_{2} $. Найдите дифференциальное уравнение, которому подчиняется равновесная стратегия игрока.

\item На аукционе второй цены продаётся участок\index{задача!о продаже участка}. Потенциальных покупателей двое. Ценность участка для каждого игрока определяется его площадью. Первый покупатель знает ширину участка $ X_{1} $, а второй — длину $X_{2}$. Совместная фукнция плотности $ X_{1} $ и $ X_{2} $ имеет вид $ f(x_{1},x_{2})=\frac{7}{8}+\frac{1}{2}x_{1}x_{2} $. Найдите равновесие Нэша.


\item Найдите равновесие Нэша в случае кнопочного аукциона\index{аукцион!кнопочный}. Сигналы $ X_{i} $ игроков имеют совместную функцию плотности $ f(x_{1},x_{2},x_{3})=7/8+x_{1}x_{2}x_{3} $ при $ x_{1},x_{2},x_{3}\in[0;1] $. Ценности определяеются по формулам:
\begin{equation}
\begin{array}{c}
V_{1}=X_{1}(X_{2}+X_{3}); \\
V_{2}=X_{2}(X_{1}+X_{3}); \\
V_{3}=X_{3}(X_{1}+X_{2}).
\end{array}
\end{equation}


\item Ценности игроков одинаково распределены, независимы, распределение ценностей дискретно\index{ценности!дискретные}: $ X_{i}$ равновероятно принимает натуральное значение от 1 до 100 включительно. Игроки одновременно делают ставки. Значения всех ценностей общеизвестны всем игрокам ещё до ставок! Разрешаются только целые неотрицательные ставки. Товар достаётся игроку, сделавшему наивысшую ставку. Если таких игроков несколько, то победитель выбирается из них равновероятно. Победитель платит сделанную им ставку. Найдите хотя бы одно равновесие Нэша в чистых стратегиях, в котором игроки не используют нестрого доминируемых стратегий.


\end{enumerate}


\section{Решение контрольной 2}

\begin{enumerate}

\item
\begin{enumerate}
\item Да, является супермодулярной. Проверяем два свойства:
\begin{enumerate}
\item $ af(\vec{x}) $ — ок.
\item $ f(\vec{x})+g(\vec{x}) $ — ок.
\end{enumerate}

\item Сразу ответ: $ f(a,b)=(n-1)(n-2)C_{n-3}^{4}C_{n-7}^{4}3a^{2}3b^{2}(a^{3}-b^{3})^{4}(b^{3})^{n-11}(1-a^{3})^{4} $.
\end{enumerate}


\item Сразу начнем с чудо-замены, $ b_{1}=b(a) $. Сначала упростим событие $ W_{1} $:\index{чудо-замена}
\begin{equation}
W_{1}=\{Bid_{2}<b(a)\}=\{b(X_{2})<b(a)\}=\{X_{2}<a\}.
\end{equation}

А теперь прибыль:
\begin{multline}
\pi(x,b(a))=\E(X_{1}X_{2}1_{W_{1}}|X_{1}=x)-b(a)\E(1_{W_{1}}|X_{1}=x)=\\
=x\E(X_{2}1_{X_{2}<a}|X_{1}=x)-b(a)\E(1_{X_{2}<a}|X_{1}=x)=\\
=x\int_{0}^{a}x_{2}\frac{f(x,x_{2})}{f(x)} \, dx_{2}-b(a)\int_{0}^{a}\frac{f(x,x_{2})}{f(x)} \, dx_{2}.
\end{multline}


Сокращаем на $ f(x) $ и берём производную по $ a $:
\begin{equation}
\frac{\partial \pi}{\partial a}=xaf(x,a)-b'(a)\int_{0}^{a}f(x,x_{2}) \, dx_{2}-b(a)f(x,a)=0.
\end{equation}


Мы хотим, чтобы оптимальной стратегий первого была $ b_{1}=b(x) $, то есть чтобы $ a=x $:
\begin{equation}
\frac{\partial \pi}{\partial a}=x^{2}f(x,x)-b'(x)\int_{0}^{x}f(x,x_{2})dx_{2}-b(x)f(x,x)=0.
\end{equation}

Остаётся подставить:
\begin{equation}
\begin{array}{c}
f(x,x)=\frac{7}{8}+\frac{1}{2}x^{2}; \\
\int_{0}^{x}f(x,x_{2})dx_{2}=\frac{7}{8}x+\frac{1}{4}x^{3}.
\end{array}
\end{equation}


\item  Никакой разницы с кнопочным аукционом в данном случае нет. Игроков-то всего два! Значит, равновесие Нэша имеет вид $ b(x)=x^{2} $. \index{аукцион!кнопочный}

Доказательство.

Если второй игрок использует такую стратегию и первый выигрывает аукцион, то его прибыль равна:
\begin{equation}
X_{1}X_{2}-X_{2}^{2}=(X_{1}-X_{2})X_{2}.
\end{equation}
Мы видим, что прибыль положительна, только если $ X_{1}>X_{2} $. Использование первым игроком функции $ b(x)=x^{2} $ будет обеспечивать его выигрыш только в ситуации $ X_{1}>X_{2} $, значит, это и есть равновесие Нэша.


\item Стратегия описывается двумя функциями: $ b^{3}(x)=2x^{2} $. Если при использовании такой стратегии игрок вышел на цене $ p $, значит, его $ x=\sqrt{p/2} $. Получаем $ b^{2}(x,p)=x^{2}+x\sqrt{p/2}$. Доказательство аналогично лекции:

Если остальные игроки используют эти стратегии и первый выигрывает аукцион, то его выигрыш равен
\begin{multline}
X_{1}(X_{2}+X_{3})-b^{2}(Y_{1},b^{3}(Y_{2}))=\\
=X_{1}(Y_{1}+Y_{2})-(Y_{1}^{2}+Y_{1}Y_{2})=(X_{1}-Y_{1})(Y_{1}+Y_{2}).
\end{multline}
Мы видим, что выигрыш положительный, только если $ X_{1}>Y_{1} $. Использование первым игроком правил $ b^{3}() $ и $ b^{2}() $ приводит к выигрышу, только если $ X_{1}>Y_{1} $, значит, это и есть равновесие.


\item  Пример равновесия Нэша. Обозначим максимальную ценность $ v_{max} $, а вторую по величине ценность — $v_{sec}$. Возможно, эти две ценности совпадают. Те игроки, чья ценность $ v_{i}<v_{max} $, делают ставку $ b_{i}=v_{i}-1 $. Если лидер один, то он делает ставку $ b=v_{sec} $, иначе каждый лидер делает ставку $ b=v_{max}-1 $.


\end{enumerate}


\chapter{Сравнение аукционов в общем случае}

%Если вы будете читать Сергея Николенко, то осторожно, там есть несколько опечаток со знаками неравенств.


Сравним доходность трёх аукционов (первой, второй цены и кнопочного) для продавца в общем случае. Предположений у нас будет всего два: аффилированность сигналов и симметричность игроков.

\section{Про симметричность}
Для наглядности три примера впереди определения:

\begin{myex} Совместная функция плотности сигналов имеет вид:
\begin{equation}
f(x_{1},x_{2},x_{3})=\frac{7}{8}+x_{1}x_{2}x_{3}
\end{equation}
Ценности определяются по формулам:
\begin{equation}
\begin{array}{c}
V_{1}=X_{1}+\sqrt{X_{2}X_{3}+1}-1 \\
V_{2}=X_{2}+\sqrt{X_{1}X_{3}+1}-1 \\
V_{3}=X_{3}+\sqrt{X_{1}X_{2}+1}-1
\end{array}
\end{equation}
Все симметрично. Ценность товара для меня может по-особому зависеть от моего сигнала, но должна одинаково зависеть от сигнала других игроков. С моей точки зрения другие игроки одинаковые, и то, что знают они, чего не знаю я, должно одинаково воздействовать на ценность товара для меня.
\end{myex}


\begin{myex} Совместная функция плотности сигналов имеет вид
\begin{equation}
f(x_{1},x_{2},x_{3})=x_{1}\left(\frac{7}{8}+\frac{1}{2}x_{2}x_{3}\right)
\end{equation}
Ценности определяются по формулам:
\begin{equation}
\begin{array}{c}
V_{1}=X_{1}+\sqrt{X_{2}X_{3}+1}-1 \\
V_{2}=X_{2}+\sqrt{X_{1}X_{3}+1}-1 \\
V_{3}=X_{3}+\sqrt{X_{1}X_{2}+1}-1
\end{array}
\end{equation}
Несимметрична функция плотности.
\end{myex}

\begin{myex} Совместная функция плотности сигналов имеет вид
\begin{equation}
f(x_{1},x_{2},x_{3})=\frac{7}{8}+x_{1}x_{2}x_{3}
\end{equation}
Ценности определяются по формулам:
\begin{equation}
\begin{array}{c}
V_{1}=X_{1}+\sqrt{X_{2}X_{3}+1}-1 \\
V_{2}=X_{2}+X_{3} \\
V_{3}=X_{3}\cdot X_{1} \\
\end{array}
\end{equation}
Несимметричны ценности.
\end{myex}

Для формальности:
\begin{mydef}
Функция $ f(a,b,c,d,e) $ симметрична относительно аргументов $ a,b,c $, если её значение не изменится при перестановке $ a,b,c $ в другом порядке.
\end{mydef}

\begin{myex} Функция симметричная относительно $ x $ и $ y $: $ f(x,y,z)=xy+z $
\end{myex}

\begin{myex} Функция симметричная относительно всех аргументов:
$$ f(w,x,y,z)=xyz+wxy+wxz+wyz $$
\end{myex}


\begin{mydef} Игроков будем называть \indef{симметричными}\index{Симметричные игроки}, если:
\begin{enumerate}
\item Совместная функция плотности $ f(x_{1},x_{2},\ldots,x_{n}) $ симметрична по всем аргументам
\item Ценность $V_{i}$ определяется по формуле:
\begin{equation}
V_{i}=u(X_{i},X_{-i})
\end{equation}
где: $X_{-i}  $ — это вектор $ (X_{1},X_{2},\ldots,X_{n}) $ в котором отсутствует $ X_{i} $, а  функция $ u(t,t_{1},t_{2},\ldots,t_{n-1}) $ симметрична по переменным $ t_{1} $,\ldots, $ t_{n-1} $, то есть по всем кроме $ t $

\end{enumerate}
\end{mydef}

\subsubsection*{Предпосылки на функцию полезности}\index{Предпосылки на функцию полезности}
Кроме того, функция полезности $ u() $ удовлетворяет условиям:
\begin{enumerate}
\item Не убывает по всем аргументам. Это означает, что сигнал $ X_{i} $ положительно связан с ценностью.
\item $ u(0,0,\ldots,0)=0 $. Это просто условие нормировки. Если все получили сигнал: «Товар просто никакой», значит, товар действительно никакой.
\end{enumerate}






\section{Еще об аффилированности}


Сперва кое-что о вероятностях\ldots

Если у нас есть случайная величина $ Z $, то мы можем построить функцию $z(y)= \E(Z|Y=y) $. Рассмотрим эту функцию в случайной точке $ Y $:
\begin{multline}
\E(z(Y))=\int_{0}^{1}z(y)f_{Y}(y)dy=\int_{0}^{1}\int_{0}^{1}z\cdot \frac{f(y,z)}{f_{Y}(y)}dz f_{Y}(y)dy=\\
=\int_{0}^{1}\int_{0}^{1}z\cdot f(y,z)dydz=\E(Z)
\end{multline}

Значит, это нам это дает способ расчета $ \E(Z) $:
\begin{equation}
\label{conditional_way}
\E(Z)=\int_{0}^{1}\E(Z|Y=y)f_{Y}(y)dx
\end{equation}

Честно говоря, этот способ мы уже использовали. Он очень мощный.

Лирическое отступление для любопытных. Если глубоко копать, то можно понять, что это не что иное, как теорема о трёх перпендикулярах из 11-го класса средней школы. Математическое ожидание случайной величины — это её проекция на множество действительных чисел. Квадратом расстояния между двумя случайными величинами при этом служит $ \E((X-Y)^{2}) $. Например, теорема Пифагора формулируется так: $ \E(X^{2})=\E(m^{2})+\E((X-m)^{2}) $. Три перпендикуляра: наклонная — это $ Y $; плоскость — это множество случайных величин, записывающихся как функция от $ X $; проекция на плоскость — это $ \E(Y|X=x) $ взятая в случайной точке $ X $; множество констант — это прямая в нашей плоскости; $ \E(Y) $ — это проекция на прямую\ldots


Аналогично, довесив условие $ X=x $ слева и справа, можно получить, что:
\begin{equation}
\label{iterated_e}
\E(Z|X=x)=\int_{0}^{1}\E(Z|Y=y\cap X=x)f_{Y|X}(y|x)dy
\end{equation}

Это не очевидно. Те, кому интересна теория вероятностей могут вывести это утверждение, остальные могут поверить.

Теперь вернемся к аффилированности:

\begin{myth}
\label{aff_order}
Если $ X_{1} $, \ldots, $ X_{n} $ аффилированы, то и $ X_{1} $, $ Y_{1} $, $ Y_{2} $, \ldots, $ Y_{n-1} $ аффилированы.
\end{myth}

\begin{proof}
Великие о-малые говорят нам, что совместная функция плотности вектора $ X_{1} $, $ Y_{1} $, $ Y_{2} $, \ldots, $ Y_{n-1} $ на участке $ y_{1}>y_{2}>\ldots>y_{n-1} $ равна:
\begin{equation}
f_{X_{1},Y_{1},\ldots,Y_{n-1}}(x_{1},y_{1},y_{2},\ldots,y_{n-1})=(n-1)!f(x_{1},y_{1},\ldots,y_{n-1})
\end{equation}
Нам нужно проверить супермодулярность логарифма:
\begin{equation}
\ln(f_{X_{1},Y_{1},\ldots,Y_{n-1}}(x_{1},y_{1},y_{2},\ldots,y_{n-1}))=\ln((n-1)!)+\ln(f(x_{1},y_{1},\ldots,y_{n-1}))
\end{equation}

Вторые смешанные производные от левой части неотрицательны в силу того, что неотрицательны вторые смешанные производные от правой части.

\end{proof}



Теоремы которые мы далее докажем будут верны для любых аффилированных случайных величин. Но мы будем иметь ввиду $ X_{1} $ и $ Y_{1} $, поэтому и будем использовать соответствующие обозначения.

\begin{myth}
\label{aff_delete}
Если из набора аффилированных величин некоторые удалить, то оставшиеся будут аффилированы
\end{myth}

\begin{proof}
Пропущено. Если будут желающие, то допишу.

%Мы докажем, что если $ Z_{1} $, $ Z_{2} $ и $ Z_{3} $ аффилированы, то $ Z_{1} $, $Z_{2} $ — аффилированы. Общий случай ничем не отличается от этого частного кроме как наличием многоточий. Удалить несколько величин — значит, несколько раз удалить одну.
\end{proof}


Из теорем \ref{aff_delete} и \ref{aff_order} следует, что $ X_{1} $ и $ Y_{1} $ аффилированы. Знание этих двух величин всегда позволяет определить, победил ли первый игрок, и сколько он платит (по крайней мере для трёх аукционов, которые мы сравниваем).

Нам надо изучать $ X_{1} $ и $ Y_{1} $, чтобы все время не писать индекс $ _{1} $ в доказательствах пока забудем про него.

Введем несколько обозначений для этой пары:
\begin{enumerate}
\item $ g(x,y) $ — их совместная функция плотности,
\item $ g(y|x)=\frac{g(x,y)}{f_{X}(x)} $ — условная функции плотности $ Y $ при заданном $ X $
\item $ G(y|x)=\P(Y\leq y|X=x)$ — условная функции распределения $ Y $ при заданном $ X $.

Конечно, верно соотношение:
\begin{equation}
G(y|x)=\P(Y\leq y|X=x)=\int_{0}^{y}g(t|x)dt
\end{equation}

\item $ R(y|x)=\frac{g(y|x)}{G(y|x)} $ — условная обратная функция риска\index{Условная обратная функция риска} $ Y $ при заданном $ X $.

Поясним её смысл. Это шансы того, что $ Y $ будет около $ y $, если известно, что $ Y\leq y $ и $ X=x $. Например, значение $ R(10,20)=30 $ можно проинтерпретировать так. Возьмем маленький $ \Delta y=0.01 $. Тогда $ \P(Y\in [9.99;10]|Y\leq 10, X=20)\approx 30\cdot 0.01=0.3 $.
\end{enumerate}

При расчете $ R(y|x) $ можно не считать $ f_{X}(x) $, так как оно сокращается:
\begin{equation}
R(y|x)=\frac{g(y|x)}{G(y|x)}=\frac{g(x,y)}{\int_{0}^{y}g(x,t)dt}
\end{equation}

Нам потребуется такой технический результат:
\begin{myth}
Если случайные величины $ X $ и $ Y $ аффилированы, и $ g(x,y) $ — их совместная функция плотности, то\footnote{Тут обычно вводят кучу определений (стохастическое доминирование, доминирование в терминах обратной доли риска и пр.), но мы не будем этого делать.}
\begin{enumerate}
\item Условная функция распределения $ G(y|x)$ — не возрастает по $ x $
\item Условная обратная функция риска  $ R(y|x)=\frac{g(y|x)}{G(y|x)} $ — не убывает по $ x $
\end{enumerate}
\end{myth}
\begin{proof}
Величины $ X $ и $ Y $ аффилированы.

Рассмотрим пару точек $ (x',y) $ и $ (x,y') $. Воспользуемся аффилированностью:
\begin{equation}
g((x',y)\wedge (x,y'))\cdot g((x',y)\vee (x,y'))\geq g(x',y)\cdot g(x,y')
\end{equation}


Пусть $ x'\geq x $ и $ y'\geq y $. Тогда:
\begin{equation}
g(x,y)\cdot g(x',y')\geq g(x',y)\cdot g(x,y')
\end{equation}

Поскольку $ g(x,y)=g(y|x)\cdot f_{X}(x) $ мы получаем:
\begin{equation}
g(y|x)\cdot f_{X}(x)\cdot g(y'|x')\cdot f_{X}(x')\geq g(y|x')\cdot f_{X}(x')\cdot g(y'|x)\cdot f_{X}(x)
\end{equation}

Убираем повторы
\begin{equation}
g(y|x)\cdot g(y'|x')\geq g(y|x')\cdot g(y'|x)
\end{equation}

Или:
\begin{equation}
\frac{g(y|x)}{g(y'|x)}\geq \frac{g(y|x')}{g(y'|x')}
\end{equation}

Интегрируем по $ y $ от $ 0 $ до $ y' $:
\begin{equation}
\frac{G(y'|x)}{g(y'|x)}\geq \frac{G(y'|x')}{g(y'|x')}
\end{equation}

Переворачиваем дробь:
\begin{equation}
\frac{g(y'|x)}{G(y'|x)}\leq \frac{g(y'|x')}{G(y'|x')}
\end{equation}

Используя условную обратную функцию риска:
\begin{equation}
R(y'|x)\leq R(y'|x')
\end{equation}

А у нас $ x\leq x' $. Это и означает, что $ R(\cdot|x) $ не убывает по $ x $.

Осталось доказать, что $ G(y'|x) $ не возрастает по $ x $. Мы докажем, что $ \ln (G(y'|x)) $ не возрастает по $ x $.

Заметим, что:
\begin{equation}
\frac{\partial \ln (G(y'|x))}{\partial y'}=\frac{g(y'|x)}{G(y'|x)}=R(y'|x)
\end{equation}

Или:
\begin{equation}
\ln(G(y'|x))=\int_{1}^{y'}R(t|x)dt
\end{equation}

Обратите внимание, что здесь несколько непривычные пределы интегрирования: не от $0$, а от $ 1 $. Связано это с тем, что интеграл должен обращаться в 0 не при $ y'=0 $, а при $ y'=1 $. Действительно, у нас регулярное распределение на $ [0;1] $, значит, $ G(1|x)=1 $ и $ \ln (G(1|x))=0 $. Заметьте, что знаки при этом совпадают: и слева отрицательное выражение, так как $ G\in (0;1) $ и справа, так как верхний предел меньше нижнего.

Давайте перепишем в привычном варианте, когда верхний предел интегрирования больше нижнего:
\begin{equation}
\ln(G(y'|x))=-\int_{y'}^{1}R(t|x)dt
\end{equation}

С ростом $ x $ подынтегрируемое выражение растет для любого $ t $, значит, растет результат интегрирования. то есть функция $\ln( G(y'|x) )$ не возрастает по $ x $.

\end{proof}









Из этих свойств следует теорема имеющая более наглядный смысл:
\begin{myth} \label{prop_affiliated}
Если $ X $ и $ Y $ аффилированы, то:
\begin{enumerate}
\item Функция $ \E(Y|X=x )$ не убывает по $ x $
\item Если $ \gamma() $ — возрастающая функция, то $ \E(\gamma(Y)|X=x) $ не убывает по $ x $
\item $ Cov(X,Y)\geq 0 $
\end{enumerate}
\end{myth}




\begin{proof}
По определению:
\begin{equation}
\E(Y|X=x)=\int_{0}^{1}yg(y|x)dy
\end{equation}

Мы можем проинтегрировать по частям ($ u=y $, $ v'=g(y|x) $) и получить:
\begin{equation}
\E(Y|X=x)=\left.yG(y|x)\right|_{y=0}^{y=1} - \int_{0}^{1}G(y|x)dy
\end{equation}

Поскольку мы работаем с регулярным на $ [0;1] $ распределением, то $ G(0|x)=0 $ и $ G(1|x)=1 $. ещё раз напомню, что выбор 0 и 1 в качестве границ распределения — это просто масштабирование для удобства и все наши доказательства проходят без изменений для случая регулярного распределения на отрезке $ [a;b] $.
\begin{equation}
\E(Y|X=x)=1 - \int_{0}^{1}G(y|x)dy
\end{equation}

Остаётся заметить, что с ростом $ x $ падает подынтегральное выражение и, следовательно, интеграл. Значит, $ \E(Y|x=x) $ возрастает.

Доказательство для произвольной $ \gamma(y) $ ничем не отличается:
\begin{equation}
\E(\gamma(Y)|X=x)=\int_{0}^{1}\gamma(y)g(y|x)dy
\end{equation}

Интегрируя по частям получаем:
\begin{equation}
\E(\gamma(Y)|X=x)=\left.\gamma(y)G(y|x)\right|_{y=0}^{y=1} - \int_{0}^{1}\gamma'(y)G(y|x)dy
\end{equation}

Или:
\begin{equation}
\E(\gamma(Y)|X=x)=1 - \int_{0}^{1}\gamma'(y)G(y|x)dy
\end{equation}
Снова замечаем, что с ростом $ x $ падает подынтегральное выражение. Вывод: $ \E(\gamma(Y)|X=x) $ возрастает по $x$.


Теперь про ковариацию. Пусть $ \E(X)=m $. Тогда:
\begin{multline}
Cov(Y,X)=Cov(Y,X-m)=\E(Y(X-m))-\E(Y)\E(X-m)=\\
\E(Y(X-m))-\E(Y)\cdot 0=\E(Y(X-m))
\end{multline}

Пользуемся условным способом расчета математического ожидания \ref{conditional_way}:
\begin{multline}
\E(Y\cdot (X-m))=\int_{0}^{1} \E(Y(X-m)|X=x)f_{X}(x)dx=\\
=\int_{0}^{1}\E(Y|X=x)(x-m)f_{X}(x)dx
\end{multline}

Теперь мы замечаем, что если бы не было сомножителя $ \E(Y|X=x) $ то интеграл бы равнялся нулю, т.к.
\begin{equation}
\int_{0}^{1}(x-m)f_{X}(x)dx=\E(X-m)=\E(X)-m=0
\end{equation}

А теперь глядим на функцию $ (x-m)f_{X}(x) $. Сначала она отрицательна, затем положительна, суммарная площадь равна 0:


\begin{tikzpicture}[domain=0:6.3]
    \draw[very thin,color=gray] (-0.1,-2.1) grid (7.1,3.1);
    \draw[->] (-0.1,0) -- (7.2,0) node[right] {$x$};
    \draw[->] (0,-0.1) -- (0,3.2) node[above] {$f(x)$};
    \draw[color=red] plot function{-sin(x)} ;
    \draw[color=blue] plot function{-(1+x/5)*sin(x)} ;
	\node[blue] at (4,2.5) {$y_2(x)=\E(Y|X=x)\cdot (x-m)f_X(x)$};
	\node[red] at (5.5,-0.5) {$y_1(x)=(x-m)f_X(x)$};
\end{tikzpicture}

Функция $ \E(Y|X=x) $ положительна и возрастает по $ x $, значит, холм растягивается сильнее, чем яма. Следовательно, интересующий нас интеграл $ \int_{0}^{1}\E(Y|X=x)(x-m)f_{X}(x)dx $ равный ковариации неотрицательный.


\end{proof}



Нам потребуется изучать функцию $ \E(V_{1}|Y_{1}=y, X_{1}=x) $.  Для краткости мы введём обозначение:

\begin{mydef}
\begin{equation}
v(x,y)=\E(V_{1}|Y_{1}=y, X_{1}=x)
\end{equation}
\end{mydef}

Самое время сделать упражнение \ref{ex_vxy}

\begin{myth}
\label{aff_multi_f}
Если $ X_{1} $, \ldots, $ X_{n} $ аффилированы, и $ g $ возрастает по всем аргументам, то $\E(g(X_{1},\ldots,X_{n})|X_{1}=x_{1},X_{2}=x_{2}) $ возрастает по $ x_{1} $ и $ x_{2} $.
\end{myth}

\begin{proof}
Пропущено. Если будут желающие, то допишу. Интуитивно: с ростом $ x_{1} $ растут условные средние остальных переменных в силу аффилированности, а с их ростом растет функция $ g $.
\end{proof}

В частности из этой теоремы следует, что $ v(x,y)=\E(V_{1}|X_{1}=x,Y_{1}=y) $ возрастает по обоим аргументам.


Теперь у нас хватает сил, чтобы решить наши три аукциона в общем виде.

\section{Решение трёх аукционов}

\begin{itemize}
\item Кнопочный аукцион.

Если вы разобрались с примером кнопочного аукциона для трёх игроков, то замена трёх на $ n $ несложная. Запишем традиционные обозначения:

\begin{itemize}
\item $ p_{1} $,\ldots,$ p_{n} $ — цены, на которых игроки покидают аукцион, упорядоченные по убыванию. Т.е., $ p_{n} $ — цена, на которой покинул аукцион самый слабый игрок, $ p_{n-1} $ — цена, на которой произошел второй выход. Заметим, что аукцион оканчивается на цене $ p_{2} $, то есть когда аукцион покидает предпоследний игрок. А $ p_{1} $ — цена, до которой был готов идти победитель, она остаётся неизвестной.
\end{itemize}


Стратегия описывается набором функций. Каждая функция говорит, до какого момента давить на кнопку, если моя ценность $ x $ и\ldots
\begin{itemize}
\item $ b^{n}(x) $ — все $ n $ игроков в игре
\item $ b^{n-1}(x,p_{n}) $ — в игре $ (n-1) $ игрок, а самый слабый вышел на $ p_{n} $
\item $ b^{n-2}(x,p_{n-1},p_{n}) $ —  в игре $ (n-2) $ игрока; самый слабый вышел на $ p_{n} $, а следующий — при цене $ p_{n-1} $
\item \ldots
\item $ b^{2}(x,p_{3},\ldots,p_{n}) $ — в игре $ 2 $ игрока, а выходы были на ценах $p_{n}$, \ldots, $ p_{3} $.
\end{itemize}

На кнопочном аукционе равновесие Нэша можно найти по алгоритму:

\begin{enumerate}
\item[Шаг 1.] В свою функцию ценности вместо всех сигналов подставляю $ x $. Получаю: $b^{n}(x)=u(x,x,x,\ldots,x)$.
\item[Шаг 2.] Предполагаю, что остальные игроки поступили также. Если я вижу, что первый выход был на цене $ p_{n} $, значит, сигнал $x_{n}  $ вышедшего игрока можно найти из уравнения:
\begin{equation}
b^{n}(x_{n})=p_{n}
\end{equation}
Учитываю эту информацию в новой функции:
\begin{equation}
b^{n-1}(x,p_{n})=u(x,x,\ldots,x,x_{n})
\end{equation}
\item[Шаг 3.] Предполагаю, что остальные игроки поступили также. Если я вижу, что второй выход был на цене $ p_{n-1} $, значит, сигнал $ x_{n-1} $ второго вышедшего можно найти из уравнения:
\begin{equation}
b^{n-1}(x_{n-1},p_{n})=p_{n-1}
\end{equation}
Учитываю эту информацию в новой функции:
\begin{equation}
b^{n-2}(x,p_{n-1},p_{n})=u(x,x,\ldots,x,x_{n-1},x_{n})
\end{equation}
\item[Шаг $ i $.]

\item[Шаг $ (n-1) $.] Предполагаю, что остальные игроки поступили также. Если я вижу, что $ (n-2) $-ой по счету выход был на цене $ p_{3} $, значит, сигнал $ x_{3} $ недавно вышедшего игрока можно найти из уравнения:
\begin{equation}
b^{3}(x_{3},p_{4},p_{5},\ldots,p_{n})=p_{3}
\end{equation}
Учитываю эту информацию в новой функции:
\begin{equation}
b^{2}(x,p_{3},\ldots,p_{n-1},p_{n})=u(x,x,x_{3},\ldots,x_{n-2},x_{n-1},x_{n})
\end{equation}

\end{enumerate}

Замечаем, что при использовании этих стратегий игроки выходят в порядке возрастания сигналов $ X_{i} $. По предположению, функция $ u $ возрастает по всем аргументам, значит, $ b^{n}(x) $ возрастает по $ x $. Значит, первым выходит игрок с наименьшим $ X_{i} $. Поскольку $ p_{n} $ одинаково для всех остающихся игроков, функция $ b^{n-1}(x,p_{n}) $ возрастает по $ x $. Значит, вторым выходит игрок с наименьшим $ X_{i} $ среди оставшихся в игре. И т.д. В частности, первый побеждает, только если его сигнал выше всех, то есть $ X_{1}>Y_{1} $.

Остаётся доказать, что это — равновесие Нэша. Пусть все игроки кроме первого используют такие функции. Что произойдет, если первый не будет использовать предлагаемую стратегию, а захочет выиграть аукцион любой ценой?

В силу того, что игроки выходят в порядке возрастания $ X_{i} $ предпоследний игрок выйдет на цене $ b^{2}(Y_{1},p_{3},\ldots,p_{n}) $. так как он использует указанную стратегию:
\begin{equation}
b^{2}(Y_{1},p_{3},\ldots,p_{n}) =u(Y_{1},Y_{1},Y_{2},Y_{3},\ldots,Y_{n-1})
\end{equation}

Выигрыш первого игрока мы упрощаем воспользовавшись тем, что $ Y_{i} $ — это $ X_{2} $,\ldots, $ X_{n} $ в другом порядке:
\begin{multline}
u(X_{1},X_{2},\ldots,X_{n})-u(Y_{1},Y_{1},Y_{2},Y_{3},\ldots,Y_{n-1})=\\
=u(X_{1},Y_{1},Y_{2},\ldots,Y_{n-1})-u(Y_{1},Y_{1},Y_{2},Y_{3},\ldots,Y_{n-1})
\end{multline}

Функция $ u $ возрастает по первому аргументу, значит, выигрыш положителен, если и только если $ X_{1}>Y_{1} $. то есть жать кнопку до выигрыша первому игроку следует если $ X_{1}>Y_{1} $. Но именно такой результат гарантирует предлагаемая стратегия. Значит, она и дает нам равновесие Нэша.


\item Аукцион первой цены.

Мы стандартным путем получаем дифференциальное уравнение, которое является необходимым условием. Итак, пусть $ b() $ — является равновесной стратегией. И пусть остальные игроки кроме первого её используют.

При стандартных предположениях о функции $ b() $ чудо-замена $ b_{1}=b(a) $ упрощает нам событие $ W_{1} $ до $ W_{1}=\{Y_{1}<a\} $:
\begin{equation}
\pi(x,b(a))=\E((V_{1}-b(a))1_{Y_{1}<a}|X_{1}=x)
\end{equation}

Далее мы пользуемся способом расчета математического ожидания через постановку условия \ref{conditional_way}. Дополнительное условие, которое мы используем — это условие по $ Y_{1}=y $:
\begin{multline}
\pi(x,b(a))=\int_{0}^{1}\E((V_{1}-b(a))1_{Y_{1}<a}|X_{1}=x,Y_{1}=y)g(y|x)dy=\\
=\int_{0}^{a}\E((V_{1}-b(a))|X_{1}=x,Y_{1}=y)g(y|x)dy=\\
=\int_{0}^{a}(v(x,y)-b(a))g(y|x)dy=\int_{0}^{a}v(x,y)g(y|x)dy-\int_{0}^{a}b(a)g(y|x)dy=\\
=\int_{0}^{a}v(x,y)g(y|x)dy-b(a)G(a|x)
\end{multline}

Берём производную по $ a $:
\begin{equation}
\frac{\partial \pi(x,b(a))}{\partial a}=v(x,a)g(a|x)-b(a)g(a|x)-b'(a)G(a|x)=0
\end{equation}

Первому игроку тоже должно быть оптимально использовать $ b(x) $, значит, $ a=x $:
\begin{equation}
v(x,x)g(x|x)-b(x)g(x|x)-b'(x)G(x|x)=0
\end{equation}

Наш диф. ур приобрел вид:
\begin{equation}
\label{b_first_de}
b'(x)=(v(x,x)-b(x))\frac{g(x|x)}{G(x|x)}=(v(x,x)-b(x))R(x|x)
\end{equation}

Мы уже говорили, что из множества решений нам нужно выбрать то, которое удовлетворяет условию $ b(0)=0 $. Давайте мы строго и в общем виде докажем, что это условие является достаточным.


\begin{myth}
Решение дифференциального уравнения:
\begin{equation}
b'(x)=(v(x,x)-b(x))\frac{g(x|x)}{G(x|x)}=(v(x,x)-b(x))R(x|x)
\end{equation}
удовлетворяющее условию $ b(0)=0 $ дает равновесие Нэша на аукционе первой цены.
\end{myth}

\begin{proof}


Построим на графике функцию $ v(x,x) $. В силу аффилированности она возрастает. В силу предпосылок на функцию $ u() $ наша $ v(x,x) $ проходит через начало координат. Заметим, что $ R(x|x)\geq 0 $. Из вида дифференциального уравнения следует, что решения убывают выше $ v(x,x) $, и возрастают ниже $ v(x,x) $.

\begin{tikzpicture}[domain=0:6]
    \draw[very thin,color=gray] (-0.1,-0.1) grid (7.1,3.1);
    \draw[->] (-0.1,0) -- (7.2,0) node[right] {$x$};
    \draw[->] (0,-0.1) -- (0,3.2) node[above] {$f(x)$};
    \draw[color=blue, dashed] plot function{sqrt(x/3)} ;
    \draw[color=red] plot function{2*sqrt(x/3)}
        node[above] {$v(x,x)$};
    \draw[color=blue, dashed] plot[id=s] function{0.05*(x-3)**2+2};
	\node [red] at (2,2.5) {$b'(x)<0$};
	\node [red] at (4,0.5) {$b'(x)>0$};
\end{tikzpicture}



Решения не удовлетворяющие условию $ b(0)=0 $ обязательно убывают при небольших $ x $. Но наше дифференциальное уравнение является необходимым условием только для случая возрастающей $ b(x) $. Мы пользовались возрастанием функции $ b() $ при исполнении чудо-замены. Решение же удовлетворяющее условию $ b(0)=0 $ позволяет оправдать чудо-замену.

Осталось доказать, что решение с $ b(0)=0 $ удовлетворяет какому-нибудь достаточному условию максимума. Мы докажем, что знак производной меняется с плюса на минус, как и положено.


Присмотримся повнимательнее к первой производной прибыли:
\begin{multline}
\frac{\partial \pi(x,b(a))}{\partial a}=v(x,a)g(a|x)-b(a)g(a|x)-b'(a)G(a|x)=\\
=(v(x,a)-b(a))g(a|x)-b'(a)G(a|x)=\\
=(v(x,a)-v(a,a)+v(a,a)-b(a))g(a|x)-b'(a)G(a|x)=\\
(v(x,a)-v(a,a))g(a|x)+(v(a,a)-b(a))g(a|x)-b'(a)G(a|x)
%G(a|x)\left((v(x,a)-b(a))R(a|x)-b'(a) \right)
\end{multline}

Функция $ b() $ является решением дифференциального уравнения \ref{b_first_de}, поэтому $ v(a,a)-b(a)=b'(a)/R(a|a) $:

\begin{multline}
\frac{\partial \pi(x,b(a))}{\partial a}=(v(x,a)-v(a,a))g(a|x)+\frac{b'(a)}{R(a|a)}g(a|x)-b'(a)G(a|x)=\\
(v(x,a)-v(a,a))g(a|x)+b'(a)g(a|x)\left(\frac{1}{R(a|a)}-\frac{1}{R(a|x)} \right)
\end{multline}

\begin{enumerate}
\item Рассмотрим $ a>x $. Во-первых, $ v(x,a)<v(a,a) $ так как $ v() $ возрастает по обоим аргументам. Во-вторых, $1/R(a|a)<1/R(a|x) $ поскольку $ R(a|x) $ возрастает по второму аргументу. Значит, справа производная отрицательна.
\item Рассмотрим $ a<x $. Во-первых, $ v(x,a)>v(a,a) $ так как $ v() $ возрастает по обоим аргументам. Во-вторых, $1/R(a|a)>1/R(a|x) $ поскольку $ R(a|x) $ возрастает по второму аргументу. Значит, слева производная положительна.
\end{enumerate}

Кстати, никаких секретов в решении линейных диф. уров первого порядка в 21 веке нет, поэтому мы можем его предъявить в явном виде:
\begin{equation}
b(x)=\int_{0}^{x}v(y,y)R(y|y)\cdot \exp\left(-\int_{y}^{x}R(t|t)dt\right) dy
\end{equation}

Желающие могут убедиться, что оно подходит и в само уравнение, и к условию $ b(0)=0 $, или получить его самостоятельно методом вариации постоянной. Но форма его настолько громоздкая, что проще решать задачи без него.
\end{proof}

В качестве побочного результата мы получили доказательство того, что $ b(x)\leq v(x,x) $.

Для последующего сравнения прибыли продавца нам потребуется функция выплат первого игрока. Вероятность того, что первый выиграет аукцион если его сигнал равен $ x $ равна $ \P(Y_{1}<x|X_{1}=x)=G(x|x)$. Поэтому:
\begin{equation}
pay^{FP}(x)=b^{FP}(x)G(x|x)
\end{equation}

Здесь мы обозначили равновесную стратегию не как $ b() $, а как $ b^{FP}() $ так как она отличается от равновесной стратегии на других аукционах.

\item Аукцион второй цены.

При решении задач мы столкнулись с тем, что аукцион второй цены в каком-то смысле правдивый, то есть ставить надо свою ценность. Когда ценность не совпадает с сигналом верен очень похожий результат:

\begin{myth} \label{NE_SP}
На аукционе второй цены равновесием Нэша будет набор стратегий: $ b(x):=v(x,x)=\E(V_{1}|X_{1}=x \cap Y_{1}=x) $
\end{myth}
\begin{proof}
Пусть остальные игроки используют предлагаемую стратегию, а первый ставит $ b_{1} $.
\begin{equation}
\pi(x,b_{1})=\E((V_{1}-b(Y_{1}))1_{W_{1}}|X_{1}=x; Bid_{1}=b_{1})
\end{equation}

Сделаем замену $ b_{1}=b(a) $, она упрощает нам событие $ W_{1} $ до $ W_{1}=\{Y_{1}<a\} $
\begin{multline}
\pi(x,b(a))=\E((V_{1}-b(Y_{1}))1_{Y_{1}<a}|X_{1}=x)=\\
=\E((V_{1}1_{Y_{1}<a}|X_{1}=x)-\E(b(Y_{1}))1_{Y_{1}<a}|X_{1}=x)
\end{multline}

Отдельно считаем вычитаемое:
\begin{equation}
\E(b(Y_{1}))1_{Y_{1}<a}|X_{1}=x)=\int_{0}^{a}b(y)g(y|x)dy=\int_{0}^{a}v(y,y)g(y|x)dy
\end{equation}

И применив к уменьшаемому формулу \ref{iterated_e}:
\begin{multline}
\E((V_{1}1_{Y_{1}<a}|X_{1}=x)=\int_{0}^{1}\E(V_{1}1_{Y_{1}<a}|X_{1}=x \cap Y_{1}=y)g(y|x)dy=\\
\int_{0}^{a}\E(V_{1}|X_{1}=x \cap Y_{1}=y)g(y|x)dy=\int_{0}^{a}v(x,y)g(y|x)dy
\end{multline}


Значит:
\begin{equation}
\pi(x,b(a))=\int_{0}^{a}(v(x,y)-v(y,y)) g(y|x)dy
\end{equation}

Если $ y<x $, то величина $ v(x,y)-v(y,y)>0 $ в силу того, что $ v(x,y) $ возрастает по $ x $. Мы хотим, максимизировать прибыль, то есть мы хотим интегрировать до тех пор, пока подынтегральное выражение положительно. то есть оптимальное $ a=x $. Остаётся заметить, что по предположению игрок делает ставку $ b_{1}=b(a) $. Но оптимальное $ a=x $, значит, оптимальная ставка равна $ b(x) $.


\end{proof}


Для последующего сравнения прибыли продавца нам потребуется функция выплат первого игрока:
\begin{equation}
pay^{SP}(x)=\E(b(Y_{1})1_{Y_{1}<x}|X_{1}=x)=\int_{0}^{x}v(t,t)g(t|x)dt
\end{equation}





\end{itemize}



\section{Теорема о сравнении доходностей}


\begin{myth}\index{Теорема о сравнении доходностей}
Если:

\begin{enumerate}
\item[RC1.] Сигналы $ X_{i} $ имеют регулярное на $ [0;1] $ распределение
\item[RC2.] Сигналы $ X_{i} $ аффилированы
\item[RC3.] Игроки симметричны, в частности:
\begin{enumerate}
\item[RC3a.] Совместная функция плотности сигналов симметрична
\item[RC3b.] Ценность игрока симметрична относительно сигналов других игроков.
\end{enumerate}
\item[RC4.] Ценность является возрастающей функцией от сигналов, $ u(0,\ldots,0)=0 $.

\end{enumerate}

То:
\begin{equation}
\E(R^{B})\geq \E(R^{SP})\geq \E(R^{FP})
\end{equation}

\end{myth}

\begin{proof}
Сначала докажем, что для продавца аукцион второй цены лучше, чем аукцион первой цены, $ \E(R^{SP})\geq \E(R^{FP}) $.


Мы снова воспользуемся дифференциальным уравнением \ref{b_first_de}:
\begin{multline}
pay^{SP}(x)=\int_{0}^{x}v(y,y)g(y|x)dy=\\
=\int_{0}^{x}(v(y,y)-b^{FP}(y))g(y|x)dy+\int_{0}^{x}b^{FP}(y)g(y|x)dy=\\
=\int_{0}^{x}b'^{FP}(y)\frac{1}{R(y|y)}g(y|x)dy+\int_{0}^{x}b^{FP}(y)g(y|x)dy=\\
=\int_{0}^{x}b'^{FP}(y)\frac{R(y|x)}{R(y|y)}G(y|x)dy+\int_{0}^{x}b^{FP}(y)g(y|x)dy\geq\\
\geq \int_{0}^{x}b'^{FP}(y)G(y|x)dy+\int_{0}^{x}b^{FP}(y)g(y|x)dy
\end{multline}
Последний переход верен в силу того, что $ y<x $. Продолжаем:

А теперь долго и пристально смотрим на эти два интеграла и берём их в уме оба сразу:
\begin{multline}
\int_{0}^{x}b'^{FP}(y)G(y|x)dy+\int_{0}^{x}b^{FP}(y)g(y|x)dy=\\
\int_{0}^{x}b^{FP}(y)g(y|x)+b'^{FP}(y)G(y|x) dy=b^{FP}(x)G(y|x)=pay^{FP}(x)
\end{multline}

Мы сравнили детерминистические функции выплат. А ожидаемый доход продавца связан с ними:
\begin{equation}
\E(R)=n\cdot \E(Pay_{1})=n\cdot \int_{0}^{1}pay(x)f(x)dx
\end{equation}
Опять же мы применяем трюк с условным подсчетом математического ожидания \ref{conditional_way}.

Теперь докажем, что для продавца кнопочный аукцион лучше, чем аукцион второй цены $ \E(R^{B})\geq \E(R^{SP}) $.

Только для целей этого доказательства введём функцию $$z(x,y)=\E(u(Y_{1},Y_{1},\ldots,Y_{n-1})|X_{1}=x,Y_{1}=y) $$.

Напомню смысл: на кнопочном аукционе самый сильный игрок (за исключением первого) жмет кнопку до $ u(Y_{1},Y_{1},\ldots,Y_{n-1}) $. Именно столько заплатит первый, если выиграет аукцион. По теореме \ref{aff_multi_f} функция $ z(x,y) $ возрастает по обоим аргументам.

Сначала мы замечаем, что $ v(y,y)=z(y,y) $:
\begin{multline}
v(y,y)=\E(V_{1}|X_{1}=y, Y_{1}=y)=\E(u(X_{1},Y_{1},\ldots,Y_{n-1})|X_{1}=y, Y_{1}=y)=\\
=\E(u(Y_{1},Y_{1},\ldots,Y_{n-1})|X_{1}=y, Y_{1}=y)=z(y,y)
\end{multline}

Если $ x>y $, то $ v(y,y)<z(x,y) $. А теперь считаем ожидаемую доходность продавца:
\begin{multline}
\E(R^{SP})=\E(b^{SP}(Y_{1})|X_{1}>Y_{1})=\E(v(Y_{1},Y_{1})|X_{1}>Y_{1})\leq\\
\leq \E(z(X_{1},Y_{1})|X_{1}>Y_{1})
\end{multline}

Заметим, что в правой части написано математическое ожидание от условного математического ожидания в случайной точке. Пользуясь идеей \ref{conditional_way} мы видим, что:
\begin{equation}
\E(z(X_{1},Y_{1})|X_{1}>Y_{1})=\E(u(Y_{1},Y_{1},\ldots,Y_{n-1})|X_{1}>Y_{1})=\E(R^{B})
\end{equation}

\end{proof}

\section{Задачи}
\begin{enumerate}
\item Аукцион второй цены с резервной ставкой. Имеется $n$ покупателей.  Сигналы независимы и равномерны на $ [0;1] $, $ V_{i}=X_{i} $. Покупатели одновременно делают ставки. Если наибольшая ставка больше $ r $, то игрок с максимальной ставкой получает товар. Если наибольшая ставка меньше $ r $, то товар остаётся у продавца. Победитель платит продавцу максимум между второй по величине ставкой и $r$.
\begin{enumerate}
\item Найдите равновесие Нэша.
\item Найдите оптимальное $ r $ для продавца, если ценности равномерны на отрезке $ [0;1] $.
\end{enumerate}


\item Аукцион первой цены с резервной ценой $ r $. Имеется $ n $ покупателей.  Сигналы независимы и равномерны на $ [0;1] $, $ V_{i}=X_{i} $. Покупатели одновременно делают ставки. Если наибольшая ставка больше $ r $, то игрок с максимальной ставкой получает товар. Если наибольшая ставка меньше $ r $, то товар остаётся у продавца. Победитель платит продавцу свою ставку. Предполагаем, что $ r $ известна покупателям.
\begin{enumerate}
\item Найдите равновесие Нэша.
\item Найдите оптимальное $ r $ для продавца, если ценности равномерны на $ [0;1] $.
\end{enumerate}


\item Аукцион второй цены с платой за вход. Имеется $ n $ покупателей.  Сигналы независимы и равномерны на $ [0;1] $, $ V_{i}=X_{i} $. Покупатели одновременно решают делать ли ставку, и если делать, то какую. Сделавший ставку платит $ w $ вне зависимости от того, победил ли он. Игрок с максимальной ставкой получает товар. Если ставок не было, то товар остаётся у продавца. Победитель платит продавцу вторую по величине ставкой. Если на аукцион вошел только один игрок, то он побеждает и  ничего кроме платы за вход не платит.
\begin{enumerate}
\item Найдите равновесие Нэша.
\item Найдите оптимальное $ w $ для продавца, если ценности равномерны на $ [0;1] $.
\end{enumerate}


\item Аукцион первой цены с платой за вход. Имеется $ n $ покупателей.  Сигналы независимы и равномерны на $ [0;1] $, $ V_{i}=X_{i} $. Покупатели одновременно решают делать ли ставку, и если делать, то какую. Сделавший ставку платит $ w $ вне зависимости от того, победил ли он. Игрок с максимальной ставкой получает товар. Если ставок не было, то товар остаётся у продавца. Победитель платит продавцу свою ставку.
\begin{enumerate}
\item Найдите равновесие Нэша.
\item Найдите оптимальное $ w $ для продавца, если ценности равномерны на $ [0;1] $.
\end{enumerate}

\item Верно ли, что $ \E(R) $ одинаково в аукционе первой и второй цены с резервной ценой?

\item Верно ли, что $ \E(R) $ одинаково в аукционе первой и второй цены с платой за вход?

\item Величины $ X_{1} $, $ X_{2} $ и $ X_{3} $ независимы и равномерны на $ [0;1] $. В аукционе второй цены участвуют два игрока: первый знает $ X_{1} $, второй — $ X_{2} $. Ценность товара общая, $ V_{1}=V_{2}=X_{1}+X_{2}+X_{3} $. Найдите равновесие Нэша и ожидаемый доход продавца.

\item По аналогии с определением условной обратной функции риска дайте определение безусловной обратной функции риска, $ R(x) $. Пусть $ X $ — случайная величина, показывающая время в часах, которое я трачу на написание одной лекции. Как можно проинтерпретировать $ R(5)=10 $?
\item Автобусы приходят на остановку согласно пуассоновскому потоку с интенсивностью $ \lambda=6 $ автобусов в час. Вася стоит некоторое время у остановки. Сколько в среднем автобусов приедет за это время? Какова вероятность, что не приедет ни одного автобуса? Рассмотрите два случая:
\begin{enumerate}
\item Вася стоит у остановки ровно 5 минут.
\item Вася стоит у остановки случайное время $ X $ (в минутах), независимое от времени прихода автобусов. Функция плотности $ X $ имеет вид $ f(x)= \frac{x}{25}$ при $ x\in [0;10] $.
\end{enumerate}
Подсказка: В первом пункте вы не замечая того нашли $ \E(N|X=5) $.
\item Найдите функции $ g(x,y)$, $ g(y|x)$, $ G(y|x)$,  $R(y|x)$ и $v(x,y)$ для случаев:
\label{ex_vxy}
\begin{enumerate}
\item Сигналы независимы, равномерны на $ [0;1] $, $ V_{i}=X_{i} $.
\item Три игрока. Сигналы независимы, равномерны на $ [0;1] $, $ V_{1}=X_{1}+X_{2}X_{3} $.
\item Три игрока. Совместная функция плотности сигналов имеет вид $ f(x_{1},x_{2},x_{3})=7/8+x_{1}x_{2}x_{3} $ при $ x_{1},x_{2},x_{3}\in[0;1] $, $ V_{1}=X_{1}+X_{2}X_{3} $.
\end{enumerate}
\end{enumerate}


\section{Решения задач}

\begin{enumerate}
\item Аукцион второй цены с резервной ставкой.

С помощью таблички доказываем, что игрокам оптимально говорить правду. Конечно, если ценность меньше $ r $, то оптимально говорить любое число ниже $ r $. Но правду оптимально говорить всегда.

Первый игрок в среднем платит:
\begin{equation}
\E(Pay_{1})=r\cdot \P(X_{1}>r>Y_{1})+\E(Y_{1}\cdot 1_{X_{1}>Y_{1}>r})
\end{equation}

Совместная функция плотности $ X_{1} $ и $ Y_{1} $ имеет вид:
\begin{equation}
g(x,y)=(n-1)y^{n-2}
\end{equation}

Значит, первое слагаемое равно:
\begin{equation}
r\P(X_{1}>r>Y_{1})=r\int_{r}^{1}\int_{0}^{r} (n-1)y^{n-2} dy dx=(1-r)r^{n}
\end{equation}

И второе слагаемое равно:
\begin{multline} \label{E_Y_1XYr}
\E(Y_{1}\cdot 1_{X_{1}>Y_{1}>r})=\int_{r}^{1}\int_{r}^{x} y\cdot (n-1)y^{n-2} dy dx=\\
=(n-1)\left(\frac{1}{n(n+1)}-\frac{\rho^{n}}{n}+\frac{\rho^{n+1}}{n+1}\right)
\end{multline}

Значит, средняя выплата первого игрока равна:
\begin{multline}
\E(Pay_{1})=(1-r)r^{n}+(n-1)\left(\frac{1}{n(n+1)}-\frac{r^{n}}{n}+\frac{r^{n+1}}{n+1}\right)=\\
=\frac{r^{n}}{n}-\frac{2r^{n+1}}{n+1}+\frac{n-1}{n(n+1)}
\end{multline}

Максимизируем по $ r $ находим, что $ r^{*}=0.5 $.


\item Аукцион первой цены с резервной ставкой. Рассуждаем за 1-го игрока:

\begin{equation}
\pi_{1}(x,b_{1})=(x-b_{1})\P(b_{1}\geq \max\{b(Y_{1}),r\})
\end{equation}

Наша задача максимизировать эту функцию выбирая произвольное $ b_{1} $.

Если $ x<r $, то нам ничего не светит, оптимально не участвовать в аукционе, то есть можно делать любую ставку меньше $ r $. Если $ x\geq r $, то оптимально участвовать в аукционе с некоторой ставкой $ r\leq b_{1}\leq x $. В частности, получаем, что $ b(r)=r$.


%Отметим, что максимум ожидаемой прибыли, то есть функция $\pi_{1}(x,b_{1}^{*}(x))$, должна плавно зависеть от $ x $. Действительно, при $ x<r $ максимум прибыли равен нулю. А при $ x=r+\varepsilon $ оптимальная ставка $ b_{1}\in[r;r+\varepsilon] $, поэтому вероятность в формуле прибыли примерно равна нулю.

Предположим, что $ x\geq r $. Тогда целевая функция упростится до старой, без резервной цены!

\begin{equation}
\pi_{1}(x,b_{1})=(x-b_{1})\P(b_{1}\geq b(Y_{1}))
\end{equation}

Делаем вывод. Если $x\geq r$, то оптимальное $ b_{1}(x) $ удовлетворяет старому дифференциальному уравнению.

Напомним, что старое уравнение было:
\begin{equation}
b'(x)x=(n-1)(x-b(x))
\end{equation}

И его решением было:
\begin{equation}
b(x)=cx^{1-n}+\frac{n-1}{n}x
\end{equation}

На этот раз $ c $ надо искать из условия $ b(r)=r $. Раньше, кстати, начальное условие было $b(0)=0 $. Отсюда находим $ c=r^{n}/n $ и частное решение:
\begin{equation}
b(x)=\frac{r^{n}}{n}x^{1-n}+\frac{n-1}{n}x, \quad x\geq r
\end{equation}

На всякий пожарный можно убедиться, что эта функция возрастает по $ x $.

Ожидаемая выплата от первого игрока:
\begin{multline}
\E(b(X_{1})1_{X_{1}\geq Y_{1},X_{1}\geq r})=\int_{r}^{1} \int_{0}^{x} b(x)g(x,y)dy dx =\\
=\int_{r}^{1} b(x) \int_{0}^{x} g(x,y)dy dx =\int_{r}^{1} b(x) \int_{0}^{x} (n-1) y^{n-2} dy dx =\\
=\int_{r}^{1}b(x) x^{n-1} dx=\frac{r^{n}}{n}-\frac{2r^{n+1}}{n+1}+\frac{n-1}{n(n+1)}
\end{multline}

Зависимость выплаты первого игрока от $ r $ такая же как на аукционе второй цены.


\item Аукцион второй цены с платой за вход.

%На аукционе второй цены с резервной ценой $ r $ реально играют только игроки с ценностью выше $ r $.

Для начала заметим, что если игрок решил делать ставку, то ему оптимально делать ставку равную ценности. Доказательство стандартное, стратегия $ b_{1}=X_{1} $ нестрого доминирует все остальные. Осталось определить, при каких $ X_{1} $ первому игроку лучше играть, а при каких  — нет.

Предполагаем, что оптимальная стратегия имеет вид: если $ x\geq \rho $, то делать ставку $ b_{1}=x $, если $ x<\rho $, то не делать ставку. Предположим кроме того, что равновесные стратегии $ b(x) $ возрастают по $ x $ при $ x\geq \rho $.
%Не участие в аукционе мы можем кодировать например, как $ b(x)=0 $ при $ x<\rho $.

Рассмотрим игрока с ценностью $ x\geq \rho $ в равновесии Нэша. Какова вероятность того, что он выиграет аукцион? Поскольку $ b(x) $ монотонно возрастает вероятность выигрыша по-прежнему:
\begin{equation}
q(x)=F^{n-1}(x),\quad x\geq \rho
\end{equation}

Игрок с ценностью $ \rho $ должен быть безразличен между ставкой $ b(\rho) $ и не участием в аукционе. Не участвуя в аукционе он получает ноль. Участвуя, он выиграет только если все остальные не участвуют, то есть он выигрывает аукцион по нулевой цене. Значит, условие безразличия имеет вид:
\begin{equation}
-w+\rho F^{n-1}(\rho)=0
\end{equation}

Применительно к нашему случаю $ F(x)=x $ получаем $ \rho=w^{1/n} $.


Ожидаемая выплата от первого игрока равна:

\begin{multline}
\E(Pay_{1})=\E(Y_{1}1_{X_{1}>Y_{1}>\rho})+w\P(X_{1}>\rho)=\\
=\E(Y_{1}1_{X_{1}>Y_{1}>\rho})+w(1-\rho)
\end{multline}

Первое слагаемое мы уже искали, см. \ref{E_Y_1XYr}:
\begin{equation}
\E(Y_{1}1_{X_{1}>Y_{1}>\rho})=(n-1)\left(\frac{1}{n(n+1)}-\frac{\rho^{n}}{n}+\frac{\rho^{n+1}}{n+1}\right)
\end{equation}

Складываем, и, как и раньше получаем:
\begin{multline}
\E(Pay_{1})=(n-1)\left(\frac{1}{n(n+1)}-\frac{\rho^{n}}{n}+\frac{\rho^{n+1}}{n+1}\right)+\rho^{n}(1-\rho)=\\
=\frac{\rho^{n}}{n}-\frac{2\rho^{n+1}}{n+1}+\frac{n-1}{n(n+1)}
\end{multline}

\item Аукцион первой цены с платой за вход.

Предположим, что оптимальная стратегия имеет вид: Если $ x\geq \rho $, то делать ставку $ b(x) $; если $ x<\rho $, то не делать ставку. Предположим, что эта $ b() $ возрастающая при $ x\geq \rho $. Как и на аукционе второй цены из этого следует, что вероятность выигрыша первого игрока при $ x\geq\rho $ равна:
\begin{equation}
q(x)=F(x)^{n-1}=x^{n-1}
\end{equation}


Если $ x=\rho $, то игроку должно быть безразлично, делать или не делать ставку. Если ему не безразлично, то значит, $ \rho $ выбрано не оптимально. Ожидаемый выигрыш первого игрока при ценности $ \rho $:
\begin{equation}
(\rho-b(\rho))F(\rho)^{n-1}-w=0
\end{equation}

Заметим, что $ b(\rho)=0 $. Действительно, игрок с пороговой ценностью $ \rho $ может выиграть только если у остальных ценность ниже $ \rho $, то есть если остальные не участвуют. А при таком условии победы оптимально ставить $ b(\rho)=0 $.

Получаем такой же порог участия как на аукционе второй цены:
\begin{equation}
\rho=w^{1/n}
\end{equation}

Рассматриваем случай $ x\geq \rho $. В этом случае прибыль совпадает со старой. Мы получаем старое дифференциальное уравнение и старое решение:
\begin{equation}
b(x)=cx^{1-n}+\frac{n-1}{n}x
\end{equation}

И начальное условие $ b(\rho)=0 $. Получаем $ c=\frac{1-n}{n}\rho^{n} $ и частное решение:
\begin{equation}
b(x)=\frac{n-1}{n}x(1-\rho^{n}x^{-n})
\end{equation}

Считаем ожидаемый платеж первого игрока:
\begin{equation}
\E(Pay_{1})=\E(b(X_{1})1_{X_{1}\geq Y_{1},X_{1}\geq \rho})+\rho^{n}\P(X_{1}>\rho)
%\int_{\rho}^{1}\int_{0}^{x} \frac{n-1}{n}x(1-\rho^{n}x^{-n}) g(x,y) dy dx+\rho^{n}(1-\rho)=\ldots=\\
%\frac{\rho^{n}}{n}-\frac{2\rho^{n+1}}{n+1}+\frac{n-1}{n(n+1)}
\end{equation}

Первый интеграл:
\begin{multline}
\E(b(X_{1})1_{X_{1}\geq Y_{1},X_{1}\geq \rho})=\int_{\rho}^{1} \int_{0}^{x} b(x)g(x,y)dy dx =\\
\int_{\rho}^{1} b(x) \int_{0}^{x} g(x,y)dy dx =\int_{\rho}^{1} b(x) \int_{0}^{x} (n-1) y^{n-2} dy dx =\\
\int_{\rho}^{1}b(x) x^{n-1} dx=(n-1)\left(\frac{1}{n(n+1)}-\frac{\rho^{n}}{n}+\frac{\rho^{n+1}}{n+1}\right)
\end{multline}

В сумме, как и раньше:
\begin{equation}
\E(Pay_{1})=\frac{\rho^{n}}{n}-\frac{2\rho^{n+1}}{n+1}+\frac{n-1}{n(n+1)}
\end{equation}

\item  Верно ли, что $ \E(R) $ одинаково в аукционе первой и второй цены с резервной ценой? Да.

\item  Верно ли, что $ \E(R) $ одинаково в аукционе первой и второй цены с платой за вход? Да.

Более того, можно заметить, что аукцион с резервной ценой $ r $ похож на аукцион с платой за вход $ w=r\cdot F(r)^{n-1} $.

\item %Величины $ X_{1} $, $ X_{2} $ и $ X_{3} $ независимы и равномерны на $ [0;1] $. В аукционе второй цены участвуют два игрока: первый знает $ X_{1} $, второй — $ X_{2} $. Ценность товара общая, $ V_{1}=V_{2}=X_{1}+X_{2}+X_{3} $. Найдите равновесие Нэша и ожидаемый доход продавца.
Можно воспользоваться теоремой \ref{NE_SP} и сказать, что равновесная стратегия:
\begin{multline}
b(x)=\E(V_{1}|X_{1}=x,Y_{1}=x)=\\
=\E(X_{1}+X_{2}+X_{3}|X_{1}=x,Y_{1}=x)=x+\E(X_{2}+X_{3}|Y_{1}=x)
\end{multline}

Внимание! Здесь есть небольшая ловушка! Игроков всего два! $ X_{3} $ — это не сигнал от Природы третьему игроку! $ X_{3} $ — это просто составляющая ценности неизвестная обоим игрокам. Поэтому здесь $ Y_{1}=X_{2} $, а не $ Y_{1}=\max\{X_{2},X_{3}\} $. Остаётся вспомнить про независимость $ X_{2} $ и $ X_{3} $ и получить:
\begin{multline}
x+\E(X_{2}+X_{3}|Y_{1}=x)=x+\E(X_{2}+X_{3}|X_{2}=x)=\\
=x+x+\E(X_{3})=2x+0.5
\end{multline}

Считаем ожидаемую выигрыш продавца:
\begin{equation}
\E(R)=2\E(\min\{X_{1},X_{2}\})+0.5=2\int_{0}^{1}y\cdot 2(1-y)dy+0.5=\ldots=\frac{7}{6}
\end{equation}

%Интуитивно. Во-первых, $ Y_{2}<x$. Во-вторых, $ Y_{2} $ когда-то раньше была одним из $ X_{i} $, то есть была равномерной на $ [0;1] $. Значит, $ \E(Y_{2}|Y_{1}=x)=x/2 $. Итого: $ b(x)=2.5x $.

\item Обратная функция риска: $ R(x)=f(x)/F(x) $, где $ f() $ — функция плотности, а $ F() $ — функция распределения случайной величины $ X $. Пусть $ X $ — случайная величина, показывающая время в часах, которое я трачу на написание одной лекции. Проинтерпретировать $ R(5)=10 $ можно с помощью небольшой $ \Delta=0.01 $ (одна сотая часа — это 36 секунд). Если известно, что прошло 5 часов после начала написания лекций, и я уже отдыхаю, то вероятность того, что я их окончил за только что истекшие 36 секунд примерно равна $ R(5)\cdot \Delta=0.1 $.
\item Обозначим $ N$ — количество пришедших автобусов, а $ X $ — время, которое Вася наблюдал за остановкой.
\begin{enumerate}
\item Количество автобусов за $ x $ минут имеет распределение Пуассона с параметром $ \lambda_{x}=0.1x $, так как в среднем 6 автобусов в час, это в среднем 0.1 автобуса в минуту. $ \E(N|X=5)=0.5 $
\item \begin{equation} \E(N)=\int_{0}^{10}\E(N|X=x)f(x)dx=\int_{0}^{10}0.1x\cdot \frac{x}{25} dx=\frac{4}{3} \end{equation}
\end{enumerate}
\item Найдите функции $ g(x,y)$, $ g(y|x)$, $ G(y|x)$,  $R(y|x)$ и $v(x,y)$ для случаев:
\begin{enumerate}
\item Сигналы независимы, равномерны на $ [0;1] $, $ V_{i}=X_{i} $.
Применяя метод о-малых находим:
\begin{equation}
g(x,y)=(n-1)y^{n-2}
\end{equation}
\begin{equation}
g(y|x)=g(x,y)/f(x)=(n-1)y^{n-2}
\end{equation}
\begin{equation}
G(y|x)=\int_{0}^{y}(n-1)t^{n-2}dt=y^{n-1}
\end{equation}
\begin{equation}
R(y|x)=\frac{n-1}{y}
\end{equation}
\begin{equation}
v(x,y)=\E(X_{1}|X_{1}=x,Y_{1}=y)=x
\end{equation}

\item Три игрока. Сигналы независимы, равномерны на $ [0;1] $, $ V_{1}=X_{1}+X_{2}X_{3} $.
Отличается только $ v(x,y) $:
\begin{multline}
v(x,y)=\E(X_{1}+X_{2}X_{3}|X_{1}=x,Y_{1}=y)=x+\E(X_{2}X_{3}|Y_{1}=y)=\\
x+\E(Y_{1}Y_{2}|Y_{1}=y)=x+y\E(Y_{2}|Y_{1}=y)
\end{multline}

Здесь мы посчитаем интуитивно, а в следующем пункте — через интегралы. Итак: если я знаю, что $ Y_{1} $, максимум из $ X_{2} $ и $ X_{3} $, равен $ y $, то оставшаяся величина $ Y_{2} $ где-то на отрезке $[0;y] $. Поскольку безусловное распределение было равномерным, то и условное будет равномерным. И условное среднее будет равно $ y/2 $. то есть $ v(x,y)=x+y^{2}/2 $.


\item

\begin{equation}
g(x,y)=2!\cdot \int_{0}^{y} 7/8+x\cdot y\cdot x_{3} dx_{3}=\ldots=\frac{7}{4}y+xy^{3}
\end{equation}

\begin{multline}
g(y|x)=\frac{g(x,y)}{f(x)}=\frac{\frac{7}{4}y+xy^{3}}{\int_{0}^{1}\int_{0}^{1} \frac{7}{8}+xx_{2}x_{3}dx_{2}dx_{3}}=\ldots\\
==\frac{\frac{7}{4}y+xy^{3}}{\frac{7}{8}+\frac{x}{4}}=\frac{14y+8xy^{3}}{7+2x}
\end{multline}

Интегрируя находим:
\begin{equation}
G(y|x)=\int_{0}^{1} g(t|x) dt=\ldots=\frac{7y^{2}+2xy^{4}}{7+2x}
\end{equation}

И взяв отношение:
\begin{equation}
R(y|x)=\frac{g(y|x)}{G(y|x)}=\frac{14y+8xy^{3}}{7y^{2}+2xy^{4}}=\frac{14+8xy^{2}}{7y+2xy^{3}}
\end{equation}

Аналогично предыдущему пункту:
\begin{equation}
v(x,y)=\ldots=x+y\E(Y_{2}|Y_{1}=y)
\end{equation}



Находим совместную функцию плотности $ X_{2} $ и $ X_{3} $:
\begin{equation}
f(x_{2},x_{3})=\int_{0}^{1} 7/8+x_{1}\cdot x_{2}\cdot x_{3} dx_{1}=\frac{7}{8}+\frac{1}{2}x_{2}x_{3}, \quad x_{2},x_{3}\in [0;1]
\end{equation}

Из этого сразу следует совместная функция плотности для $ Y_{1} $ и $ Y_{2} $:
\begin{equation}
f_{Y_{1},Y_{2}}(y_{1},y_{2})=2! f(y_{1},y_{2})=\frac{7}{4}+y_{1}y_{2}, \quad 0<y_{2}<y_{1}<1
\end{equation}

И функцию плотности для $ Y_{1} $:
\begin{equation}
f_{Y_{1}}(y_{1})=2!\int_{0}^{y_{1}} f(y_{1},x_{3}) dx_{3}=\ldots=\frac{7}{4}y_{1}+\frac{1}{2}y_{1}^{3}
\end{equation}

Далее находим условную функцию плотности:
\begin{equation}
 f_{Y_{2}|Y_{1}}(y_{2}|y_{1})=\frac{f_{Y_{1},Y_{2}}(y_{1},y_{2})}{f_{Y_{1}}(y_{1})}=\frac{\frac{7}{4}+y_{1}y_{2}}{\frac{7}{4}y_{1}+\frac{1}{2}y_{1}^{3}}, \quad 0<y_{2}<y_{1}<1
\end{equation}

И условное ожидание:
\begin{equation}
\E(Y_{2}|Y_{1}=y)=\int_{0}^{y} y_{2} \frac{\frac{7}{4}+yy_{2}}{\frac{7}{4}y+\frac{1}{2}y^{3}}dy_{2}=\frac{8 \, y^{4} + 21 \, y^{2}}{6 \, {\left(2 \, y^{3} + 7 \, y\right)}}
\end{equation}
И, наконец,
\begin{equation}
v(x,y)=x+\frac{8 \, y^{4} + 21 \, y^{2}}{6 \, {\left(2 \, y^{2} + 7\right)}}
\end{equation}

\end{enumerate}
\end{enumerate}


\section{Контрольная 3}

\begin{enumerate}


\item Пусть $  V $ — общая ценность товара для двух игроков, равномерна на $ [0;1] $. Величины $ R_{1} $ и $ R_{2} $ — независимы между собой и с $ V $ и равномерны на $ [0.5;1.5] $. Игроки получают сигналы $ X_{i}=V\cdot R_{i} $.
\begin{enumerate}
\item Найдите совместную функцию плотности $ X_{1} $ и $ X_{2} $. Верно ли, что $ X_{1} $ и $ X_{2} $ аффилированны?
\item Найдите $ v(x,y)=\E(V|X_{1}=x,Y_{1}=y) $
\item Найдите совместную функцию плотности $ X_{1} $ и $ Y_{1} $, $ g(x,y) $
\end{enumerate}


\item На аукционе продаётся картина, которая равновероятно является «Джокондой» Леонардо да Винчи или её подделкой. За неё торгуются $ n $ покупателей. Ценность картины для всех покупателей одинакова, $ V_{1}=V_{2}=\ldots=V_{n}=V $ и равна 1, если это оригинал и 0, если подделка.

Если $ V=0 $, то сигналы $ X_{i} $ условно независимы и равномерны на $ [0;1] $. Если $ V=1 $, то сигналы $ X_{i} $ условно независимы и имеют функцию плотности $ f(x|V=1)=2x $ при  $x\in [0;1] $
\begin{enumerate}
\item Найдите совместную функцию плотности всех $ X_{i} $. Верно ли, что все $ X_{i} $ аффилированны?
\item Найдите $ v(x,y)=\E(V|X_{1}=x,Y_{1}=y) $
\item Найдите совместную функцию плотности $ X_{1} $ и $ Y_{1} $, $ g(x,y) $
\end{enumerate}



\item На аукционе второй цены присутствуют $ n $ покупателей. Ценности совпадают с сигналами, $ V_{i}=X_{i} $; сигналы $ X_{i} $ независимы и равномерны на $ [0;1] $. На аукционе продаётся $k$ одинаковых чудо-швабр, $ 1<k<n $. Каждому покупателю нужна только одна чудо-швабра. Покупатели одновременно делают свои ставки. Чудо-швабры достаются по одной каждому из $ k $ покупателей с самыми высокими ставками. Каждый из $ k $ победителей платит организатору наибольшую проигравшую ставку.

Найдите равновесие Нэша.

\item На аукционе первой цены присутствуют $ n $ покупателей. Ценности совпадают с сигналами, $ V_{i}=X_{i} $; сигналы $ X_{i} $ независимы и равномерны на $ [0;1] $. На аукционе продаётся $k$ одинаковых чудо-швабр, $ 1<k<n $. Каждому покупателю нужна только одна чудо-швабра. Покупатели одновременно делают свои ставки. Чудо-швабры достаются по одной каждому из $ k $ покупателей с самыми высокими ставками. Эти $ k $ победителей платят свои ставки организатору.

Найдите дифференциальное уравнение, которому удовлетворяет равновесная стратегия.

Подсказка: Когда продавался один товар, то условие победы первого игрока — $ Y_{1}<a $, а если продаётся $ k $ товаров, то условие победы первого игрока $ Y_{?}<a $.

\item Существуют ли неаффилированные случайные величины $ X_{1} $ и $ X_{2} $ такие, что $Cov(X_{1},X_{2})>0  $?

\end{enumerate}


\section{Решение контрольной 3}

\begin{enumerate}


\item  По условию, при фиксированном $ v $ величина равномерна на $ [0.5v;1.5v] $. Длина этого отрезка равна $ v $, значит, условная функция плотности $ X_{1} $ при фиксированном $ v $ имеет вид: \index{аффилированные случайные величины}
\begin{equation}
p(x_{1}|v)=\frac{1}{v}\quad x_{1}\in [0.5v;1.5v].
\end{equation}


Так как при фиксированном $ v $ величины $X_{1}  $ и $ X_{2} $ независимы, то выписываем $ p(x_{1},x_{2}|v) $:
\begin{equation}
p(x_{1},x_{2}|v)=\frac{1}{v}\cdot \frac{1}{v}, \quad x_{1},x_{2}\in [0.5v;1.5v].
\end{equation}

Так как $ p(x_{1},x_{2},v)=p(x_{1},x_{2}|v)p(v) $:
\begin{equation}
p(x_{1},x_{2},v)=\frac{1}{v}\cdot \frac{1}{v}\cdot 1, \quad x_{1},x_{2}\in [0.5v;1.5v].
\end{equation}

Условие $ x_{1},x_{2}\in [0.5v;1.5v] $ записываем как: $ x_{1}\wedge x_{2} > 0.5v $ и $ x_{1}\vee x_{2} <1.5v $. Или как $ v\in [\frac{x_{1}\vee x_{2}}{1.5};\frac{x_{1}\wedge x_{2}}{0.5}] $. Для краткости обозначим этот интервал $[v_{min};v_{max}]$.

Интегрируем по $ v $ в указанных пределах и получаем:
\begin{equation}
p(x_{1},x_{2})=\int_{v_{min}}^{v_{max}}\frac{1}{v^{2}} \, dv=
\frac{1}{v_{min}}-\frac{1}{v_{max}}.
\end{equation}

Есть точки, где функция недифференциируема, поэтому проверять супермодулярность нужно будет по определению. Проверка супермодулярности сводится к аккуратному рассмотрению нескольких случаев.

Здесь $ Y_{1}=X_{2} $, поэтому третий пункт уже решён, осталось найти:

\begin{multline}
\E(V|X_{1}=x_{1}, X_{2}=x_{2})=\int v p(v|x_{1},x_{2}) \, dv=\\
\int v \frac{p(x_{1},x_{2},v)}{p(x_{1},x_{2})} \, dv=
\frac{\int v p(x_{1},x_{2},v) \, dv }{p(x_{1},x_{2})}.
\end{multline}

В числителе:
\begin{equation}
\int_{v_{min}}^{v_{max}}\frac{1}{v} \, dv=\ln(v_{max})-\ln(v_{min}).
\end{equation}

Значит, в итоге
\begin{equation}
v(x_{1},x_{2})=\frac{\ln(v_{max})-\ln(v_{min})}{\frac{1}{v_{min}}-\frac{1}{v_{max}}}.
\end{equation}

\item Возьмём событие $ A=\{X_{1}\in[x_{1};x_{1}+\Delta] \cap \ldots \cap X_{n}\in[x_{n};x_{n}+\Delta]\} $. Поскольку $ \P(A)>0 $, действуют старые правила:
\begin{multline}
\P(A)=\P(A\cap V=1)+\P(A\cap V=0)=\\
=\P(A|V=1)\P(V=1)+\P(A|V=0)\P(V=0)=\\
=0.5\P(A|V=1)+0.5\P(A|V=0).
\end{multline}

О-малые говорят нам\index{о-малые}, что плотности подчиняются такой же формуле, так как многомерная плотность есть вероятность поделить на $ \Delta^{n} $:
\begin{equation}
f(x_{1},\ldots,x_{n})=0.5f(x_{1},\ldots,x_{n}|V=0)+f(x_{1},\ldots,x_{n}|V=1).
\end{equation}

Поэтому совместная функция плотности имеет вид:
\begin{equation}
f(x_{1},\ldots,x_{n})=0.5\cdot 1\cdot 1 \ldots\cdot 1+0.5\cdot 2x_{1}\cdot 2x_{2}\cdot \ldots 2x_{n}=0.5+2^{n-1}x_{1}\cdot \ldots \cdot x_{n}.
\end{equation}
Проверяем вторую смешанную производную логарифма. В силу симметрии достаточно по $ x_{1} $ и $ x_{2} $:
\begin{equation}
\frac{\partial^{2}\ln(f)}{\partial x_{1}\partial x_{2}}=\ldots=\frac{0.5}{f(x_{1},\ldots,x_{2})^{2}}\geq 0.
\end{equation}


Найдём сначала третий пункт. Представим себе событие $A$:
\[
A=\{X_{1}\in [x_{1};x_{1}+\Delta], Y_{1}\in[y_{1};y_{1}+\Delta]\}.
\]
Для него $ \P(A)= 0.5\P(A|V=1)+0.5\P(A|V=0)$. О-малые\index{о-малые} говорят нам, что плотности подчиняются такой же формуле!

Поэтому находим две условные функции плотности $ X_{1} $ и $ Y_{1} $:
\begin{equation}
g(x,y|V=0)=(n-1)\cdot 1 \cdot y^{n-2}
\end{equation}
и
\begin{equation}
g(x,y|V=1)=(n-1)\cdot 2x \cdot 2y \cdot (y^{2})^{n-2}.
\end{equation}
И получаем безусловную:
\begin{equation}
g(x,y)=0.5(n-1)\cdot 1 \cdot y^{n-2}+0.5(n-1)\cdot 2x \cdot 2y \cdot (y^{2})^{n-2}.
\end{equation}

Поскольку $ V $ принимает значения только 0 и 1, то $ \E(V|A)=\P(V=1|A) $. По формуле условной вероятности:
\begin{multline}
\P(V=1|A)=\frac{\P(V=1 \cap A)}{\P(A)}=\frac{\P(A|V=1)\cdot \P(V=1)}{\P(A)}=\\
=\frac{0.5\P(A|V=1)}{\P(A)}.
\end{multline}

И в итоге искомая функция $ v(x,y) $ равна:
\begin{multline}
v(x,y)=\P(V=1|X_{1}=x,Y_{1}=y)=\\
=\frac{0.5(n-1)\cdot 2x \cdot 2y\cdot (y^{2})^{n-2}}{0.5(n-1)\cdot 1 \cdot y^{n-2}+0.5(n-1)\cdot 2x \cdot 2y \cdot (y^{2})^{n-2}}=\\
=\frac{4xy^{n-1}}{1+4xy^{n-1}}.
\end{multline}

\item Проверяем метод «Авось старое решение подойдёт». Строим табличку, как в первой лекции, и видим, что стратегия $ b(x)=x $ нестрого доминирует остальные стратегии. Единственное отличие несущественно. Выиграет ли первый игрок аукцион, зависит от сравнения его ставки и величины $m=b(Y_{k}) $, а не величины $m=b(Y_{1}) $, как в первой лекции.


\item Условие победы первого игрока: $ Y_{k}<a $. В функции прибыли мы можем убрать условие в силу независимости ценностей.
\begin{equation}
\pi(x,b(a))=(x-b(a))\E(1_{Y_{k}<a}|X_{1}=x)=(x-b(a))\P(Y_{k}<a).
\end{equation}

Применяем о-малые\index{о-малые}. Одна величина должна упасть около $ t $, $ (k-1) $ должна оказаться выше $ t $, и $ (n-1-k) $ должно оказаться ниже $ t $:
\begin{equation}
f_{Y_{k}}(t)=(n-1)\cdot C_{n-1}^{k-1}\cdot 1\cdot (1-t)^{k-1}\cdot t^{n-k-1}.
\end{equation}

Значит,
\begin{equation}
\pi(x,b(a))=(x-b(a))\E(1_{Y_{k}<a}|X_{1}=x)=(x-b(a))\int_{0}^{a}f_{Y_{k}}(t) \, dt.
\end{equation}

Получаем дифференциальное уравнение:
\begin{equation}
(x-b(x))f_{Y_{k}}(x)-b'(x)\int_{0}^{x}f_{Y_{k}}(t) \, dt=0.
\end{equation}
Всё.

А дальше можно изолировать интеграл $ \int_{0}^{x}\ldots\, dt $ в правой части, взять производную по $ x $ и избавиться от интеграла. Но это уже относится к решению дифференциального уравнения.


\item Да. Возьмём пару аффилированных случайных величин с положительной \index{аффилированные случайные величины} корреляцией. У неё функция плотности всюду удовлетворяет условию
\begin{equation}
\partial^{2}\ln (f(x_{1},x_{2}))/\partial x_{1}\partial x_{2} \geq 0.
\end{equation}
А теперь на очень-очень маленьком участке нарушим это условие. Случайные величины перестали быть аффилированными. А ковариация от этого изменится очень-очень слабо, то есть останется положительной.

Конкретный пример. Величины $ X_{1} $  и $D$ распределены  равномерно на отрезке $ [0;1] $.
\begin{equation}
X_{2}=D+
\begin{cases}
X_{1}, X_{1}>0.00001 \\
-X_{1}, X_{1}<0.00001
\end{cases}.
\end{equation}


\end{enumerate}


\section{Домашка 3}

Контрольная номер 3 оказалась чересчур сложной, поэтому студентам была предложена вместо нее домашка 3.


\begin{enumerate}


\item Техническая задача.
\begin{enumerate}
\item Выразите $ (a+c)\vee (b+c) $ через $ a\vee b $. Выразите $ (a+c)\wedge (b+c) $ через $ a\wedge b $.
\item Случайные величины $ Z_{1} $, \ldots , $ Z_{n} $ аффилированы между собой. Случайные величины $ W_{1} $, \ldots , $ W_{k} $ — аффилированы между собой. Набор случайных величин $ Z_{1} $, \ldots , $ Z_{n} $ не зависит от набора $ W_{1} $, \ldots , $ W_{k} $. Верно ли, что набор случайных величин $ Z_{1} $, \ldots , $ Z_{n} $, $ W_{1} $, \ldots , $ W_{k} $ аффилирован?
\end{enumerate}


\item Пусть $  V $ — общая ценность товара для двух игроков, равномерна на $ [1;2] $. Величины $ R_{1} $ и $ R_{2} $ — независимы между собой и с $ V $ и равномерны на $ [-0.5;0.5] $. По смыслу: $ R_{1} $ и $ R_{2} $ — это ошибки игроков при подсчете ценности товара $ V $. Игроки получают сигналы $ X_{i}=V+R_{i} $, то есть игроки знают ценность  $ V $ с ошибкой.
\begin{enumerate}
\item Найдите совместную функцию плотности $ X_{1} $ и $ X_{2} $. Верно ли, что $ X_{1} $ и $ X_{2} $ аффилированны?
\item Найдите $ v(x,y)=\E(V|X_{1}=x,Y_{1}=y) $. Найдите равновесие Нэша на аукционе второй цены.
\item Найдите совместную функцию плотности $ X_{1} $ и $ Y_{1} $, $ g(x,y) $
\end{enumerate}

Подсказка: В решении контрольной есть похожая задача. А $ g(x,y) $ можно неплохо упростить пользуясь предыдущей задачей.

Поскольку игроков всего двое, то $ g(x,y)$ — это просто совместная функция плотности $ X_{1} $ и $ X_{2} $.


\item Пусть $ R_{1} $, $ R_{2} $ и $ S $ — равномерны на $ [0;1] $ и независимы. Ценность товара для первого игрока, $ V_{1}=0.8X_{1}+0.2X_{2} $ и для второго — $ V_{2}=0.8X_{2}+0.2X_{1} $. Первый игрок получает сигнал $ X_{1}=S+R_{1} $. Второй игрок получает сигнал $ X_{2}=S+R_{2} $.
\begin{enumerate}
\item Найдите $ g(x,y) $, $ R(y|x) $ и $ v(x,y)=\E(V|X_{1}=x,Y_{1}=y) $
\item Используя предыдущие функции найдите равновесие Нэша на аукционе второй цены, первой цены и кнопочном аукционе
\end{enumerate}

% ??? (занудно) В рамках предыдущей задачи найдите $ g(y|x) $, $ G(y|x) $, $ R(y|x) $, $ R(x|x) $. Найдите равновесие Нэша на аукционе первой цены и кнопочном аукционе.

%Подсказка: Обратите внимание, что в лекции $ X_{i} $ принимает значения на $ [0;1] $. Здесь $ X_{i} \in [0.5;2.5] $, поэтому при подсчете $ G(y|x) $ нижний предел интегрирования не равен 0.


\item Продолжение задачи 2 с контрольной (можно использовать все полученные в ней результаты).
На аукционе продаётся картина, которая равновероятно является «Джокондой» Леонардо да Винчи или её подделкой. За неё торгуются $ n $ покупателей. Ценность картины для всех покупателей одинакова, $ V_{1}=V_{2}=\ldots =V_{n}=V $ и равна 1, если это оригинал и 0, если подделка.

Если $ V=0 $, то сигналы $ X_{i} $ условно независимы и равномерны на $ [0;1] $. Если $ V=1 $, то сигналы $ X_{i} $ условно независимы и имеют функцию плотности $ f(x|V=1)=2x $ при  $x\in [0;1] $

\begin{enumerate}
\item Найдите равновесие Нэша на аукционе второй цены
\item Найдите $ \E(V|X_{1}=x_{1},X_{2}=x_{2},X_{3}=x_{3}\ldots X_{n}=x_{n}) $
\item С помощью предыдущего пункта найдите функции $ b^{n}(x) $,  $ b^{n-1}(x,p_{n}) $  и $ b^{n-2}(x,p_{n-1},p_{n}) $ в равновесии Нэша на кнопочном аукционе
\end{enumerate}


%\item
% Хм: страшные интегралы лезут отовсюду..
%Товар имеет общую ценность $ V $ для трёх игроков, $ V $ равномерно на $ [0;1] $. При фиксированном $ V=v $ сигналы независимы и имеют функцию плостности:
%\begin{equation}
%f(x|v)=2v\cdot x^{2v-1}, \quad x\in[0;1]
%\end{equation}
%\begin{enumerate}
%\item Какой ожидаемый сигнал получают игроки, если $ V=0 $? $ V=0.5 $? $ V=1 $?
%\item Найдите $ g(x,y) $
%\item Найдите $ v(x,y)=\E(V|X_{1}=x,Y_{1}=y) $ и равновесие Нэша на аукционе второй цены
%\item Найдите $ \E(V|X_{1}=x_{1},X_{2}=x_{2},X_{3}=x_{3} )$ и равновесие Нэша на кнопочном аукционе
%\end{enumerate}

%Подсказка: Можно пользоваться тем, что $  $


\end{enumerate}


\section{Решение домашки 3}

\begin{enumerate}


\item
\begin{equation}
(a+c)\vee (b+c)=a\vee b +c
\end{equation}

\begin{equation}
(a+c)\wedge (b+c)=a\wedge b +c
\end{equation}

Да, набор $ Z_{1} $, \ldots , $ Z_{n} $, $ W_{1} $, \ldots , $ W_{k} $ аффилирован. В силу независимости логарифм совместной функции плотности разлагается в сумму логарифмов:
\begin{equation}
\ln f_{Z,W}(z_{1},\ldots ,z_{n},w_{1},\ldots ,w_{k})= \ln f_{Z}(z_{1},\ldots ,z_{n})+\ln f_{W}(w_{1},\ldots ,w_{k})
\end{equation}
И смешанные производные равны либо нулю, либо неотрицательны в силу аффилированности $ Z_{i} $ между собой и $ W_{i} $ между собой.

\item  Поскольку игроков всего двое, то $ g(x,y)$ — это просто совместная функция плотности $ X_{1} $ и $ X_{2} $.

Находим условную совместную плотность:
\begin{equation}
p(x_{1},x_{2}|v)=1, \quad x_{1},x_{2}\in [v-0.5;v+0.5]
\end{equation}

Значит:
\begin{equation}
p(x_{1},x_{2},v)=1, \quad x_{1},x_{2}\in [v-0.5;v+0.5],v\in [1;2]
\end{equation}

Заметим, что область, где плотность положительна, можно описать условием:
\begin{equation}
v\in [(x_{1}-0.5)\vee (x_{2}-0.5); (x_{1}+0.5)\wedge (x_{2}+0.5)]=[v_{min};v_{max}]
\end{equation}

Интегрируем по $ v $ и получаем:
\begin{equation}
p(x_{1},x_{2})=\int_{v_{min}}^{v_{max}} 1 dv= v_{max}-v_{min}=x_{1}\wedge x_{2}-x_{1}\vee x_{2}+1
\end{equation}

\begin{multline}
\E(V|X_{1}=x_{1}, X_{2}=x_{2})=\int v p(v|x_{1},x_{2})dv=\\
\int v \frac{p(x_{1},x_{2},v)}{p(x_{1},x_{2})} dv=\frac{\int v p(x_{1},x_{2},v) dv }{p(x_{1},x_{2})}
\end{multline}

В числителе:
\begin{equation}
\int_{v_{min}}^{v_{max}}vdv=\frac{v_{max}^{2}-v_{min}^{2}}{2}
\end{equation}

Значит, в итоге:
\begin{equation}
v(x_{1},x_{2})=\frac{v_{max}^{2}-v_{min}^{2}}{2\cdot (v_{max}-v_{min})}=\frac{v_{max}+v_{min}}{2}= \frac{x_{1}\wedge x_{2}+x_{1}\vee x_{2}}{2}
\end{equation}

Равновесие Нэша на аукционе второй цены:
\begin{equation}
v(x,x)=x
\end{equation}


\item Игроков всего два, значит, $ g(x,y) $ — просто совместная функция плотности $ X_{1} $ и $ X_{2} $.

\begin{equation}
p(x_{1},x_{2}|s)=1\cdot 1, \quad x_{1},x_{2}\in [s;s+1]
\end{equation}

Следовательно:
\begin{equation}
p(x_{1},x_{2},s)=1\cdot 1, \quad x_{1},x_{2}\in [s;s+1], s\in [0;1]
\end{equation}

Заметим, что область, где плотность положительна, можно описать условием:
\begin{equation}
s\in [x_{1}\vee x_{2}-1; x_{1}\wedge x_{2}]=[s_{min};s_{max}]
\end{equation}

Интегрируем по $ s $ и получаем:
\begin{equation}
p(x_{1},x_{2})=\int_{s_{min}}^{s_{max}} 1 ds= s_{max}-s_{min}=x_{1}\wedge x_{2}-x_{1}\vee x_{2}+1
\end{equation}

Плотность обращается в ноль за пределами участка $ 0\leq x_{1},x_{2}\leq 2 $, $ x_{1}-1\leq x_{2} \leq x_{1}+1 $.

Чтобы найти $ R(y|x) $ вспоминаем что это такое:
\begin{equation}
R(y|x)=\frac{g(x,y)}{\int_{0}^{y}g(x,t)dt}
\end{equation}

Возникает четыре случая для $ R(y|x) $\ldots

%\begin{equation}
%R(y|x)=
%\begin{cases}
%\frac{1+y-x}{y+0.5y^{2}-xy}, \quad x<1,y<x \\
%\frac{1+x-y}{-2x^{2}+y+yx-0.5y^{2}}, \quad x<1,y>x \\
%\frac{1+y-x}{y+0.5y^{2}-xy+0.5x^{2}+0.5-x}, \quad x>1,y<x \\
%\frac{1+x-y}{0.5+y+yx-0.5y^{2}-x-x^{2}+0.5x^{2}}, \quad x>1,y>x
%\end{cases}
%\end{equation}

%Нам нужна $ R(x|x) $. Находим, что $ g(x,x)=1 $, и $ g(x,t)=t-x+1 $ при $ t<x $ .

%\begin{equation}
%R(x|x)=
%\begin{cases}
%\frac{1}{x-0.5x^{2}}, \quad x<1 \\
%2, \quad x>1
%\end{cases}
%\end{equation}

К сожалению, в явном виде хорошего мало. Стандартная максимизация с чудо-заменой дает дифференциальное уравнение:
\begin{equation}
(0.8x-b'(x))\int_{0}^{x}p(x,x_{2})dx_{2}+x-b(x)=0
\end{equation}

Возникает два случая из-за ломаной $ p(x_{1},x_{2}) $\ldots

Если $ x\in [0;1] $, то:
\begin{equation}
(0.8x-b'(x))\cdot (x-0.5x^{2})+x-b(x)=0
\end{equation}
Из этого уравнения надо выбрать решение с $ b(0)=0 $.

Если $ x\in [1;2] $, то:
\begin{equation}
(0.8x-b'(x))\cdot 0.5+x-b(x)=0
\end{equation}
Из этого уравнения надо выбрать решение непрерывно склеивающееся с первым в точке $x=1$.


Находим $ v(x,y) $:
\begin{equation}
v(x,y)=\E(V_{1}|X_{1}=x,Y_{1}=y)=\E(V_{1}|X_{1}=x,X_{2}=y)=0.8x+0.2y
\end{equation}

Равновесие Нэша на аукционе второй цены:
\begin{equation}
b(x)=v(x,x)=x
\end{equation}
Кнопочный аукцион совпадает с аукционом второй цены.


\item В решении контрольной 3 мы получили результат:
\begin{equation}
v(x,y)=\frac{4xy^{n-1}}{1+4xy^{n-1}}
\end{equation}

Следовательно, равновесие Нэша на аукционе второй цены:
\begin{equation}
b(x)=v(x,x)=\frac{4x^{n}}{1+4x^{n}}
\end{equation}

Можно отметить, что функция растет с ростом $ x $ и падает с ростом $ n $.

Теперь рассмотрим $ A=\{X_{1}\in[x_{1};x_{1}+\Delta] \cap \ldots  \cap X_{n}\in[x_{n};x_{n}+\Delta]\} $. Как и в решении задачи с контрольной:
\begin{multline}
\E(V|A)=\P(V=1|A)=\frac{\P(V=1 \cap A)}{\P(A)}=\\
=\frac{\P(A|V=1)\cdot \P(V=1)}{\P(A)}=\frac{0.5\P(A|V=1)}{\P(A)}
\end{multline}

Согласно методу о-малых аналогичная формула справедлива для плотностей:
\begin{multline}
\E(V|X_{1}=x_{1},X_{2}=x_{2},\ldots ,X_{n}=x_{n})=\\
=\frac{0.5\cdot 2^{n}\Pi_{i=1}^{n}x_{i}}{0.5+0.5\cdot 2^{n}\Pi_{i=1}^{n}x_{i}}
=\frac{2^{n}\Pi_{i=1}^{n}x_{i}}{1+ 2^{n}\Pi_{i=1}^{n}x_{i}}
\end{multline}

Теперь частично находим стратегию на кнопочном аукционе:
\begin{equation}
b^{n}(x)=\frac{2^{n}x^{n}}{1+2^{n}x^{n}}
\end{equation}

Если все игроки используют эту функцию, то чтобы игрок вышел на цене $ p $ ценность должна равняться:
\begin{equation}
x=\frac{1}{2}\left(\frac{p}{1-p} \right)^{1/n}
\end{equation}

Подставляя один такой $ x $ в ожидаемую ценность получаем:
\begin{equation}
b^{n-1}(x,p_{n})=\frac{2^{n-1}x^{n-1}\left(\frac{p_{n}}{1-p_{n}} \right)^{1/n}}{1+2^{n-1}x^{n-1}\left(\frac{p_{n}}{1-p_{n}} \right)^{1/n}}
\end{equation}

Если второй выходит на цене $ p_{n-1} $, то его ценность была равна:
\begin{equation}
x=\frac{1}{2}\left(\frac{p_{n}}{1-p_{n}} \right)^{1/n(n-1)}\left(\frac{p_{n-1}}{1-p_{n-1}} \right)^{1/(n-1)}
\end{equation}

Значит:
\begin{equation}
b^{n-2}(x,p_{n-1},p_{n})=\frac{2^{n-2}x^{n-2}\left(\frac{p_{n}}{1-p_{n}} \right)^{1/n(n-1)}\left(\frac{p_{n-1}}{1-p_{n-1}} \right)^{1/(n-1)}}{1+2^{n-2}x^{n-2}\left(\frac{p_{n}}{1-p_{n}} \right)^{1/n(n-1)}\left(\frac{p_{n-1}}{1-p_{n-1}} \right)^{1/(n-1)}}
\end{equation}


%\item
% Хм: страшные интегралы лезут отовсюду..
%Товар имеет общую ценность $ V $ для трёх игроков, $ V $ равномерно на $ [0;1] $. При фиксированном $ V=v $ сигналы независимы и имеют функцию плостности:
%\begin{equation}
%f(x|v)=2v\cdot x^{2v-1}, \quad x\in[0;1]
%\end{equation}
%\begin{enumerate}
%\item Какой ожидаемый сигнал получают игроки, если $ V=0 $? $ V=0.5 $? $ V=1 $?
%\item Найдите $ g(x,y) $
%\item Найдите $ v(x,y)=\E(V|X_{1}=x,Y_{1}=y) $ и равновесие Нэша на аукционе второй цены
%\item Найдите $ \E(V|X_{1}=x_{1},X_{2}=x_{2},X_{3}=x_{3} )$ и равновесие Нэша на кнопочном аукционе
%\end{enumerate}

%Подсказка: Можно пользоваться тем, что $  $


\end{enumerate}



\chapter{Язык механизмов}
Как выглядит аукцион, если не вдаваться в детали? Природа случайно раздает игрокам сигналы. Затем игроки делают ставки. Затем правила аукциона определяют, кто получает товар и кто сколько платит.

Если не обращать внимания на слово «аукцион», то так выглядят абсолютно все задачи, где нужно принять решение. И в этой лекции мы покажем, как аукцион второй цены неплохо справляется со всеми этими задачами!





\section{Описание всех задач на языке механизмов}

Рассмотрим несколько примеров. Во всех примерах нужно выбрать одно решение из нескольких возможных и определить, кто сколько платит. В табличках будут находиться полезности игроков в зависимости от принятого решения. Вопрос оплаты решения мы в табличках освещать не будем. Мы также забудем пока про то, что игроки получают какие-то сигналы.\index{язык теории механизмов}


\begin{myex} Аукцион. Есть три игрока и один товар. Ценность товара для них равна $ V_{1} $, $ V_{2} $ и $ V_{3} $. В зависимости от принятого решения полезности игроков имеют вид:

\begin{tabular}{c|p{2.2 cm}p{2.2 cm}p{2.2 cm}p{2.2 cm}}
& Отдать товар игроку 1 & Отдать товар игроку 2 & Отдать товар игроку 3 & Оставить товар у продавца \\
\hline
Игрок 1 & $ V_{1} $ & 0 & 0 & 0\\
Игрок 2 & 0 & $ V_{2} $ & 0 & 0\\
Игрок 3 & 0 & 0 & $ V_{3} $ & 0\\
\end{tabular}

Можно рассмотреть различные вариации этой задачи. Например, добавить продавца как игрока или убрать решение «Оставить товар у продавца» из списка возможных.

\end{myex}


\begin{myex} \label{bridge}\index{задача!о постройке моста}
Общественное благо. Есть два города, $ A $ и $ B $, на разных берегах реки. Полезность моста для их жителей равна $ V_{A} $ и $ V_{B} $. Есть ещё администрация области, для которой мост обойдется в сумму $ c $.


\begin{tabular}{c|cc}
& Построить мост & Не строить мост \\
\hline
Жители города A & $ V_{A} $ & 0 \\
Жители города B & $ V_{B} $ & 0 \\
Администрация & $-c$ & 0 \\
\end{tabular}

Если администрация тратит не свои деньги, а, скажем, деньги из какого-то бюджета, которые ни на что, кроме моста, потратить нельзя, тогда её как игрока можно не учитывать, так как ей все равно, строить мост или не строить.

\end{myex}

\begin{myex} \label{pizza} Разносчик пиццы.\index{задача!разносчика пиццы} Есть два клиента, A и B. От ресторана до A ехать $ a $ минут, до B — $ b $ минут. От A до B ехать $ c $ минут. Разносчик может сначала посетить клиента A и затем клиента B, а может и наоборот. Удовольствие клиента от пиццы равно времени доставки со знаком минус.


\begin{tabular}{c|cc}
& сначала к А, потом к B & сначала к B, потом к A \\
\hline
Клиент А & $-a$ & $-b-c$ \\
Клиент B & $-a-c$ & $-b$ \\
\end{tabular}

При желании можно учесть и полезность разносчика пиццы. Например, как общее время доставки со знаком минус. Это будет другая игра. Наш случай означает, что разносчику все равно, сколько тратить на дорогу. Скажем, его в ресторане все равно нагрузили бы какой-нибудь работой, если бы он приехал раньше. Можно ещё добавить решения «Ехать только к А», «Ехать только к В» и «Не ехать ни к кому». Но мы будем считать, что пицца безумно вкусна и эти варианты даже не рассматриваются клиентами.

\end{myex}


\begin{myex} \label{sm_posuda}\index{задача!о мытье посуды} Мама сказала Саше и Маше помыть посуду и подмести пол. До\-пус\-тим, что неудовольствие от мытья посуды для каждого из них равно $ (-a) $, а от подметания пола — $ (-b) $.

\begin{tabular}{c|p{2.2 cm}p{2.2 cm}p{2.2 cm}p{2.2 cm}}
& Саша моет посуду, Маша — пол & Саша моет пол, Маша — посуду & Всё моет Саша & Всё моет Маша \\
\hline
Саша & $ -a $ & $ -b $ & $ -a-b $ & $ 0 $ \\
Маша & $ -b $ & $ -a $ & $ 0 $ & $ -a-b $ \\
\end{tabular}

\end{myex}


Итак, любой механизм решения задачи должен состоять из двух правил: правила, которое говорит, какое решение должно быть принято, и правила, которое говорит, кто и сколько платит.

Некая дополнительная сложность состоит в том, что в реальности часто применяются случайные механизмы решения этих задач. В частности, Саша и Маша могут просто подкинуть монетку, чтобы принять решение. Поэтому правило выбора будет говорить, какими должны быть вероятности принятия каждого из решений.

Итак, для описания механизмов нам понадобятся множества:

\begin{enumerate}
\item $ T_{i} $ — множество всех возможных сигналов\index{множество!возможных сигналов}, которые природа может послать игроку $ i $. В наших трёх предыдущих лекциях — множество возможных значений случайной величины $ X_{i} $. Ещё говорят, множество возможных типов игрока $ i $.

Пример. Пусть каждый игрок знает своё $ X_{i} $. Величины $ X_{i}  $ равномерны на $ [0;1] $. В этом случае $ T_{i}=[0;1] $. Число $ x_{1} $, конкретный сигнал, который получил первый игрок — это элемент из $ T_{1} $.
\item $ B_{i} $ — множество\footnote{~Не путайте с другими обозначениями! $ b_{i} $ — это конкретная ставка игрока $ i $, число; $ Bid_{i} $ — ставка игрока как случайная величина; $ b(x) $ — функция, которая говорит, какую ставку делать в зависимости от сигнала.}  всех возможных ходов игрока $ i $.

Например, для аукциона первой цены это список возможных ставок игрока $ i $, $ B_{i}=[0;+\infty) $. Число $ b_{1} $, конкретная ставка, которую сделал первый игрок,~— это элемент из $ B_{1} $.

\item $ B=B_{1}\times B_{2}\times \ldots B_{n} $ — декартово произведение множеств ходов отдельных игроков. Множество $B$ — это набор всех возможных сочетаний ходов для наших игроков.\index{множество!возможных ходов}

\item $ T=T_{1}\times T_{2}\times \ldots T_{n} $ — декартово произведение множеств типов отдельных игроков. Множество $T$ — это набор всех возможных сочетаний типов для наших игроков.\index{множество!возможных типов игрока}

\item $ \Delta $ — множество вероятностных распределений на списке решений.\index{множество!вероятностных распределений}

Звучит страшно, но достаточно привести в пример пару элементов из $ \Delta $, чтобы всё стало понятнее. Если мы выбираем между решениями $ a $, $ b $ и $ c $, то элементами $ \Delta $, например, будут: $ \{ $Принять решение $ a\} $, $ \{ $Принять решение $ b $ с вероятностью 0.1 и решение $c$ с вероятностью 0.9 $ \}$.

С математической точки зрения, если у нас $ k $ возможных решений, то $ \Delta $ — это все возможные векторы вероятностей:
\begin{equation}
\Delta=\left\{(p_{1},p_{2},\ldots,p_{k})|\forall p_{i}\geq 0, \sum p_{i}=1 \right\}.
\end{equation}

Можно считать, что $ \Delta $ — это множество случайных величин, значением которых является одно из решений.
У нас, правда, $ \Delta $ уже использовалась для обозначения маленького числа. Но из контекста всегда понятно, подразумевается ли под $ \Delta $ число или множество вероятностных распределений.

\end{enumerate}

В этих обозначениях, например, стратегия $ i $-го игрока — это функция $ s_{i}:T_{i}\to B_{i} $. Кстати, равновесие Нэша — это набор стратегий по одной от каждого игрока, то есть это функция $ NE:T\to B $. Действительно, если заданы типы всех игроков и задано равновесие Нэша, то мы можем понять, какие ходы будут сделаны. Естественно, равновесие Нэша — это не произвольная такая функция — нужно ещё сказать, что никому не будет выгодно отклоняться, если все игроки расскажут друг другу свои стратегии.

А механизм — это правила игры:

\begin{mydef} \indef{Механизм}\index{механизм}. Описание механизма состоит из трёх пунктов:
\begin{enumerate}
\item $ B_{i} $ — список возможных ходов игрока $ i $.
\item $Q:B\to \Delta  $ — правило распределения: функция, которая определяет вероятности принятия каждого решения в зависимости от ходов игроков.
\item $M:B\to \mathbb{R}^{n}  $ — правило платежей: функция, которая определяет, кто и сколько платит в зависимости от ходов игроков.
\end{enumerate}
\end{mydef}

Вообще говоря, список ходов $ B_{i} $, предлагаемый игроку $ i $, может быть произвольным и никак не связанным со списком $ T_{i} $ возможных состояний игрока $ i $.

\begin{myex}
Три игрока и один товар. Для игрока $ i $ товар имеет ценность $ X_{i} $, каждый игрок знает своё $ X_{i} $. Величины $ X_{i}  $ независимы и равномерны на $ [0;1] $.

В этом примере $ T_{i}=[0;1] $. Но это никак не ограничивает нас во множестве ходов. Например, мы можем попросить наших трёх игроков одновременно в разных комнатах станцевать вальс. В этом случае $B_{i} =\{ $Множество возможных вальсов$\}$.
\end{myex}

Подобные механизмы использует Дед Мороз в детском саду: «Кто расскажет самый лучший стишок\ldots» и тамада на свадьбе «Кто назовёт больше всего комплиментов невесте\ldots».

Мы же ограничимся прямыми механизмами\index{механизм!прямой}. Прямой механизм прямо спрашивает у каждого игрока: «Ты кто?» Точнее говоря, «Ты какого типа?» или «Какой сигнал послала тебе природа?». Игрок при этом может сказать правду, а может и соврать. В прямом механизме множество возможных ходов совпадает со множеством типов игрока, $ T_{i}=B_{i} $.

\begin{mydef} \indef{Прямой механизм}. В прямом механизме $ B_{i}=T_{i} $ и его описание включает в себя:
\begin{enumerate}
\item[1)] $ Q:B\to \Delta $ — правило распределения: функция, которая определяет вероятности принятия каждого решения в зависимости от объявленных игроками своих типов;\index{правило распределения}
\item[2)] $ M:B\to \mathbb{R}^{n}  $ — правило платежей: функция, которая определяет, кто и сколько платит в зависимости от объявленных игроками своих типов.\index{правило платежей}
\end{enumerate}
\end{mydef}


Рассмотрим подробнее простую ситуацию. Три игрока и один товар. Для игрока $ i $ товар имеет ценность $ X_{i} $, каждый игрок знает своё $ X_{i} $. Величины $ X_{i}  $ независимы и равномерны на $ [0;1] $. В этом случае $ T_{i}=[0;1] $.

В рамках этой ситуации рассмотрим наши три аукциона на языке механизмов.

%Чтобы разом решить проблему совпадения ставок нам потребуется ещё одно несложное обозначение. Пусть игроки сделали ставки $ \vec{b} $.
%\begin{mydef}
%Обозначим $ players(\vec{b},k) $ — вектор из нулей и единиц, где единицы стоят для тех игроков, которые сделали $ k $-ые по величине ставки. Обозначим также $ \#players(\vec{b},k) $ — количество игроков, сделавших $ k $-ую по величине ставку. Саму $ k $-ую по величине ставку обозначим буквой $ M_{k}(\vec{b}) $.
%\end{mydef}

%\begin{myex} Если $ \vec{b}=(2,3,4,5,2,5) $, то  $ players(\vec{b},1)=(0,0,0,1,0,1) $, $ players(\vec{b},2)=(0,0,1,0,0,0) $, $ \#players(\vec{b},1)=2 $, $\#players(\vec{b},2)=1$. Кроме того, $ M_{1}(\vec{b})=5 $, $ M_{2}(\vec{b})=4 $.
%\end{myex}

%Можно, например, связать с помощью скалярного произведения:
%\begin{equation}
%M_{1}=\frac{\vec{b}\cdot players(\vec{b},1) }{\#players(\vec{b},1)}
%\end{equation}

%Или:
%\begin{equation}
%\#players(\vec{b},k)=players(\vec{b},k)\cdot players(\vec{b},k)
%\end{equation}


\begin{myex} Аукцион первой цены. Товар достаётся тому, кто назвал наибольшую цену. Если таких игроков несколько — выбираем победителя из них наугад. Победитель платит названную им самим цену.\index{аукцион!первой цены}

Можно считать, что механизм прямой и множество разрешенных ходов имеет вид $B_{i}=T_{i}=[0;1] $.

Функция $ Q(\vec{b}) $ говорит нам, с какой вероятностью побеждает тот или иной игрок при заданном $ \vec{b} $.

Например, пятый игрок назвал цену выше всех:
\begin{equation}
Q(1,2,3,4,5)=(0,0,0,0,1).
\end{equation}

Или второй и третий игроки назвали наибольшую цену:
\begin{equation}
Q(1,3,3,1,2)=(0,0.5,0.5,0,0).
\end{equation}

То есть функция $ Q $ расставляет равные вероятности для игроков с наибольшей ставкой.

А соответственно функция $ M(\vec{b}) $ говорит, что только победитель платит. Но победитель выбирается наугад среди игроков с наибольшей ставкой, поэтому у всех игроков с наибольшей ставкой есть ожидаемый платеж.

Например, если пятый игрок назвал цену выше всех:
\begin{equation}
M(1,2,3,4,5)=(0,0,0,0,5).
\end{equation}

Или второй и третий игроки назвали наибольшую цену:
\begin{equation}
M(1,3,3,1,2)=(0,1.5,1.5,0,0).
\end{equation}




%Функция распределения задает вероятности получения товара каждым игроком:
%\begin{equation}
%\winpro(\vec{b})=\frac{1}{\#players(\vec{b},1)}\cdot players(\vec{b},1)
%\end{equation}

%Функция $ \mu(\vec{b}) $ должна описывать какой средний платеж должны сделать игроки, если известны их ставки. В нашем случае %платит только победитель. Но победитель выбирается наугад среди игроков с наибольшей ставкой, поэтому у всех игроков с наибольшей ставкой есть ожидаемый платеж:
%\begin{equation}
%\mu(\vec{b})=\frac{M_{1}(\vec{b})}{\#players(\vec{b},1)}\cdot players(\vec{b},1)
%\end{equation}

\end{myex}

\begin{myex} Аукцион второй цены. Товар достаётся тому, кто назвал наибольшую цену. Если таких игроков несколько — выбираем победителя из них наугад. Победитель платит вторую по величине цену.\index{аукцион!второй цены}

Можно считать, что механизм прямой и множество разрешенных ходов имеет вид $B_{i}=T_{i}=[0;1] $.

Функция $ Q(\vec{b}) $ точно такая же, как на аукционе первой цены, так как победитель~— это тот, кто назвал наибольшую ставку.

Функция $ M(\vec{b}) $ отличается. Она говорит, что победитель платит не свою, а вторую по величине ставку.

Например,
\begin{equation}
M(1,2,3,4,5)=(0,0,0,0,4).
\end{equation}

Или
\begin{equation}
M(1,3,3,1,2)=(0,1,1,0,0).
\end{equation}
Единица взялась от деления двойки (второй по величине цены) на двух потенциальных победителей.

\end{myex}

\begin{myex} Кнопочный аукцион.\index{аукцион!кнопочный}

Кнопочный аукцион не является прямым механизмом, так как множество возможных ходов существенно сложнее множества сигналов, которые может получить игрок. Полное описание этого аукциона с выписыванием функций $ Q $ и $ M $ в явном виде  занудно. Поэтому мы ограничимся описанием множеств $ B_{i} $. Зная выбор каждого игрока из его множества $ B_{i}$, мы можем определить, кто выиграл и сколько ему нужно платить, значит, функции $ Q $ и $ M $ существуют.

Представим себе, что первый игрок знает значение  своего сигнала $ X_{1} $. Ему нужно решить, до какой цены жать кнопку, пока кнопку жмут трое, то есть нужно выбрать некое число в диапазоне $ [0;1] $. Ещё нужно решить, до какой цены жать кнопку, когда осталось двое игроков, а самый слабый вышел на цене $ p $, то есть нужно выбрать некую непрерывную на $ [0;1] $ функцию.

В результате $ B_{i}=[0;1]\times C[0;1] $, то есть $ B_{i} $ — это декартово произведение отрезка $ [0;1] $ на множество непрерывных на отрезке $ [0;1] $ функций.

Для примера найдём $ Q(\vec{b})$ и $ M(\vec{b}) $ в точке  $\vec{b}=((0.7,p+p^{2}),(0.5,2p),(0.8,p+p^{3})) $. В данных условиях первым выйдет второй игрок, так как он жмет кнопку до момента времени $ t=0.5 $. Далее останутся первый и третий игрок, которые подставят в свои функции $ p=0.5 $. Значит, первый будет жать кнопку до $ t=0.75 $, а третий — до $ t=0.625 $. Аукцион окончится в $ t=0.625 $ победой первого игрока:
\begin{equation}
Q((0.7,p+p^{2}),(0.5,2p),(0.8,p+p^{3}))=(1,0,0)
\end{equation}
и
\begin{equation}
M((0.7,p+p^{2}),(0.5,2p),(0.8,p+p^{3}))=(0.625,0,0).
\end{equation}

Лишний раз стоит подчеркнуть, что $ (0.7,p+p^{2}) $ — это ход первого игрока. А стратегия игрока — это функция, которая говорит, какой элемент из $ B_{i} $ выбирать в зависимости от полученного сигнала из $ T_{i} $. Стратегия — это правило, которое каждому числу из $ T_{i}=[0;1] $ сопоставляет конкретный ход, то есть пару (число, функция) из $ B_{i} $.

\end{myex}

Мы считаем, что организаторы честно исполняют описанные в механизме функции, даже если после того, как они узнали ставки, им стало выгодно изменить механизм. Это, конечно, не всегда так, и тут уместно сделать небольшое лирическое отступление.

В поезде Москва—Амстердам перегонщиком машин из Белоруссии была рассказана следующая история\index{история перегонщика машин}. Он купил машину на аукционе за 1200 евро, перегнал, дал объявление о продаже за 2000. Звонков много, и всем он сказал, что она уже продана. Потом дал новое  объявление о продаже за 2500. Звонков много, и всем он снова сказал, что она уже продана. И так далее. Продал он её то ли за 3500, то ли за 3800, не помню.

Именно поэтому в реальности на многих аукционах есть стартовая цена, а есть резервная цена, и это не одно и то же. Стартовая цена — это цена, с которой начинаются торги. Естественно, товар не может быть продан ниже стартовой цены. Стартовую цену игроки знают, а резервная цена известна только организаторам аукциона. Если торги не доходят до резервной цены, то товар остаётся у продавца, но он получает информацию о ставках. А если бы продавец начал торг с резервной цены, то он бы не получил информацию о ставках, так как их бы не было.


\section{Правдивость и другие желательные свойства}

Когда механизм принятия решения объявлен игрокам, игроки будут выбирать свои стратегии. Какой механизм выбрать, чтобы в равновесии Нэша были достигнуты определенные цели?

А теперь чудо-замена превращается в чудо-теорему, объясняющую, почему можно изучать только прямые механизмы. Оказывается, любой механизм можно изменить так, чтобы он стал прямым, а игрокам было бы выгодно правдиво декларировать свои типы. При этом ни принимаемое решение, ни платежи никак не изменятся!

\begin{myth} \label{revelation_principle}\index{теорема!о существовании прямого механизма}
Пусть задан произвольный механизм $ (B, Q, M) $ и равновесие Нэша $NE$ в нем. Существует прямой механизм $ (Q', M') $ и равновесие Нэша $NE'$ в нем такое, что:
\begin{enumerate}
\item[1)] при любых типах игроков вероятностям принятия решений и платежи в равновесиях $ NE $ и $ NE' $ совпадают;
\item[2)] в равновесии $ NE' $ игроки правдиво сообщают свои типы.
\end{enumerate}
\end{myth}

\begin{proof}
Давайте вспомним логику нашей чудо-замены\index{чудо-замена}. Для конкретности можно представлять себе аукцион первой цены с симметричными игроками, но это нигде в доказательстве не используется.

Первый игрок, зная $ x $, максимизирует функцию $ \pi_{1}(x,b_{1}) $ по $ b_{1} $. При этом получается некое оптимальное $ b_{1}^{*} $.

Мы говорили: давайте заменим $ b_{1}=b(a) $. И будем максимизировать по $ a $. Не важно, что функция $ b() $ пока ещё неизвестна. Важно, нам заранее известен результат оптимизации по $ a $. С одной стороны, должно быть $ b_{1}^{*}=b(a^{*})$, а с другой стороны, функция $ b() $ — это равновесная стратегия, поэтому $ b_{1}^{*}=b(x) $. И при хороших свойствах $ b() $ из этого следует, что  $ a^{*}=x $.

Что будет, если мы реализуем нашу чудо-замену в реальности? То есть продавец обещает игрокам: «Вы мне говорите $ a $, а я за вас сделаю ставку $ b(a) $ на аукционе». Что тогда оптимально говорить игрокам? Игрокам оптимально говорить $ a^{*}=x $, то есть правдиво сообщать ценность товара для себя.

Если организаторы аукциона будут сначала применять функцию $ b() $ к ходам игроков, а затем использовать старый механизм, то игрокам будет выгодно правдиво декларировать свои типы.

% #TODO картинка пост-обработки ставки


Если игроки несимметричны, то при старом механизме у каждого игрока своя оптимальная стратегия и равновесие Нэша имело вид $ (b_{1}(x),b_{2}(x),\ldots,b_{n}(x)) $. В этом случае в новом прямом механизме ход первого игрока предварительно обрабатывается функцией $ b_{1}() $, ход второго — функцией $ b_{2}() $ и так далее.

Более формально: пусть равновесие $ NE $ имеет вид $ \beta: T\to B $. Смысл написанного: равновесие — это функция, которая каждому набору типов игроков ставит в соответствие набор сделанных ими ходов. Если расписывать детально, то
\[
\beta(x_{1},x_{2},\ldots,x_{n})=(b_{1}(x_{1}),b_{2}(x_{2}),\ldots,b_{n}(x_{n})).
\]

Пусть $ \vec{x} $ — произвольный набор типов игроков, $ \vec{x}\in T $. Определим прямой механизм по принципу: $ Q'(\vec{x}):=Q(\beta(\vec{x})) $ и
$ M'(\vec{x}):=M(\beta(\vec{x}))  $.

При этом автоматически оказывается, что при новом механизме равновесие Нэша $ NE' $ будет иметь вид: $ \beta'(x_{1},x_{2},\ldots,x_{n})=(x_{1},x_{2},\ldots,x_{n}) $. Действительно, если бы какому-то игроку  не было выгодно правдиво декларировать свой тип $ x_{i} $ в этой ситуации, то ему не было бы выгодно использовать стратегию $ b_{i}(x_{i}) $ в исходном непрямом механизме.

А если все игроки правдиво декларируют свои ценности, то и результат применения прямого механизма совпадает с результатом применения исходного непрямого механизма.
\end{proof}

Применим нашу теорему к аукциону первой цены.\index{аукцион!первой цены}

\begin{myex} Когда мы решали аукцион первой цены в простейшем случае, $ X_{i}=V_{i} $, ценности $ X_{i} $ независимы и равномерны на $ [0;1] $, мы установили, что оптимальная стратегия имеет вид $ b(x)=\frac{n-1}{n} x$.

Теорема говорит нам, что можно так поменять правила аукциона, что никому ни хуже, ни лучше не станет, но игроки будут говорить правду. Как это сделать?

Изменим правила аукциона. Победителем по-прежнему будем считать игрока с наибольшей ставкой. А вот платить он будет не ровно свою ставку, а свою ставку, умноженную на $ \frac{n-1}{n} $. Каждый игрок знает, что его ставка будет автоматом помножена на $ \frac{n-1}{n} $. Значит, теперь каждому выгодно говорить правду. А фактические платежи и победитель не поменялись!
\end{myex}

На аукционе второй цены игроки говорят правду и без каких-то поправок.

Теперь вместо того, чтобы изучать произвольные механизмы и произвольные равновесия Нэша в них, можно ограничиться изучением прямых механизмов и равновесий Нэша, в которых игроки правдиво заявляют свой тип. То есть можно никогда не делать разницы между множествами $ B_{i} $ и $ T_{i} $. Кроме как на свадьбе и в детском саду \Smiley.

Существует несколько свойств, которые мы хотели бы видеть у механизмов:

\begin{mydef}
\indef{Правдивость}\index{механизм!правдивый}. Прямой механизм называется правдивым, если в равновесии Нэша игроки правдиво декларируют свои сигналы.
\end{mydef}

\begin{myex}
Будем считать, что выполнены предпосылки теоремы об одинаковой доходности. Аукцион первой цены не правдив, аукцион второй цены правдив.
\end{myex}

\begin{mydef} \indef{Эффективность}\index{механизм!эффективный}. Механизм называется эффективным, если в равновесии Нэша принятое решение максимизирует суммарную полезность всех агентов.
\end{mydef}

\begin{myex} Если выполнены предпосылки теоремы об одинаковой доходности, то аукционы первой и второй цены являются эффективными механизмами. Действительно, в этом случае товар достаётся игроку с максимальной полезностью. Любое другое решение приведет к снижению совокупной полезности.
\end{myex}

Нужно подчеркнуть, что эффективность не учитывает правило платежей! Когда мы говорим об эффективности механизма, мы говорим об эффективности правила распределения.

\begin{mydef}
\indef{Индивидуальная рациональность}\index{механизм!индивидуально рациональный}. Механизм называется индивидуально рациональным, если игроки согласны участвовать в нем добровольно.
\end{mydef}

Индивидуальная рациональность учитывает правило платежей! Когда мы говорим об индивидуальной рациональности, мы учитываем и правило распределения, и правило платежей.

\begin{myex} Если выполнены предпосылки теоремы об одинаковой доходности, то аукционы первой и второй цены индивидуально рациональны, так как ожидаемый выигрыш каждого игрока неотрицательный. А пример \ref{sm_posuda} с Сашей и Машей, которым нужно помыть пол и посуду, не будет индивидуально рациональным, если только мама не предложит им какую-нибудь компенсацию в виде похода в кино.
\end{myex}

\begin{mydef} \indef{Оптимальность}\index{механизм!оптимальный}. Механизм называется оптимальным, если в равновесии Нэша организаторы получают от игроков максимально возможную ожидаемую прибыль.
\end{mydef}

Если у нас есть диктаторские полномочия, то есть мы можем заставить игроков участвовать в аукционе, то, очевидно, мы можем получить сколь угодно большую прибыль. Для этого просто надо потребовать от каждого заплатить достаточно много. Но такой аукцион не будет индивидуально рациональным. Поэтому обычно выбирают оптимальный механизм среди индивидуально рациональных.

Иногда организаторам не нужно заработать денег. Бывает, задача состоит в том, чтобы организовать игру так, чтобы в равновесии Нэша было принято некое желательное решение. Скажем, в примере со строительством моста организаторы могут быть заинтересованы в принятии решения «построить мост». То есть нам интересно принятие нужного решения, а не потоки платежей, которые возникают в связи с этим. В этом случае от механизма может требоваться:

\begin{mydef} \indef{Бюджетная сбалансированность}\index{бюджетная сбалансированность механизма}. Механизм имеет сбалансированный бюджет, если в равновесии Нэша сумма платежей всех игроков равна нулю.
\end{mydef}

Стоит сразу сказать, что не всех этих свойств можно добиться одновременно. Для некоторых задач доказано, что не существует механизма, который был бы эффективен, правдив, индивидуально рационален и бюджетно сбалансирован.


\section{Механизм VCG}

Есть универсальный механизм, который применим ко множеству ситуаций. Этот механизм есть не что иное, как аукцион второй цены. Давайте повнимательнее к нему присмотримся\ldots~Мы сознательно пока забудем про платежи и сосредоточимся только на полезности от получения товара.


%Давайте ответим на очень простой вопрос на аукционе второй цены.

Конкретный пример. У пяти игроков были ценности, равные $ (1,3,7,11,25) $. Ровно такие ставки они и сделали. Победил пятый игрок, который поставил 25.  При этом он получил от товара полезность, равную 25. Остальные четверо получили суммарную полезность 0.

А что произошло бы, если бы пятый не участвовал в аукционе? Тогда победил бы игрок с ценностью 11. При этом четверо игроков (кроме нынешнего пятого) получили бы суммарную полезность, равную 11.

Заметим, что при удалении любого другого игрока сумма полезностей остальных (без учета платежей) не поменялась бы.

Подводим итог. На аукционе второй цены выплата $i$-го игрока равна максимально достижимой суммарной полезности всех игроков, кроме $i$-го, за вычетом текущей суммарной полезности всех игроков, кроме $i$-го.

Механизм Викри—Кларка—Гровса применяет эту идею к любой задаче.

\begin{mydef} \indef{Механизм VCG}. \index{механизм!VCG}Неформальное определение:
\begin{enumerate}
\item Правило распределения: выбрать решение, максимизирующее сумму полезностей.
\item Правило платежей: игрок $ i $ платит суммарную потерю полезности остальных игроков от своего участия в игре.
\end{enumerate}
\end{mydef}

Опишем идею немножко более формально. Пусть множество возможных типов игрока $ i $ — это числовое множество $ T_{i} $, чаще всего отрезок. Мы рассматриваем только прямые механизмы, поэтому множество возможных ходов такое же, $ B_{i}=T_{i}$.

Полезность игрока зависит от его типа и принятого решения, $ v_{i}(X_{i},w) $, то есть типовая табличка имеет вид:

\begin{tabular}{c|cc}
& Решение $ w_{1} $ & Решение $ w_{2} $ \\
\hline
Игрок 1 & $v_{1}(X_{1},w_{1})$ & $v_{1}(X_{1},w_{2})$ \\
Игрок 2 & $v_{2}(X_{2},w_{1})$ & $v_{2}(X_{2},w_{2})$ \\
Игрок 3 & $v_{3}(X_{3},w_{1})$ & $v_{3}(X_{3},w_{2})$ \\
\end{tabular}

%Единственное, что нужно уточнить — это как трактовать игру без игрока $ i $. Подходящая нам трактовка такова — сигнал игрока $ i $ имеет наименьшее возможное значение, то есть $ X_{i}=\alpha_{i} $.

В теореме мы используем обозначения:

\begin{itemize}
\item $b_{i}$ — ход, сделанный игроком $ i $, поскольку механизм прямой, это есть заявленный им тип;
\item $w^{*}$ — решение, максимизирующее суммарную полезность всех игроков при типах $ (b_{1},b_{2},\ldots,b_{n}) $;
\item $w_{-i}^{*} $ — решение, максимизирующее суммарную полезность всех игроков, кроме игрока $ i $, при типах $ (b_{1},b_{2},\ldots,b_{n}) $. Заметим, что в данном случае не важно, какой тип у $ i $-го игрока, поскольку о его полезности не заботятся.
\end{itemize}

Итак:
\begin{mydef} \indef{Механизм VCG}\index{механизм!VCG}, механизм Викри—Кларка—Гровса — это прямой механизм, в котором:
\begin{itemize}
\item множество $ B_{i}=T_{i} $, то есть каждому игроку предлагают сказать свой тип;
\item правило распределения: выбирается то решение $ w $, которое максимизирует сумму полезностей игроков при задекларированных ходах. Если таких решений несколько, то оно выбирается равновероятно:
\begin{equation}
\max_{w} \sum_{i} v_{i}(b_{i},w);
\end{equation}
\item правило платежей: платеж $ i $-го игрока есть разница между: максимально возможной суммарной полезностью остальных игроков, если $ i $-ый не участвует в игре, и суммарной полезностью остальных игроков при текущем решении:
\begin{equation}
M_{i}(\vec{b})=\sum_{j\neq i} v_{j}(b_{j},w_{-i}^{*})-\sum_{j\neq i} v_{j}(b_{j},w^{*}).
\end{equation}
\end{itemize}
\end{mydef}

Мы сейчас докажем, что в механизме VCG стратегия игрока «правдиво декларировать свой тип» нестрого доминирует все остальные. Но перед доказательством разберём пару примеров:

\begin{myex}
Применение механизма VCG к разносчику пиццы, \ref{pizza}.\index{задача!разносчика пиццы}

Сначала определим, какое решение примет механизм VCG. Для этого посчитаем сумму полезностей.

\begin{tabular}{p{3 cm}|p{3 cm}p{3 cm}}
& сначала к А, потом к B & сначала к B, потом к A \\
\hline
Клиент А & $-a$ & $-b-c$ \\
Клиент B & $-a-c$ & $-b$ \\
Сумма полезностей& $-2a-c$ & $-2b-c$ \\
\end{tabular}

То есть механизм VCG выберет наикратчайший путь. Для определенности будем считать, что $ a<b<c $. В этом случае разносчик пиццы едет сначала к А, потом к В.

Теперь определим для случая $ a<b $, кто и сколько платит.

Сколько платит игрок А? Если бы A не было, то оптимальным был бы путь к В напрямую и тот получил бы полезность $ -b $. Сейчас B получает полезность $ -a-c $. Стало быть, игрок А должен заплатить $ -b-(-a-c)=a+c-b $.

Сколько платит игрок В? Если бы B не было, то оптимальным был бы путь к А напрямую и тот получил бы полезность $ -a $. Сейчас А получает полезность $ -a $. Стало быть, игрок В ничего не платит, $ -a-(-a)=0 $.
\end{myex}

\begin{myex}
Применение механизма VCG к строительству моста, \ref{bridge}\index{задача!о постройке моста}. Для примера возьмем конкретные значения $ V_{A} $, $ V_{B} $ и $ c $. Посчитаем сумму полезностей. Зная сумму, определим, какое решение примет механизм VCG.

\begin{tabular}{c|cc}
& Построить мост & Не строить мост \\
\hline
Жители города A & $ +60 $ & 0 \\
Жители города B & $ +90 $ & 0 \\
Администрация & $-100 $ & 0 \\
Сумма & +50 & 0 \\
\end{tabular}

Механизм VCG говорит, что мы строим мост.

Теперь считаем, кто и сколько платит.

Сколько платят жители А? Если бы их не было, то оптимальным было бы решение не строить мост и остальные игроки получили бы полезность 0. Сейчас остальные игроки получают в сумме $ (-10) $. Значит, жители $ A $ платят $ 0-(-10)=10 $.

Сколько платят жители В? Если бы их не было, то оптимальным было бы решение не строить мост и остальные игроки получили бы полезность 0. Сейчас остальные игроки получают в сумме $ (-40) $. Значит, жители В платят $ 0-(-40)=40 $.

Тут мы видим существенный недостаток механизма VCG, а именно бюджетную несбалансированность. Чтобы сделать то, что советует VCG, нужно взять откуда-то с потолка ещё 50 рублей. И это проблема не конкретно механизма VCG, а именно принципиальная несовместимость некоторых желательных свойств механизмов.
\end{myex}

Естественно возникает вопрос, а как механизм VCG добьется того, чтобы максимизировать суммарную полезность? Ведь для этого надо знать истинные полезности игроков! А вдруг они соврали, когда сообщали свои типы? Ответ, естественно, состоит в том, что игрокам выгодно правдиво сообщать свои типы:

\begin{myth}\index{теорема!о правдивости механизма VCG}
При использовании механизма VCG стратегия «Правдиво декларировать свой тип, $ b_{i}(x_{i})=x_{i} $» нестрого доминирует остальные стратегии игрока $ i $.
\end{myth}


\begin{proof}
Посчитаем полезность первого игрока в случае, если игроки делают ходы $ (b_{1},\ldots,b_{n}) $, необязательно правдивые.

Общая полезность с учётом решения и платежа равна:
\begin{multline}
v_{1}(x_{1},w^{*})-M_{1}=v_{1}(x_{1},w^{*})-\left(\sum_{j=2}^{n}v_{j}(b_{j},w_{-1}^{*})-\sum_{j=2}^{n}v_{j}(b_{j},w^{*})\right)=\\
=v_{1}(x_{1},w^{*})+\sum_{j=2}v_{j}(b_{j},w^{*})-\sum_{j=2}^{n}v_{j}(b_{j},w_{-1}^{*}).
\end{multline}

Первый игрок выбирает $ b_{1} $. На что оно влияет? Оно влияет только на исход $ w^{*} $! Первый игрок не в силах изменить ни  $ w_{-1}^{*} $, так как это игра без него, ни ходы $ b_{j} $ остальных игроков, то есть выбор хода $ b_{1} $ не меняет вычитаемого.

Оставшиеся два слагаемых, на которые первый игрок может влиять выбором хода $ b_{1} $, — это не что иное, как суммарная полезность всех игроков, если бы их типы были бы равны $ (x_{1},b_{2},b_{3},\ldots,b_{n}) $.

А правило распределения $ w^{*} $ обеспечивает максимизацию суммарной полезности для задекларированных типов $ (b_{1},b_{2},\ldots,b_{n}) $. Значит, если первый игрок сообщит $ b_{1}=x_{1} $, то правило $ w^{*} $ само максимизирует его полезность.

\end{proof}

Следовательно, профиль стратегий, где все игроки правдиво декларируют свой тип, является равновесием Нэша.


%Есть небольшая вариация этого механизма, механизм AVG.


%Упражнение:
%Опишите механизм AVG для аукциона трёх игроков.



Есть и другой подход к обобщению идеи аукциона второй цены. И этот подход также активно используется.

\begin{myex} Обобщенный аукцион второй цены, Generalized Second Price auction, GSP.\index{аукцион!второй цены обобщённый}

Как Google делает деньги? Около 98\% своих денег Google делает на продаже рекламных ссылок. Есть несколько рекламных мест, которые показываются, когда пользователь запрашивает в поисковике какое-нибудь слово, например «Абрикос». Эти места отличаются по престижности. А именно места отличаются по среднему количеству переходов в час, «кликов» мышкой по ссылке.

У каждого игрока есть мнение о ценности одного «клика». Желающие получить эти рекламные места одновременно делают свои ставки.

Игрок, сделавший самую большую ставку, получает самое престижное место; игрок, сделавший вторую по величине ставку, получает второе по престижности место и так далее.  При этом победитель платит за каждый «клик» не свою ставку, а вторую по величине ставку; игрок, получивший второе по престижности место, платит за каждый «клик» ставку игрока, получившего третье по престижности мес\-то и так далее. Игрок, выигравший самое непрестижное рекламное место, платит за каждый «клик» самую высокую ставку среди игроков, которые ничего не выиграли.

Есть ещё разные тонкости, но основная идея верна. Желающие могут сами поиграться на \url{adwords.google.com}. Например, за сумму в несколько долларов можно сделать страницу-сюрприз, которая будет выводиться первой при поиске на фразу «день рождения Васи Петрова».
\end{myex}


% Но все же общим корнем VCG и GSP является аукцион второй цены.



\section{Оптимальный аукцион}

А сейчас мы увидим, что в простейшей ситуации аукцион второй цены с резервной ценой оптимален.

Пусть каждый покупатель знает ценность товара для себя, то есть  $ X_{i}=V_{i} $. Кроме того, предположим, что каждая ценность имеет регулярное распределение, описываемое функцией распределения $ F_{i}(x) $.

Что произошло бы, если бы никакого аукциона не было и продавец просто предложил бы каждому покупателю купить у него товар по цене $ x $? В этом случае игрок $ i $ согласился бы на покупку, если $ X_{i}>x $, а вероятность этого равна:
\begin{equation}
\P(X_{i}>x)=1-F_{i}(x).
\end{equation}

И при отсутствии аукциона средний доход продавца от $ i $-го игрока был бы равен:
\begin{equation}
TR_{i}=x(1-F_{i}(x)).
\end{equation}

По сравнению с обычной формулой $ TR(Q)=\P(Q)\cdot Q $:
\begin{itemize}
\item $ x $ — это аналог цены $ \P(Q) $;
\item $ 1-F_{i}(x) $ — это аналог количества товара $ Q $.
\end{itemize}

Можно определить предельный доход продавца, $ TR'(Q) $:
\begin{multline}
MR_{i}(x)=\frac{d TR_{i}(x)}{d(1-F_{i}(x))}=x+(1-F_{i}(x))\frac{dx}{d(1-F_{i}(x))}=\\
=x+(1-F_{i}(x))\frac{-1}{f(x)}=x-\frac{1-F_{i}(x)}{f_{i}(x)}.
\end{multline}
Мы воспользовались тем, что производная обратной функции — это единица, делённая на производную исходной.
В результате у нас появилось:
\begin{mydef}
\indef{Предельный доход продавца}\index{предельный доход продавца} — это
\begin{equation}
MR_{i}(x)=x-\frac{1-F_{i}(x)}{f_{i}(x)}.
\end{equation}
\end{mydef}

Эта величина — скорость роста ожидаемого дохода продавца при росте вероятности сделки. Если она положительна, значит, продавец заинтересован в росте вероятности сделки, то есть в снижении $ v $. Если она отрицательна, значит, продавец заинтересован в снижении вероятности сделки, то есть в росте $ v $.

Оказывается, что величина $ MR_{i} $ возникает и при моделировании аукционов. А именно:

\begin{myth}
\begin{equation}
\E(pay_{1}(X_{1}))=\E(q_{1}(X_{1})MR_{1}(X_{1})).
\end{equation}
\end{myth}

\begin{proof}
Мы будем изучать первого игрока и поэтому опустим нижний индекс $ _1 $, чтобы было меньше писанины.

Для доказательства вспомним формулу из теоремы об одинаковой доходности первой лекции:
\begin{equation}
pay(x)=xq(x)-\int_{0}^{x}q(t) \, dt.
\end{equation}

Тогда мы считали, что все игроки используют равновесные стратегии. И при этом трактовали функции как:

\begin{itemize}
\item $ q(x) $ — вероятность того, что первый игрок выиграет, если его ценность равна $ x $;
\item $ pay(x) $ — средняя выплата от первого игрока продавцу, если его ценность равна $ x $.
\end{itemize}

Теперь представим себе, что, как в теореме \ref{revelation_principle},\index{теорема!о существовании правдивого механизма} мы заменили исходный механизм прямым, то есть продавец автоматом обрабатывает поступающие к нему сообщения о типах равновесными функциями $ b_{i} $. Тогда получается новая трактовка старых функций:

\begin{itemize}
\item $ q(x) $ — вероятность того, что первый игрок выиграет, если сообщит ценность $ x $, а остальные правдиво сообщат свои ценности;
\item $ pay(x) $ — средняя выплата от первого игрока, если он сообщит ценность $x$, а остальные правдиво сообщат свои ценности.
\end{itemize}

Поехали!
\begin{multline}
\E(pay(X_{1})=\int_{0}^{1} pay(x)f(x) \, dx=\int_{0}^{1} \left( xq(x)-\int_{0}^{x}q(t) \, dt \right) f(x) \, dx=\\
\int_{0}^{1}xq(x)f(x) \, dx-\int_{0}^{1}\int_{0}^{x}q(t) \, dt f(x) \, dx.
\end{multline}

Применим к вычитаемому формулу интегрирования по частям, получаем:
\begin{multline}
\int_{0}^{1}\int_{0}^{x}q(t) \, dt f(x) \, dx=\left. \int_{0}^{x}q(t) \, dt F(x)\right|_{0}^{1}-\int_{0}^{1}q(x)F(x) \, dx=\\
=\int_{0}^{1} q(x) \, dx-\int_{0}^{1}q(x)F(x) \, dx=\int_{0}^{1} q(x)(1-F(x)) \, dx.
\end{multline}

Подставляем полученный результат в исходную формулу:
\begin{multline}
\E(pay(X_{1})=\int_{0}^{1}xq(x)f(x) \, dx-\int_{0}^{1} q(x)(1-F(x)) \, dx=\\
=\int_{0}^{1}q(x)f(x)\cdot \left(x-\frac{1-F(x)}{f(x)} \right) \, dx=\E(q(X_{1})\cdot MR(X_{1})).
\end{multline}
\end{proof}

Польза от этой теоремы в том, что с помощью неё легко определить оптимальный аукцион.


\begin{myth}
\label{th:optimal_structure}\index{теорема!об оптимальном аукционе}\index{аукцион!оптимальный}
Предположим, что функции $ MR_{i}(x)=x-\frac{1-F_{i}(x)}{f_{i}(x)} $ для каждого игрока не убывают. Оптимальный аукцион устроен по принципу:
\begin{itemize}
\item[1.1.] Товар достаётся покупателю с наибольшим $ MR_{i}(X_{i}) $, если оно неотрицательно. Если таких покупателей несколько, то он выбирается из них равновероятно.
\item[1.2.] Если наибольшеe $ MR_{i}(x_{i})<0 $, то товар остаётся у продавца.
\item[2.] Победитель платит минимально возможную ставку, при которой он ещё остался бы победителем:
\begin{equation}
M_{i}=\inf\{ t| MR_{i}(t)\geq \max\{0, MR_{1}(X_{1}), \ldots, MR_n(X_n) \} \}.
\end{equation}
\end{itemize}
\end{myth}

\begin{proof}
Что делает оптимальный аукцион? Он должен максимизировать $ \E(R)=\E(pay_{1}(X_{1}))+\ldots+ \E(pay_{1}(X_{n}))$.

Как мы только что доказали, $\E(pay_{i}(X_{i}))=\E(q_{i}(X_{i})MR_{i}(X_{i}))$. Выбирая правила аукциона, мы не можем влиять на $ MR_{i}(X_{i}) $, так как это характеристика распределения ценностей. Мы можем только влиять на вероятности получения товара каждым из игроков, то есть на функцию $ q_{i}() $.

Предлагаемое правило распределения сделает максимум возможного! Оно помножит на 0 отрицательные $ MR_{i}(X_{i}) $. При наличии положительных $ MR_{i}(X_{i}) $ оно помножит на единицу наибольшее из них, а остальные помножит на 0. Значит, оно максимизирует ожидаемую прибыль продавца.

Остался один вопрос. А сможет ли это правило работать? Ведь, чтобы определить $ MR_{i}(X_{i}) $, надо знать настоящее $ X_{i} $. То есть осталось доказать, что при использовании этого правила игроки правдиво декларируют свои ценности.

Как и в первой лекции, построим сравнительную табличку.  Результат аукциона для первого игрока зависит не только от его собственной ставки, но и от величины $m=\max\{0,MR_{2}(X_{2}),\ldots,MR_{n}(X_{n})\}$:

\begin{tabular}{cccc}
\toprule
Значение $m$ & $\leq MR_{1}(X_{1}-\Delta)$ & $ [MR_{1}(X_{1}-\Delta);MR_{1}(X_{1})] $ & $MR_{1}(X_{1})\leq $ \\
\midrule
$ b_{1}=X_{1} $ & $ X_{1}-M_{1} $ & $ X_{1}-M_{1} $ &  0\\
$b_{1}=X_{1}-\Delta $ & $ X_{1}-M_{1} $ & 0 & 0 \\
\bottomrule
\end{tabular}

Разница только во втором столбце. В этом случае оказывается, что $MR_{1}(X_{1})\geq MR_{j}(X_{j})$ и $MR_{1}(X_{1})\geq 0$. Значит, величина $ X_{1}-M_{1}\geq 0 $.

Аналогично доказывается, что и отклоняться в положительную сторону также невыгодно.

Все платежи в матрице неотрицательные. Это означает, что механизм индивидуально рационален и игроков не надо в него затаскивать принудительно.
\end{proof}

Применим теорему \ref{th:optimal_structure} к случаю симметричных игроков.

\begin{myex} Если все функции $ MR_{i}(x) $ одинаковые и возрастают по $ x $, то товар либо достаётся игроку с наибольшим $ X_{i} $, либо не достаётся никому. Товар не достаётся никому, если\index{теорема!об оптимальном аукционе для симметричных игроков}
\begin{equation}
MR(\max\{X_{i}\})<0.
\end{equation}
Это условие можно записать и как:
\begin{equation}
\max\{X_{i}\}<MR^{-1}(0).
\end{equation}

То есть на аукционе есть победитель, если максимальная ставка достигла отметки $ MR^{-1}(0) $. Сколько платит победитель? Чтобы остаться победителем, минимальная ставка, которую нужно сделать, должна удовлетворять условию $ MR(t)\geq 0 $ и быть больше других ставок. Действительно:
\begin{multline}
M=\inf\{ t| MR(t)\geq \max\{0, MR(X_{1}), \ldots, MR(X_n) \}  \} =\\
= \inf\{ t| MR(t)\geq 0 , t\geq \max\{X_{j}\} \}.
\end{multline}

Таким образом, мы доказали, что для симметричных игроков оптимальным аукционом будет аукцион второй цены с резервной ценой, равной $ r=MR^{-1}(0) $.
\end{myex}


Стоит отметить, что если ценности независимы, но имеют разное распределение, то аукцион второй цены с резервной ценой может не быть оптимальным. Это связано с тем, что на аукционе второй цены побеждает игрок с наибольшей ценностью, а в оптимальном аукционе нужно, чтобы победил игрок с наибольшей $ MR_{i} $. Если фукнции $ MR_{i} $ отличаются, то эти условия могут не совпадать.

\begin{myex} На аукционе $ n $ игроков. Ценности независимы и равномерны на $ [0;1] $. Какими должны быть правила проведения аукциона, чтобы максимизировать ожидаемую доходность продавца?

Находим $ MR(x)$:
\begin{equation}
MR(x)=x-\frac{1-x}{1}=2x-1.
\end{equation}

Функция монотонно возрастает, поэтому оптимальный аукцион — это аукцион второй цены с резервной ценой. Находим цену из уравнения $ MR(r)=0 $. Получаем $ r=0.5 $.
\end{myex}




\section{Спасибо!}

Вот, пожалуй, и всё. Надеюсь, вам понравилось. Спасибо!\index{спасибо}



\section{Задачи}

\begin{enumerate}
\item Найдите $ \E(MR_{i}(X_{i})) $.

\item Рассмотрите задачу разносчика пиццы из лекции, $ a<b<c<1/4 $. Помимо двух\index{задача!разносчика пиццы} основных, также есть варианты: «Ехать только к А», «Ехать только к В» и «Не ехать ни к кому». Полезность заказчика от доставленной пиццы равна 1. Какое решение будет принято и сколько заплатят игроки при применении механизма VCG?

\item Рассмотрите задачу разносчика пиццы\index{задача!разносчика пиццы} с двумя игроками и с учётом самого разносчика пиццы, $ a<b<c<1/4 $, где помимо двух основных также есть варианты: «Ехать только к А», «Ехать только к В» и «Не ехать ни к кому». Полезность заказчика от доставленной пиццы равна 1. Полезность разносчика пиццы равна времени, потраченному на дорогу в одну сторону со знаком минус. Какое решение будет принято и сколько заплатят игроки при применении механизма VCG?

\item Какое решение будет принято, сколько заплатят игроки при использовании механизма VCG в задаче про посуду, \ref{sm_posuda}?

\item Аукцион по продаже интернет-рекламы.\index{аукцион!интернет-рекламы} Для каждого игрока переход по его рекламной ссылке имеет ценность $ V_{i}=X_{i} $. Продаваемые рекламные места отличаются средним количеством кликов в час. Приведите пример, показывающий, что на аукционе GSP в равновесии Нэша игроки не всегда правдиво сообщают свои ценности.

\item Аукцион «Платят все!». \index{аукцион!платят все}Покупатели одновременно делают ставки. Товар достаётся
тому, кто назвал наибольшую ставку, но платят все игроки. Каждый платит свою
ставку. Ценности товара для покупателей имеют независимое регулярное распределение с функцией плотности $ f(x)=2x $ на $ [0;1] $.

Как нужно изменить правила этой игры, чтобы игрокам было выгодно правдиво декларировать свои ценности, но при этом ни один игрок не выиграл и не проиграл?

Подсказка: смотрите список задач к первой лекции.

\item Наследство.\index{задача!о наследстве} Двум сыновьям достался земельный участок в наследство. Отец не хотел, чтобы участок был разделен, поэтому по завещанию установлены следующие правила: два брата одновременно делают ставки. Участок получает тот, кто сделал большую ставку. При этом получивший участок выплачивает свою ставку проигравшему.
Ценности участка независимы и равномерны на $[0; 1]$.

Как нужно изменить правила этой игры, чтобы игрокам было выгодно правдиво декларировать свои ценности, но при этом ни один игрок не выиграл и не проиграл?

Подсказка: смотрите список задач к первой лекции.

\item Аукцион «Победитель платит чужую среднюю». \index{аукцион!победитель платит чужую среднюю}Покупатели одновременно делают
ставки. Товар достаётся тому, кто назвал наибольшую ставку. Победитель платит
среднюю арифметическую ставок остальных игроков. Ценности товара для покупателей независимы и равномерно распределены на [0; 1].

Как нужно изменить правила этой игры, чтобы игрокам было выгодно правдиво декларировать свои ценности, но при этом ни один игрок не выиграл и не проиграл?

\item Два игрока. Ценности независимы и имеют экспоненциальные распределения с параметрами $ \lambda_{1}=1 $ и $\lambda_{2}=2  $. Сигналы совпадают с ценностями,  $ X_{i}=V_{i} $.

\begin{enumerate}
\item Предположим, что аукцион проводится по принципу аукциона второй цены с резервной ценой $ r $.
\begin{enumerate}
\item Найдите равновесие Нэша.
\item В осях $ (X_{1},X_{2}) $ изобразите три множества: товар достаётся первому игроку, товар достаётся второму игроку, товар остаётся у продавца.
\end{enumerate}
\item Разработайте оптимальный аукцион. Будет ли он отличаться от аукциона второй цены?
\begin{enumerate}
\item Найдите равновесие Нэша.
\item В осях $ (X_{1},X_{2}) $ изобразите три множества: товар достаётся первому игроку, товар достаётся второму игроку, товар остаётся у продавца.
\end{enumerate}
\end{enumerate}

\item На аукционе первой цены участвуют\index{аукцион!первой цены} $ n $ потенциальных покупателей. Продается один товар, $ V_{i}=X_{i} $, ценности равномерны на $ [0;1] $ и независимы. Доставка товара по почте стоит 0.1. Доставку оплачивает покупатель. Игроки одновременно решают, делать ли ставки, и если делать, то какие.

\begin{enumerate}
\item Найдите равновесные стратегии и ожидаемую прибыль продавца.
\item Как изменится предыдущий ответ, если доставку оплачивает продавец?
\item Решите аналогичную задачу для аукциона второй цены.
\end{enumerate}


\item Общественное благо наоборот\index{задача!об общественном благе}\index{задача!о деревне Гадюкино}. Рассмотрим задачу насильственного выбора поставщика общественного блага. Жители деревни Малое Гадюкино решили проложить к деревне новую освещённую дорогу от шоссе. Есть два варианта. Вариант А: заставить сделать всё местную администрацию по закону. От этого администрация получит ущерб в 100 тыс. рублей. При этом она и дорогу выложит, и фонари поставит. Вариант Б: воспользоваться тем, что на территории деревни есть два магазинчика. Заставить владельца более крупного магазина оплатить прокладку дороги (60 тыс. рублей), а владельца более мелкого — фонари (30 тыс. рублей). Выгода жителей деревни — случайная величина, $ X $ равномерна на $ [80;180] $ тыс. рублей:

\begin{tabular}{c|ccc}
& Не строить & Строить «под ключ» & Строить по частям \\
\hline
Администрация & 0 & $-100$ & 0\\
Фирма Б1 & 0 & 0 & $-60$\\
Фирма Б2 & 0 & 0 & $-30$\\
Жители & 0 & $X$ & $X$\\
\end{tabular}

Опишите механизм VCG применительно к этой задаче. Требуется описать, кто и сколько платит в зависимости от ставок игроков. Общее описание механизма VCG за ответ не засчитывается.

С какой вероятностью баланс механизма VCG положительный, то есть не потребуется вливать в него деньги?
%Какова вероятность того, что у механизма не сойдется баланс? Каков средний баланс от использования VCG?
%Еще чуть продумать условие\ldots


\item Рассмотрим аукцион $ n $ игроков. Ценности независимы и равномерны на $ [0;1] $, $ X_{i}=V_{i} $. Игроки одновременно делают свои ставки $ b_{i} $. Продавец считает синус каждой ставки, $ \hat{b}_{i}=\sin (b_{i}) $. И проводит обычный аукцион второй цены с реальными ставками, равными $\hat{b}_{i}  $. То есть побеждает тот, у кого $ \hat{b}_{i} $ больше, а платит он вторую по величине $ \hat{b}_{i} $.

Является ли этот механизм правдивым?




\end{enumerate}








\section{Решения задач}
\begin{enumerate}
\item По определению $ MR_{i}(x) $:
\begin{equation}
\E(MR_{i}(X_{i}))=\E(X_{i})-\int_{0}^{1}(1-F(t)) \, dt.
\end{equation}

Далее интегрируем по частям, $ u(t)=1-F(t) $, $ v'(t)=1 $, и получаем $ \E(MR_{i}(X_{i}))=0 $.

\item Табличка:\index{задача!разносчика пиццы}

\begin{tabular}{c|ccccc}
& $ \to A $ & $ \to B $ & $ \to A\to B $ & $ \to B \to A $ & Не ехать \\
\hline
A & $ 1-a $ & 0 & $ 1-a $ & $ 1-b-c $ & 0 \\
B & 0 & $ 1-b $ & $ 1-a-c $ & $ 1-b $ & 0 \\
\end{tabular}

Будет принято решение: $ \to A\to B $. Игрок B не платит ничего (так как полезность A не меняется при отсутствии игрока B). Игрок А платит $ (1-b)-(1-a-c)=a+c-b $.

\item \begin{tabular}{c|ccccc}
& $ \to A $ & $ \to B $ & $ \to A\to B $ & $ \to B \to A $ & Не ехать \\
\hline
A & $ 1-a $ & 0 & $ 1-a $ & $ 1-b-c $ & 0 \\
B & 0 & $ 1-b $ & $ 1-a-c $ & $ 1-b $ & 0 \\
Разносчик & $ -a $ & $ -b $ & $ -a-c $ & $ -b-c $ & 0 \\
\end{tabular}

Будет принято решение: $ \to A\to B $. $ A $ платит $ 2\cdot (a+c-b)>0 $, $ B $ платит $ 2\cdot (b+c-a)>0 $. Разносчик: ничего не платит.

Внимание! Задача «без разносчика» не означает, что его нет и никто пиццу не разносит. Задача «без разносчика» означает, что о нём никто не заботится!

\item Поскольку суммарная полезность во всех случаях одинакова, то механизм VCG допускает принятие любого решения. \index{задача!о мытье посуды}Например, равновероятный выбор из четырёх решений. Или выбор решения «Саша — посуда, Маша — пол». Рассмотрим вариант с равновероятным выбором любого решения. Сколько должна платить Маша? Сейчас Саша получает полезность $-0.5a-0.5b $. Если оставить в игре только Сашу, то при оптимальном решении Саша получает полезность $ 0 $. Выплата Маши равна: $0-(-0.5a-0.5b)=0.5a+0.5b$. В силу симметрии выплата Саши — такая же.

Возникает естественный вопрос: это как же так, Саша и Маша не только посуду моют, но ещё и платят? Есть два ответа. Во-первых, выбор нулевой полезности произволен. Мы с таким же результатом могли увеличить полезность Саши и Маши при каждом исходе на $ a+b $. В этом случае Саша и Маша получали бы неотрицательную полезность. Во-вторых, обратите внимание на трактовку игры «без Маши». Это не означает, что есть тот же объем работ, но сделать его может только Саша. Игра «без Маши» — это тот же объём работ при тех же игроках, но заботимся мы только о Саше.

\item Самый простой пример — с общеизвестными ценностями. Три игрока, ценности кликов для них равны: 20, 10 и 5. Два рекламных места: одно дает 10 кликов в час, другое — 9 кликов в час. Если все говорят правду, то первый игрок получает полезность: $ 10\cdot (20-10)=100 $. Если первый игрок отклонится и сделает ставку 9, то он получит: $ 9\cdot (20-5)=135 $. Значит, правду говорить невыгодно.

\item Мы знаем, что на этом аукционе равновесие Нэша: $ b(x)=\frac{n-1}{n}x^{n} $, смотрите формулу \ref{NE_all_pay} в задачах к первой лекции.

Поэтому правила новой игры должны иметь вид: каждый игрок платит не свою ставку, а свою ставку, возведенную в степень $ n $ и домноженную на $ \frac{n-1}{n} $. Товар достаётся тому, у кого ставка выше.


\item На этом аукционе равновесие Нэша имеет вид $ b(x)=x/3 $. Соответственно, измененные правила игры выглядят так: оба игрока одновременно делают ставки. Участок получает тот, кто сделал большую ставку. Победитель выплачивает проигравшему треть своей ставки.

\item Равновесие Нэша при оригинальных правилах имеет вид: $ b(x)=\frac{2(n-1)}{n}x $. Измененные правила выглядят так: товар достаётся игроку с наибольшей ставкой. Победитель платит $ \frac{2(n-1)}{n}\bar{x}_{-i} $, где $ \bar{x}_{-i} $ — средняя ставка всех игроков, кроме победителя.

\item Для аукциона второй цены: равновесие Нэша — говорить правду.

Для оптимального аукциона: $ MR(x)=x-\frac{1-(1-\exp(-\lambda x)}{\lambda \exp(-\lambda x)}=x-\frac{1}{\lambda} $, $ MR_{1}(x)=x-1 $, $ MR_{2}(x)=x-0.5 $.  Стало быть, правила таковы: если $ b_{1}<1 $ и $ b_{2}<0.5 $, то товар остаётся у продавца. Иначе товар достаётся тому, у кого $ MR $ больше, то есть товар достанется первому, если $ b_{1}-1>b_{2}-0.5 $ или
$b_{2}<b_{1}-0.5$.\index{аукцион!оптимальный}\index{аукцион!второй цены}

\begin{figure}
\hfill
\subfigure[Оптимальный аукцион]{\begin{tikzpicture}
    \draw[very thin,color=gray] (-0.1,-0.1) grid (3.1,3.1);
    \draw[->] (-0.1,0) -- (3.2,0) node[right] {$b_1$};
    \draw[->] (0,-0.1) -- (0,3.2) node[above] {$b_2$};
    \draw[color=black] (1,0)--(1,0.5) ;
    \draw[color=black] (0,0.5)--(1,0.5) ;
    \draw[color=black] (1,0.5)--(3,2.5) ;
    \node[below] at (1,0) {$1$};
    \node[left] at (0,0.5) {$0.5$};
\end{tikzpicture}}
\hfill
\subfigure[Аукцион второй цены]{\begin{tikzpicture}
    \draw[very thin,color=gray] (-0.1,-0.1) grid (3.1,3.1);
    \draw[->] (-0.1,0) -- (3.2,0) node[right] {$b_1$};
    \draw[->] (0,-0.1) -- (0,3.2) node[above] {$b_2$};
    \draw[color=black] (0.8,0)--(0.8,0.8) ;
    \draw[color=black] (0,0.8)--(0.8,0.8) ;
    \draw[color=black] (0.8,0.8)--(3,3) ;
    \node[below] at (0.8,0) {$r$};
    \node[left] at (0,0.8) {$r$};
\end{tikzpicture}}
\hfill
\end{figure}

\item Оплата доставки товара покупателем равносильна плате за участие.\index{аукцион!с оплатой доставки} Смотрим задачи из лекции 3. Если покупатель оплачивает доставку сам, то для него это как плата за участие, но продавец её не получает. Поэтому равновесные стратегии для покупателей такие же, как в аукционе с платой за вход $ w=0.1 $. Величина $ w $ продавцу не достаётся, поэтому его ожидаемый доход уменьшается на $ w(1-\rho) $ и равен:
\begin{equation}
\E(R)=n(n-1)\left(\frac{1}{n(n+1)}-\frac{\rho^{n}}{n}+\frac{\rho^{n+1}}{n+1}\right),\quad \rho=w^{1/n}.
\end{equation}


Если же продавец оплачивает доставку сам, то с точки зрения покупателей  это обычный аукцион. А доход продавца надо уменьшить на $ 0.1 $:
\begin{equation}
\frac{n-1}{n+1}-0.1.
\end{equation}

Если нарисовать эти функции, то при $ n\geq 3 $ получается, что продавцу выгоднее обещать бесплатную доставку!

\item Механизм VCG: ход делают только жители. Пусть $ X\in[80;180]$ — ход жителей. Принимается решение не строить дорогу, если $ X<90 $, и строить дорогу с помощью двух фирм Б1 и Б2, если $ X\geq 90 $.

Администрация платит 90, если дорога строится, и $ X $, если дорога не строится.

Фирма Б1 платит 0, если дорога строится, и $ X-30 $, если дорога не строится.

Фирма Б2 платит 0, если дорога строится, и $ X-60 $, если дорога не строится.

Жители получают 90, если дорога строится, и 0, если дорога не строится.

Баланс всегда неотрицательный.


\item На отрезке $ [0;1] $ синус является монотонно возрастающим. Без применения синуса стратегия «Говорить правду» нестрого доминировала остальные, то есть давала больше денег. С применением синуса стратегия «Говорить правду» дает больший синус количества денег. Но это одно и то же. Значит, стратегия «Говорить правду» по-прежнему нестрого доминирует остальные.

Можно составить табличку для сравнения ходов $ b=x $ и $ b=x-\Delta $.


\end{enumerate}


\section{Контрольная 4}

\begin{enumerate}

\item На аукционе участвуют $ n $ игроков. Ценности независимы, $ X_{i}=V_{i}$. Пусть функция распределения сигналов имеет вид $ F(x)=x^{a} $ на $ [0;1] $, где $ a $ — это некая константа, $ a\geq 1 $.
\begin{enumerate}
\item Найдите $ MR(x) $. Является ли $ MR(x) $ возрастающей?
\item Постройте оптимальный аукцион. \index{аукцион!оптимальный}
\end{enumerate}

\item Петя переезжает на новую квартиру, поэтому продаёт свои старые шкаф и комод (варианта взять их с собой у него нет).  Потенциальных покупателей двое. Первый покупатель знает значение $ X_{1} $, второй — значение $ X_{2} $. Величины  $ X_{1} $ и  $ X_{2} $ независимы и равномерны на $ [0;1] $. Полезности первого игрока: от шкафа — $ 0.5 $, от комода — $ 0.8X_{1} $, от шкафа и комода — $ 0.5+X_{1} $. Полезности второго игрока: от шкафа — $ 0.8 $, от комода — $ X_{2} $, от шкафа и комода — $ 0.8+0.8X_{2}$. \index{задача!о продаже шкафа и комода}
\begin{enumerate}
\item Чётко опишите механизм VCG применительно к этой задаче.
\item Какова средняя прибыль продавца при использовании механизма VCG?
\end{enumerate}

\item Есть $ n $ городов. Рядом с одним из них нужно построить мусоросжигательный завод\index{задача!о мусоросжигательном заводе}. Жители города, рядом с которым будет построен завод, получат отрицательную полезность $ U_{i}=-X_{i} $. Остальные получат полезность 0. Величины $ X_{i}\sim U[0;1] $ и независимы. Каждый город знает своё $ X_{i} $.
\begin{enumerate}
\item Опишите механизм VCG применительно к этой задаче. То есть предполагается, что игроки объявляют числа $ b_{i}\in [0;1] $ и механизм должен определять, у какого города строить завод и какие платежи должны сделать игроки в зависимости от $ b_{i} $.
\item Выпишите функцию плотности для компенсации, которую получают жители города, рядом с которым будет построен мусоросжигательный завод.
\item Сходится ли баланс у механизма VCG в этом случае? Если нет, то сколько в среднем нужно вложить средств извне в этот механизм?
\item Что больше, компенсация или ущерб от строительства завода в механизме VCG?
\end{enumerate}


\item Кнопочный аукцион и три игрока. Ценности $ V_{1} $, $ V_{2} $ и $ V_{3} $ равномерны на $ [0;1] $ и независимы. Первый и второй игроки знают значение своих ценностей, то есть $ X_{1}=V_{1} $ и $ X_{2}=V_{2} $. А третий игрок не знает значения своей ценности, а знает только закон распределения. \index{аукцион!кнопочный}
\begin{enumerate}
\item Что собой представляют стратегии игроков в этом случае? Почему их можно упростить?
\item Найдите равновесие Нэша.
\end{enumerate}


\end{enumerate}


\section{Решение контрольной 4}

\begin{enumerate}

\item
\begin{equation}
MR(x)=x-\frac{1-x^{a}}{ax^{a-1}}=x\left(1+\frac{1}{a}\right)-\frac{1}{ax^{a-1}}
\end{equation}
Даже без производной видно, что функция возрастает. Оптимальным будет аукцион второй цены с резервной ценой:
\begin{equation}
r=\left(\frac{1}{a+1}\right)^{1/a}
\end{equation}

\item Составляем табличку:

\begin{tabular}{c|cccc}
& (Ш,К) & (К,Ш) & (КШ,-) & (-,КШ) \\
\hline
Покупатель 1 & 0.5 & $ 0.8X_{1} $ & $ 0.5+X_{1} $ & 0 \\
Покупатель 2 & $ X_{2} $ & 0.8 & 0 & $ 0.8+0.8X_{2} $ \\
Сумма & $ 0.5+X_{2} $& $ 0.8+0.8X_{1} $ & $ 0.5+X_{1} $ & $ 0.8+0.8X_{2} $ \\
\end{tabular}

Покупатели одновременно декларируют свои значения $ X_{i} $. Мы знаем, что в механизме VCG им будет оптимально говорить правду. Механизм VCG максимизирует сумму полезностей. В данном случае мы замечаем, что $ 0.8+0.8X_{1}>0.5+X_{1} $ при любых $ X_{1} \in [0;1]$. И аналогично для $ X_{2} $. Поэтому правило выбора решения имеет вид:

Если $ X_{1}>X_{2} $, то комод — первому, и шкаф — второму. Если $ X_{1}<X_{2} $, то комод и шкаф — второму.

Осталось правило платежей:

Если $ X_{1}>X_{2} $, то первый платит $ 0.8X_{2} $, а второй — $ 0.5+0.2X_{1} $.

Если $ X_{1}<X_{2} $, то первый платит 0, а второй — $ 0.5+X_{1} $.

Получаем выручку продавца:
\begin{equation}
R=(0.5+0.2X_{1}+0.8X_{2})1_{X_{1}>X_{2}}+(0.5+X_{1})1_{X_{1}<X_{2}}
%=0.5+(0.2X_{1}+0.8X_{2})1_{X_{1}>X_{2}}+X_{1}(1-1_{X_{1}<X_{2}})=0.5+X_{1}+(0.8X_{2}-0.8X_{1})1_{X_{1}>X_{2}}
\end{equation}

Находим:
\begin{equation}
\E(X_{1}1_{X_{1}>X_{2}})=\int_{0}^{1}\int_{0}^{x_{1}}x_{1} \cdot 1 \cdot dx_{2}dx_{1}=1/3
\end{equation}

Аналогично, $ \E(X_{1}1_{X_{1}<X_{2}})=1/6 $.

Получаем, что средняя выручка равна:
\begin{equation}
\E(R)=0.5\cdot \frac{1}{2}+0.2\cdot \frac{1}{3}+0.8\cdot \frac{1}{6}+0.5\cdot \frac{1}{2}+\frac{1}{6}=\frac{13}{15}
\end{equation}

\item  Каждый город одновременно декларирует свой ущерб.

Правило принятия решения: завод построить рядом с городом, сообщившим наименьший ущерб.

Правило платежей: Город рядом с которым строят завод должен получить компенсацию в размере минимума ущербов остальных городов. Остальные города ничего не платят и не получают.

Автоматически получаем, что механизм VCG требует вливания средств извне. так как компенсация равна не самому маленькому ущербу, а ущербу второму по малости, то: компенсация всегда больше ущерба.

Функция плотности: $ p(y)=n\cdot 1\cdot (n-1)y(1-y)^{n-2} $.

Средняя компесация равна (для взятия интеграла можно сделать замену $ z=1-y $):
\begin{equation}
\E(K)=\int_{0}^{1}y\cdot n(n-1)y(1-y)^{n-2}dy=\frac{2}{n+1}
\end{equation}




\item  Поскольку третий игрок ничего не знает, а только видит, сколько игроков осталось в игре, то его стратегия описывается двумя числами, $ b_{3}^{3} $ и $ b_{3}^{2} $. Эти числа говорят, до какой цены давить кнопку, если в игре осталось три и два игрока.

Стратегия первого игрока описывается тремя функциями: $ b_{1}^{3}(x) $ — до какой цены давить кнопку, если в игре три игрока, $b_{1}^{2a}(x,p)$ — до какой цены давить кнопку, если в игре двое: я и второй; $b_{1}^{2b}(x,p)$ — до какой цены давить кнопку, если в игре двое: я и третий. Стратегия второго игрока имеет такой же вид.

Поскольку ценности независимы, то никакой полезной информации от наблюдения за ценами выхода других игроков мы не получаем. Следовательно, стратегию третьего игрока можно заменить одним числом $ b_{3} $, а стратегию первого — одной функцией $b_{1}(x)$.

Получаем аукцион второй цены. Игроки ориентируются на ожидаемый выигрыш. Поэтому с точки зрения третьего игрока его ценность равна 0.5. то есть равновесие Нэша имеет вид $ b_{3}=0.5 $; $ b_{1}(x)=x $; $ b_{2}(x)=x $.


\end{enumerate}



\section{Догонялка}

Тем, кто по уважительной причине пропустил какую-либо из контрольных предлагается догонялка:

\begin{enumerate}
\item Кнопка «Buy now!». \index{кнопка «Buy now!»}

Два игрока торгуются за товар на кнопочном аукционе с возможностью немедленной покупки товара. Ценности $ X_{i}=V_{i} $ независимы и равномерны на $ [0;1] $. Каждый игрок знает свою ценность $ X_{i} $. Продавец даёт игрокам возможность купить товар немедленно по фиксированной цене $ a $. Подробнее. В начале аукциона текущая цена равна нулю, и оба игрока жмут на свои кнопки. Текущая цена растёт с течением времени. Кто первый отпустил свою кнопку, тот проиграл. В этот момент аукцион заканчивается и победитель получает товар по текущей цене. Но в любой момент пока аукцион не закончился, любой игрок может сказать: «Покупаю по цене $ a $». В этом случае ему достаётся товар по цене $ a $, и аукцион заканчивается.
\begin{enumerate}
\item Что является стратегией игрока на этом аукционе?
\item Найдите равновесие Нэша.
\item Изменится ли ожидаемый доход продавца, если аукцион будет проводится по обычным правилам аукциона второй цены? Применима ли теорема об одинаковой доходности?
\end{enumerate}

\item Есть шесть покупателей. У продавца две чудо-швабры. \index{задача!о продаже чудо-швабр}Каждый покупатель хочет только одну чудо-швабру. Продавец решил продавать эти две чудо швабры путем двух последовательных аукционов первой цены, на каждом из которых будет выставляться одна чудо-швабра. Каждый игрок знает ценность чудо-швабры для себя, $ X_{i}=V_{i} $. Ценности независимы и равномерны на $ [0;1] $. Ценности не меняются со временем. Когда проводится второй аукцион известна только ставка, которую сделал победитель первого.

\begin{enumerate}
\item Что является стратегией игрока в этой игре?
\item Найдите равновесие Нэша
\item Верно ли, что средние цены на обоих аукционах равны?
\item Какова вероятность того, что на первом аукционе цена будет больше, чем на втором?
\item Изменится ли ожидаемый доход продавца, если чудо-швабры будут продаваться на двух последовательных аукционах второй цены? Применима ли в данном случае теорема об одинаковой доходности или её небольшая вариация?
\end{enumerate}

\item В моделях аукциона первой и второй цены с независимыми, равномерными на $ [0;1] $ ценностями покупателей сравните дисперсию выигрыша продавца.

%Можно ли сказать сделать какой-то вывод\footnote{Этот вопрос является исследовательским. Возможно он очень легкий или наоборот очень сложный — не знаю. Оценивается любое продвижение вперед.}  для произвольного регулярного распределения ценностей?

\item Может ли цена расти с ростом предложения?\index{аукцион!кнопочный}\index{нарушение закона предложения}

Рассмотрим кнопочный аукцион, в котором участвуют три игрока. Продавец продаёт две одинаковых чудо-швабры. Каждому игроку нужна только одна чудо-швабра. Ценность чудо-швабры для всех игроков одинакова и равна $ V=X_{1}+X_{2}+X_{3} $. Каждый из игроков знает только своё $ X_{i} $. Сигналы $ X_{i} $ независимы и имеют регулярное распределение $ F(t) $ на отрезке $ [0;1] $. Чудо-швабры по одной достаются тем игрокам, кто отпустил кнопку позже всех. При этом платят они за неё цену, на которой отпустил кнопку самый слабый игрок.

\begin{enumerate}
\item Найдите равновесие Нэша.
\item Найдите равновесие Нэша в случае, когда продаётся всего одна чудо-швабра.
\item Существует ли пример распределения $ F(t) $, при котором средняя цена чудо-швабры в  случае двух чудо-швабр выше, чем в случае одной чудо-швабры?
\end{enumerate}



\item Может ли цена расти с падением спроса?\index{нарушение закона спроса}

Рассмотрим кнопочный аукцион, в котором хотят участвовать три игрока. Продается одна чудо-швабра. Ценность чудо-швабры для всех игроков одинакова и равна $ V=X_{1}+X_{2}+X_{3} $. Каждый из трёх потенциальных игроков знает только своё $ X_{i} $. Сигналы $ X_{i} $ независимы и имеют регулярное распределение $ F(t) $ на отрезке $ [0;1] $. Перед началом аукциона продавец случайным образом выбирает одного игрока и говорит: «Ты мне не нравишься, поэтому ты в аукционе не участвуешь». Оставшиеся двое участвуют в аукционе.

\begin{enumerate}
\item Найдите равновесие Нэша
\item Существует ли пример распределения $ F(t) $ при котором средняя цена в случае удаления одного из игроков выше, чем в случае, когда участвуют все трое желающих?
\end{enumerate}

Подсказка: Задача 7 из лекции 3




\end{enumerate}


\section{Подсказки к догонялке}

\begin{enumerate}

\item[4.] Количество чудо-швабр обозначим буквой $k$.

$ k=1 $: $ b^{3}(x)=3x $, $ b^{2}(x,p_{3})=2x+p/3$

$ k=2 $. Поскольку аукцион заканчивается при выходе первого игрока, то стратегия определяется функцией $ b^{3}(x)$.

Поскольку мы такой аукцион не решали, то используем стандартный подход с максимизацией прибыли:
\begin{multline}
\E(Profit_{1}|X_{1}=x,Bid_{1}:=b_{1})=\\
=\E((X_{1}+X_{2}+X_{3}-b(Y_{2}))1_{b_{1}>b(Y_{2})}|X_{1}=x,Bid_{1}:=b_{1})
\end{multline}

Чудо-замена $b_{1}=b(a)$ и независимость $ X_{i} $ дают нам:
\begin{multline}
\pi_{1}(x,b(a))=\E((x+X_{2}+X_{3})-b(Y_{2}))1_{a>Y_{2}})=\\
=x\P(Y_{2}<a)+2\E(X_{2}\cdot 1_{Y_{2}<a})-\E(b(Y_{2})\cdot 1_{Y_{2}<a})
\end{multline}

Сосредоточимся на $\E(X_{2}\cdot 1_{Y_{2}<a}) $:
\begin{multline}
\E(X_{2}\cdot 1_{Y_{2}<a})=\E(X_{2}\cdot 1_{X_{2}\wedge X_{3}<a})=\\
=\E(X_{2}\cdot 1_{X_{2}<a}\cdot 1_{X_{2}<X_{3}})+\E(X_{2}\cdot 1_{X_{3}<a}\cdot 1_{X_{3}<X_{2}})
\end{multline}


\end{enumerate}


\section{Прочие задачи}

Осторожно! Эти задачи не проверялись на наличие приличного решения.

\begin{enumerate}

\item Сформулируйте и докажите вариант теоремы об одинаковой доходности для следующего случая. Ценности независимы. Каждый игрок знает свою ценность. Распределение ценностей регулярное. На торги выставлены $ k<n $ одинаковых товаров. Каждый из игроков хочет только одну единицу товара.

\item Три игрока. Ценности $ V_{1} $, $ V_{2} $ и $ V_{3} $ равномерны на $ [0;1] $ и независимы. Первый и второй игрок знают значение своих ценностей, то есть $ X_{1}=V_{1} $ и $ X_{2}=V_{2} $. А третий игрок ничего не знает!

Найдите равновесие Нэша на аукционе первой, второй цены и на кнопочном аукциоен.


\item Два игрока. Ценности независимы имеют экспоненциальное распределение с параметром $ \lambda=1 $. Сигналы совпадают с ценностями,  $ X_{i}=V_{i} $. Найдите равновесие Нэша на аукционе первой цены.




\item Аукцион «Все платят среднюю ставку». Покупатели одновременно делают ставки. Товар достаётся тому, кто назвал наибольшую ставку, но платят все игроки. Все игроки платят среднюю суммы ставок. Ценности товара для покупателей имеют независимое регулярное распределение с функцией распределения $ F() $. Найдите оптимальные стратегии игроков, $ b(x) $, средний доход продавца.



\item Сравнение правовых систем.\index{Правовые системы}

Две стороны судятся по спорному вопросу. Выиграет та сторона, которая потратит больше денег на адвокатов. Не считая расходов на адвокатов, для каждой стороны победа в суде приносит выгоду $ X_{i} $. Мы предполагаем, что $ X_{i} $ независимы и равномерны на $ [0;1] $. Предполагаем, что узнав своё $ X_{i} $, каждая сторона решает сколько потратить на адвоката.

Далее мы несколько упрощенно изложим три системы. В американской системе каждая сторона сама оплачивает издержки на адвоката независимо от исхода дела. В европейской системе проигравшая сторона платит все расходы (или фиксированный процент) выигравшей стороны. В системе предложенной Джеймсом Куэйлом (James Quayle, вице-президент США при Буше) проигравшая выплачивает выигравшей стороне сумму равную своим расходам.
\begin{enumerate}
\item Найдите равновесие Нэша в американской системе
\item Найдите равновесие Нэша в европейской системе
\item Найдите равновесие Нэша в систему Куэйла
\item Сравните ожидаемые расходы на адвокатов в разных системах
\end{enumerate}


\item Кнопочный аукцион, 4 игрока, ценность товара для всех одинакова и равна
\begin{equation}
V=X_{1}^{2}+X_{2}^{2}+X_{3}^{2}+X_{4}^{2}
\end{equation}
Каждый игрок знает своё $X_{i}$. Величины $X_{i}$ одинаково распределены и имеют некую совместную функцию плотности.

Найдите равновесие Нэша.

Автор: Руслан Дурдыев

Решение:
\begin{equation}
\begin{cases}
b^{4}(x)=4x^{2} \\
b^{3}(x,p_{4})=3x^{2}+p_{4}/4 \\
b^{2}(x,p_{3},p_{4})=2x^{2}+p_{3}/3-p_{4}/4
\end{cases}
\end{equation}

\item Каждый игрок знает ценность товара для себя, $V_{i}=X_{i}$. Ценности $X_{i}$ независимы и равномерны на $[0,5;1,5]$. Опишите правила аукциона, равновесие Нэша в котором является решением уравнения Бернулли, то есть уравнения вида:
\begin{equation}
b'(x)+a(x)b(x)=c(x)b^{n}(x)
\end{equation}

Автор: Руслан Дурдыев

Решение: Аукцион может, например, проходить по правилам: побеждает тот, кто сделает самую высокую ставку. Победитель платит квадрат своей ставки.

\begin{equation}
b'(x)+\frac{n-1}{2x}b(x)=\frac{n-1}{2}\frac{1}{b(x)}
\end{equation}

Решение уравнения, естественно, $b(x)=\sqrt{\frac{n-1}{n}x}$

\item На аукционе второй цены продаётся товар ценностью $V$ -- единой для всех покупателей, $V$ распределено равномерно на $[1;2]$. Покупатели получают сигналы $X_{i}=V+R_{i}$, где ошибки оценивания $R_{i}$ независимы и равномерны на $[-0.5;0.5]$.

Товар достаётся тому, кто сделает наибольшую ставку, а платит он наибольшую из невыигравших ставок.

\begin{enumerate}
\item Найдите равновесие Нэша и ожидаемую выручку продавца.
\item Пусть есть независимый бесплатный эксперт, к которому может обратиться любой участник. Если игрок обращается к эксперту, то после этого его сигнал равен $X_{i}=V+aR_{i}$, где $R_{i}$ независимы и равномерны на $[-0.5;0.5]$. Величина $a<1$ -- параметр, характеризующуй точность эксперта: чем меньше $a$, тем точнее эксперт.

Игра проходит теперь в два этапа: на первом этапе игроки независимо друг от друга решают, обращаться ли им за советом к эксперту. На втором этапе проходит аукцион второй цены. Ни один игрок не знает, обращались ли другие игроки за советом к эксперту.

\begin{enumerate}
\item Найдите равновесие Нэша, ожидаемую выручку продавца
\item Как изменится ответ, если за совет эксперт взимает плату $d$? Найдите ожидаемую выручку эксперта. Какое значение $d$ максимизирует выручку эксперта при заданном $a$?
\item Что происходит в случае $a=0$, то есть эксперт сообщает точное значение $V$ желающим?
\end{enumerate}

\item Продавец хочет подкупить эксперта. Игроки не знают о подкупе. Сколько готов заплатить продавец эксперту за смещение сигнала, равное $m$?

то есть игра снова проходит в два этапа, на первом этапе игроки выбирают, обращаться ли к экспрерту. Если игрок обращается к эксперту, то его сигнал будет равен $X_{i}=V+m+aR_{i}$, но игрок будет думать, что это $X_{i}=V+aR_{i}$. На втором этапе проводится аукцион второй цены.

\item Если это все хорошо решается, то предложите и решите игру со стратегическим подкупом.

\end{enumerate}

Обработка задачи, предложенной Николаем Ивановым


\item Обобщите правила кнопочного аукциона на случай продажи $k$ одинаковых чудо-швабр в ситуации, когда каждому игроку нужно не больше одной чудо-швабры.


\item У Васи пять потенциальных невест. Каждая из них выбирает свой уровень усилий, который она прикладывает для выхода замуж за Васю. Приложившая наибольшее количество усилий получает Васю замуж. Замужество приносит невесте полезность $X_{i}$, величины $X_{i}$ равномерны на $[0;1]$ и независимы. Быть второй -- самое худшее, что может быть! Вторая по усилиям претендентка страдает от зависти к первой в размере $(-X_{i})$.

Найдите равновесие Нэша.


Обработка задачи, предложенной Марией Алиевой

Цитата: «Из пяти невест выбираются две с самыми большими придаными\ldots»

\item На аукционе с двумя покупателями продаётся машина. Машина с вероятностью $p$ окажется хорошей, а с вероятностью $1-p$ — не очень. До покупки понять, какая она, игроки не могут. Каждый игрок знает свой сигнал $X_{i}$, сигналы независимы и равномерны на $[0;1]$. Ценности определяются по формуле:
\begin{equation}
V_{i}=R\cdot X_{i}
\end{equation}
Где $R$ равна 2 с вероятностью $p$ и 1 c вероятностью $1-p$.

Найдите:
\begin{enumerate}
\item $g(x,y)$, $R(y|x)$, $v(x,y)$
\item Равновесие Нэша на аукционе первой цены, второй цены и на кнопочном аукционе
\end{enumerate}

Обработка задачи, предложенной Вячеславом Савицким

Решение: $b^{FP}(x)=\frac{p+1}{2}x$, $b^{SP}(x)=(p+1)x$


\end{enumerate}



\chapter{Впечатления о курсе}



\section{Трейлер 4-ой кр}

\begin{Large}
Моделирование аукционов. Контрольная работа 4.
\end{Large}

\begin{enumerate}
\item Можно пользоваться калькулятором. Вопрос в том, нужно ли?
\item Можно решать задачи в любом порядке.
\item С собой можно принести один лист А4, где заранее могут быть написаны (именно написаны, а не напечатаны) любые формулы, теоремы или комментарии.
\item Продолжительность работы 1 час 20 минут.
\item Условия нельзя забрать с собой. Условия и решения открыто доступны на \url{auctiontheory.wordpress.com} после окончания контрольной.
\item Обсуждать задачи во время работы нельзя.
\item Человек проводящий контрольную не будет отвечать на вопросы по тексту задач.
\item Скорее всего, в задачах нет очепяток. Если, по твоему мнению, опечатка есть, то ее нужно исправить самому исходя из своего представления о хорошей задаче. При этом нужно четко отразить этот факт перед началом решения. Например, <<По-моему, в тексте есть опечатка и вместо ... должно быть ...>>. Твоя гипотеза об опечатках является личной и не подлежит обсуждению во время работы.
\item Насколько подробно все расписывать --- решай сам исходя из конкретной ситуации. Очевидно, что в примере $ 1+2+3=? $ ответ можно написать сразу, а взятие интеграла $ \int x^{5}\cos(x)dx $ требует каких-то промежуточных записей.
\item Паниковать на контрольной строжайше запрещено!
%\item Каждая из 5 задач весит 5 баллов.
\item Для каждой задачи обязательно нужно спрогнозировать свою оценку. Не надо скромничать, лучше попытаться объективно оценить свое решение.  За неверное оценивание баллы снижаться не будут, а верное оценивание даст возможность чему-то научиться. Опыт показывает, что оценка своих собственных решений позволяет резко улучшить их качество. Прогноз своей оценки пишем в табличку!
\item Не забудь подписать свою работу. Пожалуйста!

\end{enumerate}

\begin{large}
Имя:
\end{large}

\vspace{4pt}

\begin{large}
Отчество:
\end{large}

\vspace{4pt}

\begin{large}
Фамилия:
\end{large}

\vspace{4pt}

\begin{large}
Группа:
\end{large}

\vspace{4pt}

\begin{tabular}{|c|c|c|c|c|c|c|}
\hline  & Задача 1 [5] & Задача 2 [5] & Задача 3 [10] & Задача 4 [5] & Итого \\
\hline Прогноз оценки &  &  &  &  &   \\
\hline Оценка &  &  &  &  &   \\
\hline
\end{tabular}
\newpage

\begin{enumerate}

\item На аукционе участвуют $ n $ игроков. ...... . ...... . . .. ... .......... ........ ........ ........ ........ .......... ........... ............ .......... .......... .......... ...... ........ ....... ....... ..... ......
\begin{enumerate}
\item Найдите $ MR(x) $. ...... ........
\item Постройте оптимальный аукцион.
\end{enumerate}

\item Петя переезжает на новую квартиру, поэтому ..... ......... ........... ........... ........ ........... ............ ......... ............. ............. ............ .......... .......... ........... ........... ......... ........ .......... ........... ...... Потенциальных покупателей двое. Первый покупатель знает значение $ X_{1} $, второй --- значение $ X_{2} $. Величины  $ X_{1} $ и  $ X_{2} $ независимы и равномерны на $ [0;1] $. Полезности первого игрока: ..... ......... ............ .......... ........... .......... .......... ............ Полезности первого игрока: ............ ............. ............. ............. .............
\begin{enumerate}
\item Четко опишите механизм VCG применительно к этой задаче.
\item Какова средняя прибыль продавца при использовании механизма VCG?
\end{enumerate}


\item Есть $ n $ городов. ........ ............... .............. ............ .............. Жители города ........... ............. ............... .............. ................ ........ ......... получат ........... полезность ........ ......... .......... .......... ....... .........  Величины $ X_{i}\sim U[0;1] $ и независимы. Каждый город знает свое $ X_{i} $.
\begin{enumerate}
\item Опишите механизм VCG применительно к этой задаче. Т.е. предполагается, что игроки объявляют числа $ b_{i}\in [0;1] $ и механизм должен определять, ........... ........... ............. .............. ............. ............ ...........
\item Выпишите функцию плотности для .......... ............. ................ ............... ........... .......... ........... ............ .........
\item Сходится ли баланс у механизма VCG в этом случае? Если нет, то сколько в среднем нужно вложить средств извне в этот механизм?
\item Что больше: .......... или ..... .......... .......... в механизме VCG?
\end{enumerate}


%\item Есть $ 3 $ города. Рядом с одним из них нужно построить мусоросжигательный завод. Жители города рядом с которым будет построен завод получат отрицательную полезность $ U_{i}=-X_{i} $. Остальные получат полезность 0. Величины $ X_{i}\sim U[0;1] $ и независимы. Каждый город знает свое $ X_{i} $. Города одновременно называют требуемую компенсацию $ b_{i} $. Завод строится у того города, у которого $ b_{i} $ меньше. Остальные города выплачивают компенсацию поровну.
%\begin{enumerate}
%\item Найдите равновесие Нэша
%\item Как надо изменить этот механизм, чтобы он стал правдивым?
%\end{enumerate}



%\item Есть один покупатель и один продавец. Ценности товара: $ X_{1} $ --- для покупателя, $ X_{2} $ --- для продавца. Величины $ X_{i} $ независимы и равномерны на $ [0;1] $.
%\begin{enumerate}
%\item Опишите механизм VCG применительно к этой задаче. Т.е. предполагается, что игроки объявляют числа $ b_{i}\in [0;1] $ и механизм должен определять, кому отдать товар и какие платежи должны сделать игроки в зависимости от $ b_{i} $.
%\item Каков средний баланс механизма VCG в этой задаче?
%\item Предположим, что вместо VCG используется такой механизм: игроки одновременно называют желаемые цены, $ b_{1} $ и $ b_{2} $. Если $ b_{1}>b_{2} $, то обмен происходит по цене $ 0.5(b_{1}+b_{2}) $. Найдите равновесие Нэша.
%\item Верно ли, что при втором механизме обмен происходит если и только если $ X_{1}>X_{2} $?
%\end{enumerate}

%Подсказка: равновесие Нэша будет в линейных стратегиях



\item Кнопочный аукцион и три игрока. Ценности $ V_{1} $, $ V_{2} $ и $ V_{3} $ ........ ........ ............. ............. ............. .............. ........... .......... ........... .......... ........... ........... ......... ........... .......
\begin{enumerate}
\item Что собой представляют стратегии игроков в этом случае? Почему их можно упростить?
\item Найдите равновесие Нэша
\end{enumerate}
\end{enumerate}




\printindex % печать предметного указателя здесь


\bibliography{opit}




\end{document}
