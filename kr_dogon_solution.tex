\begin{enumerate}
\item
\item
\item

\item Количество чудо-швабр обозначим буквой $k$.

$ k=1 $: $ b^{3}(x)=3x $, $ b^{2}(x,p_{3})=2x+p/3$

$ k=2 $. Поскольку аукцион заканчивается при выходе первого игрока, то стратегия определяется функцией $ b^{3}(x)$.

Поскольку мы такой аукцион не решали, то используем стандартный подход с максимизацией прибыли:
\begin{equation}
E(Profit_{1}|X_{1}=x,Bid_{1}:=b_{1})=E((X_{1}+X_{2}+X_{3}-b(Y_{2}))1_{b_{1}>b(Y_{2})}|X_{1}=x,Bid_{1}:=b_{1})
\end{equation}

Чудо-замена $b_{1}=b(a)$ и независимость $ X_{i} $ дают нам:
\begin{equation}
\pi_{1}(x,b(a))=E((x+X_{2}+X_{3})-b(Y_{2}))1_{a>Y_{2}})=xP(Y_{2}<a)+2E(X_{2}\cdot 1_{Y_{2}<a})-E(b(Y_{2})\cdot 1_{Y_{2}<a})
\end{equation}

Сосредоточимся на $ E(X_{2}\cdot 1_{Y_{2}<a}) $:
\begin{equation}
E(X_{2}\cdot 1_{Y_{2}<a})=E(X_{2}\cdot 1_{X_{2}\wedge X_{3}<a})=E(X_{2}\cdot 1_{X_{2}<a}\cdot 1_{X_{2}<X_{3}})+E(X_{2}\cdot 1_{X_{3}<a}\cdot 1_{X_{3}<X_{2}})
\end{equation}

\item


\end{enumerate}
