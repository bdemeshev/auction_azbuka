\begin{enumerate}

\item[4.] Количество чудо-швабр обозначим буквой $k$. \index{задача!о продаже чудо-швабр}

$ k=1 $: $ b^{3}(x)=3x $, $ b^{2}(x,p_{3})=2x+p/3$

$ k=2 $. Поскольку аукцион заканчивается при выходе первого игрока, то стратегия определяется функцией $ b^{3}(x)$.

Поскольку мы такой аукцион не решали, то используем стандартный подход с максимизацией прибыли:
\begin{multline}
\E(Profit_{1}|X_{1}=x,Bid_{1}:=b_{1})=\\
=\E((X_{1}+X_{2}+X_{3}-b(Y_{2}))1_{b_{1}>b(Y_{2})}|X_{1}=x,Bid_{1}:=b_{1})
\end{multline}

Чудо-замена $b_{1}=b(a)$ и независимость $ X_{i} $ дают нам:
\begin{multline}
\pi_{1}(x,b(a))=\E((x+X_{2}+X_{3})-b(Y_{2}))1_{a>Y_{2}})=\\
=x\P(Y_{2}<a)+2\E(X_{2}\cdot 1_{Y_{2}<a})-\E(b(Y_{2})\cdot 1_{Y_{2}<a})
\end{multline}

Сосредоточимся на $\E(X_{2}\cdot 1_{Y_{2}<a}) $:
\begin{multline}
\E(X_{2}\cdot 1_{Y_{2}<a})=\E(X_{2}\cdot 1_{X_{2}\wedge X_{3}<a})=\\
=\E(X_{2}\cdot 1_{X_{2}<a}\cdot 1_{X_{2}<X_{3}})+\E(X_{2}\cdot 1_{X_{3}<a}\cdot 1_{X_{3}<X_{2}})
\end{multline}


\end{enumerate}
