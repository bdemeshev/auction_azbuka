\begin{enumerate}

\item Техническая задача.
\begin{enumerate}
\item Известно, что $ f(\vec{x}) $ и $ g(\vec{x}) $ — супермодулярные функции, а $ a>0 $ и $ b>0 $ — константы. Верно ли, что $ af(\vec{x})+bg(\vec{x}) $ — супермодулярная функция?

\item Пусть $ X_{1} $,\ldots, $ X_{n} $ независимы и имеют функцию плотности $ f(t)=3t^{2} $ на $ [0;1] $. Случайные $ Y_{1} $, \ldots, $ Y_{n-1} $ — это есть упорядоченные по убыванию случайные величины $ X_{2} $, \ldots, $ X_{n} $. С помощью о-малых или без неё найдите совместную функцию плотности $ f_{Y_{5},Y_{10}}(a,b) $.
\end{enumerate}

Следующие две задачи очень похожи, разница в них только в типе аукциона\ldots

\item На аукционе первой цены продаётся участок. Потенциальных покупателей двое. Ценность участка для каждого игрока определяется его площадью. Первый покупатель знает ширину участка $ X_{1} $, а второй — длину $X_{2}$. Совместная фукнция плотности $ X_{1} $ и $ X_{2} $ имеет вид $ f(x_{1},x_{2})=\frac{7}{8}+\frac{1}{2}x_{1}x_{2} $. Найдите дифференциальное уравнение, которому подчиняется равновесная стратегия игрока.

\item На аукционе второй цены продаётся участок. Потенциальных покупателей двое. Ценность участка для каждого игрока определяется его площадью. Первый покупатель знает ширину участка $ X_{1} $, а второй — длину $X_{2}$. Совместная фукнция плотности $ X_{1} $ и $ X_{2} $ имеет вид $ f(x_{1},x_{2})=\frac{7}{8}+\frac{1}{2}x_{1}x_{2} $. Найдите равновесие Нэша.


\item Найдите равновесие Нэша в случае кнопочного аукциона. Сигналы $ X_{i} $ игроков имеют совместную функцию плотности $ f(x_{1},x_{2},x_{3})=7/8+x_{1}x_{2}x_{3} $ при $ x_{1},x_{2},x_{3}\in[0;1] $. Ценности определяеются по формулам:
\begin{equation}
\begin{array}{c}
V_{1}=X_{1}(X_{2}+X_{3}) \\
V_{2}=X_{2}(X_{1}+X_{3}) \\
V_{3}=X_{3}(X_{1}+X_{2})
\end{array}
\end{equation}


\item Ценности игроков одинаково распределены, независимы, распределение ценностей дискретно: $ X_{i}$ равновероятно принимает натуральное значение от 1 до 100 включительно. Игроки одновременно делают ставки. Значения всех ценностей общеизвестны всем игрокам ещё до ставок! Разрешаются только целые неотрицательные ставки. Товар достаётся игроку, сделавшему наивысшую ставку. Если таких игроков несколько, то победитель выбирается из них равновероятно. Победитель платит сделанную им ставку. Найдите хотя бы одно равновесие Нэша в чистых стратегиях, в котором игроки не используют нестрого доминируемых стратегий.


\end{enumerate}
