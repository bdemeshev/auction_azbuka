\begin{enumerate}


\item
\begin{equation}
(a+c)\vee (b+c)=a\vee b +c
\end{equation}

\begin{equation}
(a+c)\wedge (b+c)=a\wedge b +c
\end{equation}

Да, набор $ Z_{1} $, \ldots , $ Z_{n} $, $ W_{1} $, \ldots , $ W_{k} $ аффилирован. В силу независимости логарифм совместной функции плотности разлагается в сумму логарифмов:
\begin{equation}
\ln f_{Z,W}(z_{1},\ldots ,z_{n},w_{1},\ldots ,w_{k})= \ln f_{Z}(z_{1},\ldots ,z_{n})+\ln f_{W}(w_{1},\ldots ,w_{k})
\end{equation}
И смешанные производные равны либо нулю, либо неотрицательны в силу аффилированности $ Z_{i} $ между собой и $ W_{i} $ между собой.

\item  Поскольку игроков всего двое, то $ g(x,y)$ — это просто совместная функция плотности $ X_{1} $ и $ X_{2} $.

Находим условную совместную плотность:
\begin{equation}
p(x_{1},x_{2}|v)=1, \quad x_{1},x_{2}\in [v-0.5;v+0.5]
\end{equation}

Значит:
\begin{equation}
p(x_{1},x_{2},v)=1, \quad x_{1},x_{2}\in [v-0.5;v+0.5],v\in [1;2]
\end{equation}

Заметим, что область, где плотность положительна, можно описать условием:
\begin{equation}
v\in [(x_{1}-0.5)\vee (x_{2}-0.5); (x_{1}+0.5)\wedge (x_{2}+0.5)]=[v_{min};v_{max}]
\end{equation}

Интегрируем по $ v $ и получаем:
\begin{equation}
p(x_{1},x_{2})=\int_{v_{min}}^{v_{max}} 1 dv= v_{max}-v_{min}=x_{1}\wedge x_{2}-x_{1}\vee x_{2}+1
\end{equation}

\begin{multline}
\E(V|X_{1}=x_{1}, X_{2}=x_{2})=\int v p(v|x_{1},x_{2})dv=\\
\int v \frac{p(x_{1},x_{2},v)}{p(x_{1},x_{2})} dv=\frac{\int v p(x_{1},x_{2},v) dv }{p(x_{1},x_{2})}
\end{multline}

В числителе:
\begin{equation}
\int_{v_{min}}^{v_{max}}vdv=\frac{v_{max}^{2}-v_{min}^{2}}{2}
\end{equation}

Значит, в итоге:
\begin{equation}
v(x_{1},x_{2})=\frac{v_{max}^{2}-v_{min}^{2}}{2\cdot (v_{max}-v_{min})}=\frac{v_{max}+v_{min}}{2}= \frac{x_{1}\wedge x_{2}+x_{1}\vee x_{2}}{2}
\end{equation}

Равновесие Нэша на аукционе второй цены:
\begin{equation}
v(x,x)=x
\end{equation}


\item Игроков всего два, значит, $ g(x,y) $ — просто совместная функция плотности $ X_{1} $ и $ X_{2} $.

\begin{equation}
p(x_{1},x_{2}|s)=1\cdot 1, \quad x_{1},x_{2}\in [s;s+1]
\end{equation}

Следовательно:
\begin{equation}
p(x_{1},x_{2},s)=1\cdot 1, \quad x_{1},x_{2}\in [s;s+1], s\in [0;1]
\end{equation}

Заметим, что область, где плотность положительна, можно описать условием:
\begin{equation}
s\in [x_{1}\vee x_{2}-1; x_{1}\wedge x_{2}]=[s_{min};s_{max}]
\end{equation}

Интегрируем по $ s $ и получаем:
\begin{equation}
p(x_{1},x_{2})=\int_{s_{min}}^{s_{max}} 1 ds= s_{max}-s_{min}=x_{1}\wedge x_{2}-x_{1}\vee x_{2}+1
\end{equation}

Плотность обращается в ноль за пределами участка $ 0\leq x_{1},x_{2}\leq 2 $, $ x_{1}-1\leq x_{2} \leq x_{1}+1 $.

Чтобы найти $ R(y|x) $ вспоминаем что это такое:
\begin{equation}
R(y|x)=\frac{g(x,y)}{\int_{0}^{y}g(x,t)dt}
\end{equation}

Возникает четыре случая для $ R(y|x) $\ldots

%\begin{equation}
%R(y|x)=
%\begin{cases}
%\frac{1+y-x}{y+0.5y^{2}-xy}, \quad x<1,y<x \\
%\frac{1+x-y}{-2x^{2}+y+yx-0.5y^{2}}, \quad x<1,y>x \\
%\frac{1+y-x}{y+0.5y^{2}-xy+0.5x^{2}+0.5-x}, \quad x>1,y<x \\
%\frac{1+x-y}{0.5+y+yx-0.5y^{2}-x-x^{2}+0.5x^{2}}, \quad x>1,y>x
%\end{cases}
%\end{equation}

%Нам нужна $ R(x|x) $. Находим, что $ g(x,x)=1 $, и $ g(x,t)=t-x+1 $ при $ t<x $ .

%\begin{equation}
%R(x|x)=
%\begin{cases}
%\frac{1}{x-0.5x^{2}}, \quad x<1 \\
%2, \quad x>1
%\end{cases}
%\end{equation}

К сожалению, в явном виде хорошего мало. Стандартная максимизация с чудо-заменой дает дифференциальное уравнение:
\begin{equation}
(0.8x-b'(x))\int_{0}^{x}p(x,x_{2})dx_{2}+x-b(x)=0
\end{equation}

Возникает два случая из-за ломаной $ p(x_{1},x_{2}) $\ldots

Если $ x\in [0;1] $, то:
\begin{equation}
(0.8x-b'(x))\cdot (x-0.5x^{2})+x-b(x)=0
\end{equation}
Из этого уравнения надо выбрать решение с $ b(0)=0 $.

Если $ x\in [1;2] $, то:
\begin{equation}
(0.8x-b'(x))\cdot 0.5+x-b(x)=0
\end{equation}
Из этого уравнения надо выбрать решение непрерывно склеивающееся с первым в точке $x=1$.


Находим $ v(x,y) $:
\begin{equation}
v(x,y)=\E(V_{1}|X_{1}=x,Y_{1}=y)=\E(V_{1}|X_{1}=x,X_{2}=y)=0.8x+0.2y
\end{equation}

Равновесие Нэша на аукционе второй цены:
\begin{equation}
b(x)=v(x,x)=x
\end{equation}
Кнопочный аукцион совпадает с аукционом второй цены.


\item В решении контрольной 3 мы получили результат:
\begin{equation}
v(x,y)=\frac{4xy^{n-1}}{1+4xy^{n-1}}
\end{equation}

Следовательно, равновесие Нэша на аукционе второй цены:
\begin{equation}
b(x)=v(x,x)=\frac{4x^{n}}{1+4x^{n}}
\end{equation}

Можно отметить, что функция растет с ростом $ x $ и падает с ростом $ n $.

Теперь рассмотрим $ A=\{X_{1}\in[x_{1};x_{1}+\Delta] \cap \ldots  \cap X_{n}\in[x_{n};x_{n}+\Delta]\} $. Как и в решении задачи с контрольной:
\begin{multline}
\E(V|A)=\P(V=1|A)=\frac{\P(V=1 \cap A)}{\P(A)}=\\
=\frac{\P(A|V=1)\cdot \P(V=1)}{\P(A)}=\frac{0.5\P(A|V=1)}{\P(A)}
\end{multline}

Согласно методу о-малых аналогичная формула справедлива для плотностей:
\begin{multline}
\E(V|X_{1}=x_{1},X_{2}=x_{2},\ldots ,X_{n}=x_{n})=\\
=\frac{0.5\cdot 2^{n}\Pi_{i=1}^{n}x_{i}}{0.5+0.5\cdot 2^{n}\Pi_{i=1}^{n}x_{i}}
=\frac{2^{n}\Pi_{i=1}^{n}x_{i}}{1+ 2^{n}\Pi_{i=1}^{n}x_{i}}
\end{multline}

Теперь частично находим стратегию на кнопочном аукционе:
\begin{equation}
b^{n}(x)=\frac{2^{n}x^{n}}{1+2^{n}x^{n}}
\end{equation}

Если все игроки используют эту функцию, то чтобы игрок вышел на цене $ p $ ценность должна равняться:
\begin{equation}
x=\frac{1}{2}\left(\frac{p}{1-p} \right)^{1/n}
\end{equation}

Подставляя один такой $ x $ в ожидаемую ценность получаем:
\begin{equation}
b^{n-1}(x,p_{n})=\frac{2^{n-1}x^{n-1}\left(\frac{p_{n}}{1-p_{n}} \right)^{1/n}}{1+2^{n-1}x^{n-1}\left(\frac{p_{n}}{1-p_{n}} \right)^{1/n}}
\end{equation}

Если второй выходит на цене $ p_{n-1} $, то его ценность была равна:
\begin{equation}
x=\frac{1}{2}\left(\frac{p_{n}}{1-p_{n}} \right)^{1/n(n-1)}\left(\frac{p_{n-1}}{1-p_{n-1}} \right)^{1/(n-1)}
\end{equation}

Значит:
\begin{equation}
b^{n-2}(x,p_{n-1},p_{n})=\frac{2^{n-2}x^{n-2}\left(\frac{p_{n}}{1-p_{n}} \right)^{1/n(n-1)}\left(\frac{p_{n-1}}{1-p_{n-1}} \right)^{1/(n-1)}}{1+2^{n-2}x^{n-2}\left(\frac{p_{n}}{1-p_{n}} \right)^{1/n(n-1)}\left(\frac{p_{n-1}}{1-p_{n-1}} \right)^{1/(n-1)}}
\end{equation}


%\item
% Хм: страшные интегралы лезут отовсюду..
%Товар имеет общую ценность $ V $ для трёх игроков, $ V $ равномерно на $ [0;1] $. При фиксированном $ V=v $ сигналы независимы и имеют функцию плостности:
%\begin{equation}
%f(x|v)=2v\cdot x^{2v-1}, \quad x\in[0;1]
%\end{equation}
%\begin{enumerate}
%\item Какой ожидаемый сигнал получают игроки, если $ V=0 $? $ V=0.5 $? $ V=1 $?
%\item Найдите $ g(x,y) $
%\item Найдите $ v(x,y)=\E(V|X_{1}=x,Y_{1}=y) $ и равновесие Нэша на аукционе второй цены
%\item Найдите $ \E(V|X_{1}=x_{1},X_{2}=x_{2},X_{3}=x_{3} )$ и равновесие Нэша на кнопочном аукционе
%\end{enumerate}

%Подсказка: Можно пользоваться тем, что $  $


\end{enumerate}
