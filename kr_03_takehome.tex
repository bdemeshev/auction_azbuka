\begin{enumerate}


\item Техническая задача.
\begin{enumerate}
\item Выразите $ (a+c)\vee (b+c) $ через $ a\vee b $. Выразите $ (a+c)\wedge (b+c) $ через $ a\wedge b $.
\item Случайные величины $ Z_{1} $, \ldots , $ Z_{n} $ аффилированы между собой. Случайные величины $ W_{1} $, \ldots , $ W_{k} $ — аффилированы между собой. Набор случайных величин $ Z_{1} $, \ldots , $ Z_{n} $ не зависит от набора $ W_{1} $, \ldots , $ W_{k} $. Верно ли, что набор случайных величин $ Z_{1} $, \ldots , $ Z_{n} $, $ W_{1} $, \ldots , $ W_{k} $ аффилирован?
\end{enumerate}


\item Пусть $  V $ — общая ценность товара для двух игроков, равномерна на $ [1;2] $. Величины $ R_{1} $ и $ R_{2} $ — независимы между собой и с $ V $ и равномерны на $ [-0.5;0.5] $. По смыслу: $ R_{1} $ и $ R_{2} $ — это ошибки игроков при подсчете ценности товара $ V $. Игроки получают сигналы $ X_{i}=V+R_{i} $, то есть игроки знают ценность  $ V $ с ошибкой.
\begin{enumerate}
\item Найдите совместную функцию плотности $ X_{1} $ и $ X_{2} $. Верно ли, что $ X_{1} $ и $ X_{2} $ аффилированны?
\item Найдите $ v(x,y)=\E(V|X_{1}=x,Y_{1}=y) $. Найдите равновесие Нэша на аукционе второй цены.
\item Найдите совместную функцию плотности $ X_{1} $ и $ Y_{1} $, $ g(x,y) $
\end{enumerate}

Подсказка: В решении контрольной есть похожая задача. А $ g(x,y) $ можно неплохо упростить пользуясь предыдущей задачей.

Поскольку игроков всего двое, то $ g(x,y)$ — это просто совместная функция плотности $ X_{1} $ и $ X_{2} $.


\item Пусть $ R_{1} $, $ R_{2} $ и $ S $ — равномерны на $ [0;1] $ и независимы. Ценность товара для первого игрока, $ V_{1}=0.8X_{1}+0.2X_{2} $ и для второго — $ V_{2}=0.8X_{2}+0.2X_{1} $. Первый игрок получает сигнал $ X_{1}=S+R_{1} $. Второй игрок получает сигнал $ X_{2}=S+R_{2} $.
\begin{enumerate}
\item Найдите $ g(x,y) $, $ R(y|x) $ и $ v(x,y)=\E(V|X_{1}=x,Y_{1}=y) $
\item Используя предыдущие функции найдите равновесие Нэша на аукционе второй цены, первой цены и кнопочном аукционе
\end{enumerate}

% ??? (занудно) В рамках предыдущей задачи найдите $ g(y|x) $, $ G(y|x) $, $ R(y|x) $, $ R(x|x) $. Найдите равновесие Нэша на аукционе первой цены и кнопочном аукционе.

%Подсказка: Обратите внимание, что в лекции $ X_{i} $ принимает значения на $ [0;1] $. Здесь $ X_{i} \in [0.5;2.5] $, поэтому при подсчете $ G(y|x) $ нижний предел интегрирования не равен 0.


\item Продолжение задачи 2 с контрольной (можно использовать все полученные в ней результаты).
На аукционе продаётся картина, которая равновероятно является «Джокондой» Леонардо да Винчи или её подделкой. За неё торгуются $ n $ покупателей. Ценность картины для всех покупателей одинакова, $ V_{1}=V_{2}=\ldots =V_{n}=V $ и равна 1, если это оригинал и 0, если подделка.

Если $ V=0 $, то сигналы $ X_{i} $ условно независимы и равномерны на $ [0;1] $. Если $ V=1 $, то сигналы $ X_{i} $ условно независимы и имеют функцию плотности $ f(x|V=1)=2x $ при  $x\in [0;1] $

\begin{enumerate}
\item Найдите равновесие Нэша на аукционе второй цены
\item Найдите $ \E(V|X_{1}=x_{1},X_{2}=x_{2},X_{3}=x_{3}\ldots X_{n}=x_{n}) $
\item С помощью предыдущего пункта найдите функции $ b^{n}(x) $,  $ b^{n-1}(x,p_{n}) $  и $ b^{n-2}(x,p_{n-1},p_{n}) $ в равновесии Нэша на кнопочном аукционе
\end{enumerate}


%\item
% Хм: страшные интегралы лезут отовсюду..
%Товар имеет общую ценность $ V $ для трёх игроков, $ V $ равномерно на $ [0;1] $. При фиксированном $ V=v $ сигналы независимы и имеют функцию плостности:
%\begin{equation}
%f(x|v)=2v\cdot x^{2v-1}, \quad x\in[0;1]
%\end{equation}
%\begin{enumerate}
%\item Какой ожидаемый сигнал получают игроки, если $ V=0 $? $ V=0.5 $? $ V=1 $?
%\item Найдите $ g(x,y) $
%\item Найдите $ v(x,y)=\E(V|X_{1}=x,Y_{1}=y) $ и равновесие Нэша на аукционе второй цены
%\item Найдите $ \E(V|X_{1}=x_{1},X_{2}=x_{2},X_{3}=x_{3} )$ и равновесие Нэша на кнопочном аукционе
%\end{enumerate}

%Подсказка: Можно пользоваться тем, что $  $


\end{enumerate}
