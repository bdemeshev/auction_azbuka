

\begin{enumerate}
\item Предположим, что условия теоремы об одинаковых доходностях выполнены.
\begin{enumerate}
\item Может ли выбор механизма проведения аукциона влиять на ковариацию выплат двух разных игроков?
\item  Найдите ковариацию выплат первого и второго игрока в аукционе первой цены с независимыми и равномерными на $ [0;1] $ ценностями. Подсказка: можно пользоваться тем, что средняя выплата равна $ \frac{n-1}{n(n+1)} $.
\end{enumerate}


\item «Наследство»\index{Наследство} по типу аукциона второй цены. Двум сыновьям достался земельный участок в наследство. Отец не хотел, чтобы участок был разделен, поэтому по завещанию установлены следущие правила: два брата одновременно делают ставки. Участок получает тот, кто сделал большую ставку. При этом получивший участок выплачивает проигравшему меньшую из двух ставок. Ценности участка для игроков независимы и равномерны на $ [0;1] $.

Найдите равновесие Нэша.


\item Рассмотрим аукцион второй цены. Предположим, что ценности независимы и имеют регулярное распределение. Агенты не нейтральны к риску. Их отношение к риску отражается функцией полезности $ u() $. Про $ u() $ известно, что она непрерывна, строго возрастает и для удобства $ u(0)=0 $. то есть если игрок получает товар ценностью $ x $ и платит продавцу $ m $, то его полезность равна $ u(x-m) $.

Найдите равновесие Нэша.



\item Рассмотрим аукцион второй цены с резервной ставкой $ r $. Резервная ставка — это минимальная цена за которую продавец согласен расстаться с товаром. Если все игроки сделали ставки ниже $ r $, товар остаётся у продавца, никто ничего не платит. Если хотя бы один игрок сделал ставку выше $ r $, то товар достаётся игроку сделавшему самую высокую ставку и платит он максимум между второй по величине ставкой и $ r $. Константа $ r $ общеизвестна всем игрокам. Ценности независимы и имеют регулярное распределение. Агенты нейтральны к риску.

Найдите равновесие Нэша.


\item Рассмотрим аукцион первой цены с двумя игроками. Ценности независимы и равномерны на $ [0;1] $. Но ставку можно сделать только 0 или 0.5. Если ставки игроков совпали, то товар достаётся случайно выбираемому игроку за соответствующую плату.

Найдите равновесие Нэша.

\end{enumerate}
