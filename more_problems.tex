\begin{enumerate}

\item Сформулируйте и докажите вариант теоремы об одинаковой доходности для следующего случая. Ценности независимы. Каждый игрок знает свою ценность. Распределение ценностей регулярное. На торги выставлены $ k<n $ одинаковых товаров. Каждый из игроков хочет только одну единицу товара.

\item Три игрока. Ценности $ V_{1} $, $ V_{2} $ и $ V_{3} $ равномерны на $ [0;1] $ и независимы. Первый и второй игрок знают значение своих ценностей, то есть $ X_{1}=V_{1} $ и $ X_{2}=V_{2} $. А третий игрок ничего не знает!

Найдите равновесие Нэша на аукционе первой, второй цены и на кнопочном аукциоен.


\item Два игрока. Ценности независимы имеют экспоненциальное распределение с параметром $ \lambda=1 $. Сигналы совпадают с ценностями,  $ X_{i}=V_{i} $. Найдите равновесие Нэша на аукционе первой цены.




\item Аукцион «Все платят среднюю ставку». Покупатели одновременно делают ставки. Товар достаётся тому, кто назвал наибольшую ставку, но платят все игроки. Все игроки платят среднюю суммы ставок. Ценности товара для покупателей имеют независимое регулярное распределение с функцией распределения $ F() $. Найдите оптимальные стратегии игроков, $ b(x) $, средний доход продавца.



\item Сравнение правовых систем.\index{Правовые системы}

Две стороны судятся по спорному вопросу. Выиграет та сторона, которая потратит больше денег на адвокатов. Не считая расходов на адвокатов, для каждой стороны победа в суде приносит выгоду $ X_{i} $. Мы предполагаем, что $ X_{i} $ независимы и равномерны на $ [0;1] $. Предполагаем, что узнав своё $ X_{i} $, каждая сторона решает сколько потратить на адвоката.

Далее мы несколько упрощенно изложим три системы. В американской системе каждая сторона сама оплачивает издержки на адвоката независимо от исхода дела. В европейской системе проигравшая сторона платит все расходы (или фиксированный процент) выигравшей стороны. В системе предложенной Джеймсом Куэйлом (James Quayle, вице-президент США при Буше) проигравшая выплачивает выигравшей стороне сумму равную своим расходам.
\begin{enumerate}
\item Найдите равновесие Нэша в американской системе
\item Найдите равновесие Нэша в европейской системе
\item Найдите равновесие Нэша в систему Куэйла
\item Сравните ожидаемые расходы на адвокатов в разных системах
\end{enumerate}


\item Кнопочный аукцион, 4 игрока, ценность товара для всех одинакова и равна
\begin{equation}
V=X_{1}^{2}+X_{2}^{2}+X_{3}^{2}+X_{4}^{2}
\end{equation}
Каждый игрок знает своё $X_{i}$. Величины $X_{i}$ одинаково распределены и имеют некую совместную функцию плотности.

Найдите равновесие Нэша.

Автор: Руслан Дурдыев

Решение:
\begin{equation}
\begin{cases}
b^{4}(x)=4x^{2} \\
b^{3}(x,p_{4})=3x^{2}+p_{4}/4 \\
b^{2}(x,p_{3},p_{4})=2x^{2}+p_{3}/3-p_{4}/4
\end{cases}
\end{equation}

\item Каждый игрок знает ценность товара для себя, $V_{i}=X_{i}$. Ценности $X_{i}$ независимы и равномерны на $[0,5;1,5]$. Опишите правила аукциона, равновесие Нэша в котором является решением уравнения Бернулли, то есть уравнения вида:
\begin{equation}
b'(x)+a(x)b(x)=c(x)b^{n}(x)
\end{equation}

Автор: Руслан Дурдыев

Решение: Аукцион может, например, проходить по правилам: побеждает тот, кто сделает самую высокую ставку. Победитель платит квадрат своей ставки.

\begin{equation}
b'(x)+\frac{n-1}{2x}b(x)=\frac{n-1}{2}\frac{1}{b(x)}
\end{equation}

Решение уравнения, естественно, $b(x)=\sqrt{\frac{n-1}{n}x}$

\item На аукционе второй цены продаётся товар ценностью $V$ -- единой для всех покупателей, $V$ распределено равномерно на $[1;2]$. Покупатели получают сигналы $X_{i}=V+R_{i}$, где ошибки оценивания $R_{i}$ независимы и равномерны на $[-0.5;0.5]$.

Товар достаётся тому, кто сделает наибольшую ставку, а платит он наибольшую из невыигравших ставок.

\begin{enumerate}
\item Найдите равновесие Нэша и ожидаемую выручку продавца.
\item Пусть есть независимый бесплатный эксперт, к которому может обратиться любой участник. Если игрок обращается к эксперту, то после этого его сигнал равен $X_{i}=V+aR_{i}$, где $R_{i}$ независимы и равномерны на $[-0.5;0.5]$. Величина $a<1$ -- параметр, характеризующуй точность эксперта: чем меньше $a$, тем точнее эксперт.

Игра проходит теперь в два этапа: на первом этапе игроки независимо друг от друга решают, обращаться ли им за советом к эксперту. На втором этапе проходит аукцион второй цены. Ни один игрок не знает, обращались ли другие игроки за советом к эксперту.

\begin{enumerate}
\item Найдите равновесие Нэша, ожидаемую выручку продавца
\item Как изменится ответ, если за совет эксперт взимает плату $d$? Найдите ожидаемую выручку эксперта. Какое значение $d$ максимизирует выручку эксперта при заданном $a$?
\item Что происходит в случае $a=0$, то есть эксперт сообщает точное значение $V$ желающим?
\end{enumerate}

\item Продавец хочет подкупить эксперта. Игроки не знают о подкупе. Сколько готов заплатить продавец эксперту за смещение сигнала, равное $m$?

то есть игра снова проходит в два этапа, на первом этапе игроки выбирают, обращаться ли к экспрерту. Если игрок обращается к эксперту, то его сигнал будет равен $X_{i}=V+m+aR_{i}$, но игрок будет думать, что это $X_{i}=V+aR_{i}$. На втором этапе проводится аукцион второй цены.

\item Если это все хорошо решается, то предложите и решите игру со стратегическим подкупом.

\end{enumerate}

Обработка задачи, предложенной Николаем Ивановым


\item Обобщите правила кнопочного аукциона на случай продажи $k$ одинаковых чудо-швабр в ситуации, когда каждому игроку нужно не больше одной чудо-швабры.


\item У Васи пять потенциальных невест. Каждая из них выбирает свой уровень усилий, который она прикладывает для выхода замуж за Васю. Приложившая наибольшее количество усилий получает Васю замуж. Замужество приносит невесте полезность $X_{i}$, величины $X_{i}$ равномерны на $[0;1]$ и независимы. Быть второй -- самое худшее, что может быть! Вторая по усилиям претендентка страдает от зависти к первой в размере $(-X_{i})$.

Найдите равновесие Нэша.


Обработка задачи, предложенной Марией Алиевой

Цитата: «Из пяти невест выбираются две с самыми большими придаными\ldots»

\item На аукционе с двумя покупателями продаётся машина. Машина с вероятностью $p$ окажется хорошей, а с вероятностью $1-p$ — не очень. До покупки понять, какая она, игроки не могут. Каждый игрок знает свой сигнал $X_{i}$, сигналы независимы и равномерны на $[0;1]$. Ценности определяются по формуле:
\begin{equation}
V_{i}=R\cdot X_{i}
\end{equation}
Где $R$ равна 2 с вероятностью $p$ и 1 c вероятностью $1-p$.

Найдите:
\begin{enumerate}
\item $g(x,y)$, $R(y|x)$, $v(x,y)$
\item Равновесие Нэша на аукционе первой цены, второй цены и на кнопочном аукционе
\end{enumerate}

Обработка задачи, предложенной Вячеславом Савицким

Решение: $b^{FP}(x)=\frac{p+1}{2}x$, $b^{SP}(x)=(p+1)x$


\end{enumerate}
